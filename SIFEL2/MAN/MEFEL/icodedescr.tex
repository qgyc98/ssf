%%%%%%%%%%%%%%%%%%%%%%%%%%%%%%%%%%%%%%%%%%%%%%%%%%%%%%%%%%%%%%%%%%%%%%%%%%%%%%%%%%%%%%%%%%%%%%%%%%%%%
%%%%%%%%%%%%%%%%%%%%%%%%%%%%%%%%%%%%%%%%%%%%%%%%%%%%%%%%%%%%%%%%%%%%%%%%%%%%%%%%%%%%%%%%%%%%%%%%%%%%%
\chapter{Description of the code}
%%%%%%%%%%%%%%%%%%%%%%%%%%%%%%%%%%%%%%%%%%%%%%%%%%%%%%%%%%%%%%%%%%%%%%%%%%%%%%%%%%%%%%%%%%%%%%%%%%%%%%%
%%%%%%%%%%%%%%%%%%%%%%%%%%%%%%%%%%%%%%%%%%%%%%%%%%%%%%%%%%%%%%%%%%%%%%%%%%%%%%%%%%%%%%%%%%%%%%%%%%%

\section{Allocation and Initiation}

The MEFEL starts with function mefel\_init, where all data are read
from input file or files and many of objects and arrays are allocated.

The main objects of the code are allocated at the beginning
of the function mefel\_init.
\begin{center}
\begin{tabular}{ll}
Mp  = new probdesc;  & problem description
\\
Gtm = new gtopology; & general topology
\\
Mt  = new mechtop; & mechanical topology
\\
Mm  = new mechmat; & mechamical materials
\\
Mc  = new mechcrsec; & mechanical cross sections
\\
Mb  = new mechbclc; & mechanical boundary conditions
\\
Outdm = new outdriverm; & description of output
\end{tabular}
\end{center}

\begin{itemize}
\item
Mp--$>$read (in); - reads problem description

\item
Mt--$>$read (in); - reads mechanical topology

\item
Ndofm = Gtm--$>$codenum\_generation (Out); - generates code numbers

\item
Mm--$>$read (in); - reads mechanical materials
	
\item
Mm--$>$init\_ip\_1 (); - initiates variables on integration points, namely, stress/strain state,
number of strain/stress components

\item
Mc--$>$read (in); - reads mechanical cross sections

\item
Mb--$>$read (in); - reads mechanical boundary conditions, it means
prescribed displacements, forces, moments

\item
Lsrs--$>$alloc (); - allocates arrays lhs (array with solution),
rhs (array with right hand side of the system of equations),
w (array containing eigenvalues)
 
\item
Lsrs--$>$initcond (in); - reads initial conditions

\item
Outdm--$>$read(in); - reads description of output from the code, namely description
of output to the output file, file for graphical postprocessors (e.g. GiD) and
diagrams

\item
Mm--$>$init\_ip\_2 (); - allocates arrays on integration points, namely array
for strains, stresses, other values, nonlocal values;
function also allocates the array tempr for storage of temperatures

\item
Mt--$>$alloc\_nodes\_arrays (); - allocates arrays on nodes for
strains, stresses, other values

\item
Mt--$>$give\_maxncompstr(Mm--$>$max\_ncompstrn, Mm--$>$max\_ncompstre); - determines
maximum number of strain/stress components on integration points
it is used for output

\item
Mt--$>$give\_maxncompo(Mm--$>$max\_nncompo, Mm--$>$max\_encompo); - determines
maximum number of components in array other on integration points
it is used for output

\item
Mt--$>$elemprescdisp (); - determines elements influenced by prescribed
displacements, it is used for assembling of nodal forces


\item
Mm--$>$alloceigstrains (); - allocates array eigstrains 

\item
Mm--$>$alloctempstrains (); - allocates array tempstrains

\item
Mt--$>$nodedisplacements (); - allocates array nodedispl





\end{itemize}



\section{??}

Eigenstrains are strains which are not originated by stresses.
They can be defined by relation
\begin{eqnarray}
\sigma = E (\varepsilon - \varepsilon_{eig})
\end{eqnarray}
Meaning of eigenstrains may be various. The code has subroutines
which compute nodal forces caused by eigenstrains.

The class mechmat contains arrays eigstrains and tempstrains.
The array eigstrains is intended for prescribed eigenstrains
while the array tempstrains stores strains caused by temperature.
The temperature strains are



%%%%%%%%%%%%%%%%%%%%%%%%%%%%%%%%%%%%%%%%%%%%%%%%%%%%%%%%%%%%%%%%%%%%%%%%%%%%%%%%%%%%%%%%%%%%%%%%%%%
%%%%%%%%%%%%%%%%%%%%%%%%%%%%%%%%%%%%%%%%%%%%%%%%%%%%%%%%%%%%%%%%%%%%%%%%%%%%%%%%%%%%%%%%%%%%%%%%%%%
%%%%%%%%%%%%%%%%%%%%%%%%%%%%%%%%%%%%%%%%%%%%%%%%%%%%%%%%%%%%%%%%%%%%%%%%%%%%%%%%%%%%%%%%%%%%%%%%%%%
%%%%%%%%%%%%%%%%%%%%%%%%%%%%%%%%%%%%%%%%%%%%%%%%%%%%%%%%%%%%%%%%%%%%%%%%%%%%%%%%%%%%%%%%%%%%%%%%%%%
\section{Finite Elements}
\index{finite elements}

This section deals with description of implementation of finite elements in the MEFEL code.
Theoretical background about finite elements applied in mechanics is summarized in
Section \ref{finitelements}. Characteristic matrices and vectors of nodal values are in the
centre of attention of this section.

%%%%%%%%%%%%%%%%%%%%%%%%%%%%%%%%%%%%%%%%%%%%%%%%%%%%%%%%%%%%%%%%%%%%%%%%%%%%%%%%%%%%%%%%%%%%%%%%%%%%%
\subsection{Characteristic matrices}
%%%%%%%%%%%%%%%%%%%%%%%%%%%%%%%%%%%%%%%%%%%%%%%%%%%%%%%%%%%%%%%%%%%%%%%%%%%%%%%%%%%%%%%%%%%%%%%%%%%%%

%%%%%%%%%%%%%%%%%%%%%%%%%%%%%%%%%%%%%%%%%%%%%%%%%%%%%%%%%%%%%%%%%%%%%%%%%%%%%%%%%%%%%%%%%%%%%%%%%%%%%
\subsection{Nodal vectors}
%%%%%%%%%%%%%%%%%%%%%%%%%%%%%%%%%%%%%%%%%%%%%%%%%%%%%%%%%%%%%%%%%%%%%%%%%%%%%%%%%%%%%%%%%%%%%%%%%%%%%

\subsubsection{Vectors of line, surface and volume loads}

\subsubsection{Vectors of temperature loads}

\subsubsection{Vectors of eigenstrain loads}

%%%%%%%%%%%%%%%%%%%%%%%%%%%%%%%%%%%%%%%%%%%%%%%%%%%%%%%%%%%%%%%%%%%%%%%%%%%%%%%%%%%%%%%%%%%%%%%%%%%%%

\subsection{Strains}

Strain evaluation is optional for
\begin{itemize}
\item
linear statics ({\sf problemtype} {\it linear\_statics})
\item
eigendynamics ({\sf problemtype} {\it eigen\_dynamics})
\item
forced dynamics ({\sf problemtype} {\it forced\_dynamics})
\end{itemize}
In all other types of problems, strains are evaluated.
Strains are primarily evaluated at integration points. Decomposition of strain
components into blocks leads to general situation where different numbers of
integration points are used. Therefore, only part of strain components are
evaluated at particular integration points and remaining components have to
be interpolated from other integration points. The situation is depicted in
Figure \ref{}.

Typical example is linear quadrilateral finite element for plane stress problems
where the different integrations are used in order to remove hourglas problems.
The normal strains $\varepsilon_x$ and $\varepsilon_y$ should be integrated
at 4 integration points while the shear strain $\gamma_{xy}$ should be integrated
at one integration point only. For details about this topic see reference \ref{}.




Finite elements contain several functions for strain evaluation and they are
described in the following text.

\begin{itemize}
\item
{\sf mainip\_strains}: this function computes only appropriate strain components
\end{itemize}
