\chapter{Plasticity}

stress tensor
\begin{eqnarray}
\left(\begin{array}{ccc}
\sigma_{xx} & \sigma_{xy} & \sigma_{xz}
\\
\sigma_{yx} & \sigma_{yy} & \sigma_{yz}
\\
\sigma_{zx} & \sigma_{zy} & \sigma_{zz}
\end{array}\right)
\end{eqnarray}

stress vector
\begin{eqnarray}
\mbf{\sigma} = \left(\begin{array}{c}
\sigma_x
\\
\sigma_y
\\
\sigma_z
\\
\tau_{yz}
\\
\tau_{zx}
\\
\tau_{xy}
\end{array}\right) = \left(\begin{array}{c}
\sigma_{xx}
\\
\sigma_{yy}
\\
\sigma_{zz}
\\
\sigma_{yz}
\\
\sigma_{zx}
\\
\sigma_{xy}
\end{array}\right)
\end{eqnarray}


first invariant of the stress tensor
\begin{eqnarray}
I_1 = \sigma_{ii} = \sigma_{xx} + \sigma_{yy} + \sigma_{zz} = \sigma_x + \sigma_y + \sigma_z
\end{eqnarray}

mean stress\index{mean stress}
\begin{eqnarray}
\sigma_m = \frac{1}{3} I_1 = \frac{1}{3}(\sigma_{xx} + \sigma_{yy} + \sigma_{zz}) = \frac{1}{3}(\sigma_x + \sigma_y + \sigma_z)
\end{eqnarray}

first derivatives of the first invariant of stress tensor (in vector notation)
\begin{eqnarray}
\ppd{I_1}{\sigma_x} &=& 1
\\
\ppd{I_1}{\sigma_y} &=& 1
\\
\ppd{I_1}{\sigma_z} &=& 1
\\
\ppd{I_1}{\tau_{yz}} &=& 0
\\
\ppd{I_1}{\tau_{zx}} &=& 0
\\
\ppd{I_1}{\tau_{xy}} &=& 0
\\
\end{eqnarray}

second derivates are identically equal to zero

matrix form of second derivates of $I_1^2$ (in vector notation)
\begin{eqnarray}
\npd{2}{I_1^2}{\mbf{\sigma}} = \left(\begin{array}{rrrrrr}
2 & 2 & 2 & 0 & 0 & 0
\\[2mm]
2 & 2 & 2 & 0 & 0 & 0
\\[2mm]
2 & 2 & 2 & 0 & 0 & 0
\\[2mm]
0 & 0 & 0 & 0 & 0 & 0
\\[2mm]
0 & 0 & 0 & 0 & 0 & 0
\\[2mm]
0 & 0 & 0 & 0 & 0 & 0
\end{array}\right)
\end{eqnarray}

third invariant of the stress tensor
\begin{eqnarray}
I_{3s} &=& \left|\begin{array}{ccc}
\sigma_{xx} & \sigma_{xy} & \sigma_{xz}
\\
\sigma_{yx} & \sigma_{yy} & \sigma_{yz}
\\
\sigma_{zx} & \sigma_{zy} & \sigma_{zz}
\end{array}\right| = 
\\ \nonumber
&=& \sigma_{xx} \sigma_{yy} \sigma_{zz} +  \sigma_{xy} \sigma_{yz} \sigma_{zx} + \sigma_{xz} \sigma_{yx} \sigma_{zy} -  \sigma_{xz}^2 \sigma_{yy} - \sigma_{yz}^2 \sigma_{xx} - \sigma_{zz} \sigma_{xy}^2
\end{eqnarray}


\begin{eqnarray}
\ppd{I_{3s}}{\sigma_{xx}} &=& \sigma_{yy}\sigma_{zz} - \sigma_{yz}\sigma_{zy}
\\
\ppd{I_{3s}}{\sigma_{xy}} &=& \sigma_{yz}\sigma_{zx} - \sigma_{yx}\sigma_{zz}
\\
\ppd{I_{3s}}{\sigma_{xz}} &=& \sigma_{yx}\sigma_{zy} - \sigma_{yy}\sigma_{zx}
\\
\ppd{I_{3s}}{\sigma_{yx}} &=& \sigma_{xz}\sigma_{zy} - \sigma_{xy}\sigma_{zz}
\\
\ppd{I_{3s}}{\sigma_{yy}} &=& \sigma_{xx}\sigma_{zz} - \sigma_{xz}\sigma_{zx}
\\
\ppd{I_{3s}}{\sigma_{yz}} &=& \sigma_{xy}\sigma_{zx} - \sigma_{xx}\sigma_{zy}
\\
\ppd{I_{3s}}{\sigma_{zx}} &=& \sigma_{xy}\sigma_{yz} - \sigma_{xz}\sigma_{yy}
\\
\ppd{I_{3s}}{\sigma_{zy}} &=& \sigma_{xz}\sigma_{yx} - \sigma_{xx}\sigma_{yz}
\\
\ppd{I_{3s}}{\sigma_{zz}} &=& \sigma_{xx}\sigma_{yy} - \sigma_{xy}\sigma_{yx}
\end{eqnarray}

stress deviator\index{stress deviator} (tensor notation)
\begin{eqnarray}
s_{ij} = \sigma_{ij} - \sigma_m \delta_{ij}
\end{eqnarray}
stress deviator\index{stress deviator} (vector notation)
\begin{eqnarray}
\mbf{s} = \left(\begin{array}{c}
\frac{2}{3}\sigma_{x} - \frac{1}{3}\sigma_{y} - \frac{1}{3}\sigma_{z}
\\[2mm]
\frac{2}{3}\sigma_{y} - \frac{1}{3}\sigma_{x} - \frac{1}{3}\sigma_{z}
\\[2mm]
\frac{2}{3}\sigma_{z} - \frac{1}{3}\sigma_{x} - \frac{1}{3}\sigma_{y}
\\[2mm]
\tau_{yz}
\\
\tau_{zx}
\\
\tau_{xy}
\end{array}\right)
\end{eqnarray}

first invariant of the stress deviator
\begin{eqnarray}
J_1 = 0
\end{eqnarray}

second invariant of the stress deviator
\begin{eqnarray}
J_2 = \del{1}{3}(\sigma_x^2 + \sigma_y^2 + \sigma_z^2 - \sigma_y \sigma_z - \sigma_z \sigma_x -
\sigma_x \sigma_y) + \tau_{yz}^2 + \tau_{zx}^2 + \tau_{xy}^2
\end{eqnarray}

first derivatives of the second invariant of stress deviator
\begin{eqnarray}
\ppd{J_2}{\sigma_x} &=& \del{1}{3} (2 \sigma_x - \sigma_z - \sigma_y) = s_x
\\
\ppd{J_2}{\sigma_y} &=& \del{1}{3} (2 \sigma_y - \sigma_z - \sigma_x) = s_y
\\
\ppd{J_2}{\sigma_z} &=& \del{1}{3} (2 \sigma_z - \sigma_y - \sigma_x) = s_z
\\
\ppd{J_2}{\tau_{yz}} &=& 2 \tau_{yz}
\\
\ppd{J_2}{\tau_{zx}} &=& 2 \tau_{zx}
\\
\ppd{J_2}{\tau_{xy}} &=& 2 \tau_{xy}
\end{eqnarray}

second derivatives of the second invariant of stress deviator

\begin{eqnarray}
\dpd{J_2}{\sigma_x}{\sigma_x} &=& \del{2}{3}
\\
\dpd{J_2}{\sigma_x}{\sigma_y} &=& -\del{1}{3}
\\
\dpd{J_2}{\sigma_x}{\sigma_z} &=& -\del{1}{3}
\\
\dpd{J_2}{\sigma_x}{\tau_{yz}} &=& 0
\\
\dpd{J_2}{\sigma_x}{\tau_{zx}} &=& 0
\\
\dpd{J_2}{\sigma_x}{\tau_{xy}} &=& 0
\\
\dpd{J_2}{\sigma_y}{\sigma_y} &=& \del{2}{3}
\\
\dpd{J_2}{\sigma_y}{\sigma_z} &=& -\del{1}{3}
\\
\dpd{J_2}{\sigma_y}{\tau_{yz}} &=& 0
\\
\dpd{J_2}{\sigma_y}{\tau_{zx}} &=& 0
\\
\dpd{J_2}{\sigma_y}{\tau_{xy}} &=& 0
\\
\dpd{J_2}{\sigma_z}{\sigma_z} &=& \del{2}{3}
\\
\dpd{J_2}{\sigma_z}{\tau_{yz}} &=& 0
\\
\dpd{J_2}{\sigma_z}{\tau_{zx}} &=& 0
\\
\dpd{J_2}{\sigma_z}{\tau_{xy}} &=& 0
\\
\dpd{J_2}{\tau_{yz}}{\tau_{yz}} &=& 2
\\
\dpd{J_2}{\tau_{yz}}{\tau_{zx}} &=& 0
\\
\dpd{J_2}{\tau_{yz}}{\tau_{xy}} &=& 0
\\
\dpd{J_2}{\tau_{zx}}{\tau_{zx}} &=& 2
\\
\dpd{J_2}{\tau_{zx}}{\tau_{xy}} &=& 0
\\
\dpd{J_2}{\tau_{xy}}{\tau_{xy}} &=& 2
\end{eqnarray}

matrix form of second derivatives of the second invariant of stress deviator
\begin{eqnarray}
\npd{2}{J_2}{\mbf{\sigma}} = \left(\begin{array}{rrrrrr}
 \frac{2}{3} & -\frac{1}{3} & -\frac{1}{3} & 0 & 0 & 0
\\[2mm]
-\frac{1}{3} &  \frac{2}{3} & -\frac{1}{3} & 0 & 0 & 0
\\[2mm]
-\frac{1}{3} & -\frac{1}{3} &  \frac{2}{3} & 0 & 0 & 0
\\[2mm]
          0  &           0  &           0  & 2 & 0 & 0
\\[2mm]
          0  &           0  &           0  & 0 & 2 & 0
\\[2mm]
          0  &           0  &           0  & 0 & 0 & 2
\end{array}\right)
\end{eqnarray}

\begin{eqnarray}
\ppd{J_3}{\mbf{\sigma}} = \left(\begin{array}{c}
s_y s_z - \tau_{yz}^2 + \frac{J_2}{3}
\\[2mm]
s_x s_z - \tau_{xz}^2 + \frac{J_2}{3}
\\[2mm]
s_x s_y - \tau_{xy}^2 + \frac{J_2}{3}
\\[2mm]
2(\tau_{yz}\tau_{xz} - s_z \tau_{xy})
\\[2mm]
2(\tau_{xz}\tau_{xy} - s_x \tau_{yz})
\\[2mm]
2(\tau_{xy}\tau_{yz} - s_y \tau_{xz})
\end{array}\right)
\end{eqnarray}

\begin{eqnarray}
\npd{2}{J_3}{\mbf{\sigma}} = \frac{2}{3}\left(\begin{array}{cccccc}
       s_{x} &        s_{z} &        s_{y} &    \tau_{xy} & -2 \tau_{yz} &    \tau_{xz}
\\
       s_{z} &        s_{y} &        s_{x} &    \tau_{xy} &    \tau_{yz} & -2 \tau_{xz}
\\
       s_{y} &        s_{x} &        s_{z} & -2 \tau_{xy} &    \tau_{yz} &    \tau_{xz}
\\
   \tau_{xy} &    \tau_{xy} & -2 \tau_{xy} &     -3 s_{z} &  3 \tau_{xz} &  3 \tau_{yz}
\\
-2 \tau_{yz} &    \tau_{yz} &    \tau_{yz} &  3 \tau_{xz} &     -3 s_{x} &  3 \tau_{xy}
\\
   \tau_{xz} & -2 \tau_{xz} &    \tau_{xz} &  3 \tau_{yz} &  3 \tau_{xy} &     -3 s_{y}
\end{array}\right)
\end{eqnarray}


Zienkiewicz and Cormeau in reference \cite{zienkiewicz:viscoplasticity} use the
yield function in the form
\begin{eqnarray}
f (\mbf{\sigma}) = f(\sigma_m, J_2, J_3)
\end{eqnarray}
where $\sigma_m$ is the mean\index{mean stress} stress, $J_2$ and $J_3$ are second and third
invariants of the stress deviator. Their definitions are the following
\begin{eqnarray}
\sigma_m &=& \frac{1}{3} \sigma_{ii}
\\
s_{ij} &=& \sigma_{ij} - \delta_{ij} \sigma_m
\\
J_2 &=& \frac{1}{2} s_{ij} s_{ij}
\\
J_3 &=& \frac{1}{3} s_{ij} s_{jk} s_{ki} = \det \mbf{s}
\end{eqnarray}
An alternative to the third invariant $J_3$ is the Lode \index{Lode angle} angle
\begin{eqnarray}
-\frac{\pi}{6} \leq \theta = \frac{1}{3} \sin^{-1}\left(\frac{-3 \sqrt{3} J_3}{2 J_2^{\frac{3}{2}}}\right) \leq \frac{\pi}{6}
\end{eqnarray}
In reference \cite{crisfield2}, the principal stresses can be expressed in the form
\begin{eqnarray}
\left(\begin{array}{c}
\sigma_1
\\
\sigma_2
\\
\sigma_3
\end{array}\right) = \frac{2\sqrt{J_2}}{\sqrt{3}}
\left(\begin{array}{c}
\sin (\theta + 2\pi/3)
\\
\sin \theta
\\
\sin (\theta - 2\pi/3)
\end{array}\right) + \frac{I_1}{2} \left(\begin{array}{c}
1
\\
1
\\
1
\end{array}\right)
\end{eqnarray}

In reference \cite{crisfield2}, derivative of the Lode angle with respect to the stress vector has the form
\begin{eqnarray}
\ppd{\theta}{\mbf{\sigma}} = \frac{-\sqrt{3}}{2 \cos 3\theta} \left(J_2^{-3 \over 2} \ppd{J_3}{\mbf{\sigma}} - \frac{3}{2} J_3 J_2^{-5 \over 2} \ppd{J_2}{\mbf{\sigma}}\right)
\end{eqnarray}

\section{Tresca model of plasticity}
\index{model of plasticity!Tresca}

In reference \cite{prochazka}, the yield function has the form
\begin{eqnarray}
f = 2\sqrt{J_2} \cos \alpha - \alpha_0\ ,
\end{eqnarray}
where $\alpha$ and $\alpha_0$ are Lode angles.

\section{von Mises model of plasticity}
\index{model of plasticity!von Mises}

In reference \cite{prochazka}, the yield function has the form
\begin{eqnarray}
f = \sqrt{3 J_2} - \sigma_0\ ,
\end{eqnarray}

\section{Mohr-Coulomb model of plasticity}
\index{model of plasticity!Mohr-Coulomb}

In reference \cite{prochazka}, the yield function has the form
\begin{eqnarray}
f = \frac{I_1}{3} \sin \varphi' + \sqrt{I_2}(\cos \alpha - \frac{1}{\sqrt{3}}\sin \alpha \sin \varphi') - c' \cos \varphi 
\end{eqnarray}
where $\alpha$ is the Lode angle.


\section{Drucker-Prager model of plasticity}
\index{model of plasticity!Drucker-Prager}

In reference \cite{jirasek:skripta}, the Drucker-Prager model of plasticity
is based on the yield function in the form
\begin{eqnarray}
f(\mbf{\sigma}) = 3 \alpha_{\phi} \sigma_m(\mbf{\sigma}) + \sqrt{J_2(\mbf{\sigma})} - \tau_0 = \alpha_{\phi} I_{1,s} + \sqrt{J_2(\mbf{\sigma})} - \tau_0\ ,
\end{eqnarray}
where $\alpha_{\phi}$ is the angle of internal\index{angle of internal friction} friction,
$\sigma_m(\mbf{\sigma})$ is the mean\index{mean stress} stress,
$I_{1,s}$ is the first invarinat of stress tensor and
$J_2(\mbf{\sigma})$ is second invariant of the stress deviator.

In reference \cite{belytschko:nonlin}, the Drucker-Prager model of plasticity
is based on the yield function in the form
\begin{eqnarray}
f = \bar{\sigma} - \alpha I_{1,s} - Y\ ,
\end{eqnarray}
where $\bar{\sigma}$ is the efffective\index{efffective stress} stress and
the coefficients are in the form
\begin{eqnarray}\label{eqplasticitydruckerprager1}
\alpha &=& \frac{2 \sin \phi}{3 \pm \sin \phi}\ ,
\\ \label{eqplasticitydruckerprager2}
Y &=& \frac{6 c \cos \phi}{3 \pm \sin \phi}\ ,
\end{eqnarray}
where $\phi$ is the angle of internal\index{angle of internal friction} friction
and $c$ is the cohesion.\index{cohesion}
The efffective stress has the form
\begin{eqnarray}
\bar{\sigma} = \sqrt{\frac{3}{2}s_{ij}s_{ij}}\ .
\end{eqnarray}
The plus sign in (\ref{eqplasticitydruckerprager1}) and (\ref{eqplasticitydruckerprager2})
corresponds to the inner apexes and minus signe corresponds to the outer apexes
of the Mohr-Coulomb yield surface.
The derivative with respect to the stress vector has the form
\begin{eqnarray}
\od{f}{\mbf{\sigma}} = \frac{3}{2\bar{\sigma}} \mbf{s} - \alpha \mbf{I}
\end{eqnarray}

In reference \cite{crisfield2}, the Drucker-Prager model of plasticity
is based on the yield function in the form
\begin{eqnarray}
f = D I_1 + \sqrt{J_2} - \sigma_0\ ,
\end{eqnarray}
where
\begin{eqnarray}
D &=& \frac{2 \sin \phi}{\sqrt{3}(3 \pm \sin \phi)}\ ,
\\
\sigma_0 &=& \frac{6 c \cos \phi}{\sqrt{3}(3 \pm \sin \phi)}\ .
\end{eqnarray}


In reference \cite{prochazka}, the yield function has the form
\begin{eqnarray}
f = \alpha' I_1 + \sqrt{J_2} - K'\ ,
\end{eqnarray}
where
\begin{eqnarray}
\alpha' &=& \frac{2 \sin \varphi'}{\sqrt{3}(3-\sin \varphi')}
\\
K' &=& \frac{6 c' \cos \varphi'}{\sqrt{3}(3-\sin \varphi')}
\end{eqnarray}
where $\varphi'$ is the internal friction angle and $c'$ is the cohesion.
If the parameters are defined in the form
\begin{eqnarray}
\alpha' &=& \frac{\tan \varphi'}{\sqrt{9+12 \tan \varphi'}}
\\
K' &=& \frac{3 c'}{\sqrt{9+12 \tan \varphi'}}
\end{eqnarray}
the Mohr-Coulomb model is obtained.



\section{Gurson model of plasticity}
\index{model of plasticity!Gurson}

In reference \cite{belytschko:nonlin}, the Gurson model of plasticity,
which is used in progressive microrupture modelling,
is based on the yield function in the form
\begin{eqnarray}
f = \frac{\sigma_e^2}{\bar{\sigma}^2} + 2 f^* \beta_1 \cosh \frac{\beta_2 I_{1,s}}{2\bar{\sigma}} - (\beta_1 f^*)^2 - 1\ ,
\end{eqnarray}
where
$\sigma_e=\sqrt{\frac{3}{2}s_{ij}s_{ij}}$ is the effective macroscopic stress,
$\bar{\sigma}$ is the effective stress in the matrix material,
$f^*$ is the function of the void volume fraction,
$\beta$ and $\beta_1$ are material parameters originally set to one.
The function of the void volume fraction has the form
\begin{eqnarray}
f^* &=& f\ \  {\rm for} f \leq f_c
\\
f^* &=& f_c + \frac{(f_u-f_c)(f-f_c)}{f_f-f_c}\ \  {\rm for} f>f_c
\end{eqnarray}
where
$f$ is the volume fraction,
$f_u=\frac{1}{\beta_1}$,
$f_c$ is the critical value
and $f_f$ is the void volume fraction.

First derivatives with respect to stress have the form
\begin{eqnarray}
\od{f}{\mbf{\sigma}} = \frac{3}{\bar{\sigma}^2} \mbf{s} + \frac{f^* \beta \beta_1}{\bar{\sigma}} \sinh \left(\frac{\beta_2 I_{1,s}}{2\bar{\sigma}}\right)\mbf{I} 
\end{eqnarray}

In reference \cite{crisfield2}, Tvergaard's modification of the Gurson model
for porous materials has the form
\begin{eqnarray}
f = J_2 - a_0 + a_1 \cosh a_2 \sigma_m\ ,
\end{eqnarray}
where $a_0$, $a_1$ are related to the void volume fractions.

\section{Chen model of plasticity}
\begin{eqnarray}
f_1 (\mbf{\sigma}) = J_2 + \del{1}{3} A I_1 - \tau^2 &=& 0 
\\
f_2 (\mbf{\sigma}) = J_2 -\del{1}{6} I_1^2 + \del{1}{3} A I_1 - \tau^2 &=& 0 
\end{eqnarray}

\begin{eqnarray}
\ppd{f_1}{\mbf{\sigma}} = \left(\begin{array}{c}
\frac{1}{3}(2\sigma_x - \sigma_z - \sigma_y) + \frac{1}{3} A
\\[2mm]
\frac{1}{3}(2\sigma_y - \sigma_z - \sigma_x) + \frac{1}{3} A
\\[2mm]
\frac{1}{3}(2\sigma_z - \sigma_x - \sigma_y) + \frac{1}{3} A
\\[2mm]
2 \tau_{yz}
\\[2mm]
2 \tau_{zx}
\\[2mm]
2 \tau_{xy}
\\[2mm]
\end{array}\right)
\end{eqnarray}

\begin{eqnarray}
\ppd{f_2}{\mbf{\sigma}} = \left(\begin{array}{c}
\frac{1}{3}(2\sigma_x - \sigma_z - \sigma_y) + \frac{1}{3} A - \frac{1}{3} I_1
\\[2mm]
\frac{1}{3}(2\sigma_y - \sigma_z - \sigma_x) + \frac{1}{3} A - \frac{1}{3} I_1
\\[2mm]
\frac{1}{3}(2\sigma_z - \sigma_x - \sigma_y) + \frac{1}{3} A - \frac{1}{3} I_1
\\[2mm]
2 \tau_{yz}
\\[2mm]
2 \tau_{zx}
\\[2mm]
2 \tau_{xy}
\\[2mm]
\end{array}\right)
\end{eqnarray}

\begin{eqnarray}
\npd{2}{f_1}{\mbf{\sigma}} = \left(\begin{array}{rrrrrr}
 \frac{2}{3} & -\frac{1}{3} & -\frac{1}{3} & 0 & 0 & 0
\\[2mm]
-\frac{1}{3} &  \frac{2}{3} & -\frac{1}{3} & 0 & 0 & 0
\\[2mm]
-\frac{1}{3} & -\frac{1}{3} &  \frac{2}{3} & 0 & 0 & 0
\\[2mm]
          0  &           0  &           0  & 2 & 0 & 0
\\[2mm]
          0  &           0  &           0  & 0 & 2 & 0
\\[2mm]
          0  &           0  &           0  & 0 & 0 & 2
\end{array}\right)
\end{eqnarray}

\begin{eqnarray}
\npd{2}{f_2}{\mbf{\sigma}} = \left(\begin{array}{rrrrrr}
 \frac{1}{3} & -\frac{2}{3} & -\frac{2}{3} & 0 & 0 & 0
\\[2mm]
-\frac{2}{3} &  \frac{1}{3} & -\frac{2}{3} & 0 & 0 & 0
\\[2mm]
-\frac{2}{3} & -\frac{2}{3} &  \frac{1}{3} & 0 & 0 & 0
\\[2mm]
          0  &           0  &           0  & 2 & 0 & 0
\\[2mm]
          0  &           0  &           0  & 0 & 2 & 0
\\[2mm]
          0  &           0  &           0  & 0 & 0 & 2
\end{array}\right)
\end{eqnarray}

\section{Drucker-Prager}
In reference \cite{zienkiewicz:viscoplasticity} is stated that
Drucker and Prager formulated a yield function for soil-like materials
in reference \cite{drucker:druckerprager} in the form
\begin{eqnarray}
f = \frac{6 \sin \psi}{3 - \psi} \sigma_m + \sqrt{3 J_2} - \frac{6 c \cos \psi}{3 - \sin \psi} 
\end{eqnarray}
where $c$ is the cohesion and $\psi$ is the friction angle.
If $\psi=0$, Von Mises criterion is obtained
\begin{eqnarray}
f = \sqrt{3 J_2} - 2 c
\end{eqnarray}

\section{Mohr-Coulomb yield surface}
\begin{eqnarray}
f = 2 \sigma_m \sin \psi + \left(2 \cos \theta - \frac{1}{\sqrt{3}} \sin \psi \sin \theta\right) \sqrt{J_2} - 2 c \cos \psi
\end{eqnarray}
where $c$ is the cohesion, $\psi$ is the friction angle and $\theta$ is the Lode angle.
If $\psi=0$, Tresca criterion is obtained
\begin{eqnarray}
f = 2 \cos \theta \sqrt{J_2} - 2 c
\end{eqnarray}


\section{Stress return algorithm}

plastic strains
\begin{eqnarray}
\dot{\mbf{\varepsilon}}^p = \dot{\gamma} \mbf{r}
\\
\dot{\mbf{\varepsilon}}^p = \dot{\gamma} \ppd{g}{\mbf{\sigma}}
\\
\dot{\mbf{\varepsilon}}^p = \dot{\gamma} \ppd{f}{\mbf{\sigma}}
\end{eqnarray}
stress
\begin{eqnarray}
\mbf{\sigma} = \mbf{D} (\mbf{\varepsilon} - \mbf{\varepsilon}^p)
\end{eqnarray}
hardening
\begin{eqnarray}
\dot{\mbf{q}} = \dot{\gamma} \mbf{h}
\end{eqnarray}

discrete version

increment of plastic strains
\begin{eqnarray}
\Delta \mbf{\varepsilon}^p_{n+1} = \Delta \gamma_{n+1} \ppd{g_{n+1}}{\mbf{\sigma}}
\end{eqnarray}
new stress
\begin{eqnarray}
\mbf{\sigma}_{n+1} = \mbf{D} \left(\mbf{\varepsilon}_{n+1} - \mbf{\varepsilon}^p_{n+1} - \Delta \gamma_{n} \ppd{g_{n+1}}{\mbf{\sigma}}\right)
\end{eqnarray}
\begin{eqnarray}
\mbf{\sigma}_{n} + \Delta \mbf{\sigma}_{n}^i = \mbf{D} \left(\mbf{\varepsilon}_{n+1} - \mbf{\varepsilon}^p_{n+1} -
\Delta \gamma_{n} \ppd{g_{n+1}}{\mbf{\sigma}}\right)
\end{eqnarray}


\section{Implementation comments}
Models of plasticity work in tensorial notation. Strains and stresses
are stored in 3x3 matrices, first derivatives of yield function and
plastic potential with respect to stresses are stored in 3x3
matrices, second derivatives of yield function and plastic potential
with respect to stresses are stored in 6x6 matrices. Conversion
of tensorial form to the vector (matrix) form is done in mechmat
in functions dfdsigma, dgdsigma, dfdsigmadsigma and dgdsigmadsigma.

\section{Elastic--damage--plastic models}

Step 1. Initialization
\begin{eqnarray}
k&=&0
\\
\mbf{\varepsilon}_{n+1,k}^{pl} &=& \mbf{\varepsilon}_n^{pl}
\\
\mbf{\varepsilon}_{n+1,k}^{el} &=& \mbf{\varepsilon}_{n+1}^{tot} - \mbf{\varepsilon}_{n+1,k}^{pl}
\\
\mbf{q}_{n+1,k} &=& \mbf{q}_n
\end{eqnarray}

Step 2. Update of damage parameters
\begin{eqnarray}
\tilde{\varepsilon}_{n+1} &=& f(\mbf{\varepsilon}_{n+1,k}^{el})
\\
{\rm if}\ & & \tilde{\varepsilon}_{n+1} > \kappa\ \ \ \Rightarrow \ \ \ \ \kappa=\tilde{\varepsilon}_{n+1}
\\
\omega_{n+1} &=& g(\kappa)
\end{eqnarray}

Step 3. Stress and yield function
\begin{eqnarray}
\bar{\mbf{\sigma}}_{n+1,k} &=& \mbf{D} (\mbf{\varepsilon}_{n+1}^{tot} - \mbf{\varepsilon}_{n+1,k}^{pl}) = \mbf{D} \mbf{\varepsilon}_{n+1,k}^{el}
\\
\mbf{\sigma}_{n+1,k} &=& (1-\omega_{n+1}) \bar{\mbf{\sigma}}_{n+1,k}
\\
f_{n+1,k} &=& f(\mbf{\sigma}_{n+1,k}, \mbf{q}_{n+1,k})
\\
{\rm if}\ & & f_{n+1,k}\leq \varepsilon_{tol}\ \ \ \Rightarrow \ \ \ \mbf{\varepsilon}_{n+1}^{pl} = \mbf{\varepsilon}_{n+1,k}^{pl},\ \ \mbf{q}_{n+1} = \mbf{q}_{n+1,k}\ \ {\rm go\ to}\ n+2
\end{eqnarray}

Step 4. Increment of plastic consistency parameter
\begin{eqnarray}
\Delta \lambda_{n+1,k} = \Delta \lambda_{n+1,k} (\mbf{\sigma}_{n+1,k}, \mbf{q}_{n+1,k})
\end{eqnarray}

Step 5. Update of plastic strain, elastic strain and hardening parameters
\begin{eqnarray}
\mbf{\varepsilon}_{n+1,k+1}^{pl} &=& \mbf{\varepsilon}_{n+1,k}^{pl} + \Delta \lambda_{n+1,k} \mbf{r}_{n+1,k} (\mbf{\sigma}_{n+1,k}, \mbf{q}_{n+1,k})
\\
\mbf{\varepsilon}_{n+1,k+1}^{el} &=& \mbf{\varepsilon}_{n+1} - \mbf{\varepsilon}_{n+1,k+1}^{pl}
\\
\mbf{q}_{n+1,k+1} &=& \mbf{q}_{n+1,k} + \Delta \lambda_{n+1,k} \mbf{h}_{n+1,k} (\mbf{\sigma}_{n+1,k}, \mbf{q}_{n+1,k})
\\
k &=& k+1
\\
{\rm go\ to\ } & & {\rm step\ 3}
\end{eqnarray}
