\chapter{Expansion of the MEFEL}
\section{Implementation of new finite element}
%%%%%%%%%%%%%%%%%%%%%%%%%%%%%%%%%%%%%%%%%%%%%%%%%%%%%%%%%%%%
%%%%%%%%%%%%%%%%%%%%%%%%%%%%%%%%%%%%%%%%%%%%%%%%%%%%%%%%%%%%
%%%%%%%%%%%%%%%%%%%%%%%%%%%%%%%%%%%%%%%%%%%%%%%%%%%%%%%%%%%%
%%%%%%%%%%%%%%%%%%%%%%%%%%%%%%%%%%%%%%%%%%%%%%%%%%%%%%%%%%%%
\section{Implementation of new material model}
%%%%%%%%%%%%%%%%%%%%%%%%%%%%%%%%%%%%%%%%%%%%%%%%%%%%%%%%%%%%
%%%%%%%%%%%%%%%%%%%%%%%%%%%%%%%%%%%%%%%%%%%%%%%%%%%%%%%%%%%%
\subsection{Implementation of new arbitrary model}
The main aims of any material model in the code are
\begin{itemize}
\item{evaluation of correct stresses for trial strains,}
\item{assemblage of the matrix of stiffness coefficients, $\mbf{D}$.}
\end{itemize}

Strains and stresses are stored at integration points. The actual strains are computed by any solver
implemented in the code and stored at integration points. 

Every material model must contain the following functions:
\begin{itemize}
\item{
{\sf void newmaterial::read (FILE *in)}
\newline reads necessary data of the model, {\it in} is pointer to the input file,
}
\item{
{\sf void newmaterial::matstiff (matrix \&d,long ipp,long ido)}
\newline assembles the stiffness matrix (matrix $\mbf{D}$), {\it ipp}
denotes the number of integration point, {\it ido} stands for the
index in array {\sf other}, the default value is 0,
}
\item{
{\sf void newmaterial::nlstresses (long ipp, long im, long ido)}
\newline computes correct stresses for computed strains,
{\it ipp} denotes the number of integration point,
{\it im} expresses the number of material model at the integration point (default value is 0),
{\it ido} stands for the index in array {\sf other} (default value is 0),
}
\item{
{\sf void newmaterial::updateval (long ipp,long im,long ido)}
\newline updates all material parameters when the equilibrium is reached,
{\it ipp} denotes the number of integration point,
{\it im} expresses the number of material model at the integration point (default value is 0),
{\it ido} stands for the index in array {\sf other} (default value is 0),
}
\end{itemize}

%%%%%%%%%%%%%%%%%%%%%%%%%%%%%%%%%%%%%%%%%%%%%%%%%%%%%%%%%%%%
%%%%%%%%%%%%%%%%%%%%%%%%%%%%%%%%%%%%%%%%%%%%%%%%%%%%%%%%%%%%
\subsection{Implementation of new plasticity model}
\subsubsection {Implementing new class of plasticity}
This section describes what the user needs when implements new plasticity model
to MEFEL. First of all user needs yield function $f$, the first derivative of yield function with
respect on stress tensor \ppd{f}{\sigma_{ij}}, optionally first derivative of plastic potential
with respect on stress tensor \ppd{g}{\sigma_{ij}}. In case that plasticity model
contains hardening/softening it's necessary the first derivative of yield function with
respect on hardening parameters \ppd{f}{q_i}, derivative hardening parameters with respect
on $\gamma$, $\ppd{q_i}{\gamma}=-H\ppd{g}{q_i}$ and/or user should provide function {\sf plasmodscalar}
which computes term $\ppd{f}{q_i}H\ppd{g}{q_i}$ in cutting plane method.

Each new material means the user should found new class with name of plasticity model {\sf myplasmodel}.
This class should contain material parameters as a data members of this class. Note that the MEFEL
in case plastic material suppose two materials. The first refers to plastic material and the second
one refers to elastic material. So the elastic parameters such as th Young modulus $E$ needn't be
included to the {\sf myplasmodel}. In addition the folowing methods should be defined :
\begin{itemize}
{\sf
\item
read(FILE* in)
\item
double yieldfunction(matrix \&sig, vector \&q)
\item
void deryieldfsigma(matrix \&sig, matrix \&dfds)
\item
void matstiff (matrix \&d, long ipp)
\item
void nlstresses (long ipp, long im, long ido)
\item
void updateval (long ipp, long im, long ido)
}

Optionally in case plasticity with hardening/softening :
{\sf
\item
void deryieldfq(vector \&qtr, vector \&dfq)
\item
void der\_q\_gamma(vector \&dqdg)
\item
void updateq(vector \&epsp, strastrestate ssst, vector \&q)
\item
double plasmodscalar(vector \&qtr)
}

or
{\sf
\item
void updateq(vector \&epsp, strastrestate ssst, vector \&q)
\item
double plasmodscalar(vector \&qtr)
}

Optionally in case nonassociated plasticity :
{\sf
\item
void derplaspotsigma(matrix \&sig, matrix \&dgds)
}
\end{itemize}

Function {\sf read(FILE* in)} should read material parameters for given plasticity from current
position in the opened text file given by the function parameter {\it in}.\\

Function {\sf double yieldfunction(matrix \&sig, vector \&q)} returns value of the yield function
for given plasticity model {\sf myplasmodel}. Returned value is double type. The first parameter
{\it sig} references to matrix which contains stress {\bf tensor} values $\sigma_{ij}$. The second
parameter {\it q} is vector which contains values of hardening/softening parameters.\\

Function {\sf void deryieldfsigma(matrix \&sig, matrix \&dfds)} computes values of \ppd{f}{\sigma_{ij}}
for given plasticity model {\sf myplasmodel}. Returned values are passed through the second parameter
{\it dfds}. The first parameter {\it sig} references to matrix which contains stress {\bf tensor}
values $\sigma_{ij}$. The second parameter {\it dfds} is type of matrix which contains result values
\ppd{f}{\sigma_{ij}}. This matrix is supposed to be proper allocated.\\

Function {\sf void matstiff (matrix \&d, long ipp, long ido)} assemble material stiffness matrix for given
integration point and material type. The first parameter {\it d} is allocated matrix which will be
filled up with values. The second parameter is integration point number which should be used for
access characteristic values in the integration point. The last parameter {\it ido} stands for index of the first 
internal variable in the array {\sf other} of given integration point for given material type. User may use standard 
stiffness matrix for linear elastic material or when it is available may use tangent stiffness matrix (it will reduce
number of iteration steps).\\

Function {\sf void nlstresses (long ipp, long im, long ido)} computes stresses for nonlinear material. Parameter
{\it ipp} is integration point number where the stresses should be evaluated. Parameter {\it im} is index of material type
in the array {\sf tm} for given integration point. Parameter {\it ido} stands for index of the first 
internal variable in the array {\sf other} of given integration point for given material type.
Function should work following way:
\begin{itemize}
\item
Take reached strains $\varepsilon_{ij}$ and reached {\sf eqother} values in the integration point number
{\it ipp} (See class {\sf mechmat, intpoint}). For plasticity models should contain reached plastic
strains $\varepsilon^{pl}_{ij}$ and parametr $\gamma$ for cutting plane method.
\item
Call stress-return function built in {\sf mechmat} i.e {\sf cutting\_plane} with parameters filled from
previous step. This method computes appropriate stresses and stores them into stress array in the
integration point.
\item
Store result plastic strains and coefficient $\gamma$ and possible another values to the {\sf other}
array in the integration point. 
\end{itemize}
For example see {\sf drprag::nlstresses(long ipp, long im, long ido)} in the file drprag.cpp.\\

Function {\sf void updateval (long ipp, long im, long ido)} copies values of {\sf other} array to
the eqother array in case that equilibrium for current step is reached. Parameter {\it ipp} is
integration point number whose the {\sf eqother} array should be updated. Parameter {\it im} denotes
index material type in the array {\sf tm} of given integration point. {\it ido} is index in the
{\sf other/eqother} array of the given integration point, from where the updated internal variables are stored.


In case nonassociated plasticity or plasticity with hardening/softening are neccessary following
functions :\\

Function {\sf void deryieldfq(vector \&qtr, vector \&dfq)} computes values of \ppd{f}{q_i}. The first
parameter is vector with values of hardening parameters. Second parameter is used for output result
values. Ammount and the order of these parameters is only for proposal. Prupose of this function will
be described in the {\sf plasmodscalar} function.\\

Function {\sf void der\_q\_gamma(vector \&dqdg)} computes values of \ppd{q_i}{\gamma} which means
exactly product $-H\ppd{g}{q_i}$. Parameter {\it dqdg} is output parametr for result values.
Ammount and the order of these parameters is only for proposal. Prupose of this function will be
described in the {\sf plasmodscalar} function.\\

Function {\sf double plasmodscalar(vector \&qtr)} computes $\ppd{f}{q_i}H\ppd{g}{q_i}$. This expression
is used in the cutting plane method. Both previous function should be call here. So the function
contains something like this :\\
{\tt
  double ret;\\
  vector dfq(qtr.n);\\
  vector dqg(qtr.n);\\
  deryieldfq(qtr, dfq);\\
  der\_q\_gamma(dqg);\\
  scprd(dfq, dqg, ret);\\
  return -ret;
}

When user prefers another way of representation of the expression $\ppd{f}{q_i}H\ppd{g}{q_i}$, there
is possible to write only function {\sf plasmodscalar} which supply values of this expression. Function
is called from mechmat's {\sf plasmodscalar(long ipp, long idpm, matrix \&sig, vector \&eps, vector \&qtr)} 
so there is possibility to pass parameters {\it ipp, sig, eps} to your function {\sf plasmodscalar} to obtain 
additional values from integration point.\\

Function {\sf void updateq(long ipp, long ipp, vector \&epsp, vector \&q)} 
updates hardening variables after cutting plane iteration loop has been run depending on the $\varepsilon^{pl}_{ij}$ and 
the stress/strain state. Parameter {\it q} will contain output from this function - the updated hardening parameters.
Rest parameters passed to your function {\sf plasmodscalar} from mechmat's {\sf plasmodscalar} should be used obtain 
additional necessary values from integration points.
\\

Function {\sf void derplaspotsigma(matrix \&sig, matrix \&dgds)} computes values of derivatives of plastic
potential \ppd{g}{\sigma_{ij}} for given plasticity model {\sf myplasmodel}. Returned values are passed
through the second parameter {\it dfds}. The first parameter {\it sig} references to matrix which contains
stress {\bf tensor} values $\sigma_{ij}$. The second parameter {\it dfds} is type of matrix which contains
result values \ppd{g}{\sigma_{ij}}. This matrix is supposed to be proper allocated.\\

\subsubsection {Including new plasticity class to MEFEL files}
Thus the new class {\sf myplasmodel} is written to files  myplasmodel.cpp${\rm |}$h. Now the user should
include this class to the MEFEL. Here is list of files which have to be modified :

{\bf alias.h}\\
So called material alias {\sf mattype} have to be added. Alias means the substitutive shortcut
for C++ enumaration integer number. Decision which number and name for its alias depends on the user.
Only restrictions are that the name and number have to be unique in the {\sf mattype} scope. Also note
the name haven't to be same as the name of the plasticity class. This aliases are used in the
{\tt switch} statements as values for {\tt case}. Numeric value of the aliases are used in the input
files for the MEFEL.\\

{\bf mechmat.h}\\
This file contains declaration of the {\sf mechmat} class. This class contains also pointers to arrays for each
material class. Thus the user have to include new class {\sf myplasmodel} to the {\sf mechmat} class declaration.
The name of the pointer is arbitrary expecting used names of course. Note that the user should use predeclaration
of {\sf myplasmodel} class or insert include file with {\sf myplasmodel} class declaration.\\


{\bf mechmat.cpp}\\
\begin{itemize}
\item {\sf void mechmat::readmatchar (FILE *in)}
\item {\sf void mechmat::matstiff (matrix \&d,long ipp)}
\item {\sf double mechmat::yieldfunction (long ipp, long idpm, matrix \&sig, vector \&q)}
\item {\sf void mechmat::dfdsigma (long ipp, long idpm, matrix \&sig, matrix \&dfds)}
\item {\sf void mechmat::dgdsigma (long ipp, long idpm, matrix \&sig, matrix \&dgds)}
\item {\sf void mechmat::dfdq (long ipp, long idpm, vector \&dq)}
\item {\sf double mechmat::plasmodscalar (long ipp, long idpm, vector \&qtr)}
\item {\sf void mechmat::updateq (long ipp, long idpm, double dgamma, vector \&epsp, vector \&q)}
\item {\sf void mechmat::updateipvalmat (long ipp,long im,long ido)}
\item {\sf void mechmat::computenlstresses (long ipp,long im,long ido)}
\item {\sf long mechmat::givencompother (long ipp,long im)}
\end{itemize}
Note that include file with full class declaration for given {\sf myplasmodel} should be included at the begining
of mechmat.cpp\\

In function {\sf mechmat::readmatchar (FILE *in)} the actual values of material parameters for given material type. This function
contains {\tt switch} statement where the new {\tt case} for the  {\sf myplasmodel} would be included.
It is possible to take part of {\tt case} statement from another material and copy it into new {\tt case} and then
make some modification. Folowing lines shows example of such a new {\tt case} :\\
{\tt
      if (Mespr==1)  fprintf (stdout,"$\backslash$n number of MYPLASMODEL materials \%ld",numtype);\\
      MYPLASMODELPT = new MYPLASMODEL [numtype];\\
      for (j = 0; j $<$ numtype; j++){\\
        k = numtype + 1;\\
        fscanf (in,"\%ld",\&k);\\
        if ((k $>$ numtype) $\mid\mid$ (k $<$ 1))\\
        {\\
          fprintf (stderr, "$\backslash$n$\backslash$n wrong number of MYPLASMODEL material in function\\
                   mechmat::readmatchar (\%s, line \%d).$\backslash$n",\_\_FILE\_\_,\_\_LINE\_\_);\\
        }\\
        MYPLASMODELPT[k - 1].read (in);\\
      }\\
      break;\\
}

Where user should replace MYPLASMODEL with name of the new class {\sf myplasmodel} and MYPLASMODELPT with
name of the pointer which has been used in the {\sf mechmat} class declaration in the file mechmat.hpp. This
statement block the first allocates array with new material then folows loop where the number of sets of the
parameters is read and then the {\sf read} function for the given material type is called.\\

Function {\sf mechmat::matstiff (matrix \&d,long ipp,long im,long ido)} contains {\tt switch} statement which calls appropriate function
for assembling stiffness matrix for given material. Here the new {\tt case} for the  {\sf myplasmodel} would be included.
It is possible to take part of {\tt case} statement from another material and copy it into new {\tt case} and then
make only one modification. That means replacement name of pointer to array with materials, with name which has been
used in the {\sf mechmat} class declaration in the mechmat.h. The first parameter {\it d} of this function
represents allocated result stiffnes matrix, the second one {\it ipp} is integration point number. {\it im} is index of
material type in the array {\sf tm} of given integration point, {\it ido} stands for index of the first internal variable
in the array {\sf other} or {\sf eqother} of given integration point.\\

Function {\sf mechmat::yieldfunction (long ipp, long idpm, matrix \&sig, vector \&q)} contains {\tt switch} statement which calls
appropriate function for yield function of given material. Here the new {\tt case} for the  {\sf myplasmodel} would
be included. It is possible to take part of {\tt case} statement from another material and copy it into new {\tt case}
and then make only one modification. That means replacement name of pointer to array with materials, with name which
has been used in the {\sf mechmat} class declaration in the mechmat.h. The first parameter {\it ipp} is integration
point number, {\it idpm} is index of given plasticity material type in the {\sf tm} array of given integration point, 
{\it sig} is matrix with stress tensor and {\it q} is vector with hardening parameters.\\

Function {\sf mechmat::dfdsigma (long ipp, long idpm, matrix \&sig, vector \&q, matrix \&dfds)} contains {\tt switch} statement
which calls appropriate function for derivatives of yield function with respect on stresses for given material. Here
the new {\tt case} for the  {\sf myplasmodel} would be included. It is possible to take part of {\tt case} statement
from another material and copy it into new {\tt case} and then make only one modification. That means replacement name
of pointer to array with materials, with name which has been used in the {\sf mechmat} class declaration
in the mechmat.h. Parameter {\it ipp} is integration point number, {\it idpm} is index of given plasticity material type 
in the {\sf tm} array of given integration point, {\it sig} is matrix with stress tensor and {\it q} is vector with hardening parameters.
{\it dfds} is output parameter i.e. allocated matrix with results.\\

Function {\sf mechmat::dgdsigma (long ipp, long idpm, matrix \&sig, vector \&q, matrix \&dgds)} computes derivatives of plastic
potential with respect on stresses. Implementing procedure is same as in the previous function. Denotation of function
parameters is same too.\\

Function {\sf mechmat::dfdq (long ipp, long idpm, vector \&dq)} computes derivatives of yield function with respect on hardening
parameters. Implementing procedure is same as in the previous function. If plasticity model hasn't hardening, let
the {\tt case} statement for given material empty only with keyword {\tt break}. Parameter {\it ipp} is integration
point number, {\it dq} contains values with actual hardening parameters and togeather serv as output parameter for results.
{\it q} is vector with hardening parameters.\\

Function {\sf mechmat::plasmodscalar (long ipp, long idpm, matrix \&sig, vector \&eps, vector \&epsp, vector \&qtr)} computes
expression $\ppd{f}{q_i}H\ppd{g}{q_i}$ for cutting plane method. Implementing procedure is same as in the previous function.
If plasticity model hasn't hardening, let the {\tt case} statement for given material empty only with keyword {\tt break}.
Parameter {\it ipp} is integration point number,{\it idpm} is index of given plasticity material type in the {\sf tm} array 
of given integration point, {\it sig} is matrix with stress {\bf tensor}, {\it eps} is vector of total strains, {\it epsp} is vector of plastic strains, 
{\it qtr} contains values with actual hardening parameters.\\

Function {\sf mechmat::updateq (long ipp, long idpm, double dgamma, vector \&epsp, vector \&q)}
Implementing procedure is same as in the previous function. If plasticity model hasn't hardening, let the {\tt case}
statement for given material empty only with keyword {\tt break}. Parameter {\it ipp} is integration point number,{\it idpm} is 
index of given plasticity material type in the {\sf tm} array of given integration point,
{\it dgamma} is increment of $\gamma$ parameter in the cutting plane method, {\it epsp} is plastic strains in the
trial state $\varepsilon^{pl}_{trial}$, {\it q} contains values of hardening parameters. See file mechmat.cpp, function
{\sf void mechmat::cutting\_plane (long ipp, double \&gamma, vector \&epsn, vector $\&$epsp, vector $\&$q)}.\\

Function {\sf mechmat::updateipvalmat (long ipp,long im,long ido)} contains {\tt switch} statement which calls appropriate function 
for updateing valus in the integration point {\sf eqother} array for given material. That means this function should replace 
backup values of equilibrium state from the previous step with values that has been actually computed and equilibrium state for current step
has been reached. Here the new {\tt case} for the  {\sf myplasmodel} would be included. Implementing procedure is same
as in the previous function. {\it ipp} is integration point number, {\it im} is index of material type in the array {\sf tm} of given integration point, 
{\it ido} stands for index of the first internal variable in the array {\sf other} or {\sf eqother} of given integration point.\\

Function {\sf mechmat::computenlstresses (long ipp,long im,long ido)} computes stresses in nonlinear mechanic problems for given integration
point and material type. Implementing procedure is same as in the previous function. Parameters are denoted same too.

Function {\sf mechmat::givencompother (long ipp,long im)} returns for given material type number of internal variables. That means
number of elements in the arrays {\sf other} and {\sf eqother} which are used for given material type.
Contents of this case could be partially copied form {\tt case } of another plastic
material i.e. druckerprager :\\
{\tt
    ncompo=ncompstr+1+NUMHARDPARAM;\\
    //  other[0 - (ncompstr-1)]=plastic strain components\\
    //  other[ncompstr]=gamma\\
    //  other[ncompstr+1 - (ncompstr+1+NUMHARDPARAM-1)]=hardening parameters\\
}\\
NUMHARDPARAM is not really variable or macro in MEFEL but denotes the number of hardening parameters.
This number will depend on given type of plasticity. User have to save on the first positions reached
plastic strains $\varepsilon^{pl}_{ij}$ followed by the reached value of $\gamma$ coefficient for the cutting
plane method. Rest saved values depend on the given plastic model. Generally it will be hardening parameters.
Number of strain component and number of stored hardening parameters + $\gamma$ give us number of necessary
stored values at the {\sf other} or {\sf eqother} arrays.

%%%%%%%%%%%%%%%%%%%%%%%%%%%%%%%%%%%%%%%%%%%%%%%%%%%%%%%%%%%%%%
%%%%%%%%%%%%%%%%%%%%%%%%%%%%%%%%%%%%%%%%%%%%%%%%%%%%%%%%%%%%%%
%%%%%%%%%%%%%%%%%%%%%%%%%%%%%%%%%%%%%%%%%%%%%%%%%%%%%%%%%%%%%%
%%%%%%%%%%%%%%%%%%%%%%%%%%%%%%%%%%%%%%%%%%%%%%%%%%%%%%%%%%%%%%
%  DAMAGE

\subsection{Implementation of new scalar damage  model}
\subsubsection {Implementing new class of scalar damage}
This section describes what the user needs when implements new scalar damage model
to MEFEL. First of all user needs damage parameter function $\tilde {\varepsilon}$ and damage function $f$,

Each new material means the user should found new class with name of scalar damage model {\sf myscdammodel}.
This class should contain material parameters as a data members of this class. Note that the MEFEL
in case scalar damage material suppose two materials. The first refers to damage material and the second
one refers to elastic material. So the elastic parameters such as th Young modulus $E$ needn't be
included to the {\sf myscdammodel}. In addition the folowing methods should be defined :
\begin{itemize}
\item
{\sf read(FILE* in)}
\item
{\sf void damfuncpar(long ipp, vector \&eps, vector \&kappa)}
\item
{\sf double damfunction(long ipp, vector \&kappa, vector \&eps)}
\item
{\sf void matstiff (matrix \&d, long ipp, long ido)}
\item
{\sf void nlstresses (long ipp, long im, long ido)}
\item
{\sf void updateval (long ipp, long im, long ido)}

\end{itemize}

Function {\sf read(FILE* in)} should read material parameters for given damage from current
position in the opened text file given by the function parameter {\it in}.\\

Function {\sf void damfuncpar(long ipp, vector \&eps, vector \&kappa)} returns value of the damage
function parameter $\tilde {\varepsilon}$ for given damage model {\sf myscaldammodel}.
Returned value is double type. The first parameter {\it eps} references to vector which contains strains.
The second parameter {\it kappa} is vector which contains current values of damage parameters (For
introduced models this parameter could be {\sf double} type, but {\sf vector} is prepared for another
future models which could have more then one damage parameter of damage function. In such case this
function should return {\sf vector} type instead of {\sf double}).\\

Function {\sf void damfunction(long ipp, vector  \&kappa, vector \&eps)} returns value of the damage
variable {\it d} for given damage model {\sf myscaldammodel}. Returned value is a{\sf double}.
The first parameter {\it kappa} is vector which contains current values of damage parameters (For
introduced models this parameter could be a {\sf double}, but {\sf vector} is prepared for another
future models which could have more then one damage parameter of damage function. In such case this
function should return {\sf vector} type instead of {\sf double}). The second parameter {\it eps}
references to vector which contains strains.\\

Function {\sf void matstiff (matrix  \&d, long ipp, long ido)} assemble material stiffness matrix for given
integration point and material type. The first parameter {\it d} is allocated matrix which will be
filled up with values. The second parameter is integration point number which should be used for
access characteristic values in the integration point. User may use standard stiffness matrix for
linear elastic material or when it is available he may use tangent/secant stiffness matrix (it will reduce
number of iteration steps).Parameter {\it ido} is index in the {\sf other/eqother} array of the given integration 
point, from where the internal variables are stored.
\\

Function {\sf void nlstresses (long ipp, long im, long ido)} computes stresses for nonlinear material. Parameter
{\it ipp} is integration point number where the stresses should be evaluated. Parameter {\it im} denotes
index material type in the array {\sf tm} of given integration point. {\it ido} is index in the
{\sf other/eqother} array of the given integration point, from where the internal variables are stored.

The function should work following way:
\begin{itemize}
\item
Take reached strains $\varepsilon_{ij}$ and reached {\sf eqother} values in the integration point number
{\it ipp} (See class {\sf mechmat, intpoint}). For scalar damage models should contain reached damage
parameter and damage variable  $d$.
\item
Call scalar damage solver function built in {\sf mechmat} i.e {\sf scal\_dam\_sol} with parameters filled
from previous step. This method computes damage variable and stresses.
\item
Store result stresses and coefficient $\kappa$ and possible another values to the {\sf other}
array in the integration point.
\end{itemize}
For example see {\sf scaldam::nlstresses(long ipp, long im, long ido)} in the file scaldam.cpp.\\

Function {\sf void updateval (long ipp, long im, long ido)} copies values of {\sf other} array to 
backup array {\sf eqother} in case that equilibrium for current step is reached. 
Parameter {\it ipp} is integration point number whose the {\sf eqother} array should be updated.
Parameter {\it im} denotes index material type in the array {\sf tm} of given integration point. 
{\it ido} is index in the {\sf other/eqother} array of the given integration point, from where the 
updated internal variables are stored.


\subsubsection {Including new scalar damage class to MEFEL files}
Thus the new class {\sf myscdammodel} is written to files myscaldammodel.cpp${\rm |}$h. 
Now the user should include this class to the MEFEL. Here is list of files which have to 
be modified :

{\bf alias.h}\\
So called material alias {\sf mattype} have to be added. Alias means the substitutive shortcut
for C++ enumaration integer number. Decision which number and name for its alias depends on the user.
Only restrictions are that name and number have to be unique in the {\sf mattype} scope. Also note
the name haven't to be same as the name of the scalar damage class. This aliases are used in the
{\tt switch} statements as values for {\tt case}. Numeric value of the aliases are used in the input
files for the MEFEL.\\

{\bf mechmat.h}\\
This file contains declaration of the {\sf mechmat} class. This class contains also pointers to arrays for each
material class. Thus the user have to include new class {\sf myscaldammodel} to the {\sf mechmat} class declaration.
The name of the pointer is arbitrary expecting used names of course. Note that the user should use predeclaration
of {\sf myplasmodel} class or insert include file with {\sf myscaldammodel} class declaration.\\


{\bf mechmat.cpp}\\
\begin{itemize}
\item {\sf void mechmat::readmatchar (FILE *in)}
\item {\sf void mechmat::matstiff (matrix \&d,long ipp,long im,long ido)}
\item {\sf void mechmat::damfuncpar(long ipp, long im,vector \&eps, vector \&kappa)}
\item {\sf double mechmat::damfunction(long ipp, long im, vector \&kappa, vector \&eps)}
\item {\sf void mechmat::updateipvalmat (long ipp, long im, long ido)}
\item {\sf void mechmat::computenlstresses (long ipp, long im, long ido)}
\item {\sf long mechmat::givencompother (long ipp,long im)}
\end{itemize}
Note that include file with full class declaration for given {\sf myscaldammodel} should be included at the begining
of mechmat.cpp\\

In function {\sf mechmat::readmatchar (FILE *in)} the actual values of material parameters for given material type. This function
contains {\tt switch} statement where the new {\tt case} for the  {\sf myscaldammodel} would be included.
It is possible to take part of {\tt case} statement from another material and copy it into new {\tt case} and then
make some modification. Folowing lines shows example of such a new {\tt case} :\\
{\tt
      if (Mespr==1)  fprintf (stdout,"$\backslash$n number of MYSCALDAMMODEL materials \%ld",numtype);\\
      MYSCALDAMMODELPT = new MYSCALDAMMODEL [numtype];\\
      for (j = 0; j $<$ numtype; j++){\\
        k = numtype + 1;\\
        fscanf (in,"\%ld",\&k);\\
        if ((k $>$ numtype) $\mid\mid$ (k $<$ 1))\\
        {\\
          fprintf (stderr, "$\backslash$n$\backslash$n wrong number of MYSCALDAMMODEL material in function\\
                   mechmat::readmatchar (\%s, line \%d).$\backslash$n",\_\_FILE\_\_,\_\_LINE\_\_);\\
        }\\
        MYSCALDAMMODELPT[k - 1].read (in);\\
      }\\
      break;\\
}

Where user should replace MYSCALDAMMODEL with name of the new class {\sf myscaldammodel} and MYSCALDAMMODELPT with
name of the pointer which has been used in the {\sf mechmat} class declaration in file mechmat.hpp. This
statement block first allocates array with new material then folows loop where the number of set of parameters is
read and then the {\sf read} function for the given material type is called.

Function {\sf mechmat::matstiff (matrix \&d,long ipp,long im,long ido)} contains {\tt switch} statement which calls appropriate function
for assembling stiffness matrix for given material. Here the new {\tt case} for the  {\sf myscaldammodel} would be included.
It is possible to take part of {\tt case} statement from another material and copy it into new {\tt case} and then
make only one modification. That means replacement name of pointer to array with materials, with name which has been used
in the {\sf mechmat} class declaration in the mechmat.h. The first parameter {\it d} of this function represents allocated
result stiffnes matrix, the second one {\it ipp} is integration point number.Parameter {\it im} denotes
index material type in the array {\sf tm} of given integration point. {\it ido} is index in the
{\sf other/eqother} array of the given integration point, from where the internal variables are stored.


Function {\sf mechmat::damfuncpar(long ipp, long im,vector \&eps, vector \&kappa)} contains {\tt switch} statement which calls
appropriate function for computing parameter $\tilde{\varepsilon}$ for damage function of given material. Here the new
{\tt case} for the  {\sf myscaldammodel} would be included. It is possible to take part of {\tt case} statement from
another material and copy it into new {\tt case} and then make only one modification. That means replacement
name of pointer to array with materials, with name which has been used in the {\sf mechmat} class declaration in the mechmat.h.
The first parameter {\it ipp} is integration point number. Parameter {\it im} denotes index material type in the array {\sf tm} 
of given integration point. {\it eps} is strain vector and {\it kappa} is vector with
damage parameters.

Function {\sf mechmat::damfunction(long ipp, long im,vector \&kappa, vector \&eps)} contains {\tt switch} statement which calls
appropriate damage function for given material. This function returns value of damage variable $d$. Here the new
{\tt case} for the  {\sf myscaldammodel} would be included. It is possible to take part of {\tt case} statement from
another material and copy it into new {\tt case} and then make only one modification. That means replacement name of pointer
to array with materials, with name which has been used in the {\sf mechmat} class declaration in the mechmat.h.
Parameter {\it ipp} is integration point number. Parameter {\it im} denotes index material type in the array {\sf tm} of given integration point. 
{\it kappa} is vector with damage function parameters and {\it eps} is strain vector.

Function {\sf mechmat::updateipvalmat (long ipp, long im, long ido)} contains {\tt switch} statement which calls appropriate function 
for updateing valus in the integration point {\sf eqother} array for given material. That means this function should replace backup values 
of equilibrium state from the previous step with values that has been actually computed and stored in the {\sf other} array and equilibrium state 
for current step has been reached. Here the new {\tt case} for the  {\sf myscaldammodel} would be included. Implementing procedure is same
as in the previous function.

Function {\sf computenlstresses (long ipp, long im, long ido)} computes stresses in nonlinear mechanic problems for given integration
point and material type. Implementing procedure is same as in the previous function. Parameter {\it ipp} is integration
point number. Rest parameters have the same notation and the same mean as in the previous function.

Function {\sf mechmat::givencompother (long ipp,long im)} returns for given material type number of internal variables. That means
number of elements in the arrays {\sf other} and {\sf eqother} which are used for given material type.
Contents of this case could be partially copied form {\tt case } of another scalar damage
material i.e. scaldamage :\\
{\tt
    ncompo=2;\\
    //  other[0]=equivalent strain\\
    //  other[1]=damage\\
}\\
Number of saved values depend on the given damage model. Generally it will be equivalent strains and damage.
