\chapter{Finite elements for mechanical problems}
\label{finitelements}
\index{finite elements}

The finite element method is a numerical method for solution of academic as well as real world problems.
Derivation of basic equations, matrices and vectors can be done in several ways. One possibility is
the Galerkin method which is shortly described in the following section. For more details about the
finite element method or for introduction into the method, see references
\cite{belytschko:nonlin}.

%%%%%%%%%%%%%%%%%%%%%%%%%%%%%%%%%%%%%%%%%%%%%%%%%%%%%%%%%%%%%%%%%%%%%%%%%%%%%%%%%%%%%%%%%%%%%%%%%%%
\section{Galerkin method}
%%%%%%%%%%%%%%%%%%%%%%%%%%%%%%%%%%%%%%%%%%%%%%%%%%%%%%%%%%%%%%%%%%%%%%%%%%%%%%%%%%%%%%%%%%%%%%%%%%%


Galerkin method is used for derivation of basic matrices and vectors


\section{General comments}
\label{sectgencom}
Central notion of finite element method applied in mechanical problem is stiffness matrix.
Derivation of general stiffness matrix can be done with help of Ritz or Galerkin method.
There are many ways how to compute stiffness matrices. One strategy of evaluation of
stiffness matrix is mentioned in this section.


The stiffness matrix \index{matrix!stiffness} of any finite element can be expressed in form
\begin{equation}
\mbf{K} = \int_{\Omega_e} \mbf{B}^T \mbf{D} \mbf{B} {\rm d}\Omega_e\ ,
\end{equation}
where $\mbf{B}$ denotes geometric matrix, $\mbf{D}$ stands for stiffness matrix of the material
and $\Omega_e$ denotes one finite element. Both matrices generally depend on coordinates and therefore
evaluation of the integral must be done numerically. The stiffness matrix is computed
\begin{equation}
\mbf{K} = \sum_{i=1}^{n} \mbf{B}_i^T \mbf{D}_i \mbf{B}_i w_i J_i\ ,
\end{equation}
where $n$ is the number of integration points, $w_i$ denotes the $i$-th weigth, $\mbf{B}_i$ denotes
geometric matrix \index{matrix!geometric} evaluated at the $i$-th integration point,
$\mbf{D}_i$ denotes the stiffness matrix of the material \index{matrix!stifness of material} evaluated at the
$i$-th integration point and $J_i$ stands for Jacobian evaluated at the $i$-th integration point.

Consider decomposition of the geometric matrix into several blocks. Such decomposition is enforced
in the finite element method very often with respect to various approximation functions for particular
unknowns etc. Decomposition can be written in form
\begin{equation}
\mbf{B} = \left(\begin{array}{c}
\mbf{B}_1
\\
\mbf{B}_2
\\
\vdots
\\
\mbf{B}_{n_b}
\end{array}\right)\ ,
\end{equation}
where $n_b$ means the number of blocks.
Let the number of DOFs defined on finite element is $n_d$ and the number of strain components is $n_s$.
The geometric matrix $\mbf{B}$ is from the space $R^{n_s \times n_d}$. Particluar blocks of the geometric
matrix are defined as $\mbf{B}_i \in R^{n_{s_i} \times n_d}$, where simple equality
$\sum_{i=1}^{n_b} n_{s_i} = n_s$ holds. With respect to the decomposition of the geometric matrix the
stiffness matrix of the material is also decomposed
\begin{equation}
\mbf{D} = \left(\begin{array}{cccc}
\mbf{D}_{11}    & \mbf{D}_{12} & \ldots & \mbf{D}_{1n_b}
\\
\mbf{D}_{21}    & \mbf{D}_{22} &        & \vdots
\\
\vdots          &              & \ddots &
\\
\mbf{D}_{n_b 1} & \ldots       &        & \mbf{D}_{n_b n_b}
\end{array}\right)\ .
\end{equation}
From the previous text follows that block $\mbf{D}_{ij}$ is from the space $R^{n_{s_i} \times n_{s_j}}$
The stiffness matrix of the finite element has form
\begin{equation}\label{eqstiffmatblock}
\mbf{K} = \int_{\Omega_e} \sum_{i=1}^{n_b} \sum_{j=1}^{n_b} \mbf{B}_i^T \mbf{D}_{ij} \mbf{B}_j {\rm d}\Omega_e\ .
\end{equation}
Analysis of the previous relation leads to the aim that the matrix $\mbf{K}$ is from the space $R^{n_d \times n_d}$
because $\mbf{B}_i^T \in R^{n_d \times n_{s_i}}$, $\mbf{D}_{ij} \in R^{n_{s_i} \times n_{s_j}}$ and
$\mbf{B}_j \in R^{n_{s_j} \times n_d}$.

Integration of the sum from equation (\ref{eqstiffmatblock}) cannot be done directly in general cases and therefore
numerical integration is used. Before application of numerical integration, let equation (\ref{eqstiffmatblock}) is
rewrite into the form
\begin{equation}\label{eqstiffmatblock2}
\mbf{K} = \sum_{i=1}^{n_b} \sum_{j=1}^{n_b} \int_{\Omega_e} \mbf{B}_i^T \mbf{D}_{ij} \mbf{B}_j {\rm d}\Omega_e\ ,
\end{equation}
where summation and integration were exchanged. The last formulation is better for numerical integration which can be
written as
\begin{equation}\label{eqstiffmatblockint}
\mbf{K} = \sum_{i=1}^{n_b} \sum_{j=1}^{n_b} \left(\sum_{l=1}^{m_{ij}}
\mbf{B}_i^T(\mbf{\xi}_l) \mbf{D}_{ij}(\mbf{\xi}_l) \mbf{B}_j(\mbf{\xi}_l) J(\mbf{\xi}_l) w_l\right)\ ,
\end{equation}
where $m_{ij}$ denotes the number of integration points used for the $ij$-th contribution. $\mbf{\xi}_l$
denotes coordinates (original or transformed) of the $l$-th integration point. From Equation (\ref{eqstiffmatblockint})
is clear, that particular contributions $i,j$ can be integrated by different integration scheme.
%%%%%%%%%%%%%%%%%%%%%%%%%%%%%%%%%%%%%%%%%%%%%%%%%%%%%%%%
%%%%%%%%%%%%%%%%%%%%%%%%%%%%%%%%%%%%%%%%%%%%%%%%%%%%%%%%
\section{List of implemented finite elements}
\begin{itemize}
\item{2D bar element}
\item{2D beam element}
\item{plane triangular element with linear approximation functions}
\item{plane triangular element with quadratic approximation functions}
\item{plane triangular element with rotational degrees of freedom}
\item{plane quadrilateral element with bi-linear approximation functions}
\item{plane quadrilateral element with bi-quadratic approximation functions}
\item{plane quadrilateral element with rotational degrees of freedom}
\item{subparametric plane triangular element with quadratic approximation functions and linear shape functions}
\item{plate element -- constant curve triangle}
\item{triangular plate element based on discerete Kirchhoff theory}
\item{dst element}
\item{quadrilateral plate element based on discerete Kirchhoff theory}
\item{shell triangular element}
\item{tetrahedral element with linear approximation functions}
\item{hexahedral element with tri-linear approximation functions}
\item{hexahedral element with tri-quadratic approximation functions}
\end{itemize}

%%%%%%%%%%%%%%%%%%%%%%%%%%%%%%%%%%%%%%%%%%%%%%%%%%%%%%%%%%%%%%%%%%%%
\section{Quadrilateral element for plane stress/strain ana\-lysis with bilinear approximation functions}
\label{sectrectelemlinfunct}
\index{element!quadrilateral!bi-linear}

\begin{tabular}{|l|l|}
\hline
number of blocks & 2
\\ \hline
block components & $(\varepsilon_{xx},\varepsilon_{yy}), (\gamma_{xy})$
\\ \hline
numer. integration & ($2 \times 2$ points), (1 int. point)
\\ \hline
nodal DOF & 2 displacements: $u$ in $x$ direction and $v$ in $y$ direction
\\ \cline{2-2}
 & $u,v$
\\ \hline
\end{tabular}

Approximation functions
\begin{equation}
N_1^{(1)} = \del{1}{4}(1+\xi)(1+\eta)\ ,
\end{equation}
\begin{equation}
N_2^{(1)} = \del{1}{4}(1-\xi)(1+\eta)\ ,
\end{equation}
\begin{equation}
N_3^{(1)} = \del{1}{4}(1-\xi)(1-\eta)\ ,
\end{equation}
\begin{equation}
N_4^{(1)} = \del{1}{4}(1+\xi)(1-\eta)\ .
\end{equation}
For more details about approximation functions see GEFEL ??

%%%%%%%%%%%%%%%%%%%%%%%%%%%%%%%%%%%%%%%%%%%%%%%%%%%%%%%%%%%%%%%%%%%%
\section{Quadrilateral element for plane stress/strain ana\-lysis with biquadratic approximation functions}
\index{element!quadrilateral!bi-quadratic}
\label{sectrectelemquadfun}

\begin{tabular}{|l|l|}
\hline
number of blocks & 2
\\ \hline
block components & $(\varepsilon_{xx},\varepsilon_{yy}), (\gamma_{xy})$
\\ \hline
numer. integration & ($2 \times 2$ points), (1 int. point)
\\ \hline
nodal DOF & 2 displacements: $u$ in $x$ direction and $v$ in $y$ direction
\\ \cline{2-2}
 & $u,v$
\\ \hline
\end{tabular}

Approximation functions
\begin{equation}
N_1^{(2)} = \del{1}{4}(1+\xi)(1+\eta)(\xi+\eta-1)\ ,
\end{equation}
\begin{equation}
N_2^{(2)} = \del{1}{4}(1-\xi)(1+\eta)(\eta-\xi-1)\ ,
\end{equation}
\begin{equation}
N_3^{(2)} = \del{1}{4}(1-\xi)(1-\eta)(-\xi-\eta-1)\ ,
\end{equation}
\begin{equation}
N_4^{(2)} = \del{1}{4}(1+\xi)(1-\eta)(\xi-\eta-1)\ ,
\end{equation}
\begin{equation}
N_5^{(2)} = \del{1}{2}(1-\xi^2)(1+\eta)\ ,
\end{equation}
\begin{equation}
N_6^{(2)} = \del{1}{2}(1-\xi)(1-\eta^2)\ ,
\end{equation}
\begin{equation}
N_7^{(2)} = \del{1}{2}(1-\xi^2)(1-\eta)\ ,
\end{equation}
\begin{equation}
N_8^{(2)} = \del{1}{2}(1+\xi)(1-\eta^2)\ .
\end{equation}
For more details about approximation functions see GEFEL ??

%%%%%%%%%%%%%%%%%%%%%%%%%%%%%%%%%%%%%%%%%%%%%%%%%%%%%%%%%%%%%%%%%%%%
\section{Quadrilateral element for plane stress/strain ana\-lysis with rotational degrees of freedom}
\index{element!quadrilateral!with rot. DOFs}
\label{sectrecelemrotdof}

\begin{tabular}{|l|l|}
\hline
number of blocks & 2
\\ \hline
block components & $(\varepsilon_{xx},\varepsilon_{yy}), (\gamma_{xy})$
\\ \hline
numer. integration & ($2 \times 2$ points), (1 int. point)
\\ \hline
nodal DOF & 2 displacements: $u$ in $x$ direction and $v$ in $y$ direction
\\ \cline{2-2}
 & 1 rotation around $z$ axis
\\ \cline{2-2}
 & $u,v,\omega_z$
\\ \hline
\end{tabular}

Approximation functions
\begin{equation}
N_1^{\omega, u} = \del{l_4}{16}n_x^{(4)}(1+\xi)(1-\eta^2) - \del{l_1}{16}n_x^{(1)}(1-\xi^2)(1+\eta)\ ,
\end{equation}
\begin{equation}
N_2^{\omega, u} = \del{l_1}{16}n_x^{(1)}(1-\xi^2)(1+\eta) - \del{l_2}{16}n_x^{(2)}(1-\xi)(1-\eta^2)\ ,
\end{equation}
\begin{equation}
N_3^{\omega, u} = \del{l_2}{16}n_x^{(2)}(1-\xi)(1-\eta^2) - \del{l_3}{16}n_x^{(3)}(1-\xi^2)(1-\eta)\ ,
\end{equation}
\begin{equation}
N_4^{\omega, u} = \del{l_3}{16}n_x^{(3)}(1-\xi^2)(1-\eta) - \del{l_4}{16}n_x^{(4)}(1+\xi)(1-\eta^2)\ ,
\end{equation}
\begin{equation}
N_1^{\omega, v} = \del{l_4}{16}n_y^{(4)}(1+\xi)(1-\eta^2) - \del{l_1}{16}n_y^{(1)}(1-\xi^2)(1+\eta)\ ,
\end{equation}
\begin{equation}
N_2^{\omega, v} = \del{l_1}{16}n_y^{(1)}(1-\xi^2)(1+\eta) - \del{l_2}{16}n_y^{(2)}(1-\xi)(1-\eta^2)\ ,
\end{equation}
\begin{equation}
N_3^{\omega, v} = \del{l_2}{16}n_y^{(2)}(1-\xi)(1-\eta^2) - \del{l_3}{16}n_y^{(3)}(1-\xi^2)(1-\eta)\ ,
\end{equation}
\begin{equation}
N_4^{\omega, v} = \del{l_3}{16}n_y^{(3)}(1-\xi^2)(1-\eta) - \del{l_4}{16}n_y^{(4)}(1+\xi)(1-\eta^2)\ ,
\end{equation}

%%%%%%%%%%%%%%%%%%%%%%%%%%%%%%%%%%%%%%%%%%%%%%%%%%%%%%%%%%%%%%
\section{Triangular element for plate problem ba\-sed on Mindlin theory --- CCT}
%\markright{CCT}
%%%%%%%%%%%%%%%%%%%%%%%%%%%%%%%%%%%%%%%%%%%%%%%%%%%%%%%%%%%%%%
\index{element!triangular!plate - CCT}
\label{sectcctelem}

\begin{tabular}{|l|l|}
\hline
number of blocks & 2
\\ \hline
block components & $(\kappa_{xx},\kappa_{yy},\kappa_{xy}), (\gamma_{xz},\gamma_{yz})$
\\ \hline
numer. integration & (3 int. points), (1 int. point)
\\ \hline
nodal DOF & 1 deflection and 2 rotations around $x$ and $y$ axes
\\ \cline{2-2}
 & $w, \phi_x, \phi_y$
\\ \hline
\end{tabular}

Approximation functions are
\begin{equation}
N_1^{(w)} = L_1\ ,
\end{equation}
\begin{equation}
N_2^{(w)} = L_2\ ,
\end{equation}
\begin{equation}
N_3^{(w)} = L_3\ ,
\end{equation}

\begin{equation}
N_1^{(\phi)} = L_1 L_2 \del{l_{12}}{2}s_y^{12} - L_3 L_1 \del{l_{31}}{2}s_y^{31}\ ,
\end{equation}
\begin{equation}
N_2^{(\phi)} = L_2 L_3 \del{l_{23}}{2}s_y^{23} - L_1 L_2 \del{l_{12}}{2}s_y^{12}\ ,
\end{equation}
\begin{equation}
N_3^{(\phi)} = L_3 L_1 \del{l_{31}}{2}s_y^{31} - L_2 L_3 \del{l_{23}}{2}s_y^{23}\ ,
\end{equation}

\begin{equation}
N_1^{(\chi)} = L_3 L_1 \del{l_{31}}{2}s_x^{31} - L_1 L_2 \del{l_{12}}{2}s_x^{12}\ ,
\end{equation}
\begin{equation}
N_2^{(\chi)} = L_1 L_2 \del{l_{12}}{2}s_x^{12} - L_2 L_3 \del{l_{23}}{2}s_x^{23}\ ,
\end{equation}
\begin{equation}
N_3^{(\chi)} = L_2 L_3 \del{l_{23}}{2}s_x^{23} - L_3 L_1 \del{l_{31}}{2}s_x^{31}\ .
\end{equation}

%%%%%%%%%%%%%%%%%%%%%%%%%%%%%%%%%%%%%%%%%%%%%%%%%%%%%%%%%%%%%%
\section{Triangular plate element}
%%%%%%%%%%%%%%%%%%%%%%%%%%%%%%%%%%%%%%%%%%%%%%%%%%%%%%%%%%%%%%
\index{element!triangular!DKT}
\label{sectdktplate}

Batoz, Bathe, Ho - Int. Journ. for Num. Meth. in Eng vol. 15 1980

\begin{tabular}{|l|l|}
\hline
number of blocks & 1
\\ \hline
block components & $(\kappa_x,\kappa_y,\kappa_{xy})$
\\ \hline
nodal DOF & $w,\phi_x,\phi_y$
\\ \hline
\end{tabular}

Triangular plate DKT element of type {\tt dktelem} with rotational of the normal to the 
undeformed surface. This rotational vary quadratically over element where ... are the nodal
values and at the mid-nodes. W varies cubically along the sides. Kirchhoff theory is
imposed at discrete the mid-nodes.

\begin{equation}
\phi_s + \od{w}{\mbf{s}} = 0
\end{equation}
where $\mbf{s}$ is vector of side.

%%%%%%%%%%%%%%%%%%%%%%%%%%%%%%%%%%%%%%%%%%%%%%%%%%%%%%%%%%%%%%
\section{DST element}
%%%%%%%%%%%%%%%%%%%%%%%%%%%%%%%%%%%%%%%%%%%%%%%%%%%%%%%%%%%%%%
\label{sectdstelem}
\index{element!triangular!DST} Batoz, Katili - Int. Journ. for Num. Meth. in Eng vol. 35 1992

\begin{tabular}{|l|l|}
\hline
class name & dstelem
\\ \hline
object name & Dst
\\ \hline
location & MEFEL/SRC/dst.cpp,
\\
 & MEFEL/SRC/dst.h
\\ \hline
number of blocks & 3
\\ \hline
block components & $(\kappa_x,\kappa_y,\kappa_{xy}),(\gamma_{xz}),(\gamma_{yz})$
\\ \hline
nodal DOF & $w,\phi_x,\phi_y$
\\ \hline
\end{tabular}

Triangular plate DST element of type {\tt dstelem} with the displacement field and strain-displacement relations 
are obtained with the Mindlin hypothesis. Nodal DOF are rotational of the normal to the  undeformed surface and displacement w. 
This element is based on the bending stiffness matrix and the shear stiffness matrix with independent strains gamma 
constant per element. Gamma is expressed in terms of fi using the moment equilibrium equations and the constitutive relations.

\begin{equation}
\ppd{m_x}{x} + \ppd{m_{xy}}{y} - T_x = 0
\end{equation}
\begin{equation}
\ppd{m_{xy}}{x} + \ppd{m_y}{y} + T_y = 0
\end{equation}
where $m_x, m_y m_{xy} T_x T_y$ are bending moments and shear forces.

Rotational vary quadratically over element where 
\begin{eqnarray}
\phi_x &=& \sum_{i=1}^{3} N_i\phi_{xi}+\sum_{k=4}^{6} P_k\sin{k}\alpha_{k}\\
\phi_y &=& \sum_{i=1}^{3} N_i\phi_{yi}+\sum_{k=4}^{6} P_k\cos{k}\alpha_{k}
\end{eqnarray}
where $N_1=r, N_2=1-r-s, N_3=s, P_4=4r(1-r-s)s, P_5=(1-r-s)r, P_6=4rs$
and $k=4,5,6 $ are sides (mid-nodes) between nodes $i=1,2,3$.


$w$ through element is not obtained from formulation. Only nodal displacement is imposed at discrete the mid-nodes from 
constant constrains.
Indetermined parameters $\alpha_{k}$ are obtained from the following constraints at each mid-point:
\begin{equation}
\gamma_s = \od{w}{\mbf{s}} + \phi_{s}
\end{equation}
where $\mbf{s}$ is vector of side.




%%%%%%%%%%%%%%%%%%%%%%%%%%%%%%%%%%%%%%%%%%%%%%%%%%%%%%%%%%%%%%
\section{Quadrilateral plate element}
%%%%%%%%%%%%%%%%%%%%%%%%%%%%%%%%%%%%%%%%%%%%%%%%%%%%%%%%%%%%%%
\index{element!quadrilateral!Q4}  Roufael - Computers \& structural vol. 54/5 1995
\label{sectq4elem}

\begin{tabular}{|l|l|}
\hline
number of blocks & 2
\\ \hline
block components & $(\kappa_x,\kappa_y,\kappa_{xy}),(\gamma_{xz},\gamma_{yz})$
\\ \hline
nodal DOF & $w,\phi_x,\phi_y$
\\ \hline
\end{tabular}

Quadrilateral plate Q4 element of type {\tt q4plate} with the displacement field and strain-displacement relations 
are obtained with the Mindlin hypothesis. This element is based on the bending stiffness matrix and the shear stiffness
matrix with independent strains gamma. 
Nodal DOF are rotational of the normal to the undeformed surface and displacement w and Four independent unknown parameters. 
The shape function are three'th-order for $\phi_x$, $\phi_y$, and four'th-order for w. 
\begin{eqnarray}
1 + x + y + xy + x^2 + y^2 + x^2y + xy^2 + x^3 + y^3 + x^3y + xy^3 + 0 + 0 + 0 + 0 &=& w\\
0 + 0 + 0 + 0  + 2x + 0  + 2xy + y^2  + 3x^2 + 0   + 3x^2y + y^3 +  1 + y + 0 + 0 &=& \phi_x\\
0 + 0 + 0 + 0  + 0  + 2y + x^2  + 2xy + 0   + 3y^2 + x^3  + 3xy^2 + 0 + 0 + 1 + x &=& \phi_y
\end{eqnarray}
Four independent unknown parameters is expressed in terms of using constant strains $\gamma$ per side of The element.

\begin{equation}
\gamma_s = \od{w}{\mbf{s}} + \phi_{s}
\end{equation}
where $\mbf{s}$ is vector of side.

%%%%%%%%%%%%%%%%%%%%%%%%%%%%%%%%%%%%%%%%%%%%%%%%%%%%%%%%%%%%%%
\section{Argyris's triangular plate element}
%%%%%%%%%%%%%%%%%%%%%%%%%%%%%%%%%%%%%%%%%%%%%%%%%%%%%%%%%%%%%%

\begin{eqnarray}
1
\\
x,y
\\
x^2,xy,y^2
\\
x^3,x^2y,xy^2,y^3
\\
x^4,x^3y,x^2y^2,xy^3,y^4
\\
x^5,x^4y,x^3y^2,x^2y^3,xy^4,y^5
\end{eqnarray}

\begin{eqnarray}
N_i(x,y) &=& 1 c_{i,1} + x c_{i,2} + y c_{i,3} + x^2 c_{i,4} + xy c_{i,5} + y^2 c_{i,6} +
\\
&+& x^3 c_{i,7} + x^2y c_{i,8} + xy^2 c_{i,9} + y^3 c_{i,10} +
\\
&+& x^4 c_{i,11} + x^3y c_{i,12} + x^2y^2 c_{i,13} + xy^3 c_{i,14} + y^4 c_{i,15} +
\\
&+& x^5 c_{i,16} + x^4y c_{i,17} + x^3y^2 c_{i,18} + x^2y^3 c_{i,19} + xy^4 c_{i,20} + y^5 c_{i,21}
\end{eqnarray}

Auxiliary matrix
\begin{eqnarray}
\mbf{E} = (1,x,y,x^2,xy,y^2,x^3,x^2y,xy^2,y^3,x^4,x^3y,x^2y^2,xy^3,y^4,x^5,x^4y,x^3y^2,x^2y^3,xy^4,y^5)
\end{eqnarray}
is $\mbf{E} \in R^{1 \times 21}$.
Matice b\'{a}zov\'{y}ch funkc\'{\i}
\begin{eqnarray}
\mbf{N}(x,y) = \mbf{E} \mbf{C}
\end{eqnarray}
je typu $\mbf{N} \in R^{1 \times 21}$ a $\mbf{C}$ je matice koeficient\accent23 u
$\mbf{C} \in R^{21 \times 21}$.
Pruhybov\'{a} plocha je d\'{a}na funkc\'{\i}
\begin{eqnarray}
w(x,y) = \mbf{N}(x,y) \mbf{d} = \mbf{E} \mbf{C} \mbf{d}
\end{eqnarray}
kde $\mbf{d} \in R^{21}$.

\begin{eqnarray}
\mbf{E} = (1, x, y, x^2, xy, y^2, x^3, x^2y, xy^2, y^3, x^4, x^3y, x^2y^2, xy^3, y^4, x^5, x^4y, x^3y^2, x^2y^3, xy^4, y^5)
\end{eqnarray}
\begin{eqnarray}
\mbf{E}_x = (0, 1, 0, 2x, y, 0, 3x^2, 2xy, y^2, 0, 4x^3, 3x^2y, 2xy^2, y^3, 0, 5x^4, 4x^3y, 3x^2y^2, 2xy^3, y^4, 0)
\end{eqnarray}
\begin{eqnarray}
\mbf{E}_{xx} = (0, 0, 0, 2, 0, 0, 6x, 2y, 0, 0, 12x^2, 6xy, 2y^2, 0, 0, 20x^3, 12x^2y, 6xy^2, 2y^3, 0, 0)
\end{eqnarray}
\begin{eqnarray}
\mbf{E}_y = (0, 0, 1, 0, x, 2y, 0, x^2, 2xy, 3y^2, 0, x^3, 2x^2y, 3xy^2, 4y^3, 0, x^4, 2x^3y, 3x^2y^2, 4xy^3, 5y^4)
\end{eqnarray}
\begin{eqnarray}
\mbf{E}_{yy} = (0, 0, 0, 0, 0, 2, 0, 0, 2x, 6y, 0, 0, 2x^2, 6xy, 12y^2, 0, 0, 2x^3, 6x^2y, 12xy^2, 20y^3)
\end{eqnarray}
\begin{eqnarray}
\mbf{E}_{xy} = (0, 0, 0, 0, 1, 0, 0, 2x, 2y, 0, 0, 3x^2, 4xy, 3y^2, 0, 0, 4x^3, 6x^2y, 6xy^2, 4y^3, 0)
\end{eqnarray}

\begin{eqnarray}
\mbf{E}(x_1,y_1) \mbf{c}_1 &=& 1\ ,
\\
\mbf{E}_x(x_1,y_1) \mbf{c}_1 &=& 0\ ,
\\
\mbf{E}_y(x_1,y_1) \mbf{c}_1 &=& 0\ ,
\\
\mbf{E}_{xx}(x_1,y_1) \mbf{c}_1 &=& 0\ ,
\\
\mbf{E}_{xy}(x_1,y_1) \mbf{c}_1 &=& 0\ ,
\\
\mbf{E}_{yy}(x_1,y_1) \mbf{c}_1 &=& 0\ ,
\\
\end{eqnarray}

\begin{eqnarray}
\mbf{S} \mbf{C} = \mbf{I}
\end{eqnarray}

%%%%%%%%%%%%%%%%%%%%%%%%%%%%%%%%%%%%%%%%%%%%%%%%%%%%%%%%%%%%%%
\section{Triangular shell element}
%%%%%%%%%%%%%%%%%%%%%%%%%%%%%%%%%%%%%%%%%%%%%%%%%%%%%%%%%%%%%%
\index{element!triangular!shell}
\label{secttrshellelem}

\begin{tabular}{|l|l|}
\hline
number of blocks & 3
\\ \hline
block components & $(\varepsilon_{xx},\varepsilon_{yy}), (\gamma_{xy}), (\kappa_x,\kappa_y,\kappa_{xy})$
\\ \hline
nodal DOF & $u,v,w,\phi_x,\phi_y,\phi_z$
\\ \hline
\end{tabular}

Triangular shell element of type {\tt shelltr} is composition of two finite
elements. The first element is triangular plane stress element with rotational
degrees of freedom (plelemrotlt). The second element is triangular plate element
based on discrete Kirchhoff theory (dktelem).
Composition of previously mentioned finite elements is possible only in local
coordinate system connected with shell element which will be denoted as E-system.
Plane and plate elements are defined in $xy$ plane and $z$ coordinate is constantly equal to zero.
There are also another local coordinate systems connected with nodes which will be
denoted as N-systems. They are used for modelling of complicated boundary conditions.

Generally, the transformation from local to global coordinate system is described
\begin{equation}\label{shelltransf}
\mbf{r}_g = \mbf{T} \mbf{r}_l
\end{equation}
where $\mbf{r}_g$ is vector in global system, $\mbf{r}_l$ is vector in local system and
$\mbf{T}$ is transformation matrix. Inverse relation to (\ref{shelltransf}) has form
\begin{equation}\label{shelltransfinv}
\mbf{r}_l = \mbf{T}^T \mbf{r}_g
\end{equation}

Matrix describing transformation from E-system to global one is denoted as $\mbf{T}_E$
and matrix between N-system in $i$th node and global system is denoted as $\mbf{T}_{N_i}$.
Similarly, variables connected with E-system have subscript $E$, variables connected with
N-system in $i$th node have subscript $N_i$.

Equilibrium conditions in E-system are written in form
\begin{equation}
\mbf{K}_E \mbf{r}_E = \mbf{f}_E
\end{equation}
\begin{equation}
\mbf{K}_E \mbf{T}^T_E \mbf{r}_g = \mbf{T}_E^T \mbf{f}_g
\ \ \ \Rightarrow \ \ \
\mbf{T}_E \mbf{K}_E \mbf{T}^T_E \mbf{r}_g = \mbf{f}_g
\end{equation}
Equilibrium conditions in global system are written
\begin{equation}
\mbf{K}_g \mbf{r}_g = \mbf{f}_g
\end{equation}
where $\mbf{K}_g=\mbf{T}_E \mbf{K}_E \mbf{T}^T_E$ is stiffness matrix in global system.

Nodal local coordinate systems are used for example for description of complicated boundary conditions.
Equilibrium conditions are then written in nodal coordinate systems
\begin{equation}
\mbf{K}_g \mbf{T}_{N} \mbf{r}_N = \mbf{T}_{N} \mbf{f}_N
\end{equation}
where transformation matrix $\mbf{T}_{N}$ has form
\begin{equation}
\mbf{T}_{N} = \left(\begin{array}{ccc}
\mbf{T}_{N_1} & \mbf{0} & \mbf{0}
\\
\mbf{0} & \mbf{T}_{N_2} & \mbf{0}
\\
\mbf{0} & \mbf{0} & \mbf{T}_{N_3}
\end{array}\right)
\end{equation}
Some of matrices $\mbf{T}_{N_i}$ may be equal to identity matrix when
the global coordinate system is used in particular node.
Finally, conditions in nodal systems are
\begin{equation}
\mbf{T}_{N}^T \mbf{K}_g \mbf{T}_{N} \mbf{r}_N = \mbf{f}_N
\end{equation}


Resulting equilibrium conditions are written in form
\begin{equation}
\mbf{T}_{N}^T \mbf{T}_E \mbf{K}_E \mbf{T}^T_E \mbf{T}_{N} \mbf{r}_N = \mbf{f}_N
\end{equation}

%%%%%%%%%%%%%%%%%%%%%%%%%%%%%%%%%%%%%%%%%%%%%%%%%%%%%%%%%%%%%%%%%%%%%%%%%%%%%%
\section{Quadrilateral element with bi-linear approximation functions for axisymmetric problems}
\label{secthexlinelem}
\index{element!quadrilateral!tri-linear axisymmetric}

\subsection{Edge load and resulting nodal forces}
Increased attention should be devoted to preprocessing in case of computation of nodal
forces caused by distributed load. Constant load parallel to axis of symmetry leads to
linearly changing load along radius direction. Therefore, nodal forces are computed as
integral of linear function multiplied by basis functions over appropriate edge.

\begin{eqnarray}
F_i &=& \del{1}{3} f_i l + \del{1}{6} f_j l
\\
F_j &=& \del{1}{6} f_j l + \del{1}{3} f_j l
\end{eqnarray}


%%%%%%%%%%%%%%%%%%%%%%%%%%%%%%%%%%%%%%%%%%%%%%%%%%%%%%%%%%%%%%%%%%%%%%%%%%%%%%
\section{Hexahedral element with tri-linear approximation functions}
\label{secthexlinelem}
\index{element!hexahedral!tri-linear}

%%%%%%%%%%%%%%%%%%%%%%%%%%%%%%%%%%%%%%%%%%%%%%%%%%%%%%%%%%%%%%%%%%%%%%%%%%%%%%
\section{Hexahedral element with tri-quadratic approximation functions}
\label{secthexquadelem}
\index{element!hexahedral!tri-quadratic}

\begin{eqnarray}
\label{bf3kn1}
N_{1}^{(2)} &=& \del{1}{8} (1 + \xi) (1 + \eta) (1 + \zeta)
(\xi + \eta + \zeta - 2)
\\ \label{bf3kn2}
N_{2}^{(2)} &=& \del{1}{8} (1 - \xi) (1 + \eta) (1 + \zeta)
(- \xi + \eta + \zeta - 2)
\\ \label{bf3kn3}
N_{3}^{(2)} &=& \del{1}{8} (1 - \xi) (1 - \eta) (1 + \zeta)
(- \xi - \eta + \zeta - 2)
\\ \label{bf3kn4}
N_{4}^{(2)} &=& \del{1}{8} (1 + \xi) (1 - \eta) (1 + \zeta)
(\xi - \eta + \zeta - 2)
\\ \label{bf3kn5}
N_{5}^{(2)} &=& \del{1}{8} (1 + \xi) (1 + \eta) (1 - \zeta)
(\xi + \eta - \zeta - 2)
\\ \label{bf3kn6}
N_{6}^{(2)} &=& \del{1}{8} (1 - \xi) (1 + \eta) (1 - \zeta)
(- \xi + \eta - \zeta - 2)
\\ \label{bf3kn7}
N_{7}^{(2)} &=& \del{1}{8} (1 - \xi) (1 - \eta) (1 - \zeta)
(- \xi - \eta - \zeta - 2)
\\ \label{bf3kn8}
N_{8}^{(2)} &=& \del{1}{8} (1 + \xi) (1 - \eta) (1 - \zeta)
(\xi - \eta - \zeta - 2)
\\ \label{bf3kn9}
N_{9}^{(2)} &=& \del{1}{4} (1 - \xi^2) (1 + \eta) (1 + \zeta)
\\ \label{bf3kn10}
N_{10}^{(2)} &=& \del{1}{4} (1 - \xi) (1 - \eta^2) (1 + \zeta)
\\ \label{bf3kn11}
N_{11}^{(2)} &=& \del{1}{4} (1 - \xi^2) (1 - \eta) (1 + \zeta)
\\ \label{bf3kn12}
N_{12}^{(2)} &=& \del{1}{4} (1 + \xi) (1 - \eta^2) (1 + \zeta)
\\ \label{bf3kn13}
N_{13}^{(2)} &=& \del{1}{4} (1 + \xi) (1 + \eta) (1 - \zeta^2)
\\ \label{bf3kn14}
N_{14}^{(2)} &=& \del{1}{4} (1 - \xi) (1 + \eta) (1 - \zeta^2)
\\ \label{bf3kn15}
N_{15}^{(2)} &=& \del{1}{4} (1 - \xi) (1 - \eta) (1 - \zeta^2)
\\ \label{bf3kn16}
N_{16}^{(2)} &=& \del{1}{4} (1 + \xi) (1 - \eta) (1 - \zeta^2)
\\ \label{bf3kn17}
N_{17}^{(2)} &=& \del{1}{4} (1 - \xi^2) (1 + \eta) (1 - \zeta)
\\ \label{bf3kn18}
N_{18}^{(2)} &=& \del{1}{4} (1 - \xi) (1 - \eta^2) (1 - \zeta)
\\ \label{bf3kn19}
N_{19}^{(2)} &=& \del{1}{4} (1 - \xi^2) (1 - \eta) (1 - \zeta)
\\ \label{bf3kn20}
N_{20}^{(2)} &=& \del{1}{4} (1 + \xi) (1 - \eta^2) (1 - \zeta)
\end{eqnarray}
