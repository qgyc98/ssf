\chapter{Mesh components - nodes, edges, surfaces, elements}

The finite element method is based on a mesh of finite elements
which are defined with the help of mesh nodes. The finite element
mesh can be viewed as a hierarchy of nodes which are connected
by edges. The edges define surfaces and the surfaces define elements.
Knowledge of interactions among the nodes, edges, surfaces and elements
is essential and it is described in this chapter. Nodes are represented
by the class gnode (see \ref{gtopologygnode}) and elements are represented by
class gelement (see \ref{gtopologygelement}).


\section{Mesh formats}
\subsection{JKTK format}

\begin{itemize}
\item
number of nodes (denoted by nn)
\item
nn rows follows, each row has the form:
x coordinate, y coordinate, z coordinate, node property
\item
number of elements (denoted by ne), type of elements (see siftop\_elem\_type.h)
\item
ne rows follows, each row has the form:
number of nodes on one element (denoted by nne), nne nodes, element property
\end{itemize}

\begin{tabular}{llll}
16      &         &         &
\\
0.0e+00 & 0.0e+00 & 0.0e+00 & 0 
\\
0.0e+00 & 2.0e+00 & 0.0e+00 & 0 
\\
0.0e+00 & 4.0e+00 & 0.0e+00 & 0 
\\
0.0e+00 & 6.0e+00 & 0.0e+00 & 0 
\\
2.0e+00 & 0.0e+00 & 0.0e+00 & 3 
\\
2.0e+00 & 2.0e+00 & 0.0e+00 & 4 
\\
2.0e+00 & 4.0e+00 & 0.0e+00 & 4 
\\
2.0e+00 & 6.0e+00 & 0.0e+00 & 1 
\\
4.0e+00 & 0.0e+00 & 0.0e+00 & 3 
\\
4.0e+00 & 2.0e+00 & 0.0e+00 & 4 
\\
4.0e+00 & 4.0e+00 & 0.0e+00 & 4 
\\
4.0e+00 & 6.0e+00 & 0.0e+00 & 1 
\\
6.0e+00 & 0.0e+00 & 0.0e+00 & 2 
\\
6.0e+00 & 2.0e+00 & 0.0e+00 & 2 
\\
6.0e+00 & 4.0e+00 & 0.0e+00 & 2 
\\
6.0e+00 & 6.0e+00 & 0.0e+00 & 2 
\\
\end{tabular}

\begin{tabular}{rrrrrr}
9 &  5 &    &    &    &
\\
4 &  6 &  2 &  1 &  5 & 0
\\
4 &  7 &  3 &  2 &  6 & 0
\\
4 &  8 &  4 &  3 &  7 & 0
\\
4 & 10 &  6 &  5 &  9 & 0
\\
4 & 11 &  7 &  6 & 10 & 0
\\
4 & 12 &  8 &  7 & 11 & 0
\\
4 & 14 & 10 &  9 & 13 & 0
\\
4 & 15 & 11 & 10 & 14 & 0
\\
4 & 16 & 12 & 11 & 15 & 0
\\
\end{tabular}

\section{Nodes-elements}

Relationship between nodes and elements is the most important and the most
used relationship from the studied topic. Elements contain array of
nodes which define them. The oposite relationship is not explicitly
defined in the code but it can be built. The most important procedures
dealing with node-element relationships are:

\begin{itemize}
\item
void gelement::give\_nodes (ivector \&nod) - function returns nodes
of the element stored in the vector nod
\item
void gtopology::give\_nodes (long eid,ivector \&nod) - function returns
nodes of the eid-th element, the nodes are stored in nod
\end{itemize}


\section{Edges-elements}

File ordering.cpp in GEFEL contains functions which assemble array edgenod for
the edg-th edge of the element. edgenod[i]=j means, the i-th node of the edge
has the number j, where j is from interval 0-nne (nne is the number of nodes
on one element).

\begin{itemize}
\item
void lintriangle\_edgnod (long *edgenod,long edg) - function works for triangular element
with linear approximation functions
\item
void linquadrilat\_edgnod (long *edgenod,long edg) - function works for quadrilateral
element with bi-linear approximation functions
\item
void quadquadrilat\_edgnod (long *edgenod,long edg) - function works for quadrilateral
element with bi-quadratic approximation functions
\end{itemize}

The class gtopology contains function
void gtopology::give\_edge\_nodes (long eid,long edid,long *nodes)
which assembles array nodes for the eid-th element and the edid-th edge.
Array nodes contains node numbers.




\section{Surfaces-elements}

File ordering.cpp in GEFEL contains functions which assemble array surfnod for
the surf-th surface of the element. surfnod[i]=j means, the i-th node of the surface
has the number j, where j is from interval 0-nne (nne is the number of nodes
on one element).

\begin{itemize}
\item
void linhexahedral\_surfnod (long *surfnod,long surf) - function works for hexahedral element
with tri-linear approximation functions
\item
void quadhexahedral\_surfnod (long *surfnod,long surf) - function works for hexahedral
element with tri-quadratic approximation functions
\end{itemize}

The class gtopology contains function
void gtopology::give\_surf\_nodes (long eid,long surfid,long *nodes)
which assembles array nodes for the eid-th element and the surfid-th surface.
Array nodes contains node numbers.

