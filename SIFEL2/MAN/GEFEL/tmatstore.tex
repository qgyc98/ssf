\chapter{Matrix storage}
There are many various types of matrices which are used in numerical
mathematics and appropriate matrix storage is important for efficient
algorithms. There several possibilities of sorting matrices.
Especially two main properties of matrices are extremely important
in computer manipulation, symmetry and density/sparsity of matrices. Many
matrices in numerical mathematics are symmetric and therefore only
half of the matrix is stored (skyline, symmetric compressed rows).
The second important property is the density/sparsity of matrix. There is no
exact definition of dense or sparse matrix. But it is known fact in
numerical solution of ordinary or partial differential equations that
matrices arisen from finite element method, finite difference method and
finite volume method are sparse, because there are only tens of nonzero entries
in one row what is small number in comparison with the total number of rows.

\noindent
The most important storage types of matrices are:
\begin{itemize}
\item{dense matrix}\index{matrix storage!dense}
\item{skyline}\index{matrix storage!skyline}
\item{double skyline}\index{matrix storage!double skyline}
\item{compressed rows}\index{matrix storage!compressed rows}
\item{symmetrix compressed rows}\index{matrix storage!symm. comp. rows}
\end{itemize}

Following matrix will be used as an example of storage systems.
\begin{equation}\label{prmat}
A=\left(
\begin{tabular}{rrrrrrrrrrrr}
  15 &   0 &  -4 &   3 &  -2 &   1 &   0 &   0 &   0 &   0 &  -4 &  0
\\
   0 &  23 &   3 &  -6 &  -1 &   4 &   0 &   0 &   0 &   0 &   0 & -19
\\
  -4 &   3 &  15 &   0 &  -4 &   0 &  -2 &  -1 &  -4 &  -3 &  -2 &   1
\\
   3 &  -6 &   0 &  23 &   0 & -19 &   1 &   4 &  -3 &  -6 &  -1 &   4
\\
  -2 &  -1 &  -4 &   0 &  15 &   0 &  -4 &   3 &  -2 &   1 &  -4 &  -3
\\
   1 &   4 &   0 & -19 &   0 &  23 &   3 &  -6 &  -1 &   4 &  -3 &  -6
\\
   0 &   0 &  -2 &   1 &  -4 &   3 &   7 &  -3 &  -2 &  -1 &   0 &   0
\\
   0 &   0 &  -1 &   4 &   3 &  -6 &  -3 &  11 &   1 &  -9 &   0 &   0
\\
   0 &   0 &  -4 &  -3 &  -2 &  -1 &  -2 &   1 &   7 &   3 &   0 &   0
\\
   0 &   0 &  -3 &  -6 &   1 &   4 &  -1 &  -9 &   3 &  11 &   0 &   0
\\
  -4 &   0 &  -2 &  -1 &  -4 &  -3 &   0 &   0 &   0 &   0 &  15 &   0
\\
  0 & -19 &   1 &   4 &  -3 &  -6 &   0 &   0 &   0 &   0 &   0 &  23
\end{tabular}
\right)
\end{equation}


\section{Dense matrix storage}
\index{matrix storage!dense}
\label{sectdensematstor}
Dense matrix storage is very simple technique and no special comments are necessary.
All matrix entries are stored and no additional informations are necessary.

\section{Skyline}
\index{matrix storage!skyline}
\label{sectskylinestor}

Skyline storage is applicable for symmetric matrices with reasonable bandwidth.
For each column, the diagonal element is stored and then all matrix entries in the
column up to the farthest nonzero offdiagonal entry are stored into one array
of type double. It is clear that the upper triangular elements and the diagonal
elements will be stored for really dense matrix. Conversly, only the diagonal
entries will be stored for diagonal matrix. The skyline storage is efficient for
matrices with narrow bandwidth. It is default matrix storage for mechanical
problems.

\begin{table}[tbh]
\begin{center}
Array A contains matrix entries
\\[3mm]
\begin{tabular}{|c|c|c|c|c|c|c|c|c|c|}
\hline
 {\bf 15} & {\bf 23} &  {\bf 15} &   3 &  -4 & {\bf 23} &   0 &  -6 &   3 & {\bf 15}
\\ \hline
   0 &  -4 &  -1 &  -2 & {\bf 23} &   0 & -19 &   0 &   4 &   1 
\\ \hline
 {\bf 7} &   3 &  -4 &   1 &  -2 & {\bf 11} &  -3 &  -6 &   3 &   4 
\\ \hline
  -1 &  {\bf 7} &   1 &  -2 &  -1 &  -2 &  -3 &  -4 & {\bf 11} &   3 
\\ \hline
  -9 &  -1 &   4 &   1 &  -6 &  -3 & {\bf 15} &   0 &   0 &   0 
\\ \hline
   0 &  -3 &  -4 &  -1 &  -2 &   0 &  -4 & {\bf 23} &   0 &   0 
\\ \hline
   0 &   0 &   0 &  -6 &  -3 &   4 &   1 & -19 & \multicolumn{2}{c}{}
\\ \cline{1-8}
\end{tabular}
\\[5mm]
Array ADR contains addresses of the diagonal elements
\\[3mm]
\begin{tabular}{|c|c|c|c|c|c|c|c|c|c|c|c|c|}
\hline
0 & 1 & 2 & 5 & 9 & 14 & 20 & 25 & 31 & 38 & 46 & 57 & 68
\\ \hline
\end{tabular}
\caption{Skyline storage of matrix}
\label{tabskymat}
\end{center}
\end{table}

\section{Double skyline}
\index{matrix storage!double skyline}
\label{sectdoubleskylinestor}

Double skyline is generalization of the simple skyline for nonsymmetric matrices.
Diagonal entries are stored and then all row either column entries up to farthest
nonzero offdiagonal entries are stored.

\section{Compressed rows}
\index{storage!compressed rows}
\label{sectcrstor}

Storage of all matrix entries does not respect the sparsity of stored matrix. Skyline storage
respects envelope of the stored matrix but it stores many zero entries which are inside of the envelope.
If some iterative method will be used for solution of equation system, only nonzero entries are required
because iterative methods are usually based on matrix-vector multiplication. Therefore compressed rows
storage was introduced many years ago. Only nonzero matrix entries are stored row after row into
one-dimensional array. Appropriate column indices are stored into one-dimensional integer array. Third
and last necessary integer array contains addresses of first numbers in rows.

The sizes of arrays are not known in advance and should be computed to avoid dynamic allocation.
In connection with finite element method, the sizes of array can be computed as follows. Nonzero
matrix entry can be placed only on position where matrix of one finite element has nonzero entry.
Positions of entries in the matrix of the system is defined by unknowns ordering and therefore
list of adjacent nodes (unknowns) must be created. Lists of all adjacent finite elements to each node
are assembled first. Lists of adjacent nodes of each node are created from lists of adjacent finite
elements. Number of nonzero entries in one row of the matrix is defined by the list of adjacent nodes
of node where the required unknown is defined.

The illustration matrix is stored in compressed rows format and three basic arrays are summarized
in Table \ref{tabcrmat}. The diagonal entries are characterized by bold letters.

\begin{table}[tbh]
\begin{center}
Array K contains matrix entries,
\\
array CI contains column indices.
\\[3mm]
\begin{tabular}{|c|c|c|c|c|c|c|c|c|c|c|c|}
\hline
{\bf 15} &  -4 &   3 &  -2 &   1 &  -4 & {\bf 23} &   3 &  -6 &  -1 &   4 & -19
\\ \hline
1 & 3 & 4 & 5 & 6 & 11 & 2 & 3 & 4 & 5 & 6 & 12
\\ \hline \hline
  -4 &   3 & {\bf 15} &  -4 &  -2 &  -1 &  -4 &  -3 &  -2 &   1 &   3 &  -6
\\ \hline
1 & 2 & 3 & 5 & 7 & 8 & 9 & 10 & 11 & 12 & 1 & 2
\\ \hline \hline
 {\bf 23} & -19 &   1 &   4 &  -3 &  -6 &  -1 &   4 &  -2 &  -1 &  -4 & {\bf 15}
\\ \hline
4 & 6 & 7 & 8 & 9 & 10 & 11 & 12 & 1 & 2 & 3 & 5
\\ \hline \hline
  -4 &   3 &  -2 &   1 &  -4 &  -3 &   1 &   4 & -19 & {\bf 23} &   3 &  -6
\\ \hline
7 & 8 & 9 & 10 & 11 & 12 & 1 & 2 & 4 & 6 & 7 & 8
\\ \hline \hline
  -1 &   4 &  -3 &  -6 &  -2 &   1 &  -4 &   3 &  {\bf 7} &  -3 &  -2 &  -1
\\ \hline
9 & 10 & 11 & 12 & 3 & 4 & 5 & 6 & 7 & 8 & 9 & 10
\\ \hline \hline
  -1 &   4 &   3 &  -6 &  -3 & {\bf 11} &   1 &  -9 &  -4 &  -3 &  -2 &  -1
\\ \hline
3 & 4 & 5 & 6 & 7 & 8 & 9 & 10 & 3 & 4 & 5 & 6
\\ \hline \hline
  -2 &   1 &  {\bf 7} &   3 &  -3 &  -6 &   1 &   4 &  -1 &  -9 &   3 & {\bf 11}
\\ \hline
7 & 8 & 9 & 10 & 3 & 4 & 5 & 6 & 7 & 8 & 9 & 10
\\ \hline \hline
  -4 &  -2 &  -1 &  -4 &  -3 & {\bf 15} & -19 &   1 &   4 &  -3 &  -6 & {\bf 23}
\\ \hline
1 & 3 & 4 & 5 & 6 & 11 & 2 & 3 & 4 & 5 & 6 & 12
\\ \hline
\end{tabular}
\\[5mm]
Array ADR contains addresses of the first nonzero entries in rows.
\\[3mm]
\begin{tabular}{|c|c|c|c|c|c|c|c|c|c|c|c|c|}
\hline
0 & 6 & 12 & 22 & 32 & 42 & 52 & 60 & 68 & 76 & 84 & 90 & 96
\\ \hline
\end{tabular}
\caption{Compressed rows storage of test matrix.}
\label{tabcrmat}
\end{center}
\end{table}

%%%%%%%%%%%%%%%%%%%%%%%%%%%%%%%%%%%%%%%%%%%%%%%%%%%%%%%%%%%%%%%%%%%%%%%%%%%%%%%%
%%%%%%%%%%%%%%%%%%%%%%%%%%%%%%%%%%%%%%%%%%%%%%%%%%%%%%%%%%%%%%%%%%%%%%%%%%%%%%%%
\section{Symmetric compressed rows}
\index{storage!symmetric compressed rows}
\label{sectscrstor}

Compressed rows storage is the most efficient storage for general matrices. Numerical methods used
in engineering and scientific problems lead usually to symmetric matrices which deserve special attention.
In case of symmetric matrices, it is not necessary to store all row entries but only entries from the first
one in the row up to the diagonal entry. This is modification of compressed rows storage which is called
symmetric compressed rows. Three arrays are necessary for this storage. Only nonzero entries from the
lower part including diagonal of the matrix are stored into one-dimensional array. Appropriate column indices
are stored into integer array and third integer array contains addresses of the first entries in the rows.

Similarly as in compressed rows storage, the sizes of three array should be computed in advance.
The strategy is the same as for compressed rows but only nonzero entries of lower triangular part of the matrix
are considered. The illustration matrix is stored in compressed rows storage and necessary arrays are
summarized in Table \ref{tabscrmat}. Diagonal entries are once again characterized by bold letters.

\begin{table}[tbh]
\begin{center}
K contains matrix entries
\\
CI contains column indices
\\[3mm]
\begin{tabular}{|c|c|c|c|c|c|c|c|c|c|}
\hline
{\bf 15} & {\bf 23} &  -4 &   3 & {\bf 15} &   3 &  -6 & {\bf 23} &  -2 &  -1 
\\ \hline
1 & 2 & 1 & 2 & 3 & 1 & 2 & 4 & 1 & 2
\\ \hline \hline
  -4 & {\bf 15} &   1 &   4 & -19 & {\bf 23} &  -2 &   1 &  -4 &   3 
\\ \hline
3 & 5 & 1 & 2 & 4 & 6 & 3 & 4 & 5 & 6
\\ \hline \hline
{\bf 7} &  -1 &   4 &   3 &  -6 &  -3 & {\bf 11} &  -4 &  -3 &  -2 
\\ \hline
7 & 3 & 4 & 5 & 6 & 7 & 8 & 3 & 4 & 5
\\ \hline \hline
  -1 &  -2 &   1 &  {\bf 7} &  -3 &  -6 &   1 &   4 &  -1 &  -9 
\\ \hline
6 & 7 & 8 & 9 & 3 & 4 & 5 & 6 & 7 & 8
\\ \hline \hline
   3 & {\bf 11} &  -4 &  -2 &  -1 &  -4 &  -3 & {\bf 15} & -19 &   1 
\\ \hline
9 & 10 & 1 & 3 & 4 & 5 & 6 & 11 & 2 & 3
\\ \hline \cline{1-4}
   4 &  -3 &  -6 & {\bf 23} & \multicolumn{6}{c}{}
\\ \cline{1-4}
4 & 5 & 6 & 12 & \multicolumn{6}{c}{}
\\ \cline{1-4}
\end{tabular}
\\[5mm]
ADR contains addresses of the first entries in rows.
\\[3mm]
\begin{tabular}{|c|c|c|c|c|c|c|c|c|c|c|c|c|}
\hline
0 & 1 & 2 & 5 & 8 & 12 & 16 & 21 & 27 & 34 & 42 & 48 & 54
\\ \hline
\end{tabular}
\caption{Symmetric compressed rows storage of test matrix.}
\label{tabscrmat}
\end{center}
\end{table}
