\documentclass[12pt]{book}
\newcommand{\mbf}[1]{\mbox{\boldmath$#1$}}
% definice zlomku
\newcommand{\del}[2]{\mbox{$\displaystyle\frac{#1}{#2}$}}
% definice prvni parcialni derivace funkce
\newcommand{\ppd}[2]{\del{\partial{#1}}{\partial{#2}}}
% definice druhe parcialni derivace funkce podle x a y
\newcommand{\dpd}[3]{\del{\partial^{2}\>\!\!{#1}}{\partial{#2}\ \partial{#3}}}
% definice n-te parcialni derivace funkce
\newcommand{\npd}[3]{\del{\partial^{#1}\>\!\!{#2}}{\partial{#3}^{#1}}}
% definice prvni derivace funkce
\newcommand{\od}[2]{\del{{\rm d}\>\!{#1}}{{\rm d}{#2}}}
% definice n-te obycejne derivace funkce
\newcommand{\nod}[3]{\del{{\rm d}^{#1}\>\!\!{#2}}{{\rm d}{#3}^{#1}}}
\usepackage{makeidx}
\makeindex

\setcounter{secnumdepth}{4}
\setcounter{totalnumber}{4}
\oddsidemargin=5mm
\evensidemargin=5mm
\textwidth=160mm
\topmargin=-2.0cm
\textheight=250mm
\headheight=1.5cm


\begin{document}
\tableofcontents
\chapter {PREPROCESSOR MANUAL}

\section{INTRODUCTION}

Each preprocessor program is based on assigning properties
to nodes (ndof, local coord. system, load, ...) or elements
(load, material, element type, ....) by the property number
generated in the mesh generator (T3D or JKTK). The way how
the generators assign property numbers to each node or each
element or element edges is described in their manual.
User may use \# character for comments. Some sections would
be enclosed with begin - end keywords as shown in the preprocessor
file template.
\\
\\
Template of preprocessor file :\\ \\
{\it
\# section with description of input files for the preprocessor\\
\\
begin\\
 \# section with a problem description \\
end\\
\\ 
 \# number of load cases \\
\\ 
begin \\
 \# section which assigns nodal properties \\
end \\
\\
begin \\
 \# section which assigns element properties\\
end\\
\\
begin\\
 \# section with a output description \\
end\\
}

\section {THE SECTION WITH DESCRIPTION OF INPUT FILES FOR PREPROCESSOR}
This section of the input file should contain following data :
\begin{itemize}
\item topology file name - preprocessor accepts topology in JKTK format and
   T3D format
\item file name of material database       (file format is described later)
\item file name of cross-section database  (file format is described later)
\item indicator of topology file format - 0 = JKTK, 1 = T3D
\item indicator whether edge numbers are present in the topology file.
   0 = without edge numbers,
   1 = with edge numbers
\end{itemize}

\section{THE SECTION WITH PROBLEM DESCRIPTION}
This section is introduced with begin keyword and finished with end keyword.
It contains data with problem description in same format as in the ??FEL
(MEFEL,TRFEL,...). Data format can be found in the file probdesc?.cpp, function
read.


\section{THE NUMBER OF LOAD CASES}
The number of load cases should be after the keyword end of the previous section. In case
 of the TRFEL preprocessor, there would be number of transported medias.



\section{THE SECTION WHICH ASSIGNS NODAL PROPERTIES}
The section is introduced with the begin keyword and finished with the end keyword.
The list of keywords could be found in the file GEFEL/intools.cpp - variable Propkw
at the begining of file. Line in this section looks like this one : \\
\\
$i$ $kwd$ $kwdpar$\\
\\
Where :
\begin{itemize}
\item i      = property number of nodes which user wants to assign given parameters
\item kwd    = keyword which denotes type of assigned parameters
\item kwdpar = parameters of given keyword\\ \\
\end{itemize}

For nodes it is possible to use following keywords at the {\bf MEFEL} preprocessor:
\begin{itemize}
\item ndofn = number of dofs in the node.
        Keyword parameters : integer number which means number of dofn
\item bocon = boundary condition in the node.\\
        Keyword parameters :\\
        $nbc$ $dir_1$ $con_1$ . . . $dir_{nbc}$ $con_{nbc}$\\
        Where :
        \begin{itemize}
        \item nbc = number of conditions (integer number)
        \item dir = direction of condition (that means a dof in which the condition is prescribed)
              (positive int. number for static condition,
               negative int. number for dynamic condition)
        \item con = condition\\
              dir $>$ 0 - condition is real number
                         (zero means a support, nonzero means a prescribed displacement)\\
              dir $<$ 0 - condition is string with function for the parser\\
        \end{itemize}
        The number of nodal dofs have to be assigned before the user use this keyword.        
\item nload = nodal load.\\
        Keyword parameters :\\
        $nlc$ $f_1$ $f_2$ . . . $f_{ndofn}$\\
        Where :
        \begin {itemize}
        \item nlc = load case number
        \item f   = load values for each dofn
        \end{itemize}
        The number of nodal dofs have to be assigned before user use this keyword.
\item crsec = cross-section in the node.\\
        Keyword parameters :\\
        $ct$ $ci$\\
        Where :
        \begin{itemize}
        \item ct = cross-section type number by the alias.h
        \item ci = cross-section index number which denotes index of
             cross-section parameters in the cross-section database file.
        \end{itemize}
\item lcsys = local coordinate system in the node.\\
        Keyword parameters :\\
        $nv$ $base\_vec_1$ . . . $base\_vec_{nv}$\\
        Where :
        \begin{itemize}
        \item nv = number of base vectors
        \item base\_vec = values of base vectors
        \end{itemize}
\item dloadn = dynamic nodal load.\\
         Keyword parameters :\\
         $nlc$ $func_1$ . . . $func_{ndofn}$\\
         Where :
         \begin{itemize}
         \item nlc  = load case number
         \item func = strings with function for parser, which returns
                load value depending on time. This string should be
                prescribed in the each dof.
         \end{itemize}
         Number of dofn have to be assigned before user use this keyword.

\item inicon = initial condition in the nodes. It is used esspecially for
         dynamics and nonlinear statics.\\
         Keyword parameters :\\
         $nlc$ $con_1$ . . . $con_{ndofn}$\\
         Where :
         \begin{itemize}
         \item nlc  = load case number
         \item con  = condition values for each dofn
         \end{itemize}
         Number of dofn have to be assigned before user use this keyword.
\item comcn = common code numbers in the nodes.\\
        Keyword parameters :\\
        $ndir$ $dir_1$ . . .  $dir_{ndir}$\\
        Where :
        \begin{itemize}
        \item ndir = number of direction in which the common code numbers
               will be prescribed.
        \item dir  = integer number which denotes direction in which code number
               will be same for given nodes.
        \end{itemize}
        Number of dofn have to be assigned before user use this keyword.
\item ntemp = Temperature load in the node.\\
        Keyword parameters :\\
        $nlc$ $val$\\
        Where :
        \begin{itemize}
        \item nlc = load case number
        \item val = temperature value.\\ \\
        \end{itemize} 
\end{itemize} %end of MEFEL nodal properties keywords items


For nodes it is possible to use following keywords for {\bf TRFEL} preprocessor:
\begin{itemize}
\item bocon = boundary condition in the node.\\
        Keyword parameters :\\
        $ndir$ $dir_1$ $tf_1$ $con_1$ . . . $dir_{ndir}$ $tf_{ndir}$ $con_{ndir}$\\
        Where :\\
        \begin{itemize}
        \item ndir = number of condition (integer number)
        \item dir  = direction of condition (dof in which the condition is prescribed)
        \item tf   = type of time function for condition by the GEFEL/galias.h
               It is possible to use static func, parser funct table func. or
               set of parser func.
        \item con  = condition.
               \begin {itemize}
               \item tf = stat $\to$ con = real number.
               \item tf = pars $\to$ con = string with function for parser
               \item tf = tab  $\to$ con = table of function values. See GEFEL/tablefunct.h$\mid$cpp\\
                    con = $size$ $ityp$ $x_1$ $y_1$ . . . $x_{size}$ $y_{size}$\\
                    Where :
                    \begin{itemize}
                    \item size = number of values in the table
                    \item ityp = integer which defines type interpolation of function
                           See tablefunct.h
                    \item x, y = values for each point in the table.
                    \end{itemize}
               \item tf = $par\_set$ $\to$ con = several strings for parser.See GEFEL/gfunct.h$\mid$cpp\\
               con = $neq$ $x_1$ $func_1$\\
                         .\\
                         .\\
                         $x_{neq}$ $func_{neq}$\\
               Where :
               \begin{itemize}
               \item neq = number of function
               \item x   = limit values
               \item func = string with function for parser. Note that each string
                      have to be on extra line.
               \end{itemize}
              \end{itemize} %end of con item
        \end{itemize} % end of bocon item
\item crsec = cross-section in the node.\\
        Keyword parameters :\\
        $ct$ $ci$\\
        Where :
        \begin{itemize}
        \item ct = cross-section type number by the aliast.h
        \item ci = cross-section index number which denotes index of
             cross-section parameters in the cross-section database file.
        \end{itemize}
\item inicon = initial condition in the nodes. It is used esspecially for\\
         Keyword parameters :\\
         $nlc$ $con_1$ . . . $con_{ndofn}$\\
         Where :
         \begin{itemize}
         \item nlc  = load case number
         \item con  = condition values for each dofn
         \end{itemize}
         Number of dofn have to be assigned before user use this keyword.
\item comcn = common code numbers in the nodes.\\
        Keyword parameters :\\
        $ndir$ $dir_1$ . . .  $dir_{ndir}$\\
        Where :
        \begin{itemize}
        \item ndir = number of direction in which the common code numbers
               will be prescribed.
        \item dir  = integer number which denotes direction in which code number
               will be same for given nodes.
        \end{itemize}
        Number of dofn have to be assigned before user use this keyword.
\item ntemp = source in the node\\
        Keyword parameters :\\
        $nlc$ $tf$ $val$\\
        Where :
        \begin{itemize}
        \item nlc = load case number
        \item tf  = type of time function for condition by the GEFEL/galias.h
               It is possible to use static func, parser funct table func. or
               set of parser func.
        \item val  = source value.
               \begin{itemize}
               \item tf = stat $\to$ val = real number.
               \item tf = pars $\to$ val = string with function for parser
               \item tf = tab  $\to$ val = table of function values. See GEFEL/tablefunct.h$\mid$cpp\\
                    val = $size$ $ityp$ $x_1$ $y_1$ . . . $x_{size}$ $y_{size}$\\
                    Where :
                    \begin {itemize}
                    \item size = number of values in the table
                    \item ityp = integer which defines type interpolation of function
                           See tablefunct.h
                    \item x, y = values for each point in the table.
                    \end{itemize} 
               \item tf = $par_{set}$ $\to$ val = several strings for parser.\\
                     val = $neq$ $x_1$ $func_1$\\
                         .\\
                         .\\
                         .\\
                         $x_{neq}$ $func_{neq}$\\
               Where :
                 \begin{itemize}
                 \item neq = number of function
                 \item x   = limit values
                 \item func = string with function for parser. Note that each string
                        have to be on extra line.
                 \end{itemize}
               \end{itemize} %end of tf items
           \end{itemize} %end of ntemp items
\end{itemize} %end of nodal properties keywords items for TRFEL


\section {SECTION WHICH ASSIGNS ELEMENT PROPERTIES}
The section is introduced with begin keyword and finished with end keyword.
List of keywords could be found in the file GEFEL/intools.cpp - variable Propkw
at the beginig of file. Line in this section looks like this one :\\

$i$ $kwd$ $kwdpar$\\

Where :\\
\begin{itemize}
\item i = property number of elements which user wants to assign given parameters
          In some cases this number means edge number, as will be described latter.
\item kwd    = keyword which denotes type of assigned parameters
\item kwdpar = parameters of given keyword\\ \\
\end{itemize}

For elements it is possible to use following keywords for {\bf MEFEL}:
\begin{itemize}
\item eltype = element type number by the alias.h\\
         Keyword parameters :\\
         $sel$ $tel$ $[sss]$\\
         Where :
         \begin{itemize}
         \item sel = shape element number which is defined in the file
               GEFEL/siftop\_element\_types.h . This number has been defined
               due to mixed meshs where different types of elements
               have the same property number.
         \item tel = element type number by the alias.h
         \item sss = this parameter defines planestress/planestrain state
               for plane elements by the alias.h . THIS PARAMETER IS
               USED ONLY FOR PLANE ELEMENTS so the square brackets in the
               keyword parameters record is used in this manual to emphasize
               optionallity.
         \end{itemize}
\item crsec  = cross-section in the element.\\
         Keyword parameters :\\
         $ct$ $ci$\\
         Where :
         \begin {itemize}
         \item ct = cross-section type number by the alias.h
         \item ci = cross-section index number which denotes index of
              cross-section parameters in the cross-section database file.
         \end{itemize}
\item mater = material in the element.\\
        Keyword parameters :\\
        $ct$ $ci$\\
        Where :
        \begin{itemize}
        \item ct = material type number by the alias.h
        \item ci = material index number which denotes index of
             material parameters in the material database file.
        \end{itemize}
\item eload =  plane/volume load of element.\\
         Keyword parameters :\\
         $nlc$ $ndl$ $f_{1,1}$ . . . $f_{1,ndl}$ . . . . . . $f_{nne,1}$ . . . $f_{nne,ndl}$\\
         Where :
         \begin{itemize}
         \item nlc = load case number
         \item ndl = number of load direction
         \item f   = load value for each direction in the each element node.
         \end{itemize}
\item loadedge = static load of element edge. In this case the property number
           means edge number.\\
           Keyword parameters :\\
           $nlc$ $ndir$ $xa$ $ya$ $za$ $fa_1$ . . . $fa_{ndir}$ $xb$ $yb$ $zb$ $fb_1$ . . . $fb_{ndir}$\\
           Where :
           \begin{itemize}
           \item nlc = load case number.
           \item ndir = number of load direction.
           \item xa, ya, za = x,y,z coordinates of the first point of the loaded edge.
           \item fa  = load values in the each direction for the first point.
           \item xb, yb, zb = x,y,z coordinates of the second point of the loaded
                        edge.
           \item fb  = load values in the each direction for the second point.
           \end{itemize}
\item sscomp = stress/strain computation. This keyword defines where the stress/strains
         should be computed.\\
         Keyword parameters :\\
         $sel$ $ep$ $[ps]$
         Where :
         \begin{itemize}
         \item sel = shape element number which is defined in the file
               GEFEL/siftop\_element\_types.h . This number has been defined
               due to mixed meshs where different types of elements
               have the same property number.
         \item ep = integer number which defines wher the laues are computed. See alias.h\\
               \begin{itemize}  
               \item ep = 0 - nowher
               \item ep = 1 - integration points
               \item ep = 2 - nodes
               \item ep = 3 - user defined point set
               \end{itemize}
         \item ps = set of user defined points. This value is required ONLY when ep = 3,
              so the square brackets in the keyword parameters record is used in
              this manual to emphasize optionallity.
              See pointset.cpp for format of point set data - function read.
         \end{itemize}
\item dloadel = dynamic element load. Not implemented yet.\\ \\
\end{itemize}
For elements it is possible to use following keywords for {\bf TRFEL}:
\begin{itemize}
\item eltype = element type number by the aliast.h\\
         Keyword parameters :\\
         $sel$ $tel$\\
         Where :
         \begin{itemize}
         \item sel = shape element number which is defined in the file
               GEFEL/siftop\_element\_types.h . This number has been defined
               due to mixed meshs where different types of elements
               have the same property number.
         \item tel = element type number by the aliast.h
         \end{itemize}
\item crsec  = cross-section in the element.\\
         Keyword parameters :\\
         $ct$ $ci$\\
         Where :
         \begin{itemize}
         \item ct = cross-section type number by the aliast.h
         \item ci = cross-section index number which denotes index of
              cross-section parameters in the cross-section database file.
         \end{itemize}
\item mater = material in the element.\\
        Keyword parameters :\\
        $ct$ $ci$\\
        Where :
        \begin{itemize}
        \item ct = material type number by the aliast.h
        \item ci = material index number which denotes index of
             material parameters in the material database file.
        \end{itemize}
\item loadedge = boundary condition on element edge. In this case the property number
           means edge number.\\
           Keyword parameters :\\
           $nlc$ $tc$ $tf$ $xa$ $ya$ $za$ $fa$ $xb$ $yb$ $zb$ $fb$ $[ktr]$\\
           Where :
           \begin{itemize}
           \item nlc = load case number.
           \item tc  = type of condition
                 \begin{itemize}
                 \item tc = 2  - condition means prescribed flux
                 \item tc = 3  - condition means prescribed transmition
                 \end{itemize}
           \item tf  = type of time function for condition by the GEFEL/galias.h
                 It is possible to use static func, parser funct table func. or
                 set of parser func.
           \item xa, ya, za = x,y,z coordinates of the first point of the loaded edge.
           \item fa  = condition value.
                 \begin{itemize}
                 \item tf = stat $\to$ fa = real number.
                 \item tf = pars $\to$ fa = string with function for parser
                 \item tf = tab  $\to$ fa = table of function values. See GEFEL/tablefunct.h$\mid$cpp\\
                      fa = $size$ $ityp$ $x_1$ $y_1$ . . . $x_{size}$ $y_{size}$\\
                      Where :
                      \begin{itemize}
                      \item size = number of values in the table
                      \item ityp = integer which defines type interpolation of
                             function. See tablefunct.h
                      \item x, y = values for each point in the table.
                      \end{itemize}
                 \item tf = $par\_set$ $\to$ fa = several strings for parser.See GEFEL/gfunct.h$\mid$cpp\\
                 fa = $neq$ $x_1$ $func_1$\\
                           .\\
                           .\\
                           .\\
                          $x_{neq}$ $func_{neq}$\\
                 Where :
                 \begin{itemize}
                 \item neq = number of function
                 \item x   = limit values
                 \item func = string with function for parser. Note that each string
                        have to be on extra line.
                 \end{itemize}
                \end{itemize}%end of tf items
           \item xb, yb, zb = x,y,z coordinates of the second point of the loaded edge.
           \item fb  = condition values for the second point. Syntax is same as for fa.
           \item ktr = coefficient of transmition. This parameter is need only
                 in case that tc = 3. so the square brackets in the
                 keyword parameters record is used in this manual to emphasize
                 optionallity.
          \end{itemize}%end of loadedge items
\end{itemize}%end of TRFEL element properties keywords

\section{THE SECTION WITH OUTPUT DESCRIPTION}
This section is introduced with begin keyword and finished with end keyword.
It contains data with output description in same format as in the ??FEL
(MEFEL,TRFEL,...). Data format can be found in the file outdriver?.cpp, function
read.

\section {MATERIAL DATABASE FILE FORMAT}
This file contains information about material properties. It can be used one
file of material database for different task but only for one type of problem i.e user
should have one file for mechanics and one file for transport problems.
User may use \# character for comments.
File format :\\
{\it
ntm\\
$tm_1$ $ni_1$\\
$line_1$\\
$line_2$\\
.\\
.\\
$line_{ni,1}$\\
.\\
.\\ \\

.\\
$tm_{ntm}$ $ni_{ntm}$\\
$line_1$\\
$line_2$\\
.\\
.\\
$line_{ni,ntm}$\\\\
}
Where :
\begin{itemize}
\item ntm  = number of different type of materials
\item tm   = number which denotes type of material by the alias?.h
\item ni   = number of different sets of material parameters for given material type
\item line = line with material parameters
\end{itemize}

\section {CROSS-SECTION DATABASE FILE FORMAT}
File format is same as in the material database. Only lines contains data about
cross-section instead of material.
\end{document}
