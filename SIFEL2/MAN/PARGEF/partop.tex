\chapter{Topology}

%%%%%%%%%%%%%%%%%%%%%%%%%%%%%%%%%%%%%%%%%%%%%%%%%%%%%
%%%%%%%%%%%%%%%%%%%%%%%%%%%%%%%%%%%%%%%%%%%%%%%%%%%%%
%%%%%%%%%%%%%%%%%%%%%%%%%%%%%%%%%%%%%%%%%%%%%%%%%%%%%
%%%%%%%%%%%%%%%%%%%%%%%%%%%%%%%%%%%%%%%%%%%%%%%%%%%%%
\section{Schur complement method}

in the input file
\begin{tabular}{ll}
internal nodes & ltg=0
\\
boundary nodes & ltg>0
\\
internal nodes & top--$>$gnodes[i].ai=ltg[i]=0
\\
boundary nodes & top--$>$gnodes[i].ai=ltg[i]>0;
\end{tabular}


in the code
\begin{tabular}{ll}
internal nodes & ltg=-1
\\
boundary nodes & ltg>-1
\\
internal nodes & top--$>$gnodes[i].ai=0
\\
boundary nodes & top--$>$gnodes[i].ai>0
\end{tabular}

%%%%%%%%%%%%%%%%%%%%%%%%%%%%%%%%%%%%%%%%%%%%%%%%%%%%%
%%%%%%%%%%%%%%%%%%%%%%%%%%%%%%%%%%%%%%%%%%%%%%%%%%%%%
%%%%%%%%%%%%%%%%%%%%%%%%%%%%%%%%%%%%%%%%%%%%%%%%%%%%%
%%%%%%%%%%%%%%%%%%%%%%%%%%%%%%%%%%%%%%%%%%%%%%%%%%%%%
\section{FETI-DP method}

in the input file
\begin{tabular}{ll}
internal nodes & ltg=0
\\
boundary nodes & ltg>0
\\
corner nodes   & ltg<0
\\
internal nodes & top--$>$gnodes[i].ai=ltg[i]=0
\\
boundary nodes & top--$>$gnodes[i].ai=ltg[i]=0
\\
corner nodes   & top--$>$gnodes[i].ai=0-ltg[i]>0
\end{tabular}


in the code
\begin{tabular}{ll}
internal nodes & ltg=-1
\\
boundary nodes & ltg>-1
\\
corner nodes   & ltg<-2
\\
internal nodes & top--$>$gnodes[i].ai=0
\\
boundary nodes & top--$>$gnodes[i].ai=0
\\
corner nodes   & top--$>$gnodes[i].ai>0
\end{tabular}

%%%%%%%%%%%%%%%%%%%%%%%%%%%%%%%%%%%%%%%%%%%%%%%%%%%%%
%%%%%%%%%%%%%%%%%%%%%%%%%%%%%%%%%%%%%%%%%%%%%%%%%%%%%
%%%%%%%%%%%%%%%%%%%%%%%%%%%%%%%%%%%%%%%%%%%%%%%%%%%%%
%%%%%%%%%%%%%%%%%%%%%%%%%%%%%%%%%%%%%%%%%%%%%%%%%%%%%
\chapter{Partop}

description of the class partop

   class deals with topology in parallel computations
   it is superior class to the class gtopology which is used
   for particular subdomains
   the class partop manages several classes gtopology
   
   
   notation:

   node multiplicity - number of subdomains which share the node

   internal node - node with multiplicity 1, (node inside a subdomain,
		   node not lying on inter-subdomain boundary)

   boundary node - node with multiplicity at least 2, (node lying
                   on inter-subdomain boundary)

   coarse node - artificial node defined on master which collects appropriate
                 boundary nodes

   
   global ordering - node ordering before mesh partitioning
   global node numbers - node numbers used in undecomposed mesh
                        (node numbers used before decomposition/partitioning)

   local ordering - ordering of subdomains (without respect to remaining subdomains)
   local node numbers - node numbers used in decomposed mesh
                        (node numbers used after decomposition/partitioning)

   coarse ordering - ordering of boundary (interface) nodes only
   coarse node number - node numbers of boundary nodes only, ordering
                        of all boundary nodes in coarse problem

   group ordering - ordering of selected nodes only
   group node number - only selected nodes are ordered on the whole problem, it is similar to
                       the coarse ordering


   mesh description:
   md = 1 - all nodes have their global node number
   md = 2 - only boundary nodes have coarse number, internal nodes are denoted by -1
   md = 3 - all nodes have their global node number, boundary nodes have their global number multiplied by -1


\section{Schur complement method}

\begin{itemize}
\item
ptop--$>$initiation (top,ltg)
\newline
auxiliary variable ai on gnodes are set up with respect to the array ltg
top--$>$gnodes[i].ai = ltg[i];

\item
ptop--$>$numbers\_of\_all\_nodes\_on\_subdomains (domproc,out);
\newline
function establishes maximum number of nodes on one subdomain maxnn and a list
of nodes on particular subdomains nnsd which is located on the master processor
nnsd[i]=j - j nodes are defined on the i-th subdomain

\item
ptop--$>$coupled\_dofs (top,domproc,out);
\newline
   ncdof - the number of coupled DOFs
   coupdof - (M) array containing number of coupled DOFs on subdomains
   coupdofmas - (M) array containing suspicious indicators of coupled DOFs
   nbcdof - the number of boundary/interface coupled DOFs

\item
ptop--$>$assemble_multip (ltg,domproc,out,proc_name);
   maxnbn - maximum number of boundary/interface nodes on subdomain
   nbn - number of boundary/interface nodes
   tnbn - total number of boundary/interface nodes in the problem
   bnodes - (M)
   bmultip - (M)
   maxnbn - maximum number of boundary/interface nodes on subdomain
   tnnp - (M) total number of nodes in the problem
   allnodes - (M)
   amultip - (M) 







\item
ptop--$>$compute\_multiplicity (ltg,domproc,out);
\newline
tnnp - total number of nodes in the problem
maxnbn - maximum number of boundary nodes
nbn - number of boundary nodes on each subdomain
tnbn - total number of boundary nodes
nbnd - array containing numbers of boundary nodes on subdomains
tnnp - total number of nodes in problem
multip - array of boundary node multiplicity (on the master)
nodmultip - array of node multiplicity (on all processors)

multip[i]=j - the i-th node belongs to j subdomains
allnodes[i][j]=k - the j-th node on the i-th subdomain has global number k
nodmultip[i]=j - the i-th node belong to j subdomains
nbnd[i]=j - the i-th domain contains j boundary nodes


\item
ptop--$>$find\_boundary\_nodes (ltg,domproc,out);
\newline
lgnbn - global numbers of boundary nodes
lnbn - local numbers of boundary nodes
nin - number of interanl nodes
lnin - local numbers of internal nodes

lgnbn[i]=j - the i-th boundary node has global number k
lnbn[i]=j - the i-th boundary node has local number j
lnin[i]=j - the i-th internal node has local number j


\item
ptop--$>$rewrite\_ltg (ltg);

\item
selnodschur = new selnodes (nproc,myrank,ndom,top--$>$nn,schurnodes,md,out,mespr);
nsn - number of selected nodes
lsnl - list of selected nodes (local numbers)
lsng - list of selected nodes (global numbers)

\item
selnodschur--$>$number\_of\_selected\_nodes (domproc,out);

maxnsn - maximum number of selected nodes
nsndom - number of selected nodes on subdomains
nsndom[i]=j - j nodes are selected on the i-th subdomain

\item
selnodschur--$>$nodes\_on\_master (domproc,out);

gnn - group node numbers (see partop.h)
gnn[i][j]=k - the j-th selected node on the i-th subdomain has group number k

\item
selnodschur--$>$node\_multiplicity (out);

nodmultip - node multiplicity
nodmultip[i]=j - the i-th selected node is shared by j subdomains

\item
selnodschur--$>$number\_all\_dofs (top,domproc,out);

maxndof
ndofdom
ndofdom[i]=j - the i-th subdomain contains j DOFs

\item
selnodschur--$>$ndofn\_on\_master (top,domproc,out);
ndofnmas
ndofnmas[i][j]=k - the j-th node on the i-th subdomain contains k DOFs
 

\item
selnodschur--$>$dof\_indicators (top,domproc,out);
dofmas
dofmas[i][j][k]=l - the k-th DOF at the j-th selected node on the i-th subdomain has value l

\item
selnodschur--$>$schur\_ordering (top,out);
tndof - total number of DOFs in reduced problem
ndof - number of DOFs which contribute to the reduced problem
cndom - code numbers on master
cndom[i][j]=k - the j-th DOF on the i-th subdomain has group code number k

nedodelano:
ldof - list of code numbers which contribute to the coarse problem




\end{itemize}


\section{FETI Method}

all previous functions are called in the same order as in the Schur complement method

\begin{itemize}
\item
selnodfeti--$>$group\_local\_nodes (out);
ljn - list of joint nodes to selected nodes assumed as coarse nodes
ljn[i][j]=k - the j-th node connected to the i-th coarse node has local number k
ljn contains tnsn rows and nodmultip columns
lsn - list of subdomain numbers which contain connected nodes to coarse nodes
lsn[i][j]=k - the j-th node connected to the i-th coarse node belongs to the k-th subdomain
lsn contains tnsn rows and nodmultip columns

\item
selnodfeti--$>$dof\_feti (out);
doffeti - code numbers / indicators for FETI method
doffeti[i][j][k]=l - the k-th DOF on the j-th connected node to the i-th coarse node has code number / indicator l
 
\item
selnodfeti--$>$number\_contrib (domproc,out);
ncn, ncndom - 
number of contributing nodes in the FETI method
ncndom[i]=j - the i-th subdomain contributes to the coarse problem by j nodes


\item
selnodfeti--$>$contrib\_dofs (top,domproc,out);
???

\end{itemize}
