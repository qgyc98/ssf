\chapter{Numerical solution of nonlinear problems}
\label{numsolofnonlin}

%%%%%%%%%%%%%%%%%%%%%%%%%%%%%%%%%%%%%%%%%%%%%%%%%%%%%%%%
%%%%%%%%%%%%%%%%%%%%%%%%%%%%%%%%%%%%%%%%%%%%%%%%%%%%%%%%
\section{Method of solution}

Coupled mechanical and transport processes after discretization by finite element method are described by system of
ordinary differential equations which can be written in the form
\begin{eqnarray}\label{final_eq_short}
\mbf{K}\mbf{r} + \mbf{C} \od{\mbf{r}}{t} = \mbf{F},
\end{eqnarray}
where $\mbf{K}$ is stiffness-conductivity matrix, $\mbf{C}$ is capacity matrix, $\mbf{r}$ is the vector of nodal values
and $\mbf{F}$ is vector of prescribed forces and fluxes.
Numerical solution of system of ordinary differential equations (\ref{final_eq_short}) is based on expressions
for unknown values collected in the vector $\mbf{r}$ at time $n+1$
\begin{equation}\label{defdn}
\mbf{r}_{n+1} = \mbf{r}_n + \Delta t \mbf{v}_{n+\alpha}\ ,
\end{equation}
where the vector $\mbf{v}_{n+\alpha}$ has form
\begin{equation}\label{defvn}
\mbf{v}_{n+\alpha} = (1-\alpha)\mbf{v}_n + \alpha \mbf{v}_{n+1}\ .
\end{equation}
The vector $\mbf{v}$ contains time derivatives of unknown variables (time derivatives of the vector $\mbf{r}$).
The equation (\ref{final_eq_short}) is expressed in time $n+1$ and with help of previously defined vectors
one can find
\begin{equation}\label{diskr}
\left(\mbf{C} + \Delta t \alpha \mbf{K}\right)\mbf{v}_{n+1} = \mbf{F}_{n+1} -
\mbf{K}\left(\mbf{r}_n + \Delta t (1-\alpha)\mbf{v}_n\right)
\end{equation}
This formulation is suitable because explicit or implicit computational scheme can be set by parameter $\alpha$.
Advantage of the explicit algorithm is based on possible efficient solution of the system of equations because
parameter $\alpha$ can be equal to zero and capacity matrix $\mbf{C}$ can be diagonal. Therefore the solution
of the system is extremely easy. The disadvantage of such method is its conditional stability. It means, that
the time step must satisfy stability condition what usually leads to very short time step.

\begin{table}
\begin{center}
\begin{tabular}{|lll|}
\hline
Initial vectors                 & \hspace{5mm} & $\mbf{r}_0, \mbf{v}_0$
\\
(defined by initial conditions) & &
\\[3mm] \hline
do until & & $i \leq n$
\\
 & & ($n$ is number of time steps)
\\[3mm]
predictor & & $\tilde{\mbf{r}}_{i+1} = \mbf{r}_i + (1-\alpha)\Delta t \mbf{v}_i$
\\[3mm]
right hand side vector & & $\mbf{y}_{i+1} = \mbf{f}_{i+1} - \mbf{K}\tilde{\mbf{r}}_{i+1}$
\\[3mm]
matrix of the system & & $\mbf{A} = \mbf{C} + \alpha \Delta t \mbf{K}$
\\[3mm]
solution of the system & & $\mbf{v}_{i+1} = \mbf{A}^{-1} \mbf{y}_{i+1}$
\\[3mm]
new approximation & & $\mbf{r}_{i+1} = \tilde{\mbf{r}}_{i+1} + \alpha \Delta t \mbf{v}_{i+1}$
\\[3mm] \hline
\end{tabular}
\caption{Algorithm for linear problem.}
\label{linalgtab}
\end{center}
\end{table}

Previously described algorithm is valid for linear problems and one system of linear
algebraic equations must be solved in every time step. The situation is more complicated for
nonlinear problems where nonlinear system of algebraic equations must be solved in every time
step. The high complexity of the problems leads to the application of the Newton-Raphson method
as the most popular method for such cases.

There are several possibilities how to apply and implement solver of nonlinear algebraic
equations. We prefer equilibrium of forces and fluxes (computed and prescribed) in nodes. This strategy is
based on the equation
\begin{equation}\label{fluxequilib}
\mbf{f}_{int} + \mbf{f}_{ext} = \mbf{0}
\end{equation}
where vectors $\mbf{f}_{int}$ and $\mbf{f}_{ext}$ contain internal values and prescribed values.
With respect of nonlinear feature of material laws used in
the analysis, the equation (\ref{fluxequilib}) is not valid after computation of new values
from the equations summarized in Table \ref{linalgtab}. There are nonequilibriated forces and fluxes which
must be suppressed.

When new values in nodes are computed, the strains and gradients of approximated functions can be established.
It is done with help of matrices $\mbf{B}_{\varepsilon}$ and $\mbf{B}_{g}$ where particular partial derivatives are collected.
Concise expression of strains and gradients are written as
\begin{equation}
\mbf{g} = \mbf{B}_{g} \mbf{r}\ ,\ \ \ \ \
\mbf{\varepsilon} = \mbf{B}_{\varepsilon} \mbf{r}\ .
\end{equation}
New stresses and fluxes are obtained from constitutive relations
\begin{equation}
\mbf{\sigma} = \mbf{D}_{\sigma} \mbf{\varepsilon}\ ,\ \ \ \ \
\mbf{q} = \mbf{D}_{q} \mbf{g}\ ,
\end{equation}
where $\mbf{D}_{\sigma}$ is the stiffness matrix of the material and $\mbf{D}_{q}$ is the matrix of conductivity
coefficients.
Real nodal forces and discrete fluxes on element are computed from the relations
\begin{equation}
\mbf{f}_{int}^{e} = \int_{\Omega_e} \mbf{B}^T_{\varepsilon} \mbf{\sigma} {\rm d}\Omega_e\ ,\ \ \ \ \
\mbf{f}_{int}^{e} = \int_{\Omega_e} \mbf{B}^T_{g} \mbf{q} {\rm d}\Omega_e\ .
\end{equation}
They are compared with the prescribed nodal forces and discrete fluxes and their differences create
the vector of residuals $\mbf{R}$. Increments of the vector $\mbf{v}$ are computed from the equation
\begin{equation}\label{incr}
\left(\mbf{C} + \Delta t \alpha \mbf{K}\right)\Delta \mbf{v}_{n+1} = \mbf{R}\ .
\end{equation}
If the matrices $\mbf{\mbf{C}}$ and $\mbf{\mbf{K}}$ are updated after every computation of new increment
from equation (\ref{incr}), the full Newton-Raphson method is used. If the matrices are updated only once
after every time step, the modified Newton-Raphson method is used.

%%%%%%%%%%%%%%%%%%%%%%%%%%%%%%%%%%%%%%%%%%%%%%%%%%%%%%%%%%%
%%%%%%%%%%%%%%%%%%%%%%%%%%%%%%%%%%%%%%%%%%%%%%%%%%%%%%%%%%%
\section{Numerical difficulties}

Transport processes lead to very general systems of algebraic equations. In mechanical
problems, the finite element method results into symmetric systems of equations (it is not
true e.g. for nonassociated theory of plasticity). The system is usually positive definite.

In contrary, the transport processes lead to nonsymmetric indefinite systems. It means
that usual methods like $\mbf{LDL}^T$ decomposition does not work for such problems and more
general $\mbf{LU}$ decomposition must be used. It seems to us that there are problems where
the pivoting is necessary. In this article we describe only examples where pivoting is not
used and the elimination methods work. Our first tests with iterative methods indicate also
difficulties. Usual conjugate gradient method must be replaced by GMRES method and suitable
preconditioner must be find and used. GMRES without preconditioner has serious troubles and we
have found several examples which the method has not solved because of convergence deterioration.

One possible solution of numerical difficulties can be based on appropriate choice of units
used in material models. Other strategy can start from suitable scaling of the matrix.

\chapter{Climat conditions}

\section{Heat flux density}
climatcond.cpp line 1078
give\_heat\_flux 

climatcond.cpp line 1534
condit\_vapour\_diffusion\_heat\_flux

$T_E$ - exterior temperature
$T$ unknown temperature of structure
$T_p$ average auxiliary temperature
\begin{eqnarray}
T_p = \frac{1}{2}(T_E + T)
\end{eqnarray}
$T_0=273.15$ K Reference temperature for enthalpy (0 C)
$h_v$ is the specific enthalpy of water vapor
\begin{eqnarray}
h_v = c_v (T_p-T_0) + h_{v,0}
\end{eqnarray}
$h_{v,0}=2256000.0$ J - specific enthalpy of water vapour at 0 C
$c_v = 1617$ is the specific heat capacity of water vapour
\begin{eqnarray}
q = - h_v q_{diff}
\end{eqnarray}


climatcond.cpp line 2093
condit\_vapour\_diffusion\_moisture\_flux

$p_{v,sat}$ - pressure of saturated water vapour of exterior
\begin{eqnarray}
p_{v,sat} = e^{23.5771 - \frac{4042.9}{T_e - 37.58}}
\end{eqnarray}
$T_e$ temperature of eterier

water vapour pressure of exterier
\begin{eqnarray}
p_{v,e} = p_{v,sat} \varphi_{e}
\end{eqnarray}
$\varphi_{e}$ relative humidity of exterior
moisture flux density for $p_{v,e}<p_v$
\begin{eqnarray}
q = \beta (p_v - p_{v,e})
\end{eqnarray}
$p_v$ water vapout pressure in structure (unknown)
moisture flux density for $p_{v,e}>p_v$
\begin{eqnarray}
q = \frac{w_{sat} - w}{w_{sat}}\beta (p_v - p_{v,e})
\end{eqnarray}
$w$ is the volumetric moisture content (m$^3$/m$^3$),
$w_{sat}$ is the saturated volumetric moisture content (m$^3$/m$^3$),

climatcond.cpp line 1281
condit\_heat\_flux

\begin{eqnarray}
q = \alpha (T - T_e)
\end{eqnarray}
$T$ unknown temperature
$T_e$ temperature of exterior

\subsection{heat flux generated by moisture}

\begin{eqnarray}
q &=& - h_v q_{diff} = (c_v (T_p-T_0) + h_{v,0}) \left(\frac{w_{sat} - w}{w_{sat}}\beta (p_v - p_{v,e})\right) =
\\
&=& \left(c_v (\frac{1}{2}(T_E + T)-T_0) + h_{v,0}\right)\left(\frac{w_{sat} - w}{w_{sat}}\beta (p_v - p_{v,e})\right) =
\\
&=& \left((c_v (\frac{1}{2}(T_E + T)-T_0) + h_{v,0})\frac{w_{sat} - w}{w_{sat}}\beta\right)(p_v - p_{v,e})
\end{eqnarray}


\section{Moisture flux density}

climatcond.cpp 1556
give\_moisture\_flux 

climatcond.cpp 1685
condit\_rain\_moisture\_flux

\begin{eqnarray}
q_a = \beta_r (w_{sat}-w)
\end{eqnarray}
$\beta_r$ exchange coefficient connected with rain
$w$ is the volumetric moisture content (m$^3$/m$^3$),
$w_{sat}$ is the saturated volumetric moisture content (m$^3$/m$^3$),
\begin{eqnarray}
q_r = k_w k_r r
\end{eqnarray}
$r$ (l/m$^2$/s) amount of water per square meter per second caused by rain
$k_w$ coefficient connected with wind
$k_r$ coefficient connected with rain

moisture flux for $q_r<q_a$
\begin{eqnarray}
q = q_r
\end{eqnarray}
else
\begin{eqnarray}
q = q_a
\end{eqnarray}


climatcond.cpp 1685
condit\_vapour\_diffusion\_moisture\_flux
vize v\'{y}\v{s}e

\begin{eqnarray}
\end{eqnarray}
\begin{eqnarray}
\end{eqnarray}
\begin{eqnarray}
\end{eqnarray}
\begin{eqnarray}
\end{eqnarray}
\begin{eqnarray}
\end{eqnarray}
\begin{eqnarray}
\end{eqnarray}
\begin{eqnarray}
\end{eqnarray}
