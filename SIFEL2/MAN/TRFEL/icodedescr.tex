%%%%%%%%%%%%%%%%%%%%%%%%%%%%%%%%%%%%%%%%%%%%%%%%%%%%%%%%%%%%%%%%%%%%%%%%%%%%%%%%%%%%%%%%%%%%%%%%%%%%%
%%%%%%%%%%%%%%%%%%%%%%%%%%%%%%%%%%%%%%%%%%%%%%%%%%%%%%%%%%%%%%%%%%%%%%%%%%%%%%%%%%%%%%%%%%%%%%%%%%%%%
\chapter{Description of the code}
%%%%%%%%%%%%%%%%%%%%%%%%%%%%%%%%%%%%%%%%%%%%%%%%%%%%%%%%%%%%%%%%%%%%%%%%%%%%%%%%%%%%%%%%%%%%%%%%%%%%%%%
%%%%%%%%%%%%%%%%%%%%%%%%%%%%%%%%%%%%%%%%%%%%%%%%%%%%%%%%%%%%%%%%%%%%%%%%%%%%%%%%%%%%%%%%%%%%%%%%%%%

\section{Boundary conditions}

\begin{itemize}
\item
bc=0 - no boundary condition
\item
bc=1 - prescribed values (Dirichlet boundary condition)
\item
bc=2 - prescribed flux (Neumann boundary condition)
\item
bc=3 - climatic data condition, everything is defined by fluxes - Neumann condition
\item
bc=4 - climatic data condition ???
\item
bc=5 - climatic data condition, some boundary conditions are in the Neumann type, some of them in the Newton type
\item
bc=30 - transmission boundary condition (Newton/Cauchy boundary condition),
boundary values are identical with values used in the problem
\item
bc=31 - relative humidity is used in problem while pressures are
prescribed on boundary
\item
bc=32 - prestup, pri kterem se nebere v uvahu dodatecna matice vodivosti,
vypada to jako predepsany tok
\item
bc=33 - Tomasova specialni okrajova podminka pro Karluv most
\end{itemize}

transmission indicator - indicator of boundary condition on element
defined on elementt.cpp
\begin{itemize}
\item
transi=0 - default value, no boundary conditions
\item
transi=2 - prescribed fluxes (defined directly or due to climatic conditions)
\item
transi=3 - prescribed transmission
\item
transi=4 - element contains prescribed fluxes as well as prescribed transmission (e.g. there is one edge with Neumann condition and one edge with Newton condition)
\end{itemize}
transi is defined in loadelt.cpp


\section{Allocation and Initiation}

The TRFEL starts with function trfel\_init, where all data are read
from input file or files and many of objects and arrays are allocated.

The main objects of the code are allocated at the beginning
of the function trfel\_init.
\begin{center}
\begin{tabular}{ll}
Tp  = new probdesct;  & problem description
\\
Gtt = new gtopology; & general topology
\\
Tt  = new transtop; & transport topology
\\
Tm  = new transmat; & transport materials
\\
Tc  = new transcrsec; & transport cross sections
\\
Tb  = new transbclc; & transport boundary conditions
\\
Outdt = new outdrivert; & description of output
\end{tabular}
\end{center}

\begin{itemize}
\item
Tp--$>$read (in); -  reads problem description

\item
Tt--$>$read (in); -  reads transport topology

\item
Ndoft = Gtt--$>$codenum\_generation (Outt); -  generates code numbers

\item
Tm--$>$read (in); - reads transport materials

\item
Tm--$>$intpointalloc (); - allocates array of integration points

\item
Tm--$>$intpointinit (); - initiates variables on integration points,
it defines number of blocks, number of components, type of materials and material id

\item
Tc--$>$read (in); - reads transport cross sections

\item
Tb--$>$read (in); - reads transport boundary conditions, it means
Dirichlet, Neumann and Newton

\item
Tt--$>$elemprescvalue(); - determines elements influenced by prescribed
values, it is used for assembling of nodal fluxes

\item
Tb--$>$elemsource (); - determines elements with sources

\item
Lsrst--$>$alloc (); - allocates arrays lhs (array with solution),
rhs (array with right hand side of the system of equations),

\item
Lsrst--$>$initcond (in); - reads initial conditions

\item
Outdt--$>$read(in); - reads description of output from the code, namely description
of output to the output file, file for graphical postprocessors (e.g. GiD) and
diagrams

\item
Tt--$>$alloc\_nodes (); - allocates arrays grad and flux

\end{itemize}


\section{Vector of right hand side}

globmatt.cpp trfel\_right\_hand\_side (long lcid,double *rhs,long n)


assemble\_init (rhs);

Tb->lc[i].assemble(i,av);


loadcaset.cpp assemble (long lcid,double *rhs)

  source\_contrib (lcid,rhs);

      elem\_neumann\_vector (lv,lcid,eid,i);



      elem\_newton\_vector (lv,lcid,eid,i);

     elem\_neumann\_vector (lv,lcid,eid,i); -> linbart::res\_convection\_vector -> linbart::convection\_vector ->   Tb->lc[lcid].elemload[leid].give\_nodval (lcid,nodval);
-> loadelt::give\_nodval (long lcid,vector \&fe)


v globmatu je smycka pres pocet promennych, 
ve smycce se vola load case, kde je assemble,
assemble obsahuje smycku pres prvky, na nichz je okrajova podminka,
v kazdem volani se vleze do neumanna na prvku, jen jednou, protoze
tok i-te veliciny je tam predepsan celou svou hodnotou
v kazdem volani assemble se vleze do res\_transmission\_vector, kde
je smycka pres promenne, protoze prestup se pocita jako soucet
prispevku ze vsech promennych
napr. $q_T = \kappa (T-T) + \beta (p-p) + ...$.