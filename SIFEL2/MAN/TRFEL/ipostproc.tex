\chapter{Structure of input file}

There are many tools for generation of input files for TRFEL code. These tools
suppress importance of input files because user can create them without exact
knowledge of their structure. But efficient work with the code is based on
good knowledge of the input file.

\section{Main parts}

There are several main parts of input file:

\begin{itemize}
\item{part containing general information about solved problem}
\item{part defining nodes of finite element mesh}
\item{part defining elements of finite element mesh}
\item{part defining boundary conditions}
\end{itemize}

%%%%%%%%%%%%%%%%%%%%%%%%%%%%%%%%%%%%%%%%%%%%%%%%%%%%%%%%%%%
\section{General information}

The first part contains basic data about solved problem. Attributes of the class {\sf probdesc}
are read there. The first part can be divided into part independent on type of problem and part
dependent on it.

\subsection{Part independent on problem type}

This part of input file is common for all implemented problems in the TRFEL. Attributes read
in this part are summarized in the following table.

\begin{center}
\begin{tabular}{|l|l|}
\hline
attribute & description
\\ \hline
{\it name} & description of solved problem, serves only for user
\\
{\it outfile} & name of the output file (with path)
\\
{\it Mespr} & indicator of message printing
\\
 & {\it Mespr}=0 - no print, {\it Mespr}=1 - print of control data
\\
{\it grout} & indicator of graphic output
\\
 & {\it grout}=0 - no graphic output, {\it grout}=1 - graphic output
\\
{\it tprob} & type of problem, see Section \ref{sectproblemtype}
\\
{\it straincomp} & computation of strains, see Section \ref{sectstraincomp}
\\
{\it strainaver} & strain averaging, see Section \ref{sectstraincomp}
\\
{\it stresscomp} & computation of stresses, see Section \ref{sectstresscomp}
\\
{\it stressaver} & stress averaging, see Section \ref{sectstresscomp}
\\
{\it reactcomp} & computation of reactions, see Section \ref{sectreactcomp}
\\
{\it adaptivity} & application of adaptivity analysis
\\
{\it stochasticcalc} & deterministic/stochastic indicator, see Section \ref{sectdetstochcontr}
\\ \hline
\end{tabular}
\end{center}

\subsection{Parts dependent on type of problem}

Other attributes of the class {\sf probdesc} depend on type of problem. Particular cases
are described in following subsections.

\subsubsection{Linear static problem}

\begin{center}
\begin{tabular}{|l|l|}
\hline
attribute & description
\\ \hline
{\it tstorsm} & type of storage of stiffness matrix, see Section \ref{}
\\
{\it ssle} & controller of solver of systems of linear equations, see Section \ref{}
\\ \hline
\end{tabular}
\end{center}

\subsubsection{Nonlinear static problem}

\begin{center}
\begin{tabular}{|l|l|}
\hline
attribute & description
\\ \hline
{\it tnlinsol} & type of solver of system of nonlinear equations, see Section \ref{sectnonlinsolvertype}
\\ \hline
\multicolumn{2}{c}{arc-lenght method}
\\ \hline
{\it nial} & maximum number of increments
\\
{\it niilal} & maximum number of iterations in inner loop
\\
{\it erral} & required norm of vector of unbalanced forces
\\
{\it dlal} & length of the arc at the beginning
\\
{\it dlminal} & minimum length of the arc
\\
{\it dlmaxal} & maximum length of the arc
\\
{\it psial} & control parameter of displacement-loading
\\
{\it hdbackupal} & control of backup and recover of arc-length, see Section \ref{sectcontracrmet}
\\
{\it displnorm} & type of norm used for increments of displacements, see Section \ref{sectdisplacementnorm}
\\ \hline
\end{tabular}
\end{center}


%%%%%%%%%%%%%%%%%%%%%%%%%%%%%%%%%%%%%%%%%%%%%%%%%%%%%%%%%%%
\section{Part defining nodes}

Part of input file defining nodes of finite element mesh has simple structure. Nodes are described by attributes
of the class {\sf mechtop}. There are several quantities defined at nodes: coordinates, number of degrees of
freedom at node, type of cross section (see Section \ref{sectcrsectype}) and local coordinate system.

Total number of nodes is stored in attribute {\it nn} of the class {\sf mechtop}.
Each node is defined by one row in input file. There are following data: number of node (node id), all three coordinates
(that is $x,\ y,$ and $z$, for all problems including onedimensional), number of degrees of freedom defined at node,
type of cross section a local coordinate system.

Example of a typical line has form

\noindent
{\tt 5  3.0 2.0 0.0   2  10 1  0}

\noindent
5 is number of node, coordinates of node are $x=3.0,\ y=2.0,\ z=0.0$, there are 2 DOFs at node, type of
cross section is 10 (plane stress cross section), id of cross section is 1, 0 means no local coordinate
system. If local coordinate system is required, zero is exchanged by 2 or 3 which indicate number of vectors
of local basis. In such case components of basis vectors must be written.

Number of constrained nodes is stored in attribute {\it ncn} of the class {\sf mechtop}.
Constrained nodes are mentioned at a special list. One constrained node is described by one row.
There are following data: number of node (node id), indicators of support. Number of indicators
is equal to number of DOFs at node. 0 means constrained DOF, 1 means free DOF.

All part defining nodes has form similar to this example:

\begin{center}
{\tt
\begin{tabular}{lllllllll}
9 &              &              &     &   &    &   &
\\
1 & 0.0          & 0.0          & 0.0 & 2 & 10 & 1 & 0
\\
2 & -0.632455532 & 1.897366596  & 0.0 & 2 & 10 & 1 & 0
\\
3 &-1.264911064  & 3.794733192  & 0.0 & 2 & 10 & 1 & 0
\\
4 & 2.846049894  & 0.9486832981 & 0.0 & 2 & 10 & 1 & 0
\\
5 & 2.213594362  & 2.846049894  & 0.0 & 2 & 10 & 1 & 0
\\
6 & 1.58113883   & 4.74341649   & 0.0 & 2 & 10 & 1 & 0
\\
7 & 5.692099788  & 1.897366596  & 0.0 & 2 & 10 & 1 & 0
\\
8 & 5.059644256  & 3.794733192  & 0.0 & 2 & 10 & 1 & 0
\\
9 & 4.427188724  & 5.692099788  & 0.0 & 2 & 10 & 1 & 0
\\
\\
3 &              &              &     &   &    &   &
\\
1 & 0            & 0            &     &   &    &   &
\\
2 & 0            & 0            &     &   &    &   &
\\
3 & 0            & 0            &     &   &    &   &
\\
\end{tabular}
}
\end{center}

where 9 nodes are defined and 3 of them are supported.


%%%%%%%%%%%%%%%%%%%%%%%%%%%%%%%%%%%%%%%%%%%%%%%%%%%%%%%%%%%
%%%%%%%%%%%%%%%%%%%%%%%%%%%%%%%%%%%%%%%%%%%%%%%%%%%%%%%%%%%
%%%%%%%%%%%%%%%%%%%%%%%%%%%%%%%%%%%%%%%%%%%%%%%%%%%%%%%%%%%
%%%%%%%%%%%%%%%%%%%%%%%%%%%%%%%%%%%%%%%%%%%%%%%%%%%%%%%%%%%
\chapter{Generation of input file}

%\section{Possibilities}

\section{File with topology}
Topology file should contain all informations about used finite element mesh.
It means nodes and elements. The nodes are described by their 3 coordinates
(3 coordinates must be always used) and by a property. The property serves
for definition of many values like the thickness of plate, density of material etc.
The elements are described by the number of nodes of element and by a property.
The property serves for definition of element type, material type etc. The topology
file does not contain type of finite element (like triangular element with linear
approximation functions for plane stress) but it contains only geometrical
informations (it is triangular element with 3 nodes). The type of finite element
is defined during preprocessing phase.

\section{Example}

Compute displacements of a brick which is clamped at one face and loaded at the
opossite one. The sizes of the brick are 10 m ($x$ direction), 2 m ($y$ direction) and
3 m ($z$ direction). There are 6 elements along the $x$ direction, 2 elements
along the $y$ direction and 3 elements along the $z$ direction.

This very simple mesh could be generated by our simple generator ghex from the
PRG/PREP/GENER

The statement is

ghex brick.top 10 2 3 \hspace{4mm} 6 2 3

There are 84 nodes and 36 elements. The nodes are divided into 5 groups with respect
to their property. All finite elements have property 0.


