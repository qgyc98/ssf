\chapter{Approaches for heat and mass transfer in porous media}
\label{approaches}


%%%%%%%%%%%%%%%%%%%%%%%%%%%%%%%%%%%%%%%%%%%%%%%%%%%%%%%%
%%%%%%%%%%%%%%%%%%%%%%%%%%%%%%%%%%%%%%%%%%%%%%%%%%%%%%%%
\section{K\"unzel's and Kiessl's approach}
\label{kunzelkiessl}

...

%%%%%%%%%%%%%%%%%%%%%%%%%%%%%%%%%%%%%%%%%%%%%%%%%%%%%%%%
%%%%%%%%%%%%%%%%%%%%%%%%%%%%%%%%%%%%%%%%%%%%%%%%%%%%%%%%
\section{Lewis' and Schrefler's approach}
\label{seclewisschrefler}

This section presents a review of basic notions and fundamentals of Lewis' and Schrefler's approach to the moisture and heat transfer in deforming medium. 
Governing equations of phenomenological model correspond to generalized Biot's theory and are derived in detail in~\cite{lewis}.

%%%%%%%%%%%%%%%%%%%%%%%%%%%%%%%%%%%%%%%%%%%%%%%%%%%%%%%%
\subsection{Fully coupled nonlinear heat and moisture transfer in deforming porous medium}
\subsubsection{Constitutive relations}

Moisture in materials can be present as moist air, water and ice or in some intermediate state as adsorbed phase 
on the pore walls, respectively. Since it is in general not possible to distinguish the different aggregate states, 
the water content $w$ is defined as the ratio of the total moisture weight (kg/kg) to the dry weight of the material. 
The degree of saturation $S_w$ is a function of capillary pressure $p^c$ and temperature $T$, 
which is determined experimentally
\begin{equation}\label{Sw}
S_w = S_w(p^c,T).
\end{equation}
The capillary pressure $p^c$ id defined as
\begin{equation}\label{pc}
p^c = p^g - p^w,
\end{equation}
where $p^w > 0$ is the pressure of the liquid phase (water). 

The pressure of the moist air, $p^g > 0$, in the pore system is usually considered as the pressure in a perfect 
mixture of two ideal gases - dry air, $p^{ga}$, and water vapor, $p^{gw}$, i.e.,
\begin{equation}\label{ungas}
p^g = p^{ga} + p^{gw}=\Big( \frac{\rho^{ga}}{M_a} + \frac{\rho^{gw}}{M_w} \Big)T 
R = \frac{\rho^{g}}{M_g}T R.
\end{equation}
In this relation $\rho^{ga}, \rho^{gw}$ and $\rho^{g}$ stand for the respective intrinsic phase densities, $T$ 
is the absolute temperature, and $R$ is the universal gas constant.
 
Identity \eqref{ungas} defining the molar mass of the moist air, $M_g$, in terms of the molar masses of individual 
constituents is known as Dalton's law. The capillary pressure is larger the smaller the capillary radius is. It is shown 
thermodynamically that the capillary pressure can be expressed unambiguously by the relative humidity $RH$ using 
the Kelvin-Laplace law
\begin{equation}\label{kelvinlapl}
RH = \frac{p^{gw}}{p^{gws}} = {\rm exp} \Big( - \frac{p^c M_w}{\rho^w R T} \Big).
\end{equation}
The water vapor saturation pressure, $p^{gws}$, is a function of the temperature only and can be expressed by the 
Clausius-Clapeyron equation
\begin{equation}\label{clausius}
p^{gws}(T) = p^{gws}(T_0) {\rm exp} \Big[ - \frac{M_w \Delta h_{\rm vap}}{R} \Big(\frac{1}{T} - 
\frac{1}{T_0} \Big) \Big],
\end{equation}
where $T_0$ is a reference temperature and $\Delta h_{\rm vap}$ is the specific enthalpy of saturation.

Materials having heat capacities is the term deliberately used 
to emphasize the similarity to the description of the moisture retention. It is simply expressed as
\begin{equation}
H =H(T),
\end{equation}
where $H$ is the mass specific enthalpy (J.kg$^{-1}$), $T$ - temperature (K).

It is not common to write the enthalpy in an absolute way as here. Instead, changes of enthalpy are 
described in a differential way, which leads to the definition of the specific heat capacity as the slope 
of the $H - T$ curve, i.e.
\begin{equation}\label{cap}
C_p = \Big( \frac{\partial H}{\partial T}\Big)_{p={\rm const.}}.
\end{equation}
The heat capacity varies insignificantly with temperature. It is customary, however, to correct this term 
for the presence of the fluid phases and to introduce the effective heat capacity as
\begin{equation}
\big( \rho C_p \big)_{\rm eff} = \rho_s C_{ps} + \rho_w C_{pw} + \rho_g C_{pg}.
\end{equation}\\

\subsubsection{Transfer equations}

The mass averaged relative velocities, $\tenss{v} ^{\alpha} - \tenss{v} ^s$, are expressed 
by the generalized form of {\bf {Darcy's law}} \cite{lewis}
\begin{equation}\label{darcy}
n S_{\alpha} \big( \tenss{v} ^{\alpha} - \tenss{v} ^s \big) = \frac{k^{r\alpha} \tenss{k}_{sat}}
{\mu^{\alpha}} \big( - {\rm grad} p^{\alpha} + \rho^{\alpha} \tenss{g} \big),
\end{equation}
where $\alpha = w$ for the liquid phase and $\alpha = g$ for the gaseous phase.

Dimensionless relative permeabilities $k^{r\alpha} \in \langle 0,1 \rangle$ are functions of degree of~
saturation
\begin{equation}
k^{r\alpha} = k^{r\alpha} (S_w)\qquad ({\rm m \cdot s}^{-1}).
\end{equation}

In Equation (\ref{darcy}), $\tenss{k}_{sat}$ (m$^2$) is the square (3x3) intrinsic permeability matrix 
 and $\mu^{\alpha}$ is the dynamic viscosity (kg.m$^{-1}$.s$^{-1}$). The intrinsic mass densities $\rho^{\alpha}$ 
are related to the volume averaged mass densities $\rho_{\alpha}$ through the relation
\begin{equation}
\rho_{\alpha} = n S_{\alpha} \rho^{\alpha}.
\end{equation}
The relative permeability $k^{rw}$ goes to zero, when water saturation $S_w$ approaches $S_{irr}$, 
which is the limiting value of $S_w$ as the suction stress approaches infinity (\cite{fatt}).

Diffusive-dispersive mass flux (kg.m$^{-2}$.s$^{-1}$) of the water vapor ($gw$) in the gas
 ($g$) is the second driving mechanism. It is governed by {\bf {Fick's law}}
\begin{equation}\label{fick}
\tenss{J}^{gw}_g = n S_g \rho^{gw} \big( \tenss{v} ^{gw} - \tenss{v} ^g \big) = - \rho^g \tenss{D}^{gw}_g 
{\rm grad} \Big( \frac{\rho^{gw}}{\rho^g} \Big),
\end{equation}
where $\tenss{D}^{gw}_g$ (m$^2$.s$^{-1}$) is the effective dispersion tensor. It can be shown \\ \cite{lewis} that
\begin{equation}
\tenss{J}^{gw}_g = - \rho^g \frac{M_a M_w}{M^2_g} \tenss{D}^{gw}_g {\rm grad} \Big( \frac{\rho^{gw}}{\rho^g} \Big)
= \rho^g \frac{M_a M_w}{M^2_g} \tenss{D}^{ga}_g {\rm grad} \Big( \frac{\rho^{ga}}{\rho^g} \Big) = - \tenss{J}^{ga}_g.
\end{equation}
Recall that $\tenss{D}^{gw}_g = \tenss{D}^{ga}_g = \tenss{D}_g$.
Here, $\tenss{J}_g^{ga}$ is the diffusive-dispersive mass flux of the dry air in the gas.

Conduction of heat in normal sense comprises radiation as well as convective heat transfer on a 
microscopic level. The generalized version of {\bf {Fourier's law}} is used to describe the conduction
 heat transfer
\begin{equation}\label{fourier}
\tenss{q} = - \tenss{\chi}_{\rm eff} {\rm grad} T,
\end{equation}
where $\tenss{q}$ is the heat flux (W.m$^{-2}$), $\tenss{\chi}_{\rm eff}$ is the effective thermal conductivity matrix 
(W.m$^{-1}$.K$^{-1}$).

The thermal conductivity increases with increasing temperature due to the non-linear 
behavior of the microscopic radiation, which depends on difference of temperatures raised to the 
$4^{\rm{th}}$ power. Presence of water also increases the thermal conductivity. A suitable formula
reflecting this effect can be found in \cite{lewis}.\\

\subsubsection{Deformation of solid skeleton. Concept of effective stress}

The stresses in the grains, $\tenss{\sigma}^s$, can be expressed using a standard averaging technique 
in terms of the stresses in the liquid phase, $\tenss{\sigma}^w$, the stresses in the gas, $\tenss{\sigma}^g$, 
and the effective stresses between the grains, $\tenss{\sigma}^{ef}$. The equivalence conditions for the internal 
stresses and for the total stress $\tenss{\sigma}$ lead to the expression~\cite{bitt}.
\begin{equation}\label{d3}
\tenss{\sigma} = \tenss{\sigma}^{ef} + S_w \tenss{\sigma}^w + S_g \tenss{\sigma}^g + \Delta \tenss{\tau}.
\end{equation}
Assuming that the shear stress $\tenss{\tau}$ in fluids is negligible converts the latter equation into the form
\begin{equation}\label{d4}
\tenss{\sigma} = \tenss{\sigma}^{ef} - p^s \tenss{m},
\end{equation}
where
\begin{equation}\label{d5}
\tenss{\sigma} = \big\{ \sigma_x,\sigma_y,\sigma_z,\tau_{yz},\tau_{zx},\tau_{xy}\big\}^{\rm T}, \quad
\tenss{m} = \big\{ 1,1,1,0,0,0 \big\}^{\rm T},
\end{equation}
and
\begin{equation}\label{d6}
p^s = S_w p^w + S_g p^g.
\end{equation}

Deformation of a porous skeleton associated with the grain rearrangement can be expressed using 
the constitutive equation written in the rate form
\begin{equation}\label{d7}
\dot{\tenss{\sigma}}^{ef} = \tenss{D}_{sk} \big(\dot{\tenss{\varepsilon}} - \dot{\tenss{\varepsilon}}_0 \big).
\end{equation}
The dots denote differentiation with respect to time, $\tenss{D}_{sk} = \tenss{D}_{sk}(\dot{\tenss{\varepsilon}}, 
{\tenss{\sigma}}^{ef}, T) $ is the tangential matrix of the porous skeleton and $\dot{\tenss{\varepsilon}}_0$ 
represents the strains that are not directly associated with stress changes (e.g., temperature effects, shrinkage, 
swelling, creep). It also comprises the strains of the bulk material due to changes of the pore pressure
\begin{equation}\label{d8}
\dot{\tenss{\varepsilon}} =  -\tenss{m} \Big( \frac{\dot{p}^s}{3K_s} \Big),
\end{equation}
where $K_s$ is the bulk modulus of the solid material (matrix).

When admitting only this effect and combining Equations (\ref{d4}), (\ref{d7}) and (\ref{d8}), we get
\begin{equation}\label{d9}
\dot{\tenss{\sigma}} = \dot{\tenss{\sigma}}^{ef} - \dot{p}^s \tenss{m} = \tenss{D}_{sk} \dot{\tenss{\varepsilon}}
- \alpha \tenss{m} \dot{p}^s = \dot{\tenss{\sigma}}'' - \alpha \tenss{m} \dot{p}^s,
\end{equation}
where
\begin{equation}\label{d10}
\alpha = \frac{1}{3} \tenss{m}^{\rm T} \Big( \tenss{I} - \frac{\tenss{D}_{sk}}{3K_m} \Big) \tenss{m} = 1 - 
\frac{K_{sk}}{K_s} < 1,
\end{equation}
and $K_{sk} = \tenss{m}^{\rm T} \tenss{D}_{sk} \tenss{m}/9$ is the bulk modulus of the porous skeleton. 
For a material without any pores, $K_{sk} = K_s$. For cohesive soils, $K_{sk} << K_s$ and $\alpha = 1$. 
The above formulas are also applicable to long-term deformation of rocks, for which  $\alpha \leq 0.5$, 
and this fact strongly affects Equation (\ref{d9}) \cite{zienkiewicz83}.

Changes of the effective stress along with temperature and pore pressure changes produce change of the solid density 
$\dot{\rho}^s$. To derive the respective material relation for this quantity, we start from the mass conservation equation 
for the solid phase. In the second step we introduce the constitutive relationship for the mean effective stress expressed 
in terms of quantities describing the deformation of the porous skeleton. After some manipulations we arrive at the 
searched formula
\begin{equation}\label{d15}
(1 - n)\frac {\dot{\rho}^s}{\rho^s} = (\alpha - n)\Big(\frac {\dot{p}^s}{K_s} - \beta_s \dot{T}\Big) + 
(\alpha - 1){\rm div} \tenss{v}^s,
\end{equation}
where $\beta_s$ is the thermal expansion coefficient of the solid phase.

Similar approach applied to the mass conservation equation of the liquid phase leads to the following constitutive equation
\begin{equation}\label{d16}
\frac {\dot{\rho}^w}{\rho^w} = \frac {\dot{p}^w}{K_w} - \beta_s \dot{T},
\end{equation}
where $K_w$ is the bulk modulus of water and $\beta_w$ is the thermal expansion coefficient of this phase.\\

\subsubsection{Set of governing equations}

The complete set of equations describing the coupled moisture and heat transport in deforming porous media comprises 
the linear balance (equilibrium) equation formulated for a multi-phase body, the energy balance equation and 
the continuity equations for the liquid water and gas.

{\it Continuity equation for the dry air}
\begin{eqnarray}\label{air}
\frac{\partial}{\partial t}\Big(\varphi(1-S_w)\rho^{ga}\Big) + \alpha(1-S_w)\rho^{ga} {\rm div} \tenss{\dot{u}} - 
{\rm div} \Big( \rho^{ga}\frac{k^{rg}\tenss{k}_{sat}}{{\mu}^{g}} {\rm grad}p^g\Big) + \nonumber\\
+{\rm div} \Big( \rho^g \frac{M_a M_w}{M^2_g} \tenss{D}_{\rm eff} {\rm grad} \Big( \frac{p^{gw}}{p^g} \Big) \Big) = 0,
\end{eqnarray}
where $\tenss{\dot{u}}$ ($\tenss{\dot{u}} = \tenss{v}^s$) is the velocity of solid.

{ \it Continuity equation for the water species}
\begin{eqnarray}\label{water}
\frac{\partial}{\partial t}\Big(\varphi(1-S_w)\rho^{gw}\Big) + \alpha(1-S_w)\rho^{gw} {\rm div} \tenss{\dot{u}} - 
{\rm div} \Big( \rho^{gw}\frac{k^{rg}\tenss{k}_{sat}}{{\mu}^{g}} {\rm grad}p^g\Big) + \nonumber\\
-{\rm div} \Big( \rho^g \frac{M_a M_w}{M^2_g} \tenss{D}_{\rm eff} {\rm grad} \Big( \frac{p^{gw}}{p^g} \Big) \Big) = \nonumber\\
=-\frac{\partial}{\partial t}\Big(\varphi S_w\rho^{w}\Big) - \alpha S_w \rho^{w} {\rm div} \tenss{\dot{u}} + 
{\rm div} \Big( \rho^{w}\frac{k^{rw}\tenss{k}_{sat}}{{\mu}^{w}} ({\rm grad}p^g - {\rm grad}p^c - \rho^w\tenss{g})\Big)
\end{eqnarray}

{ \it Energy balance equation}
 \begin{eqnarray}\label{heat}
\big( \rho C_p \big)_{\rm eff} \frac{\partial T}{\partial t} - {\rm div}\big(\lambda_{\rm eff}{\rm grad}T\big) + \nonumber\\
- \Big(C_{pw} \rho^w \frac{k^{rw}\tenss{k}_{sat}}{{\mu}^{w}} ({\rm grad}p^g - {\rm grad}p^c - \rho^w\tenss{g}) + 
C_{pg} \rho^{gw}  \frac{k^{rg}\tenss{k}_{sat}}{{\mu}^{g}} {\rm grad}p^g\Big) {\rm grad}T = \nonumber\\
=\Delta h_{\rm vap} \Big[\frac{\partial}{\partial t}\Big(\varphi S_w\rho^{w}\Big) + \alpha S_w \rho^{w} {\rm div} \tenss{\dot{u}} -
{\rm div} \Big( \rho^{w}\frac{k^{rw}\tenss{k}_{sat}}{{\mu}^{w}} ({\rm grad}p^g - {\rm grad}p^c - \rho^w\tenss{g})\Big)\Big]
\end{eqnarray}

The { \it equilibrium equation} (the linear balance equation) must yet be introduced to complete a set of governing 
equations
\begin{equation}\label{d19}
{\rm div} \big(\tenss{\sigma} - \tenss{m}(p^g - S_w p^c)\big)+ \rho \tenss{g} = \tenss{\rm 0}
\end{equation}
with density of the multi-phase medium defined as
\begin{equation}\label{d20}
\rho = (1-n)\rho^s + nS_w\rho^w + nS_g\rho^g = \rho_s + \rho_w + \rho_g.
\end{equation}

{ \it Initial and boundary conditions}

The initial conditions specify the full fields of gas pressure, capillary or water pressure, 
temperature and displacement and velocities:
\begin{equation}\label{init}
p^g = p^g_0, \qquad p^c = p^c_0, \qquad T = T_0, \qquad \tenss{u} = \tenss{u}_0, 
\quad {\rm and}  \quad \dot{\tenss{u}} = \dot{\tenss{u}}_0, \quad {\rm at}\quad t = 0. 
\end{equation}
The boundary conditions can be imposed values on $\Gamma^1_{\pi}$ or fluxes on $\Gamma^2_{\pi}$, where the boundary
$\Gamma = \Gamma^1_{\pi} + \Gamma^2_{\pi}$.
\begin{equation}\label{bound1}
p^g = \overline{p}^g \quad {\rm on} \quad \Gamma^1_{g}, \quad p^c = \overline{p}^c \quad {\rm on} \quad \Gamma^1_{c},
\quad T = \overline{T} \quad {\rm on} \quad \Gamma^1_{T}, \quad \tenss{u} = \overline{\tenss{u}}  \quad {\rm on} \quad \Gamma^1_{u}.
\end{equation}
The volume averaged flux boundary conditions for water species and dry air conservation equations 
and energy equation to be imposed at the interface between the porous medium and the surrounding fluid are as follows
\begin{eqnarray}\label{bound2}
\big(\rho^{ga}\tenss{J}^{ga} - \rho^{g}\tenss{J}^{gw}\big)\cdot\tenss{n} &=& q_{ga} \quad {\rm on} \quad \Gamma^2_{g}\nonumber\\
\big(\rho^{gw}\tenss{J}^{ga} + \rho^{w}\tenss{J}^{w} + \rho^{g}\tenss{J}^{gw}\big)\cdot\tenss{n} &=& 
\beta_c(\rho^{gw} -\rho^{gw}_{\infty}) + q_{gw} + q_{w}\quad {\rm on} \quad \Gamma^2_{c}\\
-\big(\rho^{w}\tenss{J}^{w}\Delta h_{\rm vap} - \lambda_{\rm eff}{\rm grad}T)\cdot\tenss{n} &=& 
\alpha_c(T - T_{\infty}) + q_{T}\quad {\rm on} \quad \Gamma^2_{T}\nonumber
\end{eqnarray}
where $\tenss{n}$ is the unit normal vector of the surface of the porous medium, $\rho^{gw}_{\infty}$ and $T_{\infty}$ 
are the mass concentration of water vapor and temperature in the undisturbed gas phase far away from the interface, 
and $q_{ga}$, $q_{gw}$, $q_w$ and $q_T$ are the imposed air flux, the imposed vapor flux, the imposed liquid flux and 
the imposed heat flux, respectively.

The traction boundary conditions for displacement field are:
\begin{equation}\label{bound3}
\sigma\cdot\tenss{n} = \tenss{t} \quad {\rm on} \quad \Gamma^2_{u}
\end{equation}
where $\tenss{t}$ is the imposed traction.

\subsubsection{Discretization of governing equations}

A weak formulation of the governing equations \eqref{air} to \eqref{d19} is obtained by applying Galerkin's method
of weighted residuals. For the numerical solution, the governing equations are discretized in space by means of the 
finite element method, yielding a non-symmetric and non-linear system of partial differential equations:
\begin{eqnarray}\label{final1}
\tenss{K}_{uu}\tenss{u} + \tenss{K}_{ug}\tenss{p}_g + \tenss{K}_{uc}\tenss{p}_c + \tenss{K}_{ut}\tenss{T} &=& \tenss{F}_u,\nonumber\\
\tenss{C}_{gg}\tenss{\dot{p}}_g + \tenss{C}_{gc}\tenss{\dot{p}}_c + \tenss{C}_{gt}\tenss{\dot{T}} +\tenss{C}_{gu}\tenss{\dot{u}} + 
\tenss{K}_{gg}\tenss{p}_g + \tenss{K}_{gc}\tenss{p}_c + \tenss{K}_{gt}\tenss{T} &=& \tenss{F}_g,\nonumber\\
\tenss{C}_{cg}\tenss{\dot{p}}_g + \tenss{C}_{cc}\tenss{\dot{p}}_c + \tenss{C}_{ct}\tenss{\dot{T}} + \tenss{C}_{cu}\tenss{\dot{u}} + 
\tenss{K}_{cg}\tenss{p}_g + \tenss{K}_{cc}\tenss{p}_c + \tenss{K}_{ct}\tenss{T} &=& \tenss{F}_c,\\
\tenss{C}_{tg}\tenss{\dot{p}}_g + \tenss{C}_{tc}\tenss{\dot{p}}_c + \tenss{C}_{tt}\tenss{\dot{T}} + \tenss{C}_{tu}\tenss{\dot{u}} + 
\tenss{K}_{tg}\tenss{p}_g + \tenss{K}_{tc}\tenss{p}_c + \tenss{K}_{tt}\tenss{T} &=& \tenss{F}_t.\nonumber
\end{eqnarray}
The equations \eqref{final1} can be rewritten in concise form as
\begin{equation}\label{final2}
\tenss{K}(\tenss{X})\tenss{X} + \tenss{C}(\tenss{X})\tenss{\dot{X}} = \tenss{F}(\tenss{X}),
\end{equation}
where $\tenss{X}^{\rm T} = \{\tenss{p}^g,\tenss{p}^c,\tenss{T},\tenss{u}\}$, $\tenss{C}(\tenss{X})$ is ``the general capacity matrix'', 
$\tenss{K}(\tenss{X})$ is ``the general conductivity matrix'' and are obtained together with $\tenss{F}(\tenss{X})$ 
by assembling the sub-matrices indicated in equations \eqref{final1}. The dot denotes the time derivative.

%%%%%%%%%%%%%%%%%%%%%%%%%%%%%%%%%%%%%%%%%%%%%%%%%%%%%%%%
\subsection{Fully coupled nonlinear heat and moisture transfer in nondeforming porous medium}

...

%%%%%%%%%%%%%%%%%%%%%%%%%%%%%%%%%%%%%%%%%%%%%%%%%%%%%%%%%%%
\subsection{Coupled nonlinear heat and moisture transfer - simplified solution}
\label{seccoupl}

In this paragraph, ...

\subsubsection{Energy balance equation}

The complete set of equations describing the coupled moisture and heat transport in porous media comprises 
the energy balance equation and the continuity equations for the liquid water and gas. 
The {\it energy balance equation} has already been derived in Paragraph~\ref{ssec1_2_3} (Eq. (\ref{masstempr})).

The effect of evaporation (latent heat) is represented by the last term appearing on the right-hand 
side of Eq. (\ref{masstempr}) as a source term. Expressing the rate of evaporation by means of the balance 
equation for vapor and combining Eq. (\ref{masstempr}) through (\ref{fourier}) we arrive at the resulting 
form of the equation of energy conservation:
\begin{eqnarray}\label{enrg_con_first}
( \rho C)_{\rm eff} \frac{\partial T}{\partial t} &=& {\rm div} \Big\{ \Big(\chi_{\rm eff} (S, 
T) + \Delta H^{gw} \delta^{gw}(S) \frac{\partial p^{gw}}{\partial T} \Big) {\rm grad} T +\\ 
& &\qquad \quad + \Delta H^{gw} \delta^{gw}(S) \frac{\partial p^{gw}}{\partial S} {\rm grad} S \Big\} 
- \frac{\partial}{\partial t}
\Big[\Delta H^{gw} n(1-S) \rho^{gw} \Big].\nonumber
\end{eqnarray}
Since $h = \rho^{gw}/\rho^{gws}$, the source term takes this form
\begin{eqnarray}
\frac{\partial}{\partial t} \Big[\Delta H^{gw} n(1-S) \rho^{gw} \Big] &=& \Delta H^{gw} n h \rho^{gws} 
\frac{\partial S}{\partial t} + \Delta H^{gw} n (1 - S)\rho^{gws} \frac{\partial h}{\partial t}\\
&=& \Delta H^{gw} n \rho^{gws} \Big(h + (1-S)\frac{\partial h}{\partial w} \frac{\partial w}{\partial S} \Big) 
\frac{\partial S}{\partial t} = \Delta H^{gw} b(S) \frac{\partial S}{\partial t}\nonumber.
\end{eqnarray}
When dealing with the coupled moisture and heat transfer, three quantities can be used to describe the moisture field. 
Denote them by $\varPsi$; 
$\varPsi = S, w, h$. 
Eq. (\ref{enrg_con_first}) can be then rewritten into this contracted form
\begin{eqnarray}
( \rho C)_{\rm eff} \frac{\partial T}{\partial t} = {\rm div} \big\{ a_{T T} (T, \Psi) {\rm grad} T + 
a_{T \Psi} (T, \Psi) {\rm grad} \Psi \big\} - \Delta H^{gw} b(\Psi)\frac{\partial \Psi}{\partial t},
\end{eqnarray}
\begin{eqnarray}
a_{T T} (T, \Psi) &=& \chi_{\rm eff} (\Psi,T) + \Delta H^{gw} \delta^{gw}(\Psi) \frac{\partial p^{gw}}{\partial T},
\nonumber\\
a_{T S} (T, \Psi) &=& \Delta H^{gw} \delta^{gw}(\Psi) \frac{\partial p^{gw}}{\partial \Psi}.\nonumber
\end{eqnarray}

\subsubsection{Liquid and gas transport}
\label{ssec1_4_2}

If moisture convection is neglected, the liquid and gas (moist air) transport and 
the vapor diffusion taking place in the gas are the remaining driving mechanisms.
The transport of the vapor phase is governed by Fick's law (\ref{fick})
\begin{eqnarray}\label{fick_2}
\tenss{J}^{gw}_s &=& n(1-S_w) \rho^{gw} \big( \tenss{v} ^{gw} - \tenss{v}^s \big) = 
- \frac{k^{gw}}{\nu^{gw}} {\rm grad} p^{gw}\nonumber\\
&=&-\delta^{gw}(S) {\rm grad} p^{gw}\nonumber\\
&=&-\delta^{gw}(S)\Big[\frac{\partial p^{gw}}{\partial T} {\rm grad} T + \frac{\partial p^{gw}}{\partial S} {\rm grad} 
S \Big].
\end{eqnarray}
The flux of the liquid phase (Darcy's law (\ref{darcy})) is expressed as
\begin{eqnarray}\label{darcy_2}
\tenss{J}_s^{w} &=& n S_w \rho^w \big( \tenss{v} ^{w} - \tenss{v}^s \big) = - \frac{K^w(s)}{g} {\rm grad} p^w\nonumber\\
&=& - \frac{K^w(S)}{g} {\rm grad} (p^{gw} - p^c)\nonumber\\
&=& - \frac{K^w(S)}{g}\Big[\Big(
\frac{\partial p^{gw}}{\partial T} - \frac{\partial p^{c}}{\partial T}\Big) {\rm grad} T
+ \Big(\frac{\partial p^{gw}}{\partial S} - \frac{\partial p^{c}}{\partial S}\Big) {\rm grad} S \Big].
\end{eqnarray}
It remains to calculate the partial derivatives of $p^{gw}$ and $p^c$ with respect to $T$ and $S$.

As $p^{gw} = h p^{gws}$ we have
\begin{eqnarray}
\frac{\partial p^{gw}}{\partial T} = \frac{\partial}{\partial T} \Big(h p^{gws}(T)\Big)\Big|_{h = {\rm const}} 
= h \frac{\partial p^{gws}(T)}{\partial T}.
\end{eqnarray}
Making use of the Kelvin-Laplace law (\ref{kelvinlapl}) yields
\begin{eqnarray}
\frac{\partial p^{c}}{\partial T} = \frac{\partial}{\partial T} \Big[-\frac{\rho^w R T}{M_w}\ln h \Big]_{h = {\rm const}} 
= -\frac{\rho^w R}{M_w}\ln h.
\end{eqnarray}
If $h$ approaches one, $S > S_{\rm irr}$ (II. and III. region) then 
\begin{eqnarray}\label{region}
\frac{\partial p^{gw}}{\partial T} \rightarrow \frac{{\rm d} p^{gws}(T)}{{\rm d} T} 
\qquad {\rm and} \quad \frac{\partial p^{c}}{\partial T}\rightarrow 0.
\end{eqnarray}
It is apparent from (\ref{w_S}) that
\begin{eqnarray}
\frac{{\rm d} w(S)}{{\rm d} S}  &=& \frac{n(\rho^w - \rho^g)}{(1-n)\rho^s} 
\approx \frac{n}{1-n}\frac{(\rho^w - \rho^{gw})}{\rho^s} \doteq {\rm const}.
\end{eqnarray}
Proceeding in a standard way, we find the partial derivative
\begin{eqnarray}
\frac{\partial p^{gw}}{\partial S}\Big|_{T = {\rm const}} &=& 
\frac{\partial p^{gw}}{\partial h} \cdot \frac{\partial h}{\partial w}
\cdot \frac{\partial w}{\partial S}\nonumber\\
&=& \frac{\partial p^{gw}}{\partial w} \frac{n}{1-n}\frac{(\rho^w - \rho^{gw})}{\rho^s}
\end{eqnarray}
where
\begin{eqnarray}
\frac{\partial p^{gw}}{\partial w} = p^{gws}(T) \frac{{\rm d} h(w)}{{\rm d} w}.
\end{eqnarray}
There are two possibilities how to evaluate the partial derivative of $p^c$ with respect to $\varPsi = S, w, h$:
\begin{itemize}
\item{by starting from Leverett's formula (\ref{leverett}) to get}
\begin{eqnarray}
\frac{\partial p^{c}}{\partial \varPsi} = \frac{\partial p^{c}(S)}{\partial S} \cdot \frac{\partial S}{\partial \varPsi}
\end{eqnarray}
\item{or by exploiting the Kelvin - Laplace equation (\ref{kelvinlapl})}
\end{itemize}
\begin{eqnarray}
\frac{\partial p^{c}}{\partial \varPsi} = \frac{\partial p^{c}}{\partial h}\Big|_{T = {\rm const}} 
\cdot \frac{\partial h}{\partial \varPsi},
\quad {\rm where} \quad \frac{\partial p^{c}}{\partial h}\Big|_{T = {\rm const}} = -\frac{\rho^w R T}{M_w} 
\frac{{\rm d}(\ln h)}{{\rm d} h} = 
- \frac{\rho^w R T}{M_w} \frac{1}{h}
\end{eqnarray}


\subsubsection{Mass balance equation for two-phase zone}
\label{ssec_1_4_3}

The mass balance equation for the liquid phase (\ref{masswater}) reads 
\begin{eqnarray}
\frac{\partial \big( n S_w \rho^w\big)}{\partial t} + {\rm div} \big[n S_w \rho^w \tenss{v}^w\big]
 = -\dot{m}. \nonumber
\end{eqnarray}
The effect of dry air can be neglected in the transition region for $S > S_{\rm irr}$ (two-phase zone). 
Then $S_{gw} \doteq 1-S_{w} = 1-S$ and the mass balance equation for vapor assumes this form
\begin{eqnarray}\label{massvapor_2}
\frac{\partial \big[n(1- S_w) \rho^{gw}\big]}{\partial t} + {\rm div} \big[n (1-S_w)
\rho^{gw} \tenss{v}^{gw}\big] = \dot{m}.
\end{eqnarray}
By adding Eqs. (\ref{masswater}) and (\ref{massvapor_2}) considering Eq. (\ref{w_S}) and substituting from 
Eqs. (\ref{fick_2}) and (\ref{darcy_2}), 
we arrive at the mass balance equation in two-phase zone of a porous material:
\begin{eqnarray}\label{twophase}
\rho_s \frac{\partial w}{\partial t} = \frac{\partial w(\varPsi)}{\partial \varPsi} 
\cdot \frac{\partial \varPsi}{\partial t} = 
{\rm div} \big( a_{\varPsi T}(\varPsi, T) {\rm grad} T + a_{\varPsi \varPsi}(\varPsi, T){\rm grad} \varPsi \big),
\end{eqnarray}
where $\rho_s = (1 - n)\rho^s$ (see Eq.~\eqref{w_S}), $\varPsi = S, w, h$, and
\begin{eqnarray}\label{a_phi}
a_{\varPsi T}(\varPsi, T) &=& \Big[\Big( \delta^{gw}(\varPsi) + \frac{K^w(\varPsi)}{g} \Big) 
\frac{\partial p^{gw}}{\partial T} - \frac{K^w(\varPsi)}{g} \frac{\partial p^{c}}{\partial T} \Big]_{\varPsi = 
{\rm const}}\nonumber\\
a_{\varPsi \varPsi}(\varPsi, T) &=& \Big[\Big( \delta^{gw}(\varPsi) + \frac{K^w(\varPsi)}{g} \Big) 
\frac{\partial p^{gw}}{\partial \varPsi} - \frac{K^w(\varPsi)}{g} \frac{\partial p^{c}}{\partial \varPsi} \Big]_{T = 
{\rm const}}\quad .
\end{eqnarray}
If water vapor diffusion is the only driving mechanism, the preceding equations (\ref{a_phi}) simplify as
\begin{eqnarray}
a_{\varPsi T}(\varPsi, T) &=& \delta^{gw}(\varPsi) \frac{\partial p^{gw}}{\partial T}\Big|_{\varPsi = 
{\rm const}}\nonumber\\
a_{\varPsi \varPsi}(\varPsi, T) &=& \delta^{gw}(\varPsi) \frac{\partial p^{gw}}{\partial \varPsi} \Big|_{T = 
{\rm const}}\quad .
\end{eqnarray}
In the two-phase zone $S$ is always greater than $S_{\rm irr}$ (i.e. $h > 0.9$) and, in addition to Eqs. (\ref{region}), 
$\partial p^{gw}/\partial S \rightarrow 0$.%, which always simplifies Eqs. (\ref{a_phi}).


\subsubsection{FEM formulation for coupled moisture and heat transfer}
\label{ssec_1_4_4}

The balance equation for moisture transfer $(\Psi = w(\tenss{x},t))$ reads.
\begin{eqnarray}\label{couple}
\rho_s \frac{\partial w}{\partial t} &=& - {\rm div} \big(\tenss{J}_s^{gw}+ \tenss{J}_s^w\big) \nonumber\\
&=& {\rm div} \Big\{ \delta^{gw}(w) \Big[ \frac{\partial p^{gw}}{\partial T} {\rm grad} T 
+ \frac{\partial p^{gw}}{\partial w} {\rm grad} w \Big] + \nonumber\\ &+& \frac{K^w(w)}{g} \Big[\Big(
\frac{\partial p^{gw}}{\partial T} - \frac{\partial p^{c}}{\partial T}\Big) {\rm grad} T
+ \Big(\frac{\partial p^{gw}}{\partial w} - \frac{\partial p^{c}}{\partial w}\Big) {\rm grad} w \Big]\Big\}.
\end{eqnarray}
The fluid is transferred across interfaces with the surrounding environment by means of convection 
fluxes, $\tenss{\nu}^T\tenss{J}_c^{\alpha} \dots$ $\alpha = w, gw$. This phenomenon can be expressed 
by the boundary conditions pertaining to Eq. (\ref{couple}):
\begin{itemize}
\item{either given values (an essential condition)}
\begin{eqnarray}
w=\overline{w} \qquad {\rm on} \quad \Gamma_w = \Gamma_{1w}
\end{eqnarray}
\item{or imposed fluxes (natural boundary conditions)}
\begin{eqnarray}
\tenss{\nu}^T \tenss{J}_c^{gw} = - \tenss{\nu}^T {\delta}^{gw}(w) \Big[ \frac{\partial 
p^{gw}}{\partial T} {\rm grad} T + \frac{\partial p^{gw}}{\partial w} {\rm grad} w \Big]
= \overline{q}^{gw} \quad \Gamma^{gw}
\end{eqnarray}
\end{itemize}
and
\begin{eqnarray}
\tenss{\nu}^T \tenss{J}_c^{w} = - \tenss{\nu}^T \frac{K^w(w)}{g} \Big[\Big(
\frac{\partial p^{gw}}{\partial T} - \frac{\partial p^{c}}{\partial T}\Big) {\rm grad} T
+ \Big(\frac{\partial p^{gw}}{\partial w} - \frac{\partial p^{c}}{\partial w}\Big) {\rm grad} w \Big]
= \overline{q}^{w} \quad {\rm on} \quad \Gamma^{w}\nonumber.
\end{eqnarray}
\begin{eqnarray}
{\Gamma^{gw} \cup \Gamma^w = \Gamma_{2w}}\nonumber
\end{eqnarray}	
The water vapor flux prescribed in the direction of the outward normal $\tenss{\nu}$ 
on $\Gamma^{gw}$ 
is obtained from the following formula \cite{ctu}
\begin{eqnarray}
\overline{q}^{gw} = \beta^{gw} (p_{\rm surf}^{gw} - p_{\rm air}).
\end{eqnarray}

The equation of energy conservation is of the form
\begin{eqnarray}\label{enrg_con}
( \rho C)_{\rm eff} \frac{\partial T}{\partial t} &=& {\rm div} \Big\{ \Big(\chi_{\rm eff} (w, 
T) +  \Delta H^{gw} \delta^{gw}(w) \frac{\partial p^{gw}}{\partial T} \Big) {\rm grad} T +\\
&& \qquad  +\Delta H^{gw} \delta^{gw}(w) \frac{\partial p^{gw}}{\partial w} {\rm grad} w \Big\} +  \Delta H^{gw} b(w) 
\frac{\partial}{\partial t}\nonumber
\end{eqnarray}
where
\begin{eqnarray}
 \Delta H^{gw} b(w) =  \Delta H^{gw} n \rho^{gws} \Big( h(w) + (1-S)\frac{{\rm d} h}{{\rm d} w}\Big )
\overline{q}^{gw} = \beta^{gw} (p_{\rm surf}^{gw} - p_{\rm air}).
\end{eqnarray}
The boundary conditions belonging to Eq. (\ref{enrg_con}) are
\begin{itemize}
\item{either the given values}
\begin{eqnarray}
T = \overline{T} \qquad {\rm on} \quad \Gamma_{T} = \Gamma_{1 T}
\end{eqnarray}
\item{or imposed flux. It is given as the sum of three individual fluxes - a convection exchange 
of heat between outer surfaces and the air, the absorption of a fraction of the sun radiation 
heat, and the loss of heat due to latent heat of moisture vaporization. The result reads}
\end{itemize}
\begin{eqnarray}
\tenss{\nu}^T q = - \tenss{\nu}^T \chi_{\rm eff}(w,T) {\rm grad} T  = 
\big(\beta_c + \beta_r(T)\big) \big(T_{\rm surf} - T_{\rm air}\big) + \beta_c \tenss{\nu}^T 
\tenss{J}_c^{gw}= \overline{q}_{T}\nonumber\\
{\rm on} \quad \Gamma^{T} = \Gamma_{2 T}.
\end{eqnarray}\\

{\bf Moisture transfer}\\

The balance equation \eqref{couple} for moisture transport is written under the assumption 
that the water vapor flux is the only driving mechanism (sorption isotherm - region I):
\begin{eqnarray}\label{dif}
-\rho_s \frac{\partial w}{\partial t} + {\rm div} \Big(k_{w T} {\rm grad} T +
k_{w w} {\rm grad} w \Big) = 0, \quad k_{w T} = \delta^{gw}  \frac{\partial p^{gw}}
{\partial T}, \quad k_{w w} = \delta^{gw} \frac{\partial p^{gw}}{\partial w}
\end{eqnarray}
with the following boundary conditions
\begin{eqnarray}
w =\overline{w} \qquad {\rm on} \quad \Gamma_{1w} \nonumber
\end{eqnarray}
\begin{eqnarray}\label{bound}
\tenss{\nu}^T \tenss{J}_c^{gw} = 
- \tenss{\nu}^T {\delta}^{gw} \Big[ \frac{\partial p^{gw}}{\partial T} 
{\rm grad} T + \frac{\partial p^{gw}}{\partial w} 
{\rm grad} w \Big]= \overline{q}^{gw}
 \qquad {\rm on} \quad \Gamma_{2w}.
\end{eqnarray}

Applying {\it Galerkin's method} to Eqs. (\ref{dif}) and (\ref{bound}) we obtain
\begin{eqnarray}
\int_{\Omega}\Big\{ -\rho_s\frac{\partial w}{\partial t} + {\rm div} \Big(k_{w T} {\rm grad} T + 
k_{ww} {\rm grad} w\Big)\Big\} \delta w {\rm d} \Omega &+&\nonumber\\ 
+\int_{\Gamma_{2w}}\Big\{ - \Big(k_{w T} \frac{\partial T}{\partial \vec{n}} + 
k_{ww} \frac{\partial w}{\partial \vec{n}}\Big)
+ \overline{q}^{gw}\Big\}\delta w {\rm d} \Gamma &=& 0.
\end{eqnarray}
Now we convert the first integral via {\it Gauss theorem}:
\begin{eqnarray}\label{fem_coupm}
- \int_{\Omega} \rho_s \frac{\partial w}{\partial t} \delta w {\rm d} \Omega + \int_{\Omega} \Big(k_{wT} {\rm grad} T + 
k_{ww} {\rm grad} w\Big) \delta w {\rm d} \Omega + \\
+ \int_{\Gamma_{1w}} \Big(k_{wT} \frac{\partial T}{\partial \vec{n}} + 
k_{ww} \frac{\partial w}{\partial \vec{n}}\Big) \delta w {\rm d} \Gamma
+ \int_{\Gamma_{2w}} \overline{q}^{gw} \delta w {\rm d} \Gamma = 0\nonumber.
\end{eqnarray}
Since $\delta w = 0$ on $\Gamma_1$, it holds
\begin{eqnarray}
\int_{\Gamma_{1w}} \Big(k_{wT} \frac{\partial T}{\partial \vec{n}} + 
k_{ww} \frac{\partial w}{\partial \vec{n}}\Big) \delta w {\rm d} \Gamma = 0.
\end{eqnarray}
Eq.~\eqref{fem_coupm} thus leads to the following form
\begin{eqnarray}\label{fem_couplm}
\int_{\Omega} \rho_s \frac{\partial w}{\partial t} \delta w {\rm d} \Omega + \int_{\Omega} \Big(k_{wT} {\rm grad} T + 
k_{ww} {\rm grad} w\Big) {\rm grad} (\delta w) {\rm d} \Omega
= \int_{\Gamma_{2w}} \overline{q}^{gw} \delta w {\rm d} \Gamma.
\end{eqnarray}

In the finite element method (FEM) moisture $w$ and temperature $T$ are approximated as:
\begin{eqnarray}\label{fem_apr}
T = \tenss{N}_T \tenss{r}_T, \qquad w = \tenss{N}_w \tenss{r}_w,
\end{eqnarray}
where $\tenss{N}$ is a matrix of shape functions and $\tenss{r}$ is a column matrix 
of nodal values of $w$ or $T$, respectively.

Using approximation (\ref{fem_apr}) in Eq. (\ref{fem_couplm}), we arrive at a set of the first 
order differential equations
\begin{eqnarray}\label{fem_couplm_matr}
\tenss{K}_{wT} \tenss{r}_{T} + \tenss{K}_{ww}\tenss{r}_w + \tenss{C}_{ww} \frac{{\rm d}\tenss{r}_w}{{\rm d}t} = 
\overline{\tenss{q}}^{gw}.
\end{eqnarray}\\

{\bf Heat transfer}\\

The FEM - based solution of the equation of energy conservation can be treated in a similar manner. 
Equation (\ref{enrg_con}) leads to the simplified form
\begin{eqnarray}
- (\rho C)_{\rm eff} \frac{\partial T}{\partial t}  + 
{\rm div} \Big(\chi_{\rm eff} {\rm grad} T\Big) = 0
\end{eqnarray}
with the following boundary conditions
\begin{eqnarray}
T = \overline{T} \qquad {\rm on} \quad \Gamma_{1 T}\nonumber
\end{eqnarray}
and
\begin{eqnarray}
\tenss{\nu}^T q = - \tenss{\nu}^T \chi_{\rm eff} {\rm grad} T =
\big(\beta_c + \beta_r(T)\big) \big(T_{\rm surf}
- T_{\rm air}(\tenss{x},t_{i+1}) \big) + \beta_c \tenss{\nu}^T 
\tenss{J}_c^{gw} = \overline{q}_{T}\nonumber\\ {\rm on} \quad \Gamma_{2 T}.
\end{eqnarray}
	
Application of {\it Galerkin's method} yields:
\begin{eqnarray}\label{1_79}
\int_{\Omega}\Big\{ (\rho C)_{\rm eff} \frac{\partial T}{\partial t}  
+ {\rm div}\Big(\chi_{\rm eff} {\rm grad} T\Big)\Big\} \delta T {\rm d} \Omega
+ \int_{\Gamma_{2T}}\Big\{ - \chi_{\rm eff} \frac{\partial T}{\partial \vec{n}} + 
\overline{q}_{T}\Big\}\delta T {\rm d} \Gamma = 0.
\end{eqnarray}
Converting the first integral of Eq.~\eqref{1_79} via {\it Gauss theorem} gives:
\begin{eqnarray}
-\int_{\Omega}\Big( (\rho C)_{\rm eff} \frac{\partial T}{\partial t}\Big) \delta T {\rm d} \Omega
-\int_{\Omega} \Big(\chi_{\rm eff}{\rm grad} T \Big) {\rm grad}(\delta T) {\rm d} \Omega \nonumber\\
+ \int_{\Gamma_{1T}} \chi_{\rm eff} \frac{\partial T}{\partial \vec{n}}\delta T {\rm d} \Gamma
+ \int_{\Gamma_{2T}}\overline{q}_{T}\delta T {\rm d} \Gamma = 0.
\end{eqnarray}
Since
\begin{eqnarray}
\int_{\Gamma_{1T}} \chi_{\rm eff} \frac{\partial T}{\partial \vec{n}}\delta T {\rm d} \Gamma
 = 0,
\end{eqnarray}
we finally get
\begin{eqnarray}\label{fem_couplh}
\int_{\Omega}\Big( (\rho C)_{\rm eff} \frac{\partial T}{\partial t}\Big) \delta T {\rm d} \Omega
+\int_{\Omega} \Big(\chi_{\rm eff}{\rm grad} T \Big) {\rm grad}(\delta T) {\rm d} \Omega
= \int_{\Gamma_{2T}}\overline{q}_{T}\delta T {\rm d} \Gamma.
\end{eqnarray}
Temperature $T$ is approximated in the same way as in Eq.~\eqref{fem_apr}:
\begin{eqnarray}\label{fem_ap}
T = \tenss{N}_T \tenss{r}_T
\end{eqnarray}
After some manipulations we arrive at a set of the first order differential equations for heat transfer
\begin{eqnarray}\label{fem_couplh_matr}
\tenss{K}_{TT} \tenss{r}_{T} + \tenss{C}_{TT} \frac{{\rm d}\tenss{r}_{T}}{{\rm d}t} 
=\overline{\tenss{q}}_{T}.
\end{eqnarray}

Numerical solution of this coupled problem consists in integration of
the moisture and heat transfer equations \eqref{fem_couplm_matr} and \eqref{fem_couplh_matr} together:
\begin{eqnarray}\label{final_eq}
\left[ \begin{array}{cc}
\tenss{K}_{ww} & \tenss{K}_{wT} \\
\tenss{0} & \tenss{K}_{TT}
\end{array} \right]
\left\{ \begin{array}{c}
\tenss{r}_w \\
\tenss{r}_{T}
\end{array} \right\} + 
\left[ \begin{array}{cc}
\tenss{C}_{ww} & \tenss{0} \\
\tenss{0} & \tenss{C}_{TT}
\end{array} \right]
\left\{ \begin{array}{c}
\dot{\tenss{r}}_w \\
\dot{\tenss{r}}_{T}
\end{array} \right\} = 
\left\{ \begin{array}{c}
\tenss{q}^{gw} \\
\tenss{q}_{T}
\end{array} \right\}.
\end{eqnarray}
For simplicity, Eq.~\eqref{final_eq} is rewritten into the following form
\begin{eqnarray}\label{final_eq_short}
\tenss{K} + \tenss{C}\frac{{\rm d}\tenss{r}}{{\rm d} t} = \overline{\tenss{q}},
\end{eqnarray}
where matrix $\tenss{K}$ is the general non-symmetric conductivity matrix and $\tenss{C}$ is the general capacity matrix.

\newpage
%%%%%%%%%%%%%%%%%%%%%%%%%%%%%%%%%%%%%%%%%%%%%%%%%%%%%%%%
\subsection{Appendix}

The matrices appearing in Equation \eqref{final1} are shown here in detail, using the notation of Lewis and Schrefler~\cite{lewis}.
\begin{eqnarray}
\tenss{K}_{uu} &=&\\
\tenss{K}_{ug} &=&\\
\tenss{K}_{uc} &=&\\
\tenss{K}_{ut} &=&\\
\tenss{F}_u &=&\\
\tenss{C}_{gg} &=&\\
\tenss{C}_{gc} &=&\\
\tenss{C}_{gt} &=&\\
\tenss{C}_{gu} &=&\\ 
\tenss{K}_{gg} &=&\\
\tenss{K}_{gc} &=&\\
\tenss{K}_{gt} &=&\\
\tenss{F}_g &=&\\
\tenss{C}_{cg} &=&\\
\tenss{C}_{cc} &=&\\
\tenss{C}_{ct} &=&\\
\tenss{C}_{cu} &=&\\ 
\tenss{K}_{cg} &=&\\
\tenss{K}_{cc} &=&\\
\tenss{K}_{ct} &=&\\
\tenss{F}_c &=&\\
\tenss{C}_{tg} &=&\\
\tenss{C}_{tc} &=&\\
\tenss{C}_{tt} &=&\\
\tenss{C}_{tu} &=&\\ 
\tenss{K}_{tg} &=&\\
\tenss{K}_{tc} &=&\\
\tenss{K}_{tt} &=&\\
\tenss{F}_t &=&
\end{eqnarray}

The matrices appearing in Equation \eqref{fem_couplm_matr} and \eqref{fem_couplh_matr}:
\begin{eqnarray}
\tenss{K}_{wT} &=& \int_{\Omega} k_{wT} \Big(\frac{\partial \tenss{N}_{T}^T}{\partial x}
\frac{\partial \tenss{N}_{T}}{\partial x} 
+ \frac{\partial \tenss{N}_{T}^T}{\partial y}\frac{\partial \tenss{N}_{T}}{\partial y} + 
\frac{\partial \tenss{N}_{T}^T}{\partial z}\frac{\partial \tenss{N}_{T}}{\partial z}\Big){\rm d}\Omega,\\
\tenss{K}_{ww} &=& \int_{\Omega} k_{ww} \Big(\frac{\partial \tenss{N}_w^T}{\partial x}
\frac{\partial \tenss{N}_w}{\partial x} 
+ \frac{\partial \tenss{N}_w^T}{\partial y}\frac{\partial \tenss{N}_w}{\partial y} + 
\frac{\partial \tenss{N}_w^T}{\partial z}\frac{\partial \tenss{N}_w}{\partial z}\Big){\rm d}\Omega,\\
\tenss{C}_{ww} &=& \int_{\Omega} \rho_s \tenss{N}_w^T \tenss{N}_w {\rm d}\Omega,\\
\overline{\tenss{q}}^{gw} &=& \int_{\Gamma_2} \tenss{N}_q^T  \overline{q}^{gw}{\rm d}\Gamma,\\
\tenss{K}_{TT} &=& \int_{\Omega} \big(\chi_{\rm eff}\big) \Big(\frac{\partial \tenss{N}_{T}^T}{\partial x}
\frac{\partial \tenss{N}_{T}}{\partial x} 
+ \frac{\partial \tenss{N}_{T}^T}{\partial y}\frac{\partial \tenss{N}_{T}}{\partial y} + 
\frac{\partial \tenss{N}_{T}^T}{\partial z}\frac{\partial \tenss{N}_{T}}{\partial z}\Big){\rm d}\Omega,\\
\tenss{C}_{TT} &=& \int_{\Omega} (\rho C)_{\rm eff} \tenss{N}_{T}^T \tenss{N}_{T} {\rm d}\Omega,\\
\overline{\tenss{q}}^{T} &=& \int_{\Gamma_2} \tenss{N}_q^T  \overline{q}_{T}{\rm d}\Gamma.
\end{eqnarray}

\newpage
%%%%%%%%%%%%%%%%%%%%%%%%%%%%%%%%%%%%%%%%%%%%%%%%%%%%%%%%
\subsection{Nomenclature}

