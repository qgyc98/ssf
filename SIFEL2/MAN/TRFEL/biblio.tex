\begin{thebibliography}{99}

\addcontentsline{toc}{chapter}{\numberline {}Bibliography}

\bibitem{bazant_t}
Z. P. Bazant and W. Thonguthai (1979) Pore pressure in heated concrete walls: theorethical prediction.
Magazine of Concrete Research, {\bf 31(107)}, 67-76.

\bibitem{bau}
H. Bau and K. E. Torrance (1982) Boiling in low-permeability porous materials. Int. J. Heat Mass Transfer {\bf 25}, 45-55.

\bibitem{rahli}
O. Rahli, F. Topin, L. Tadrist and J. Pantaloni (1996) Analysis of heat transfer 
with liquid-vapor phase change in a forced-flow fluid moving through porous media.
Int. J. Heat Mass Transfer, {\bf 18(39)}, 3959-3975.

\bibitem{thomas}
H. R. Thomas, M. R. Sansom, G. Volkaert, P. Jacobs and M. Kumnan (1994)
An experimental and numerical investigation of the hydration of compacted powder boom clay.
Num. Meth. in Geomechanical Engineering. Smith (ed.), Balkema, Rotterdam, 135-141

\bibitem{lewis}
R. W. Lewis and B. A. Schrefler (1998) The finite element method in static and dynamic deformation and consolidation 
of porous media. John Wiley \& Sons, Chiester-Toronto (492)

\bibitem{boer} 
R. de Boer (1996)
Highlights in the historical development of the porous media theory: toward a consistent macroscopic theory. 
Appl. Mech. Rev., {\bf 49}, 201-262

\bibitem{woltman}
R. Woltman (1974) Beitr${\rm \ddot{a}}$ge zur hydraulischen Architektur.
Johann Christian Dietrich, {\bf 3}, G${\rm\ddot{o}}$tingen.

\bibitem{fillunger}
P. Fillunger (1913) Der Auftrieb in Talsperren. 
${\rm\ddot{O}}$sterr. Wochenschrift f${\rm\ddot{u}}$r den ${\rm\ddot{o}}$ffentl. Baudienst, {\bf 19}, 532-556, 567-570.

\bibitem{terzaghi}
K. von Terzaghi (1923)
Die Berechnung der Durchl${\rm\ddot{a}}$ssigkeitsziffer des Tones aus dem Verlauf der hydrodynamischen 
Spannungserscheinungen. Sitzungsber. Akad. Wiss. Math. - Naturwiss., Section IIa, {\bf 132}, (3/4), 125-138.

\bibitem{biot1}
M. A. Biot (1941) General theory of three-dimensional consolidation. J. Appl. Phys., {\bf 12}, 155-164.

\bibitem{biot2}
M. A. Biot (1956) General solution of the equation of elasticity and consolidation for a porous material.
J. Appl. Mech. {\bf 23}, 91-96.

\bibitem{hassanizadeh78}
M. Hassanizadeh and W. G. Gray (1979) General conservation equations for multiphase systems: 1, averaging procedure.
Adv. Water Resources, {\bf 2}, 131-144.

\bibitem{hassanizadeh80}
M. Hassanizadeh and W. G. Gray (1980) 
General conservation equations for multiphase systems: 2, mass, momenta, energy and entropy equations.
Adv.  Water Resources, {\bf 2}, 191-203.

\bibitem{kuklik1}
P. Kukl\ii k, J. Mare\v{s} and M. \v{S}ejnoha (1999)
Evaluation of the modified CAM clay model with reference to isotropic consolidation.
CTU REPORTS, {\bf 4(3)}, 47-49.

\bibitem{zienkiewicz77}
O. C. Zienkiewicz, C. Humperson and R. W. Lewis (1977)
A unified approach to soil mechanics problems including plasticity and viscoplasticity. 
Finite Elements in Geomechanics, G. Gudehus (ed.), Willey, London Ch. 4.

\bibitem{zienkiewiczI}
O. C. Zienkiewicz, A. H. C. Chan, M. Pastor, D. K. Paul and T. Shiomi (1990)
Static and dynamic behaviour of soils: a random approach to quantitative solutions. I. Fully saturated problems.
Proc. R. Soc. Lond. A, {\bf 429}, 285-309.

\bibitem{zienkiewiczII}
O. C. Zienkiewicz, Y. M. Xie, B. A. Shrefler, A. Ledesma and N. Bicanic (1990)
Static and dynamic behaviour of soils: a rational approach to quantitative solutions.
II. Semi-saturated problems, Proc. R. Soc. Lond. A, {\bf 429}, 311-321.

\bibitem{kiessl}
K. Kiessl (1983) Kapillarer und dampff${\rm\ddot{o}}$rmiger Feuchtetransrport in mehrschichtigen Bauteilen. PhD-thesis
University of Essen, Essen.

\bibitem{pedersen}
C. R. Pedersen (1990) Combined heat and moisture transfer in exposed bulding constructions. PhD-thesis
Technical University of Denmark, Lingby.

\bibitem{kunzel}
H. M. K${\rm\ddot{u}}$nzel and K. Kiessl (1997) 
Calculation of heat and moisture transfer in exposed building components.
Int. J. Heat Mass Transfer, {\bf 40}, 159-167.

\bibitem{bear}
J. Bear (1972) Dynamics of Fluids in Porous Media. American Elsevier, New York.

\bibitem{welty}
J. R. Welty, C. E. Wicks and R. E. Wilson (1969) Fundamentals of heat and mass transfer.
John Wiley \& Sons, New York.

\bibitem{degroot}
S. R. Degroot and P. Mazur (1962) Non-equilibrium thermodynamics.
North Holland-Interscience Publisher, New York.

\bibitem{lykov}
A. V. Lykov and Ju. A. Mikhailov (1961) Theory of energy and mass transfer (translated from Russian).
Englewood Cliffs, Prentice Hall, N.J.

\bibitem{fatt}
I. Fatt and W. A. Klikoff (1959) Effect of fractional wettability on multiphase flow through porous media. 
Note No. 2043, AIME Trans., {\bf 216}, 246.

\bibitem{brooks}
R. N. Brooks and A. T. Corey (1966) Properties of porous media affectivity fluid flow.
J. Irrig. Drain. Div. Am. Sc. Civ. Engng, {\bf 92(IR 2)}, 61-68.

\bibitem{monte}
J. L. Monte and R. J. Kritzen (1976) One-dimensional mathematical model for large-strain consolidation.
Geotechnique, {\bf 26}, 3, 495-510.

\bibitem{kuklik2}
P. Kukl\ii k, J. Mare\v{s} and M. \v{S}ejnoha (1999)
The structural strength of soil from the isotropic consolidation point of view.
Computational Mechanics for the Next Millenium, Wang, Lee, Ang (eds.), {\bf 2}, Elsevier, Singapore, 797-802.

\bibitem{dluzevski}
J. M. Dluzevski (1998)
Large strain consolidation for elastoplastic soils. Application of Numerical Methods to Geotechnical Problems.
Cividini (ed.), Springer Verlag, Wien-New York, 473-482.

\bibitem{bazant_n}
Z. P. Bazant, L. J. Najjar (1972) Nonlinear water diffusion in nonsaturated concrete
structural analysis program. Mat\'eriaux et constructions, RILEM, Paris, 
{\bf 5(25)}, 8-9.

\bibitem{bitt}
Z. Bittnar and J. \v Sejnoha (1996) Numerical methods in structural
mechanics. ASCE Press, NY.

\bibitem{zienkiewicz83}
O. C. Zienkiewicz (1983) Basic formulas of static and dynamic behaviour of soils and other porous media.
Institute of Numerical Methods in Engineering. University College of Swansea

\bibitem{ctu}
T. Krej\v{c}\ii , T. Nov\'y, L. Sehnoutek and J. \v{S}ejnoha (2001) Structure - Subsoil Interaction in view of Transport 
Processes in Porous Media. CTU Reports, {\bf 1(5)}

\bibitem{bazant88}
(1988) Mathematical Modeling of Creep and Shrinkage of Concrete. 
John Wiley \& Sons, Edited by Zden\v{e}k P. Ba\v{z}ant, Chichester, New York, Brisbane, Toronto, Singapore.

\bibitem{bazant_ch}
Z. P. Bazant and Chern (1985a) Concrete creep at variable humidity: constitutive law and mechanism. Materials and 
Structures (RILEM, Paris), {\bf 18}, Jan., 1-20.

\bibitem{cervenka}
Vladimir Cervenka, Rolf Elingehausen, Radomir Pukl (1990/1) SBETA Computer Program For Nonlinear Finite Element Analysis 
Of Reinforced Concrete Structures. ISBN 3-9801833-6-X, Mitteilungen des Instituts f${\rm\ddot{u}}$r Werkstoffe 
im Bauwesen, 2-5.

\bibitem{cerny}
O. Hrstka, R. \v{C}ern\'y, P. Rovnan\ii kov\'a (1999) Hygrothermal Stress Induced Problems in Large Scale Sprayed 
Concrete Structures. Specialist Techniques and Materials for Concrete Construction. R. K. Dhir, N. A. Hendersen (eds.), 
Thomas Telford, London, 103-109.

\bibitem{bazant1}
Z. P. Bazant and S. T. Wu (1973) Dirichlet series function for aging
concrete. J. of the Engineering Mechanics Division, {\bf 99(EM2)}, 367-387.

\bibitem{bazant2}
Z. P. Bazant (1982) Input of creep and shrinkage characteristics for
structural analysis program. Materiaux et constructions, RILEM, Paris, 
{\bf 15(88)}, 283-290.

\bibitem{bazant3}
Z. P. Bazant and L. Panulla (1978) Practical prediction of time
dependent deformations of concrete, Part I,II. Materiaux et
constructions, RILEM, Paris, {\bf 11(65)}, 307-328.

\bibitem{gawin1}
D. Gawin, C. E. Majorana, B. A. Schrefler (1999) Numerical analysis of hygro-thermic behaviour and damage of concrete 
at high temperature. Mechanics of Cohesiv-Frictional Materials, {\bf 4}, 37-74.

\bibitem{gawin2}
D. Gawin, C. E. Majorana, B. A. Schrefler (1996) Hygro-thermic and mechanical behaviour of concrete 
at high temperature. Advances in Computational Mechanics, Zhong W., Cheng G., Li X. (eds.). International 
Academic Publishers: Beijing, 221-242.

\bibitem{pesavento}
Francesco Pesavento (2000) NONLINEAR MODELING OF CONCRETE AS MULTIPHASE POROUS MATERIAL IN HIGH TEMPERATURE CONDITION. 
PhD-thesis, Universit\`a degli Studi di Padova

\bibitem{leonov}
A. I. Leonov (1976) Nonequilibrium thermodynamics and rheology of viscoelastic polymer media. J. Rheol., {\bf 15}, 85. 

\bibitem{tenchev}
Tenchev, R. T., Li, L. Y. \& Purkiss, J. A. (2001) Finite Element Analysis of Coupled Heat and Moisture
Transfer in Concrete Subjected to fire, {\it Numerical Heat Transfer, Part A}; {\bf 39}: pp 685 - 710.

\bibitem{glas}
Colin T. Davie, Chris J. Pierce and Nenad Bi\'cani\'c (2004) COUPLED HEAT AND MOISTURE TRANSPORT IN CONCRETE AT ELEVATED
TEMPERATURES - EFFECT OF CAPILLARY PRESSURE AND ADSORBE3D WEATER, Department of Civil Engineering, University of Glasgow

\end{thebibliography}
