%%%%%%%%%%%%%%%%%%%%%%%%%%%%%%%%%%%%%%%%%%%%%%%%%%%%%%%%%%%%%%%%%%%%%%%%%%%%%%%%%%%%%%%%%%%%%%%%%%%%%
%%%%%%%%%%%%%%%%%%%%%%%%%%%%%%%%%%%%%%%%%%%%%%%%%%%%%%%%%%%%%%%%%%%%%%%%%%%%%%%%%%%%%%%%%%%%%%%%%%%%%
\chapter{Description of TRFEL Philosophy}
%%%%%%%%%%%%%%%%%%%%%%%%%%%%%%%%%%%%%%%%%%%%%%%%%%%%%%%%%%%%%%%%%%%%%%%%%%%%%%%%%%%%%%%%%%%%%%%%%%%%%
%%%%%%%%%%%%%%%%%%%%%%%%%%%%%%%%%%%%%%%%%%%%%%%%%%%%%%%%%%%%%%%%%%%%%%%%%%%%%%%%%%%%%%%%%%%%%%%%%%%%%

\section{Strategy of nonstationary problems}

tento odstavec opravit??!!

Nonlinear static problems are described by (\ref{eqnnonlinstat}). Dependence of stiffness matrix on actual
configuration represented by vector of displacements $\mbf{d}$ is the main difficulty. System of nonlinear
algebraic equations (\ref{eqnnonlinstat}) is solved by the arc-length method or by the Newton-Raphson method.
Increments of displacements and loads are computed from the attained state which is in equilibrium. New
vectors of displacements and loads define new state which is generally not in equilibrium. Therefore vector
of residuals is assembled and corrections of mentioned vectors are computed. The last step is repeated until
equilibrium state is obtained.


%%%%%%%%%%%%%%%%%%%%%%%%%%%%%%%%%%%%%%%%%%%%%%%%%%%%%%%%%%%%%%%%%%%%%%%%%%%%%%%%%%%%%%%%%%%%%%%%%%%%%
\section{Integration Points}
\index{integration points}

Integration points play significant role in the code. They contain information about used material models,
actual and previous values of unknowns, important material parameters etc.


There are the following arrays at every integration point
\begin{itemize}
\item{{\sf tm} - contains type of material or materials,}
\item{{\sf idm} - contains identification number of material or materials,}
\item{{\sf av} - contains actual values of unknowns,}
\item{{\sf pv} - contains values of unknowns from previous tim step,}
\item{{\sf fluxes} - contains flux components,}
\item{{\sf grad} - contains gradients components,}
\item{{\sf other} - contains all other parameters, which are not in the equilibrium during the iteration,}
\end{itemize}
\index{array!other}\index{array!eqother}\index{array!stress}\index{array!strain}\index{array!nonloc}

tento odstavec opravit??!!

The array {\sf other} contains all necessary material parameters. Elastic models do not use it. Plastic
models require to store plastic strains, consistency parameter and optionally hardening parameters.
Damage models require to store damage parameter etc. Stored parameters in the array {\sf other} are
defined by used material model. The material model should describe an order of parameters in the
array {\sf other}.

The array {\sf eqother} contains the same parameters as the array {\sf other}. The array {\sf other}
is changed in each iteration and contains values of material parameters which can be unequilibriated.
On the other hand, the array {\sf eqother} contains values of material parameters which are
equilibriated and this array is changed only when the new global equilibrium is attained.
Components of the array {\sf eqother} are changed by the function {\sf void mechmat::updateipval (void)}
which is called by the solver (arc-length, Newton-Raphson method etc.).

The situation is more complicated in the case of combination of several material models. Since it is
not clear the maximum number of combined material models, the arrays {\sf other} and {\sf eqother}
are not multiplied. The arrays are allocated longer. The number of components of the arrays is
equal to the sum of number of material parameters of combined models. The number of components
of the arrays can be obtained by the function {\sf long mechmat::givencompother (long ipp,long im)}.
The {\it ipp} denotes the number of integration point and {\it im} stands for the number of
material model used at the integration point (it is index in the array {\sf tm}).
The defualt value of {\it im} is 0.

%%%%%%%%%%%%%%%%%%%%%%%%%%%%%%%%%%%%%%%%%%%%%%%%%%%%%%%%%%%%%%%%%%%%%%%%%%%%%%%%%%%%%%%%%%%%%%%%%%%%%
%%%%%%%%%%%%%%%%%%%%%%%%%%%%%%%%%%%%%%%%%%%%%%%%%%%%%%%%%%%%%%%%%%%%%%%%%%%%%%%%%%%%%%%%%%%%%%%%%%%%%
\section{Combination of Material Models}

The code contains many material models in the following areas:
\begin{itemize}
\item{elasticity}
\item{plasticity}
\item{damage}
\item{viscosity}
\item{microplane models}
\end{itemize}

In order to keep the code general in the area of material modelling, the code enables many combinations
of implemented material models. For this purpose, artificial material models are introduced. They are called
artificial because they do not contain any material parameters. Their main and only purpose is that they
call appropriate functions of combined models.

The array {\sf tm} of the class {\sf intpoints} play important role in the case of combination of several
material models. The main material model is located at the first position. This model controls whole
computation. It means, that it calls appropriate methods of classes, stores and restores required data etc.
The artificial material models are always located at the first position of the array {\sf tm}.

The type elastic material model is always located at the last position of the array {\sf tm}. The only exception
is the case with temperature load which requires a model of thermal dilatancy. The type of the thermal dilatancy
is located at the last position of the array {\sf tm} and the type of elastic model is at the last but one position.
The type of elastic material can be obtained by the function {\sf long intpoints::gemid(void)}. It returns
the index of the elastic model in the array {\sf tm}.

%%%%%%%%%%%%%%%%%%%%%%%%%%%%%%%%%%%%%%%%%%%%%%%%%%%%%%%%%%%%%%%%%%%%%%%%%%%%%%%%%%%%%%%%%%%%%%%%%%%%%
%%%%%%%%%%%%%%%%%%%%%%%%%%%%%%%%%%%%%%%%%%%%%%%%%%%%%%%%%%%%%%%%%%%%%%%%%%%%%%%%%%%%%%%%%%%%%%%%%%%%%
\section{Nonlinear nonstationary computations}

Nonlocal computations are used for various problems which suffer by difficulties during the classical
local approximation. The typical example is description of damage which has to be localized in a narrow
band.

The application of material models based on nonlocal approximation is clearly more complicated than
the classical material models. For better understanding, the order of method calling is described in this section.

Solver of nonlinear algebraic equations (arc-length, Newton-Raphson) calls function
\newline {\sf internal\_forces (lcid,fi);} from the file {\sf globmat.cpp}.
\newline With respect to the nonlocal model, the function
\newline {\sf nonloc\_internal\_forces (lcid,intfor);} from the file {\sf globmat.cpp}
\newline is called. This function calls the function
\newline {\sf local\_values(lcid, i, 0, 0);}
\newline with respect to the type of finite element. The finite element calls
\newline {\sf Mm$\rightarrow$computenlstresses (ipp);} from the file {\sf mechmat.cpp}
\newline in order to obtain correct stresses. With respect to the used material model, the function
\newline {\sf nlstresses (ipp);} or {\sf nonloc\_nlstresses (ipp,im,ido);}
\newline is called. Then the code is going back to the function
\newline {\sf nonloc\_internal\_forces (long lcid,double *intfor)} from the file {\sf globmat.cpp}
\newline where the averaging is performed. Then the function
\newline {\sf nonloc\_internal\_forces (lcid, i, 0, 0, ifor);}
\newline is called with respect to the used element. This function calls
\newline {\sf Mm$\rightarrow$compnonloc\_nlstresses (ipp);} from the file {\sf mechmat.cpp}
\newline With respect to the type of material model, the function
\newline {\sf nonloc\_nlstresses (ipp,im,ido);}
\newline is called. After its execution, the stresses are known and nodal forces
are computed at the element level. The resulting vector of nodal forces is obtained
after execution
\newline {\sf nonloc\_internal\_forces (long lcid,double *intfor)}

%%%%%%%%%%%%%%%%%%%%%%%%%%%%%%%%%%%%%%%%%%%%%%%%%%%%%%%%%%%%%%%%%%%%%%%%%%%%%%%%%%%%%%%%%%%%%%%%%%%%%
%%%%%%%%%%%%%%%%%%%%%%%%%%%%%%%%%%%%%%%%%%%%%%%%%%%%%%%%%%%%%%%%%%%%%%%%%%%%%%%%%%%%%%%%%%%%%%%%%%%%%
\section{Quantity sources}

Source of studied quantity is one term in transport equation.

Source of quantity is defined in object {\it lc} of the class {\sf loadcaset}
which represents load case. The source is described by the class {\sf sourcet}.

Source can be defined at nodes as well as at elements. List of nodes where source
of quantity is defined is stored in an array denoted {\it lnqs} which is located in
objects {\it lc}. List of elements with defined source is stored in an array denoted
{\it leqs} which is in located in objects {\it lc}. Contributions from source terms
are computed on elements and therefore elements with sources have to be denoted.
Function {\sf loadcaset::elemsource ()} searches for elements with sources.
It deals only with sources defined at nodes because the situation is clear in case of
sources defined at element (list of such elements is directly available, the list
is denoted {\it leqs}). 

 The class {\sf loadcaset} contains array {\it lenqs} where
list of numbers of nodes with sources are stored for each element. {\it lenqs[eid][nid]}
contains number of row in array {\it lnqs} for


If {\it lenqs} returns value -1, no source is defined at the node.

%%%%%%%%%%%%%%%%%%%%%%%%%%%%%%%%%%%%%%%%%%%%%%%%%%%%%%%%%%%%%%%%%%%%%%%%%%%%%%%%%%%%%%%%%%%%%%%%%%%%%
%%%%%%%%%%%%%%%%%%%%%%%%%%%%%%%%%%%%%%%%%%%%%%%%%%%%%%%%%%%%%%%%%%%%%%%%%%%%%%%%%%%%%%%%%%%%%%%%%%%%%
\section{Problems with Discontinuities}

Transport problems with discontinuities prescribed in space can be solved
by problem type discont\_nonlin\_nonstat\_problem=62. Discontinuity prescribed
in space can be found e.g. at material interfaces. It is assumed that
discontinuity leads on element edges in two dimensional problems or on
element surfaces in three dimensional elements. Therefore, approximated
functions (such as temperature, humidity, etc.) are smooth on elements
but there are jumps at interface nodes and of course along interface edges or surfaces.
More precisely, there are jumps in nodal values.

Additional informations have to be read in the case of transport problems
with discontinuitites. There must be a list of elements influenced by
discontinuity. Elements adjacent to discontinuity from one side must
be mentioned in the list. For each element, there must be number of
nodes with discontinuity (with jumps) on particular element. There must
also be a list of node numbers for each element. These informations
are read before the classical element reading. The part
containing these informations is located after node reading
and before the classical element reading.

When the previously mentioned data are read, variable discont located
at the class elementt is changed from its default value -1 to some
nonnegative value. The variable describes number of element influenced
by discontinuity. This number is generally different from element number
because it is assigned only to elements influenced by the jumps. Elements
not influenced by the jumps have the variable discont equal to -1.
This value is assigned in the constructor.

In contrast to the classical computation, nodal values of elements influenced
by discontinuity are recalculated with the help of appropriate material model.
Function nodalvalues in file globmatt.cpp contains case
discont\_nonlin\_nonstat\_problem, where function discont\_val is used.
Final value transformation is located in function     
Tm->values\_transformation in the file transmat.cpp

