\documentclass[12pt]{book}
\newcommand{\mbf}[1]{\mbox{\boldmath$#1$}}
% definice zlomku
\newcommand{\del}[2]{\mbox{$\displaystyle\frac{#1}{#2}$}}
% definice prvni parcialni derivace funkce
\newcommand{\ppd}[2]{\del{\partial{#1}}{\partial{#2}}}
% definice druhe parcialni derivace funkce podle x a y
\newcommand{\dpd}[3]{\del{\partial^{2}\>\!\!{#1}}{\partial{#2}\ \partial{#3}}}
% definice n-te parcialni derivace funkce
\newcommand{\npd}[3]{\del{\partial^{#1}\>\!\!{#2}}{\partial{#3}^{#1}}}
% definice prvni derivace funkce
\newcommand{\od}[2]{\del{{\rm d}\>\!{#1}}{{\rm d}{#2}}}
% definice n-te obycejne derivace funkce
\newcommand{\nod}[3]{\del{{\rm d}^{#1}\>\!\!{#2}}{{\rm d}{#3}^{#1}}}
\usepackage{makeidx}
\usepackage{graphicx}
\makeindex

\setcounter{secnumdepth}{4}
\setcounter{totalnumber}{4}
\oddsidemargin=5mm
\evensidemargin=5mm
\textwidth=160mm
\topmargin=-2.0cm
\textheight=250mm
\headheight=1.5cm


\begin{document}
\tableofcontents
%%%%%%%%%%%%%%%%%%%%%%%%%%%%%%%%%%%%%%%%%%%%%%%%%%%%%%%%%%%%%%%%%%%%%%%%%%
%%%%%%%%%%%%%%%%%%%%%%%%%%%%%%%%%%%%%%%%%%%%%%%%%%%%%%%%%%%%%%%%%%%%%%%%%%
%%%%%%%%%%%%%%%%%%%%%%%%%%%%%%%%%%%%%%%%%%%%%%%%%%%%%%%%%%%%%%%%%%%%%%%%%%
%%%%%%%%%%%%%%%%%%%%%%%%%%%%%%%%%%%%%%%%%%%%%%%%%%%%%%%%%%%%%%%%%%%%%%%%%%
\chapter{MESH GENERATION}

This chapter describes available simple mesh generators especially for the SIFEL code.
The mesh generators are not comparable with real mesh generators because they have
strongly limited branch of application but they are very comfortable for simple domain shapes
and simple boundary conditions.

%%%%%%%%%%%%%%%%%%%%%%%%%%%%%%%%%%%%%%%%%%%%%%%%%%%%%%%%%%%%%%%%%%%%%%%%%%
%%%%%%%%%%%%%%%%%%%%%%%%%%%%%%%%%%%%%%%%%%%%%%%%%%%%%%%%%%%%%%%%%%%%%%%%%%
%%%%%%%%%%%%%%%%%%%%%%%%%%%%%%%%%%%%%%%%%%%%%%%%%%%%%%%%%%%%%%%%%%%%%%%%%%
%%%%%%%%%%%%%%%%%%%%%%%%%%%%%%%%%%%%%%%%%%%%%%%%%%%%%%%%%%%%%%%%%%%%%%%%%%
\section{SIFEL mesh file format}

SIFEL Mesh format has the following form

\noindent
{\tt
nn\\
x y z np\\
:\\
x y z np\\
ne et\\
m n1 n2 ... nm ep\\
:\\
m n1 n2 ... nm ep\\
}

\noindent
Notation:

\noindent
\begin{tabular}{ll}
nn & number of nodes
\\
x,y,z & node coordinates
\\
np & property of the node
\\
ne & number of elements
\\
et & type of element
\\
m & number of nodes on one element
\\
n1, ..., nm & node numbers on element
\end{tabular}

\noindent
Example of file with mesh in the SIFEL mesh format. The length of the
rectangle in the $x$-direction is 4 m, in the $y$-direction is 3 m.
There are 2 elements along both directions. The mesh contains 9 nodes
and 4 quadrilateral elements.
\noindent
{\tt
9\\
0.0000000000e+00 0.0000000000e+00 0.0000000000e+00 0 \\
0.0000000000e+00 1.5000000000e+00 0.0000000000e+00 0 \\
0.0000000000e+00 3.0000000000e+00 0.0000000000e+00 0 \\
2.0000000000e+00 0.0000000000e+00 0.0000000000e+00 3 \\
2.0000000000e+00 1.5000000000e+00 0.0000000000e+00 4 \\
2.0000000000e+00 3.0000000000e+00 0.0000000000e+00 1 \\
4.0000000000e+00 0.0000000000e+00 0.0000000000e+00 2 \\
4.0000000000e+00 1.5000000000e+00 0.0000000000e+00 2 \\
4.0000000000e+00 3.0000000000e+00 0.0000000000e+00 2 \\
4 5\\
4 5 2 1 4 0\\
4 6 3 2 5 0\\
4 8 5 4 7 0\\
4 9 6 5 8 0\\
}

%%%%%%%%%%%%%%%%%%%%%%%%%%%%%%%%%%%%%%%%%%%%%%%%%%%%%%%%%%%%%%%%%%%%%%%%%%
%%%%%%%%%%%%%%%%%%%%%%%%%%%%%%%%%%%%%%%%%%%%%%%%%%%%%%%%%%%%%%%%%%%%%%%%%%
%%%%%%%%%%%%%%%%%%%%%%%%%%%%%%%%%%%%%%%%%%%%%%%%%%%%%%%%%%%%%%%%%%%%%%%%%%
%%%%%%%%%%%%%%%%%%%%%%%%%%%%%%%%%%%%%%%%%%%%%%%%%%%%%%%%%%%%%%%%%%%%%%%%%%
\section{Two dimensional mesh generation}

For simple code testing of two dimensional problems, several simple mesh generators on rectangular
domain have been developed. Namely, generators of quadrilaterals with four nodes, quadrilaterals
with eight nodes, triangles with three nodes and triangles with 6 nodes have been implemented.


%%%%%%%%%%%%%%%%%%%%%%%%%%%%%%%%%%%%%%%%%%%%%%%%%%%%%%%%%%%%%%%%%%%%%%%%%%
%%%%%%%%%%%%%%%%%%%%%%%%%%%%%%%%%%%%%%%%%%%%%%%%%%%%%%%%%%%%%%%%%%%%%%%%%%
%%%%%%%%%%%%%%%%%%%%%%%%%%%%%%%%%%%%%%%%%%%%%%%%%%%%%%%%%%%%%%%%%%%%%%%%%%
%%%%%%%%%%%%%%%%%%%%%%%%%%%%%%%%%%%%%%%%%%%%%%%%%%%%%%%%%%%%%%%%%%%%%%%%%%
\subsection{Generation of rectangles with four nodes}

The mesh generator of rectangles with four nodes on rectangular domain is in the file gensifquad.cpp.
Executable file is denoted {\tt gquad}. The code generates rectangular elements on a rectangular domain
in the $xy$ plane. The length of the domain in the $x$ direction is lx while the length in the $y$ direction
is ly. All $z$ coordinates are equal to zero. Nodes on left edge ($x$=0.0) of the rectangle
are denoted by property 0. Nodes on top edge ($y$=ly) of the rectangle are denoted by property 1.
Nodes on right edge ($x$=lx) of the rectangle are denoted by property 2. Nodes on bottom edge ($y$=0.0)
of the rectangle are denoted property 3. All remaining nodes are denoted by property 4.
All elements are denoted by property 0.

\vspace{2mm}
\noindent
The command has the form

{\tt gquad file\_name lx ly nx ny write\_edge\_num}

\vspace{2mm}
\noindent
where

\vspace{2mm}
\noindent
\begin{center}
\begin{tabular}{ll}
file\_name & name of file with mesh data
\\
lx & length of the rectangle in the $x$ direction
\\
ly & length of the rectangle in the $y$ direction
\\
nx & number of elements along the $x$ direction
\\
ny & number of elements along the $y$ direction
\\
write\_edge\_num & switch for edge number output
\\
\end{tabular}
\end{center}

\noindent
Switch for edge number output is 0 or 1. 0 means no edge number output
while 1 means write edge numbers.

\vspace{3mm}
\noindent
{\bf Example}

\noindent
Rectangular domain with the length in the $x$ direction is 4 m and in the
$y$ direction is 3 m. Two elements are required along both sides.

\noindent
Command

{\tt gquad quad.top 4 3 2 2 0}

\noindent
Output

\noindent
{\tt
9\\
0.0000000000e+00 0.0000000000e+00 0.0000000000e+00 0 \\
0.0000000000e+00 1.5000000000e+00 0.0000000000e+00 0 \\
0.0000000000e+00 3.0000000000e+00 0.0000000000e+00 0 \\
2.0000000000e+00 0.0000000000e+00 0.0000000000e+00 3 \\
2.0000000000e+00 1.5000000000e+00 0.0000000000e+00 4 \\
2.0000000000e+00 3.0000000000e+00 0.0000000000e+00 1 \\
4.0000000000e+00 0.0000000000e+00 0.0000000000e+00 2 \\
4.0000000000e+00 1.5000000000e+00 0.0000000000e+00 2 \\
4.0000000000e+00 3.0000000000e+00 0.0000000000e+00 2 \\
4 5\\
4 5 2 1 4 0\\
4 6 3 2 5 0\\
4 8 5 4 7 0\\
4 9 6 5 8 0\\
}

%%%%%%%%%%%%%%%%%%%%%%%%%%%%%%%%%%%%%%%%%%%%%%%%%%%%%%%%%%%%%%%%%%%%%%%%%%
%%%%%%%%%%%%%%%%%%%%%%%%%%%%%%%%%%%%%%%%%%%%%%%%%%%%%%%%%%%%%%%%%%%%%%%%%%
%%%%%%%%%%%%%%%%%%%%%%%%%%%%%%%%%%%%%%%%%%%%%%%%%%%%%%%%%%%%%%%%%%%%%%%%%%
%%%%%%%%%%%%%%%%%%%%%%%%%%%%%%%%%%%%%%%%%%%%%%%%%%%%%%%%%%%%%%%%%%%%%%%%%%
\subsection{Generation of rectangles with eight nodes}

The mesh generator of rectangles with eight nodes on rectangular domain is in the file gensifquadq.cpp.
Executable file is denoted {\tt gquadq}. The code generates rectangular elements on a rectangular domain
in the $xy$ plane. The length of the domain in the $x$ direction is lx while the length in the $y$ direction
is ly. All $z$ coordinates are equal to zero. Nodes on left edge ($x$=0.0) of the rectangle
are denoted by property 0. Nodes on top edge ($y$=ly) of the rectangle are denoted by property 1.
Nodes on right edge ($x$=lx) of the rectangle are denoted by property 2. Nodes on bottom edge ($y$=0.0)
of the rectangle are denoted property 3. All remaining nodes are denoted by property 4.
All elements are denoted by property 0.

\vspace{2mm}
\noindent
The command has the form

{\tt gquadq file\_name lx ly nx ny write\_edge\_num}

\vspace{2mm}
\noindent
where

\vspace{2mm}
\noindent
\begin{center}
\begin{tabular}{ll}
file\_name & name of file with mesh data
\\
lx & length of the rectangle in the $x$ direction
\\
ly & length of the rectangle in the $y$ direction
\\
nx & number of elements along the $x$ direction
\\
ny & number of elements along the $y$ direction
\\
write\_edge\_num & switch for edge number output
\\
\end{tabular}
\end{center}

\noindent
Switch for edge number output is 0 or 1. 0 means no edge number output
while 1 means write edge numbers.

\vspace{3mm}
\noindent
{\bf Example}

\noindent
Rectangular domain with the length in the $x$ direction is 4 m and in the
$y$ direction is 3 m. Two elements are required along both sides.

\noindent
Command

{\tt gquadq quadq.top 4 3 2 2 0}

\noindent
Output

\noindent
{\tt
21\\
0.0000000000e+00 0.0000000000e+00 0.0000000000e+00 0\\
0.0000000000e+00 7.5000000000e-01 0.0000000000e+00 0\\
0.0000000000e+00 1.5000000000e+00 0.0000000000e+00 0\\
0.0000000000e+00 2.2500000000e+00 0.0000000000e+00 0\\
0.0000000000e+00 3.0000000000e+00 0.0000000000e+00 0\\
1.0000000000e+00 0.0000000000e+00 0.0000000000e+00 3\\
1.0000000000e+00 1.5000000000e+00 0.0000000000e+00 4\\
1.0000000000e+00 3.0000000000e+00 0.0000000000e+00 1\\
2.0000000000e+00 0.0000000000e+00 0.0000000000e+00 3\\
2.0000000000e+00 7.5000000000e-01 0.0000000000e+00 4\\
2.0000000000e+00 1.5000000000e+00 0.0000000000e+00 4\\
2.0000000000e+00 2.2500000000e+00 0.0000000000e+00 4\\
2.0000000000e+00 3.0000000000e+00 0.0000000000e+00 1\\
3.0000000000e+00 0.0000000000e+00 0.0000000000e+00 3\\
3.0000000000e+00 1.5000000000e+00 0.0000000000e+00 4\\
3.0000000000e+00 3.0000000000e+00 0.0000000000e+00 1\\
4.0000000000e+00 0.0000000000e+00 0.0000000000e+00 2\\
4.0000000000e+00 7.5000000000e-01 0.0000000000e+00 2\\
4.0000000000e+00 1.5000000000e+00 0.0000000000e+00 2\\
4.0000000000e+00 2.2500000000e+00 0.0000000000e+00 2\\
4.0000000000e+00 3.0000000000e+00 0.0000000000e+00 2\\
4 6\\
8 11 3 1 9 7 2 6 10 0\\
8 13 5 3 11 8 4 7 12 0\\
8 19 11 9 17 15 10 14 18 0\\
8 21 13 11 19 16 12 15 20 0\\
}

%%%%%%%%%%%%%%%%%%%%%%%%%%%%%%%%%%%%%%%%%%%%%%%%%%%%%%%%%%%%%%%%%%%%%%%%%%
%%%%%%%%%%%%%%%%%%%%%%%%%%%%%%%%%%%%%%%%%%%%%%%%%%%%%%%%%%%%%%%%%%%%%%%%%%
%%%%%%%%%%%%%%%%%%%%%%%%%%%%%%%%%%%%%%%%%%%%%%%%%%%%%%%%%%%%%%%%%%%%%%%%%%
%%%%%%%%%%%%%%%%%%%%%%%%%%%%%%%%%%%%%%%%%%%%%%%%%%%%%%%%%%%%%%%%%%%%%%%%%%
\subsection{Generation of triangles with three nodes}

The mesh generator of triangles with three nodes on rectangular domain is in the file gensiftria.cpp.
Executable file is denoted {\tt gtria}. The code generates triangular elements on a rectangular domain
in the $xy$ plane. The length of the domain in the $x$ direction is lx while the length in the $y$ direction
is ly. All $z$ coordinates are equal to zero. Nodes on left edge ($x$=0.0) of the rectangle
are denoted by property 0. Nodes on top edge ($y$=ly) of the rectangle are denoted by property 1.
Nodes on right edge ($x$=lx) of the rectangle are denoted by property 2. Nodes on bottom edge ($y$=0.0)
of the rectangle are denoted property 3. All remaining nodes are denoted by property 4.
All elements are denoted by property 0.

\vspace{2mm}
\noindent
The command has the form

{\tt gtria file\_name lx ly nx ny write\_edge\_num}

\vspace{2mm}
\noindent
where

\vspace{2mm}
\noindent
\begin{center}
\begin{tabular}{ll}
file\_name & name of file with mesh data
\\
lx & length of the rectangle in the $x$ direction
\\
ly & length of the rectangle in the $y$ direction
\\
nx & number of elements along the $x$ direction
\\
ny & number of elements along the $y$ direction
\\
write\_edge\_num & switch for edge number output
\\
\end{tabular}
\end{center}

\noindent
Switch for edge number output is 0 or 1. 0 means no edge number output
while 1 means write edge numbers.

\vspace{3mm}
\noindent
{\bf Example}

\noindent
Rectangular domain with the length in the $x$ direction is 4 m and in the
$y$ direction is 3 m. Two elements are required along both sides.

\noindent
Command

{\tt gtria tria.top 4 3 2 2 0}

\noindent
Output

\noindent
{\tt
9\\
0.0000000000e+00 0.0000000000e+00 0.0000000000e+00 0\\
0.0000000000e+00 1.5000000000e+00 0.0000000000e+00 0\\
0.0000000000e+00 3.0000000000e+00 0.0000000000e+00 0\\
2.0000000000e+00 0.0000000000e+00 0.0000000000e+00 3\\
2.0000000000e+00 1.5000000000e+00 0.0000000000e+00 4\\
2.0000000000e+00 3.0000000000e+00 0.0000000000e+00 1\\
4.0000000000e+00 0.0000000000e+00 0.0000000000e+00 2\\
4.0000000000e+00 1.5000000000e+00 0.0000000000e+00 2\\
4.0000000000e+00 3.0000000000e+00 0.0000000000e+00 2\\
8 3\\
3 5 1 4 0\\
3 5 2 1 0\\
3 6 2 5 0\\
3 6 3 2 0\\
3 8 4 7 0\\
3 8 5 4 0\\
3 9 5 8 0\\
3 9 6 5 0\\
}





%%%%%%%%%%%%%%%%%%%%%%%%%%%%%%%%%%%%%%%%%%%%%%%%%%%%%%%%%%%%%%%%%%%%%%%%%%
%%%%%%%%%%%%%%%%%%%%%%%%%%%%%%%%%%%%%%%%%%%%%%%%%%%%%%%%%%%%%%%%%%%%%%%%%%
%%%%%%%%%%%%%%%%%%%%%%%%%%%%%%%%%%%%%%%%%%%%%%%%%%%%%%%%%%%%%%%%%%%%%%%%%%
%%%%%%%%%%%%%%%%%%%%%%%%%%%%%%%%%%%%%%%%%%%%%%%%%%%%%%%%%%%%%%%%%%%%%%%%%%
\subsection{Generation of hexahedra with eight nodes}

The mesh generator of hexahedra with eight nodes on hexahedral domain (regular rectangular prism) is in the file gensifhex.cpp.
Executable file is denoted {\tt ghex}. The code generates hexahedral elements on a hexahedral domain.
The length of the rectangular prism in the $x$ direction is lx, the length in the $y$ direction is ly and
the length in the $z$ direction is lz. Nodes on surface $x$=0.0 of the brick domain are denoted by property 0.
Nodes on surface $z$=lz of the rectangular prism are denoted by property 1.
Nodes on surface $x$=lx of the rectangular prism are denoted by property 2.
Nodes on surface $z$=0.0 of the rectangular prism are denoted property 3.
All remaining nodes are denoted by property 4.
All elements are denoted by property 0.


\vspace{2mm}
\noindent
The command has the form

{\tt ghex file\_name lx ly lz nx ny nz}

\vspace{2mm}
\noindent
where

\vspace{2mm}
\noindent
\begin{center}
\begin{tabular}{ll}
file\_name & name of file with mesh data
\\
lx & length of the prism in the $x$ direction
\\
ly & length of the prism in the $y$ direction
\\
lz & length of the prism in the $z$ direction
\\
nx & number of elements along the $x$ direction
\\
ny & number of elements along the $y$ direction
\\
nz & number of elements along the $z$ direction
\end{tabular}
\end{center}

\vspace{3mm}
\noindent
{\bf Example}

\noindent
Regular quadrangular prism domain with the length in the $x$ direction equal to 5 m, in the
$y$ direction equal to 4 m and in the $z$ direction equal to 3 m. Two elements are required along each side.

\noindent
Command

{\tt ghex hex.top 5 4 3 2 2 2}

\noindent
Output

\noindent
{\tt
27\\
0.0000000000e+00 0.0000000000e+00 0.0000000000e+00 0\\
0.0000000000e+00 0.0000000000e+00 1.5000000000e+00 0\\
0.0000000000e+00 0.0000000000e+00 3.0000000000e+00 0\\
0.0000000000e+00 2.0000000000e+00 0.0000000000e+00 0\\
0.0000000000e+00 2.0000000000e+00 1.5000000000e+00 0\\
0.0000000000e+00 2.0000000000e+00 3.0000000000e+00 0\\
0.0000000000e+00 4.0000000000e+00 0.0000000000e+00 0\\
0.0000000000e+00 4.0000000000e+00 1.5000000000e+00 0\\
0.0000000000e+00 4.0000000000e+00 3.0000000000e+00 0\\
2.5000000000e+00 0.0000000000e+00 0.0000000000e+00 3\\
2.5000000000e+00 0.0000000000e+00 1.5000000000e+00 4\\
2.5000000000e+00 0.0000000000e+00 3.0000000000e+00 1\\
2.5000000000e+00 2.0000000000e+00 0.0000000000e+00 3\\
2.5000000000e+00 2.0000000000e+00 1.5000000000e+00 4\\
2.5000000000e+00 2.0000000000e+00 3.0000000000e+00 1\\
2.5000000000e+00 4.0000000000e+00 0.0000000000e+00 3\\
2.5000000000e+00 4.0000000000e+00 1.5000000000e+00 4\\
2.5000000000e+00 4.0000000000e+00 3.0000000000e+00 1\\
5.0000000000e+00 0.0000000000e+00 0.0000000000e+00 2\\
5.0000000000e+00 0.0000000000e+00 1.5000000000e+00 2\\
5.0000000000e+00 0.0000000000e+00 3.0000000000e+00 2\\
5.0000000000e+00 2.0000000000e+00 0.0000000000e+00 2\\
5.0000000000e+00 2.0000000000e+00 1.5000000000e+00 2\\
5.0000000000e+00 2.0000000000e+00 3.0000000000e+00 2\\
5.0000000000e+00 4.0000000000e+00 0.0000000000e+00 2\\
5.0000000000e+00 4.0000000000e+00 1.5000000000e+00 2\\
5.0000000000e+00 4.0000000000e+00 3.0000000000e+00 2\\
8 13\\
8 14 5 2 11 13 4 1 10 0\\
8 15 6 3 12 14 5 2 11 0\\
8 17 8 5 14 16 7 4 13 0\\
8 18 9 6 15 17 8 5 14 0\\
8 23 14 11 20 22 13 10 19 0\\
8 24 15 12 21 23 14 11 20 0\\
8 26 17 14 23 25 16 13 22 0\\
8 27 18 15 24 26 17 14 23 0\\
}


%%%%%%%%%%%%%%%%%%%%%%%%%%%%%%%%%%%%%%%%%%%%%%%%%%%%%%%%%%%%%%%%%%%%%%%%%%
%%%%%%%%%%%%%%%%%%%%%%%%%%%%%%%%%%%%%%%%%%%%%%%%%%%%%%%%%%%%%%%%%%%%%%%%%%
%%%%%%%%%%%%%%%%%%%%%%%%%%%%%%%%%%%%%%%%%%%%%%%%%%%%%%%%%%%%%%%%%%%%%%%%%%
%%%%%%%%%%%%%%%%%%%%%%%%%%%%%%%%%%%%%%%%%%%%%%%%%%%%%%%%%%%%%%%%%%%%%%%%%%
\subsection{Generation of hexahedra with twenty nodes}

The mesh generator of hexahedra with twenty nodes on hexahedral domain (regular rectangular prism)
or on a tube segment is in the file gensifhexq.cpp.
Executable file is denoted {\tt ghexq}. The code generates hexahedral elements on a rectangular prism
or on a tube segment.
In the case of prism, the following informations are required.
The length of the rectangular prism in the $x$ direction is lx, the length in the $y$ direction is ly and
the length in the $z$ direction is lz. Nodes on surface $x$=0.0 of the brick domain are denoted by property 0.
All remaining nodes are denoted by property 4.
All elements are denoted by property 0.
In the case of tube segment, the following informations are required.
The radius of the tube $r$, the length of the segment $l$, the thickness of the tube $t$ and
the angle $\alpha$.


\vspace{2mm}
\noindent
In the case of prism, the command has the form

{\tt ghexq file\_name  1  lx ly lz nx ny nz}

\vspace{2mm}
\noindent
where

\vspace{2mm}
\noindent
\begin{center}
\begin{tabular}{ll}
file\_name & name of file with mesh data
\\
1 & indicates a prism domain
\\
lx & length of the prism in the $x$ direction
\\
ly & length of the prism in the $y$ direction
\\
lz & length of the prism in the $z$ direction
\\
nx & number of elements along the $x$ direction
\\
ny & number of elements along the $y$ direction
\\
nz & number of elements along the $z$ direction
\end{tabular}
\end{center}

\vspace{2mm}
\noindent
In the case of tube segment, the command has the form

{\tt ghexq 2 file\_name r l t nr nl nt alpha}

\vspace{2mm}
\noindent
where

\vspace{2mm}
\noindent
\begin{center}
\begin{tabular}{ll}
file\_name & name of file with mesh data
\\
2 & indicates a tube segment
\\
r & radius of the tube segment
\\
l & length of the tube segment
\\
t & thisckness of the tube segment
\\
nr & number of elements along the $r$ direction
\\
nl & number of elements along the $l$ direction
\\
nt & number of elements along the thickness
\\
alpha & angle of the segment
\\
\end{tabular}
\end{center}

\vspace{3mm}
\noindent
{\bf Example 1}

\noindent
Regular quadrangular prism domain with the length in the $x$ direction equal to 5 m, in the
$y$ direction equal to 4 m and in the $z$ direction equal to 3 m. Two elements are required along each side.

\noindent
Command

{\tt ghexq prism.top  1  5 4 3 2 2 2}

\noindent
Output

\noindent
{\tt
81\\
0.0000000000e+00 0.0000000000e+00 0.0000000000e+00 0\\
0.0000000000e+00 0.0000000000e+00 7.5000000000e-01 0\\
0.0000000000e+00 0.0000000000e+00 1.5000000000e+00 0\\
0.0000000000e+00 0.0000000000e+00 2.2500000000e+00 0\\
0.0000000000e+00 0.0000000000e+00 3.0000000000e+00 0\\
0.0000000000e+00 1.0000000000e+00 0.0000000000e+00 0\\
0.0000000000e+00 1.0000000000e+00 1.5000000000e+00 0\\
0.0000000000e+00 1.0000000000e+00 3.0000000000e+00 0\\
0.0000000000e+00 2.0000000000e+00 0.0000000000e+00 0\\
0.0000000000e+00 2.0000000000e+00 7.5000000000e-01 0\\
0.0000000000e+00 2.0000000000e+00 1.5000000000e+00 0\\
0.0000000000e+00 2.0000000000e+00 2.2500000000e+00 0\\
0.0000000000e+00 2.0000000000e+00 3.0000000000e+00 0\\
0.0000000000e+00 3.0000000000e+00 0.0000000000e+00 0\\
0.0000000000e+00 3.0000000000e+00 1.5000000000e+00 0\\
0.0000000000e+00 3.0000000000e+00 3.0000000000e+00 0\\
0.0000000000e+00 4.0000000000e+00 0.0000000000e+00 0\\
0.0000000000e+00 4.0000000000e+00 7.5000000000e-01 0\\
0.0000000000e+00 4.0000000000e+00 1.5000000000e+00 0\\
0.0000000000e+00 4.0000000000e+00 2.2500000000e+00 0\\
0.0000000000e+00 4.0000000000e+00 3.0000000000e+00 0\\
1.2500000000e+00 0.0000000000e+00 0.0000000000e+00 1\\
1.2500000000e+00 0.0000000000e+00 1.5000000000e+00 1\\
1.2500000000e+00 0.0000000000e+00 3.0000000000e+00 1\\
1.2500000000e+00 2.0000000000e+00 0.0000000000e+00 1\\
1.2500000000e+00 2.0000000000e+00 1.5000000000e+00 1\\
1.2500000000e+00 2.0000000000e+00 3.0000000000e+00 1\\
1.2500000000e+00 4.0000000000e+00 0.0000000000e+00 1\\
1.2500000000e+00 4.0000000000e+00 1.5000000000e+00 1\\
1.2500000000e+00 4.0000000000e+00 3.0000000000e+00 1\\
2.5000000000e+00 0.0000000000e+00 0.0000000000e+00 1\\
2.5000000000e+00 0.0000000000e+00 7.5000000000e-01 1\\
2.5000000000e+00 0.0000000000e+00 1.5000000000e+00 1\\
2.5000000000e+00 0.0000000000e+00 2.2500000000e+00 1\\
2.5000000000e+00 0.0000000000e+00 3.0000000000e+00 1\\
2.5000000000e+00 1.0000000000e+00 0.0000000000e+00 1\\
2.5000000000e+00 1.0000000000e+00 1.5000000000e+00 1\\
2.5000000000e+00 1.0000000000e+00 3.0000000000e+00 1\\
2.5000000000e+00 2.0000000000e+00 0.0000000000e+00 1\\
2.5000000000e+00 2.0000000000e+00 7.5000000000e-01 1\\
2.5000000000e+00 2.0000000000e+00 1.5000000000e+00 1\\
2.5000000000e+00 2.0000000000e+00 2.2500000000e+00 1\\
2.5000000000e+00 2.0000000000e+00 3.0000000000e+00 1\\
2.5000000000e+00 3.0000000000e+00 0.0000000000e+00 1\\
2.5000000000e+00 3.0000000000e+00 1.5000000000e+00 1\\
2.5000000000e+00 3.0000000000e+00 3.0000000000e+00 1\\
2.5000000000e+00 4.0000000000e+00 0.0000000000e+00 1\\
2.5000000000e+00 4.0000000000e+00 7.5000000000e-01 1\\
2.5000000000e+00 4.0000000000e+00 1.5000000000e+00 1\\
2.5000000000e+00 4.0000000000e+00 2.2500000000e+00 1\\
2.5000000000e+00 4.0000000000e+00 3.0000000000e+00 1\\
3.7500000000e+00 0.0000000000e+00 0.0000000000e+00 1\\
3.7500000000e+00 0.0000000000e+00 1.5000000000e+00 1\\
3.7500000000e+00 0.0000000000e+00 3.0000000000e+00 1\\
3.7500000000e+00 2.0000000000e+00 0.0000000000e+00 1\\
3.7500000000e+00 2.0000000000e+00 1.5000000000e+00 1\\
3.7500000000e+00 2.0000000000e+00 3.0000000000e+00 1\\
3.7500000000e+00 4.0000000000e+00 0.0000000000e+00 1\\
3.7500000000e+00 4.0000000000e+00 1.5000000000e+00 1\\
3.7500000000e+00 4.0000000000e+00 3.0000000000e+00 1\\
5.0000000000e+00 0.0000000000e+00 0.0000000000e+00 1\\
5.0000000000e+00 0.0000000000e+00 7.5000000000e-01 1\\
5.0000000000e+00 0.0000000000e+00 1.5000000000e+00 1\\
5.0000000000e+00 0.0000000000e+00 2.2500000000e+00 1\\
5.0000000000e+00 0.0000000000e+00 3.0000000000e+00 1\\
5.0000000000e+00 1.0000000000e+00 0.0000000000e+00 1\\
5.0000000000e+00 1.0000000000e+00 1.5000000000e+00 1\\
5.0000000000e+00 1.0000000000e+00 3.0000000000e+00 1\\
5.0000000000e+00 2.0000000000e+00 0.0000000000e+00 1\\
5.0000000000e+00 2.0000000000e+00 7.5000000000e-01 1\\
5.0000000000e+00 2.0000000000e+00 1.5000000000e+00 1\\
5.0000000000e+00 2.0000000000e+00 2.2500000000e+00 1\\
5.0000000000e+00 2.0000000000e+00 3.0000000000e+00 1\\
5.0000000000e+00 3.0000000000e+00 0.0000000000e+00 1\\
5.0000000000e+00 3.0000000000e+00 1.5000000000e+00 1\\
5.0000000000e+00 3.0000000000e+00 3.0000000000e+00 1\\
5.0000000000e+00 4.0000000000e+00 0.0000000000e+00 1\\
5.0000000000e+00 4.0000000000e+00 7.5000000000e-01 1\\
5.0000000000e+00 4.0000000000e+00 1.5000000000e+00 1\\
5.0000000000e+00 4.0000000000e+00 2.2500000000e+00 1\\
5.0000000000e+00 4.0000000000e+00 3.0000000000e+00 1\\
8 14\\
20 41 11 3 33 39 9 1 31 26 7 23 37 40 10 2 32 25 6 22 36   0\\
20 43 13 5 35 41 11 3 33 27 8 24 38 42 12 4 34 26 7 23 37   0\\
20 49 19 11 41 47 17 9 39 29 15 26 45 48 18 10 40 28 14 25 44   0\\
20 51 21 13 43 49 19 11 41 30 16 27 46 50 20 12 42 29 15 26 45   0\\
20 71 41 33 63 69 39 31 61 56 37 53 67 70 40 32 62 55 36 52 66   0\\
20 73 43 35 65 71 41 33 63 57 38 54 68 72 42 34 64 56 37 53 67   0\\
20 79 49 41 71 77 47 39 69 59 45 56 75 78 48 40 70 58 44 55 74   0\\
20 81 51 43 73 79 49 41 71 60 46 57 76 80 50 42 72 59 45 56 75   0\\
}




%%%%%%%%%%%%%%%%%%%%%%%%%%%%%%%%%%%%%%%%%%%%%%%%%%%%%%%%%%%%%%%%%%%%%%%%%%
%%%%%%%%%%%%%%%%%%%%%%%%%%%%%%%%%%%%%%%%%%%%%%%%%%%%%%%%%%%%%%%%%%%%%%%%%%
%%%%%%%%%%%%%%%%%%%%%%%%%%%%%%%%%%%%%%%%%%%%%%%%%%%%%%%%%%%%%%%%%%%%%%%%%%
%%%%%%%%%%%%%%%%%%%%%%%%%%%%%%%%%%%%%%%%%%%%%%%%%%%%%%%%%%%%%%%%%%%%%%%%%%
\subsection{Generation of tetrahedra with four nodes}

The mesh generator of tetrahedra with four nodes on hexahedral domain (regular quadrangular prism) is in the file gensiftetras.cpp.
Executable file is denoted {\tt gtetra}. The code generates tetrahedral elements on a hexahedral domain.
The length of the domain in the $x$ direction is lx, the length in the $y$ direction is ly and
the length in the $z$ direction is lz. Nodes on surface $x$=0.0 of the brick domain are denoted by property 0.
Nodes on surface $z$=lz of the brick domain are denoted by property 1.
Nodes on surface $x$=lx of the brick domain are denoted by property 2.
Nodes on surface $z$=0.0 of the brick domain are denoted property 3.
All remaining nodes are denoted by property 4.
All elements are denoted by property 0.
A mesh of hexahedral elements is created first and then each element is divided to 6 tetrahedra.

\vspace{2mm}
\noindent
The command has the form

{\tt gtetra file\_name lx ly lz nx ny nz}

\vspace{2mm}
\noindent
where

\vspace{2mm}
\noindent
\begin{center}
\begin{tabular}{ll}
file\_name & name of file with mesh data
\\
lx & length of the prism in the $x$ direction
\\
ly & length of the prism in the $y$ direction
\\
lz & length of the prism in the $z$ direction
\\
nx & number of elements along the $x$ direction
\\
ny & number of elements along the $y$ direction
\\
nz & number of elements along the $z$ direction
\end{tabular}
\end{center}

\vspace{3mm}
\noindent
{\bf Example}

\noindent
Regular quadrangular prism domain with the length in the $x$ direction equal to 5 m, in the
$y$ direction equal to 4 m and in the $z$ direction equal to 3 m. Two rows of elements are required along each side.

\noindent
Command

{\tt gtetra tetra.top 5 4 3 2 2 2}

\noindent
Output

\noindent
{\tt
27\\
0.0000000000e+00 0.0000000000e+00 0.0000000000e+00 0\\
0.0000000000e+00 0.0000000000e+00 1.5000000000e+00 0\\
0.0000000000e+00 0.0000000000e+00 3.0000000000e+00 0\\
0.0000000000e+00 2.0000000000e+00 0.0000000000e+00 0\\
0.0000000000e+00 2.0000000000e+00 1.5000000000e+00 0\\
0.0000000000e+00 2.0000000000e+00 3.0000000000e+00 0\\
0.0000000000e+00 4.0000000000e+00 0.0000000000e+00 0\\
0.0000000000e+00 4.0000000000e+00 1.5000000000e+00 0\\
0.0000000000e+00 4.0000000000e+00 3.0000000000e+00 0\\
2.5000000000e+00 0.0000000000e+00 0.0000000000e+00 3\\
2.5000000000e+00 0.0000000000e+00 1.5000000000e+00 4\\
2.5000000000e+00 0.0000000000e+00 3.0000000000e+00 1\\
2.5000000000e+00 2.0000000000e+00 0.0000000000e+00 3\\
2.5000000000e+00 2.0000000000e+00 1.5000000000e+00 4\\
2.5000000000e+00 2.0000000000e+00 3.0000000000e+00 1\\
2.5000000000e+00 4.0000000000e+00 0.0000000000e+00 3\\
2.5000000000e+00 4.0000000000e+00 1.5000000000e+00 4\\
2.5000000000e+00 4.0000000000e+00 3.0000000000e+00 1\\
5.0000000000e+00 0.0000000000e+00 0.0000000000e+00 2\\
5.0000000000e+00 0.0000000000e+00 1.5000000000e+00 2\\
5.0000000000e+00 0.0000000000e+00 3.0000000000e+00 2\\
5.0000000000e+00 2.0000000000e+00 0.0000000000e+00 2\\
5.0000000000e+00 2.0000000000e+00 1.5000000000e+00 2\\
5.0000000000e+00 2.0000000000e+00 3.0000000000e+00 2\\
5.0000000000e+00 4.0000000000e+00 0.0000000000e+00 2\\
5.0000000000e+00 4.0000000000e+00 1.5000000000e+00 2\\
5.0000000000e+00 4.0000000000e+00 3.0000000000e+00 2\\
 48 7\\
 4  14 5 11 10  0\\
 4  14 5 13 10  0\\
 4  5 2 11 10  0\\
 4  5 2 1 10  0\\
 4  5 4 13 10  0\\
 4  5 1 4 10  0\\
 4  15 6 12 11  0\\
 4  15 6 14 11  0\\
 4  6 3 12 11  0\\
 4  6 3 2 11  0\\
 4  6 5 14 11  0\\
 4  6 2 5 11  0\\
 4  17 8 14 13  0\\
 4  17 8 16 13  0\\
 4  8 5 14 13  0\\
 4  8 5 4 13  0\\
 4  8 7 16 13  0\\
 4  8 4 7 13  0\\
 4  18 9 15 14  0\\
 4  18 9 17 14  0\\
 4  9 6 15 14  0\\
 4  9 6 5 14  0\\
 4  9 8 17 14  0\\
 4  9 5 8 14  0\\
 4  23 14 20 19  0\\
 4  23 14 22 19  0\\
 4  14 11 20 19  0\\
 4  14 11 10 19  0\\
 4  14 13 22 19  0\\
 4  14 10 13 19  0\\
 4  24 15 21 20  0\\
 4  24 15 23 20  0\\
 4  15 12 21 20  0\\
 4  15 12 11 20  0\\
 4  15 14 23 20  0\\
 4  15 11 14 20  0\\
 4  26 17 23 22  0\\
 4  26 17 25 22  0\\
 4  17 14 23 22  0\\
 4  17 14 13 22  0\\
 4  17 16 25 22  0\\
 4  17 13 16 22  0\\
 4  27 18 24 23  0\\
 4  27 18 26 23  0\\
 4  18 15 24 23  0\\
 4  18 15 14 23  0\\
 4  18 17 26 23  0\\
 4  18 14 17 23  0\\
}








































%%%%%%%%%%%%%%%%%%%%%%%%%%%%%%%%%%%%%%%%%%%%%%%%%%%%%%%%%%%%%%%%%%%%%%%%%%
%%%%%%%%%%%%%%%%%%%%%%%%%%%%%%%%%%%%%%%%%%%%%%%%%%%%%%%%%%%%%%%%%%%%%%%%%%
%%%%%%%%%%%%%%%%%%%%%%%%%%%%%%%%%%%%%%%%%%%%%%%%%%%%%%%%%%%%%%%%%%%%%%%%%%
%%%%%%%%%%%%%%%%%%%%%%%%%%%%%%%%%%%%%%%%%%%%%%%%%%%%%%%%%%%%%%%%%%%%%%%%%%
\chapter{SPECIAL MESH TOOLS}

Extensive number of possible problems solvable in the SIFEL led to development of several
special mesh tools. The tools are described in this chapter.


%%%%%%%%%%%%%%%%%%%%%%%%%%%%%%%%%%%%%%%%%%%%%%%%%%%%%%%%%%%%%%%%%%%%%%%%%%
%%%%%%%%%%%%%%%%%%%%%%%%%%%%%%%%%%%%%%%%%%%%%%%%%%%%%%%%%%%%%%%%%%%%%%%%%%
%%%%%%%%%%%%%%%%%%%%%%%%%%%%%%%%%%%%%%%%%%%%%%%%%%%%%%%%%%%%%%%%%%%%%%%%%%
%%%%%%%%%%%%%%%%%%%%%%%%%%%%%%%%%%%%%%%%%%%%%%%%%%%%%%%%%%%%%%%%%%%%%%%%%%
%%%%%%%%%%%%%%%%%%%%%%%%%%%%%%%%%%%%%%%%%%%%%%%%%%%%%%%%%%%%%%%%%%%%%%%%%%
\section{Problem of floating subdomains}

Floating subdomains are used in problems of interaction between fibres and composite
matrix. The composite matrix is represented by one subdomain which is properly constrained.
It means, that there are no rigid body motions. Fibres are represented by additional
subdomains which may be or may not be constrained. Generally, fibres may contain
rigid body motions.

The problem of floating subdomain requires merging of at least two files with meshes.
For this purpose, the code flsubmesh.cpp has been developed. Auxiliary steps have to be
performed before application of the code flsubmesh. The meshes of composite matrix
and fibres have to be stored in the files with an identical name. The name has to be
provided by a number. The file containing the mesh of the composite matrix has to be
provided by zero. The files containing the meshes of the fibres have to be provided
by increasing numbers. 

\vspace{2mm}
\noindent
The command has the form

{\tt flsubmesh meshes n finmesh.top}

\vspace{2mm}
\noindent
where

\vspace{2mm}
\noindent
\begin{center}
\begin{tabular}{ll}
flsubmesh & executable file
\\
meshes & name of files with meshes without suffixes
\\
n & number of fibres
\\
finmesh.top & name of the file with final mesh
\\
\end{tabular}
\end{center}


\vspace{3mm}
\noindent
{\bf Example}

\noindent
Let a composite matrix contains 3 fibres. Mesh of the composite matrix is stored
in the file mesh0.top. Mesh of the first fibre is stored in the file mesh1.top,
second fibre in the file mesh2.top and the third fibre in the file mesh3.top.
Final file with meshes is denoted by finalmesh.top. The command has the form


{\tt flsubmesh mesh 3 finalmesh.top}



%%%%%%%%%%%%%%%%%%%%%%%%%%%%%%%%%%%%%%%%%%%%%%%%%%%%%%%%%%%%%%%%%%%%%%%%%%
%%%%%%%%%%%%%%%%%%%%%%%%%%%%%%%%%%%%%%%%%%%%%%%%%%%%%%%%%%%%%%%%%%%%%%%%%%
%%%%%%%%%%%%%%%%%%%%%%%%%%%%%%%%%%%%%%%%%%%%%%%%%%%%%%%%%%%%%%%%%%%%%%%%%%
%%%%%%%%%%%%%%%%%%%%%%%%%%%%%%%%%%%%%%%%%%%%%%%%%%%%%%%%%%%%%%%%%%%%%%%%%%
%%%%%%%%%%%%%%%%%%%%%%%%%%%%%%%%%%%%%%%%%%%%%%%%%%%%%%%%%%%%%%%%%%%%%%%%%%
\chapter{MESH PARTITIONING}

Parallel computing based on domain decomposition requires mesh partitioning. This
chapter describes mesh partitioning by METIS library, especially by the code partdmesh.

%%%%%%%%%%%%%%%%%%%%%%%%%%%%%%%%%%%%%%%%%%%%%%%%%%%%%%%%%%%%%%%%%%%%%%%%%%
%%%%%%%%%%%%%%%%%%%%%%%%%%%%%%%%%%%%%%%%%%%%%%%%%%%%%%%%%%%%%%%%%%%%%%%%%%
%%%%%%%%%%%%%%%%%%%%%%%%%%%%%%%%%%%%%%%%%%%%%%%%%%%%%%%%%%%%%%%%%%%%%%%%%%
%%%%%%%%%%%%%%%%%%%%%%%%%%%%%%%%%%%%%%%%%%%%%%%%%%%%%%%%%%%%%%%%%%%%%%%%%%
%%%%%%%%%%%%%%%%%%%%%%%%%%%%%%%%%%%%%%%%%%%%%%%%%%%%%%%%%%%%%%%%%%%%%%%%%%
\section{Application of the code partdmesh}

Mesh partitioning of an original finite element mesh consists of several steps
described in this section.

\begin{itemize}
\item
generation of original finite element mesh
\item
generation of file for the code partdmesh
\item
generation of files based on output of partdmesh
\end{itemize}

\subsection{Generation of original finite element mesh}
Generation of the original finite element mesh may be done in various ways.
Application of the simple mesh generators described in ??? is one
possibility. The result of this step is a file with original finite
element mesh in the SIFEL mesh format described in ???

\subsection{Generation of file for the code partdmesh}

The code partdmesh requires an input file in exactly given form. The form is the following:

\noindent
{\tt
ne mt
n1 n2 ... nm
:
n1 n2 ... nm
}

\begin{tabular}{ll}
ne & number of elements
\\
mt & type of elements with respect to METIS
\\
n1, n2, ..., nm & node numbers on elements
\end{tabular}


\subsection{Generation of files based on output of partdmesh}
The code partdmesh generates output file where for all elements the number of subdomain, which contains them, is indicated.
This is enough description of the mesh partitioning. The code metispost generates nd files with meshes, where
nd is the number of subdomains. 





\end{document}

