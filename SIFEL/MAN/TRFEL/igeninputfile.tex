%%%%%%%%%%%%%%%%%%%%%%%%%%%%%%%%%%%%%%%%%%%%%%%%%%%%%%%%%%%
%%%%%%%%%%%%%%%%%%%%%%%%%%%%%%%%%%%%%%%%%%%%%%%%%%%%%%%%%%%
%%%%%%%%%%%%%%%%%%%%%%%%%%%%%%%%%%%%%%%%%%%%%%%%%%%%%%%%%%%
%%%%%%%%%%%%%%%%%%%%%%%%%%%%%%%%%%%%%%%%%%%%%%%%%%%%%%%%%%%
\chapter{Generation of input file}

%\section{Possibilities}

\section{File with topology}
Topology file should contain all informations about used finite element mesh.
It means nodes and elements. The nodes are described by their 3 coordinates
(3 coordinates must be always used) and by a property. The property serves
for definition of many values like the thickness of plate, density of material etc.
The elements are described by the number of nodes of element and by a property.
The property serves for definition of element type, material type etc. The topology
file does not contain type of finite element (like triangular element with linear
approximation functions for plane stress) but it contains only geometrical
informations (it is triangular element with 3 nodes). The type of finite element
is defined during preprocessing phase.

\section{Example}

Compute displacements of a brick which is clamped at one face and loaded at the
opossite one. The sizes of the brick are 10 m ($x$ direction), 2 m ($y$ direction) and
3 m ($z$ direction). There are 6 elements along the $x$ direction, 2 elements
along the $y$ direction and 3 elements along the $z$ direction.

This very simple mesh could be generated by our simple generator ghex from the
PRG/PREP/GENER

The statement is

ghex brick.top 10 2 3 \hspace{4mm} 6 2 3

There are 84 nodes and 36 elements. The nodes are divided into 5 groups with respect
to their property. All finite elements have property 0.


