\chapter{Expansion of the TRFEL}
\section{Implementation of new finite element}
%%%%%%%%%%%%%%%%%%%%%%%%%%%%%%%%%%%%%%%%%%%%%%%%%%%%%%%%%%%%
%%%%%%%%%%%%%%%%%%%%%%%%%%%%%%%%%%%%%%%%%%%%%%%%%%%%%%%%%%%%
%%%%%%%%%%%%%%%%%%%%%%%%%%%%%%%%%%%%%%%%%%%%%%%%%%%%%%%%%%%%
%%%%%%%%%%%%%%%%%%%%%%%%%%%%%%%%%%%%%%%%%%%%%%%%%%%%%%%%%%%%
\section{Implementation of new material model and theory}
%%%%%%%%%%%%%%%%%%%%%%%%%%%%%%%%%%%%%%%%%%%%%%%%%%%%%%%%%%%%
%%%%%%%%%%%%%%%%%%%%%%%%%%%%%%%%%%%%%%%%%%%%%%%%%%%%%%%%%%%%
\subsection{Implementation of new arbitrary model and theory}
The main aims of any material model and theory in the code are
\begin{itemize}
\item{evaluation of correct fluxes for trial gradients,}
\item{assemblage of the matrices of conductivity coefficients, $\mbf{D}$.}
\item{assemblage of capacity coefficients, $\mbf{C}$.}
\item{assemblage of right-hand side (if it is needed), $\mbf{f}$.}
\end{itemize}

Gradients and fluxes and uknown variables are stored at integration points. The actual gradients and actual unknown variables 
are computed by any solver implemented in the code and stored at integration points. 

Every material model must contain the following functions:
\begin{itemize}
\item{
{\sf void newmaterial::read (FILE *in)}
\newline reads necessary data of the model, {\it in} is pointer to the input file,
}
\item{
{\sf double newmaterial::condcoeff (double x, long ipp)}
\newline assembles the conductivity coefficient ($k$) for unknown variable $x$, 
{\it x} denotes unknown variable (conductivity coefficient is generaly function of unknown variable $x$)
{\it ipp} denotes the number of integration point, {\it ido} stands for the
}

\item{
{\sf double newmaterial::capcoeff (double x, long ipp)}
\newline assembles the capacity coefficient ($c$) for unknown variable $x$,
{\it x} denotes unknown variable (capacity coefficient is generaly function of unknown variable $x$)
{\it ipp} denotes the number of integration point
}

\item{
{\sf void newmaterial::flux (long ipp)}
\newline computes correct flux for computed gradient, if flux is not a function of conductivity matrix,
{\it ipp} denotes the number of integration point
}

\end{itemize}

%%%%%%%%%%%%%%%%%%%%%%%%%%%%%%%%%%%%%%%%%%%%%%%%%%%%%%%%%%%%
%%%%%%%%%%%%%%%%%%%%%%%%%%%%%%%%%%%%%%%%%%%%%%%%%%%%%%%%%%%%
\subsection{Implementation of new model for one medium transfer}

Needed functions:
\begin{itemize}
\item{
{\sf void newmaterial::read (FILE *in)}
\newline reads necessary data of the model
}
\item{
{\sf double newmaterial::condcoeff (double x, long ipp)}
\newline assembles the conductivity coefficient ($k$) for unknown variable
}

\item{
{\sf double newmaterial::capcoeff (double x, long ipp)}
\newline assembles the capacity coefficient ($c$) for unknown variable
}

\item{
{\sf void newmaterial::flux (long ipp)}
\newline computes correct flux for computed gradient, if flux is not a function of conductivity matrix (coefficient).
This function is not needed, if the flux is defined with this mentioned (standard) way.
For example, heat flux is defined (see eq. \eqref{fourier}) using conductivity coefficient (Fourier's law):
\begin{equation}
\tenss{q} = - \tenss{\chi}_{\rm eff} {\rm grad} T.\nonumber
\end{equation}
}
\end{itemize}

optional:
\begin{itemize}
\item 
{\sf void bound\_nodval (double \&new\_nodval,double \&bv,double trr,long nn,long bc)}
\newline assembles new value prescribed on the boundary for media transfer (Cauchy's boundary condition)

\item 
{\sf void trans\_coeff (double \&trc,long nn,long bc)}
\newline assembles new transfer coeffcient on the boundary for media transfer (Cauchy's boundary condition)

\item 
{\sf void bound\_flux (double \&flux,double \&bv,double trr,long nn,long bc)}
\newline assembles correct flux on the boundary for media transfer (Cauchy's boundary condition)

doplnit???\\


Explanation???\\

\end{itemize}

\subsubsection {Implementing new class of new model for one medium transfer}
This section describes what the user needs when implements new material model or theory
to TRFEL. 

First of all, user needs the problem derived in FEM form \eqref{onemed} in \ref{onemedtrans}:

\begin{eqnarray}
\tenss{K}_{11}\tenss{r}_1 + \tenss{C}_{11}\dot{\tenss{r}}_1 = {\tenss{q}}_1,\nonumber
\end{eqnarray}
where matrix $\tenss{K}$ is the general conductivity matrix (it contains conductivity coefficients), 
$\tenss{C}$ is the general capacity matrix (it contains capacity coefficients), respectively.

Each new material or theory means the user should found new class with name of material model or theory {\sf mymatmodel}.
This class should contain material parameters as a data members of this class. 

In addition the folowing functions should be defined :

\begin{itemize}
{\sf
\item
void read(FILE* in)
\item
double condcoeff (double x, long ipp)
\item
double capcoeff (double x, long ipp)
\item
void flux (long ipp)\\\\
{\rm optional:}
\item 
{void bound\_nodval (double \&new\_nodval,double \&bv,double trr,long nn,long bc)}
\item 
{void trans\_coeff (double \&trc,long nn,long bc)}
\item 
{void bound\_flux (double \&flux,double \&bv,double trr,long nn,long bc)}
}
\end{itemize}

Function {\sf void read (FILE* in)} reads necessary material parameters from input file {\it in}.\\

Function {\sf double condcoeff (double x, long ipp)} assembles material conductivity coefficient for given
integration point and material type. The first parameter {\it x} is allocated matrix which will be
filled up with values. The second parameter is integration point number which should be used for
access characteristic values in the integration point.\\

Function {\sf double capcoeff (double x, long ipp)} assembles material capacity coefficient for given
integration point and material type. The first parameter {\it x} is allocated matrix which will be
filled up with values. The second parameter is integration point number which should be used for
access characteristic values in the integration point.\\

Function {\sf void flux (long ipp)} computes flux for nonlinear material model. Parameter
{\it ipp} is integration point number where the flux should be evaluated.

Function should work following way:
\begin{itemize}
\item
Takes reached gradients ${\rm grad}_{ij}$ and reached {\sf eqother} values in the integration point number
{\it ipp} (See class {\sf transmat, intpoint}). 
\item
Computes appropriate fluxes and stores them into fluxes array in the integration point.
\end{itemize}

Optional function:\\

Function {\sf void bound\_nodval (double \&new\_nodval,double \&bv,double trr,long nn,long bc)}
assembles new value prescribed on the boundary for media transfer (Cauchy's boundary condition)\\

doplnit???\\

Function {\sf void trans\_coeff (double \&trc,long nn,long bc)}
assembles new transfer coeffcient on the boundary for media transfer (Cauchy's boundary condition)\\

 
doplnit???\\

Function {\sf void bound\_flux (double \&flux,double \&bv,double trr,long nn,long bc)}
assembles correct flux on the boundary for media transfer (Cauchy's boundary condition)

doplnit???\\


Explanation???\\


\subsubsection {Including new model class to TRFEL files}
Thus the new class {\sf mymatmodel} is written to files  mymatmodel.cpp${\rm |}$h. Now the user should
include this class to the TRFEL. Here is list of files which have to be modified:\\

{\bf alias.t}\\
So called material alias {\sf mattype} have to be added. Alias means the substitutive shortcut
for C++ enumaration integer number. Decision which number and name for its alias depends on the user.
Only restrictions are that the name and number have to be unique in the {\sf mattype} scope. Also note
the name haven't to be same as the name of the plasticity class. This aliases are used in the
{\tt switch} statements as values for {\tt case}. Numeric value of the aliases are used in the input
files for the TRFEL.\\

doplnit upravit???\\


{\bf transmat.h}\\
This file contains declaration of the {\sf transmat} class. This class contains also pointers to arrays for each
material class. Thus the user have to include new class {\sf mymatmodel} to the {\sf transmat} class declaration.
The name of the pointer is arbitrary expecting used names of course. Note that the user should use predeclaration
of {\sf mypmatmodel} class or insert include file with {\sf mymatmodel} class declaration.\\

{\bf transmat.cpp}
\begin{itemize}
\item {\sf void transmat::readmatchar (FILE *in)}
\end{itemize}
Note that include file with full class declaration for given {\sf mymatmodel} should be included at the begining
of transmat.cpp\\

In function {\sf transmat::readmatchar (FILE *in)} the actual values of material parameters for given material type. This function
contains {\tt switch} statement where the new {\tt case} for the  {\sf mymatmodel} would be included.
It is possible to take part of {\tt case} statement from another material and copy it into new {\tt case} and then
make some modification. Folowing lines shows example of such a new {\tt case} :\\
{\tt
      if (Mespr==1)  fprintf (stdout,"$\backslash$n number of MYMATMODEL materials \%ld",numtype);\\
      MYMATMODELPT = new MYMATMODEL [numtype];\\
      for (j = 0; j $<$ numtype; j++){\\
        k = numtype + 1;\\
        fscanf (in,"\%ld",\&k);\\
        if ((k $>$ numtype) $\mid\mid$ (k $<$ 1))\\
        {\\
          fprintf (stderr, "$\backslash$n$\backslash$n wrong number of MYMATMODEL material in function\\
                   transmat::readmatchar (\%s, line \%d).$\backslash$n",\_\_FILE\_\_,\_\_LINE\_\_);\\
        }\\
        MYMATMODELPT[k - 1].read (in);\\
      }\\
      break;\\
}

Where user should replace MYMATMODEL with name of the new class {\sf mymatmodel} and MYMATMODELPT with
name of the pointer which has been used in the {\sf transmat} class declaration in the file transmat.hpp. This
statement block the first allocates array with new material then folows loop where the number of sets of the
parameters is read and then the {\sf read} function for the given material type is called.\\

{\bf onemedium.cpp}
\begin{itemize}
\item {\sf double med1::double cond\_k (long ipp)}
\item {\sf double med1::double cap\_c (long ipp)}
\item {\sf void med1::rhs\_f\_vec\_nodval (double \&new\_nodval,double \&bv,double trr,long nn,long bc,long ipp)}
\item {\sf void med1::rhs\_f\_matrix (double \&trc,long nn,long bc,long ipp)}
\item {\sf void med1::rhs\_f\_flux (double \&new\_nodval,double \&bv,double trr,long nn,long bc,long ipp)}

\end{itemize}


In function {\sf double med1::double cond\_k (long ipp)}

doplnit upravit???\\


In function {\sf double med1::double cap\_c (long ipp)}

doplnit upravit???\\

In function {\sf void med1::rhs\_f\_vec\_nodval (double \&new\_nodval,double \&bv,double trr,long nn,long bc,long ipp)}

doplnit upravit???\\

In function {\sf void med1::rhs\_f\_matrix (double \&trc,long nn,long bc,long ipp)}

doplnit upravit???\\

In function {\sf void med1::rhs\_f\_flux (double \&new\_nodval,double \&bv,double trr,long nn,long bc,long ipp)}

doplnit upravit???\\

%%%%%%%%%%%%%%%%%%%%%%%%%%%%%%%%%%%%%%%%%%%%%%%%%%%%%%%%%%%%
%%%%%%%%%%%%%%%%%%%%%%%%%%%%%%%%%%%%%%%%%%%%%%%%%%%%%%%%%%%%
\subsection{Implementation of new model for two media transfer}

Needed functions:
\begin{itemize}
\item{
{\sf void newmaterial::read (FILE *in)}
\newline reads necessary data of the model
}



\item{
{\sf double newmaterial::cond\_coeff\_11 (double x$_1$, double x$_2$, long ipp)}
\newline assembles the conductivity coefficient ($k_{11}$) for first unknown variable of first unknown variable
}

\item{
{\sf double newmaterial::cap\_coeff\_11 (double x$_1$, double x$_2$, long ipp)}
\newline assembles the capacity coefficient ($c_{11}$) for first unknown variable of first unknown variable
}



\item{
{\sf double newmaterial::cond\_coeff\_12 (double x$_1$, double x$_2$, long ipp)}
\newline assembles the conductivity coefficient ($k_{12}$) for first unknown variable of second unknown variable
}

\item{
{\sf double newmaterial::cap\_coeff\_12 (double x$_1$, double x$_2$, long ipp)}
\newline assembles the capacity coefficient ($c_{12}$) for first unknown variable of second unknown variable
}



\item{
{\sf double newmaterial::cond\_coeff\_21 (double x$_1$, double x$_2$, long ipp)}
\newline assembles the conductivity coefficient ($k_{21}$) for second unknown variable of first unknown variable
}

\item{
{\sf double newmaterial::cap\_coeff\_21 (double x$_1$, double x$_2$, long ipp)}
\newline assembles the capacity coefficient ($c_{21}$) for second unknown variable of first unknown variable
}



\item{
{\sf double newmaterial::cond\_coeff\_22 (double x$_1$, double x$_2$, long ipp)}
\newline assembles the conductivity coefficient ($k_{22}$) for second unknown variable of second unknown variable
}

\item{
{\sf double newmaterial::cap\_coeff\_22 (double x$_1$, double x$_2$, long ipp)}
\newline assembles the capacity coefficient ($c_{22}$) for second unknown variable of second unknown variable
}



\item{
{\sf void newmaterial::flux\_1 (long ipp)}
\newline computes correct flux for first unknown of all (2) computed gradients
}

doplnit upravit???

\item{
{\sf void newmaterial::flux\_2 (long ipp)}
\newline computes correct flux for second unknown of all (2) computed gradients
}

doplnit upravit???


\end{itemize}


\subsubsection {Implementing new class of new model for two media transfer}
This section describes what the user needs when implements new material model or theory
to TRFEL. First of all user need derived ... equation FEM like in section 1.

doplnit upravit???


Each new material means the user should found new class with name of plasticity model {\sf mymatmodel}.
This class should contain material parameters as a data members of this class. 

In addition the folowing methods should be defined :

doplnit upravit???

\begin{itemize}
{\sf
\item
read(FILE* in)
\item
void matstiff (matrix \&d, long ipp)
\item
void nlstresses (long ipp, long im, long ido)
}
\end{itemize}

Function {\sf void matstiff (matrix \&d, long ipp, long ido)} assemble material stiffness matrix for given
integration point and material type. The first parameter {\it d} is allocated matrix which will be
filled up with values. The second parameter is integration point number which should be used for
access characteristic values in the integration point. The last parameter {\it ido} stands for index of the first 
internal variable in the array {\sf other} of given integration point for given material type. User may use standard 
stiffness matrix for linear elastic material or when it is available may use tangent stiffness matrix (it will reduce
number of iteration steps).\\

Function {\sf void nlstresses (long ipp, long im, long ido)} computes stresses for nonlinear material. Parameter
{\it ipp} is integration point number where the stresses should be evaluated. Parameter {\it im} is index of material type
in the array {\sf tm} for given integration point. Parameter {\it ido} stands for index of the first 
internal variable in the array {\sf other} of given integration point for given material type.

Function should work following way:
\begin{itemize}
\item
Take reached strains $\varepsilon_{ij}$ and reached {\sf eqother} values in the integration point number
{\it ipp} (See class {\sf transmat, intpoint}). For plasticity models should contain reached plastic
strains $\varepsilon^{pl}_{ij}$ and parametr $\gamma$ for cutting plane method.
\item
Call stress-return function built in {\sf transmat} i.e {\sf cutting\_plane} with parameters filled from
previous step. This method computes appropriate stresses and stores them into stress array in the
integration point.
\item
Store result plastic strains and coefficient $\gamma$ and possible another values to the {\sf other}
array in the integration point. 
\end{itemize}
For example see {\sf drprag::nlstresses(long ipp, long im, long ido)} in the file drprag.cpp.\\

Function {\sf void updateval (long ipp, long im, long ido)} copies values of {\sf other} array to
the eqother array in case that equilibrium for current step is reached. Parameter {\it ipp} is
integration point number whose the {\sf eqother} array should be updated. Parameter {\it im} denotes
index material type in the array {\sf tm} of given integration point. {\it ido} is index in the
{\sf other/eqother} array of the given integration point, from where the updated internal variables are stored.


\subsubsection {Including new model class to TRFEL files}
Thus the new class {\sf mymatmodel} is written to files  mymatmodel.cpp${\rm |}$h. Now the user should
include this class to the TRFEL. Here is list of files which have to be modified:\\

{\bf alias.t}\\
So called material alias {\sf mattype} have to be added. Alias means the substitutive shortcut
for C++ enumaration integer number. Decision which number and name for its alias depends on the user.
Only restrictions are that the name and number have to be unique in the {\sf mattype} scope. Also note
the name haven't to be same as the name of the plasticity class. This aliases are used in the
{\tt switch} statements as values for {\tt case}. Numeric value of the aliases are used in the input
files for the TRFEL.\\

{\bf transmat.h}\\
This file contains declaration of the {\sf transmat} class. This class contains also pointers to arrays for each
material class. Thus the user have to include new class {\sf mymatmodel} to the {\sf transmat} class declaration.
The name of the pointer is arbitrary expecting used names of course. Note that the user should use predeclaration
of {\sf mypmatmodel} class or insert include file with {\sf mymatmodel} class declaration.\\

{\bf transmat.cpp}\\
\begin{itemize}
\item {\sf void transmat::readmatchar (FILE *in)}
\item {\sf void transmat::matcond (matrix \&d,long ipp,long ri,long ci,long ncomp)}
\item {\sf double transmat::capcoeff (long ipp,long ri,long ci)}
\item {\sf void transmat::computenlfluxes (long lcid,long ipp)}
\end{itemize}
Note that include file with full class declaration for given {\sf mymatmodel} should be included at the begining
of transmat.cpp\\

In function {\sf transmat::readmatchar (FILE *in)} the actual values of material parameters for given material type. This function
contains {\tt switch} statement where the new {\tt case} for the  {\sf mymatmodel} would be included.
It is possible to take part of {\tt case} statement from another material and copy it into new {\tt case} and then
make some modification. Folowing lines shows example of such a new {\tt case} :\\
{\tt
      if (Mespr==1)  fprintf (stdout,"$\backslash$n number of MYMATMODEL materials \%ld",numtype);\\
      MYMATMODELPT = new MYMATMODEL [numtype];\\
      for (j = 0; j $<$ numtype; j++){\\
        k = numtype + 1;\\
        fscanf (in,"\%ld",\&k);\\
        if ((k $>$ numtype) $\mid\mid$ (k $<$ 1))\\
        {\\
          fprintf (stderr, "$\backslash$n$\backslash$n wrong number of MYMATMODEL material in function\\
                   transmat::readmatchar (\%s, line \%d).$\backslash$n",\_\_FILE\_\_,\_\_LINE\_\_);\\
        }\\
        MYMATMODELPT[k - 1].read (in);\\
      }\\
      break;\\
}

Where user should replace MYMATMODEL with name of the new class {\sf mymatmodel} and MYMATMODELPT with
name of the pointer which has been used in the {\sf transmat} class declaration in the file transmat.hpp. This
statement block the first allocates array with new material then folows loop where the number of sets of the
parameters is read and then the {\sf read} function for the given material type is called.\\

In function {\sf void transmat::matcond (matrix \&d,long ipp,long ri,long ci,long ncomp)}

doplnit upravit???


In function {\sf double transmat::capcoeff (long ipp,long ri,long ci)}


doplnit upravit???


In function {\sf void transmat::computenlfluxes (long lcid,long ipp)}


doplnit upravit???


{\bf twomedia.cpp}\\
\begin{itemize}
\item {\sf double med2::double cond\_kcc (long ipp)}
\item {\sf double med2::double cap\_cc (long ipp)}
\item {\sf double med2::double cond\_kct (long ipp)}
\item {\sf double med2::double cap\_ct (long ipp)}
\item {\sf double med2::double cond\_ktc (long ipp)}
\item {\sf double med2::double cap\_tc (long ipp)}
\item {\sf double med2::double cond\_ktt (long ipp)}
\item {\sf double med2::double cap\_tt (long ipp)}

doplnit prave strany a toky ???

\end{itemize}

In function {\sf double med2::double cond\_kcc (long ipp)}

doplnit upravit???


In function {\sf double med2::double cap\_cc (long ipp)}

doplnit upravit???

In function {\sf double med2::double cond\_kct (long ipp)}

doplnit upravit???

In function {\sf double med2::double cap\_ct (long ipp)}

doplnit upravit???

In function {\sf double med2::double cond\_ktc (long ipp)}

doplnit upravit???

In function {\sf double med2::double cap\_tc (long ipp)}

doplnit upravit???

In function {\sf double med2::double cond\_ktt (long ipp)}

doplnit upravit???

In function {\sf double med2::double cap\_tt (long ipp)}

doplnit upravit???


doplnit prave strany a toky ???

%%%%%%%%%%%%%%%%%%%%%%%%%%%%%%%%%%%%%%%%%%%%%%%%%%%%%%%%%%%%
%%%%%%%%%%%%%%%%%%%%%%%%%%%%%%%%%%%%%%%%%%%%%%%%%%%%%%%%%%%%
\subsection{Implementation of new model for three media transfer}


Needed functions:
\begin{itemize}
\item{
{\sf void newmaterial::read (FILE *in)}
\newline reads necessary data of the model
}



\item{
{\sf double newmaterial::cond\_coeff\_11 (double x$_1$, double x$_2$, double x$_3$, long ipp)}
\newline assembles the conductivity coefficient ($k_{11}$) for first unknown variable of first unknown variable
}

\item{
{\sf double newmaterial::cap\_coeff\_11 (double x$_1$, double x$_2$, double x$_3$, long ipp)}
\newline assembles the capacity coefficient ($c_{11}$) for first unknown variable of first unknown variable
}

\item{
{\sf double newmaterial::cond\_coeff\_12 (double x$_1$, double x$_2$, double x$_3$, long ipp)}
\newline assembles the conductivity coefficient ($k_{12}$) for first unknown variable of second unknown variable
}

\item{
{\sf double newmaterial::cap\_coeff\_12 (double x$_1$, double x$_2$, double x$_3$, long ipp)}
\newline assembles the capacity coefficient ($c_{12}$) for first unknown variable of second unknown variable
}

\item{
{\sf double newmaterial::cond\_coeff\_13 (double x$_1$, double x$_2$, double x$_3$, long ipp)}
\newline assembles the conductivity coefficient ($k_{13}$) for first unknown variable of third unknown variable
}

\item{
{\sf double newmaterial::cap\_coeff\_13 (double x$_1$, double x$_2$, double x$_3$, long ipp)}
\newline assembles the capacity coefficient ($c_{13}$) for first unknown variable of third unknown variable
}



\item{
{\sf double newmaterial::cond\_coeff\_21 (double x$_1$, double x$_2$, double x$_3$, long ipp)}
\newline assembles the conductivity coefficient ($k_{21}$) for second unknown variable of first unknown variable
}

\item{
{\sf double newmaterial::cap\_coeff\_21 (double x$_1$, double x$_2$, double x$_3$, long ipp)}
\newline assembles the capacity coefficient ($c_{21}$) for second unknown variable of first unknown variable
}

\item{
{\sf double newmaterial::cond\_coeff\_22 (double x$_1$, double x$_2$, double x$_3$, long ipp)}
\newline assembles the conductivity coefficient ($k_{22}$) for second unknown variable of second unknown variable
}

\item{
{\sf double newmaterial::cap\_coeff\_22 (double x$_1$, double x$_2$, double x$_3$, long ipp)}
\newline assembles the capacity coefficient ($c_{22}$) for second unknown variable of second unknown variable
}

\item{
{\sf double newmaterial::cond\_coeff\_23 (double x$_1$, double x$_2$, double x$_3$, long ipp)}
\newline assembles the conductivity coefficient ($k_{23}$) for second unknown variable of third unknown variable
}

\item{
{\sf double newmaterial::cap\_coeff\_23 (double x$_1$, double x$_2$, double x$_3$, long ipp)}
\newline assembles the capacity coefficient ($c_{23}$) for second unknown variable of third unknown variable
}



\item{
{\sf double newmaterial::cond\_coeff\_31 (double x$_1$, double x$_2$, double x$_3$, long ipp)}
\newline assembles the conductivity coefficient ($k_{31}$) for third unknown variable of first unknown variable
}

\item{
{\sf double newmaterial::cap\_coeff\_31 (double x$_1$, double x$_2$, double x$_3$, long ipp)}
\newline assembles the capacity coefficient ($c_{31}$) for third unknown variable of first unknown variable
}

\item{
{\sf double newmaterial::cond\_coeff\_32 (double x$_1$, double x$_2$, double x$_3$, long ipp)}
\newline assembles the conductivity coefficient ($k_{32}$) for third unknown variable of second unknown variable
}

\item{
{\sf double newmaterial::cap\_coeff\_32 (double x$_1$, double x$_2$, double x$_3$, long ipp)}
\newline assembles the capacity coefficient ($c_{32}$) for third unknown variable of second unknown variable
}

\item{
{\sf double newmaterial::cond\_coeff\_33 (double x$_1$, double x$_2$, double x$_3$, long ipp)}
\newline assembles the conductivity coefficient ($k_{33}$) for third unknown variable of third unknown variable
}

\item{
{\sf double newmaterial::cap\_coeff\_33 (double x$_1$, double x$_2$, double x$_3$, long ipp)}
\newline assembles the capacity coefficient ($c_{33}$) for third unknown variable of third unknown variable
}




\item{
{\sf void newmaterial::flux\_1 (long ipp)}
\newline computes correct flux for first unknown of all (3) computed gradients
}

doplnit upravit???


\item{
{\sf void newmaterial::flux\_2 (long ipp)}
\newline computes correct flux for second unknown of all (3) computed gradients
}

doplnit upravit???


\item{
{\sf void newmaterial::flux\_3 (long ipp)}
\newline computes correct flux for third unknown of all (3) computed gradients
}

doplnit upravit???


\end{itemize}

\subsubsection {Implementing new class of new model for three media transfer}

\subsubsection {Including new model class to TRFEL files}

This section describes what the user needs when implements new material model or theory
to TRFEL. First of all user need derived ... equation FEM like in section 1.

doplnit upravit???


Each new material means the user should found new class with name of plasticity model {\sf mymatmodel}.
This class should contain material parameters as a data members of this class. 

In addition the folowing methods should be defined :

doplnit upravit???

\begin{itemize}
{\sf
\item
read(FILE* in)
\item
void matstiff (matrix \&d, long ipp)
\item
void nlstresses (long ipp, long im, long ido)
}
\end{itemize}

Function {\sf void matstiff (matrix \&d, long ipp, long ido)} assemble material stiffness matrix for given
integration point and material type. The first parameter {\it d} is allocated matrix which will be
filled up with values. The second parameter is integration point number which should be used for
access characteristic values in the integration point. The last parameter {\it ido} stands for index of the first 
internal variable in the array {\sf other} of given integration point for given material type. User may use standard 
stiffness matrix for linear elastic material or when it is available may use tangent stiffness matrix (it will reduce
number of iteration steps).\\

Function {\sf void nlstresses (long ipp, long im, long ido)} computes stresses for nonlinear material. Parameter
{\it ipp} is integration point number where the stresses should be evaluated. Parameter {\it im} is index of material type
in the array {\sf tm} for given integration point. Parameter {\it ido} stands for index of the first 
internal variable in the array {\sf other} of given integration point for given material type.

Function should work following way:
\begin{itemize}
\item
Take reached strains $\varepsilon_{ij}$ and reached {\sf eqother} values in the integration point number
{\it ipp} (See class {\sf transmat, intpoint}). For plasticity models should contain reached plastic
strains $\varepsilon^{pl}_{ij}$ and parametr $\gamma$ for cutting plane method.
\item
Call stress-return function built in {\sf transmat} i.e {\sf cutting\_plane} with parameters filled from
previous step. This method computes appropriate stresses and stores them into stress array in the
integration point.
\item
Store result plastic strains and coefficient $\gamma$ and possible another values to the {\sf other}
array in the integration point. 
\end{itemize}
For example see {\sf drprag::nlstresses(long ipp, long im, long ido)} in the file drprag.cpp.\\

Function {\sf void updateval (long ipp, long im, long ido)} copies values of {\sf other} array to
the eqother array in case that equilibrium for current step is reached. Parameter {\it ipp} is
integration point number whose the {\sf eqother} array should be updated. Parameter {\it im} denotes
index material type in the array {\sf tm} of given integration point. {\it ido} is index in the
{\sf other/eqother} array of the given integration point, from where the updated internal variables are stored.


\subsubsection {Including new model class to TRFEL files}
Thus the new class {\sf mymatmodel} is written to files  mymatmodel.cpp${\rm |}$h. Now the user should
include this class to the TRFEL. Here is list of files which have to be modified:\\

{\bf alias.t}\\
So called material alias {\sf mattype} have to be added. Alias means the substitutive shortcut
for C++ enumaration integer number. Decision which number and name for its alias depends on the user.
Only restrictions are that the name and number have to be unique in the {\sf mattype} scope. Also note
the name haven't to be same as the name of the plasticity class. This aliases are used in the
{\tt switch} statements as values for {\tt case}. Numeric value of the aliases are used in the input
files for the TRFEL.\\

{\bf transmat.h}\\
This file contains declaration of the {\sf transmat} class. This class contains also pointers to arrays for each
material class. Thus the user have to include new class {\sf mymatmodel} to the {\sf transmat} class declaration.
The name of the pointer is arbitrary expecting used names of course. Note that the user should use predeclaration
of {\sf mypmatmodel} class or insert include file with {\sf mymatmodel} class declaration.\\

{\bf transmat.cpp}\\
\begin{itemize}
\item {\sf void transmat::readmatchar (FILE *in)}
\item {\sf void transmat::matcond (matrix \&d,long ipp,long ri,long ci,long ncomp)}
\item {\sf double transmat::capcoeff (long ipp,long ri,long ci)}
\item {\sf void transmat::computenlfluxes (long lcid,long ipp)}
\end{itemize}
Note that include file with full class declaration for given {\sf mymatmodel} should be included at the begining
of transmat.cpp\\

In function {\sf transmat::readmatchar (FILE *in)} the actual values of material parameters for given material type. This function
contains {\tt switch} statement where the new {\tt case} for the  {\sf mymatmodel} would be included.
It is possible to take part of {\tt case} statement from another material and copy it into new {\tt case} and then
make some modification. Folowing lines shows example of such a new {\tt case} :\\
{\tt
      if (Mespr==1)  fprintf (stdout,"$\backslash$n number of MYMATMODEL materials \%ld",numtype);\\
      MYMATMODELPT = new MYMATMODEL [numtype];\\
      for (j = 0; j $<$ numtype; j++){\\
        k = numtype + 1;\\
        fscanf (in,"\%ld",\&k);\\
        if ((k $>$ numtype) $\mid\mid$ (k $<$ 1))\\
        {\\
          fprintf (stderr, "$\backslash$n$\backslash$n wrong number of MYMATMODEL material in function\\
                   transmat::readmatchar (\%s, line \%d).$\backslash$n",\_\_FILE\_\_,\_\_LINE\_\_);\\
        }\\
        MYMATMODELPT[k - 1].read (in);\\
      }\\
      break;\\
}

Where user should replace MYMATMODEL with name of the new class {\sf mymatmodel} and MYMATMODELPT with
name of the pointer which has been used in the {\sf transmat} class declaration in the file transmat.hpp. This
statement block the first allocates array with new material then folows loop where the number of sets of the
parameters is read and then the {\sf read} function for the given material type is called.\\

In function {\sf void transmat::matcond (matrix \&d,long ipp,long ri,long ci,long ncomp)}

doplnit upravit???


In function {\sf double transmat::capcoeff (long ipp,long ri,long ci)}


doplnit upravit???


In function {\sf void transmat::computenlfluxes (long lcid,long ipp)}


doplnit upravit???


{\bf threemedia.cpp}\\
\begin{itemize}
\item {\sf double med3::double cond\_kcc (long ipp)}
\item {\sf double med3::double cap\_cc (long ipp)}
\item {\sf double med3::double cond\_kcg (long ipp)}
\item {\sf double med3::double cap\_cg (long ipp)}
\item {\sf double med3::double cond\_kct (long ipp)}
\item {\sf double med3::double cap\_ct (long ipp)}

\item {\sf double med3::double cond\_kgc (long ipp)}
\item {\sf double med3::double cap\_gc (long ipp)}
\item {\sf double med3::double cond\_kgg (long ipp)}
\item {\sf double med3::double cap\_gg (long ipp)}
\item {\sf double med3::double cond\_kgt (long ipp)}
\item {\sf double med3::double cap\_gt (long ipp)}

\item {\sf double med3::double cond\_ktc (long ipp)}
\item {\sf double med3::double cap\_tc (long ipp)}
\item {\sf double med3::double cond\_ktg (long ipp)}
\item {\sf double med3::double cap\_tg (long ipp)}
\item {\sf double med3::double cond\_ktt (long ipp)}
\item {\sf double med3::double cap\_tt (long ipp)}

doplnit prave strany a toky ???

\end{itemize}



In function {\sf double med2::double cond\_kcc (long ipp)}

doplnit upravit???


In function {\sf double med2::double cap\_cc (long ipp)}

doplnit upravit???

In function {\sf double med3::double cond\_kcg (long ipp)}

doplnit upravit???

In function {\sf double med3::double cap\_cg (long ipp)}


doplnit upravit???


In function {\sf double med2::double cond\_kct (long ipp)}

doplnit upravit???

In function {\sf double med2::double cap\_ct (long ipp)}

doplnit upravit???


In function {\sf double med3::double cond\_kgc (long ipp)}

doplnit upravit???

In function {\sf double med3::double cap\_gc (long ipp)}

doplnit upravit???

In function {\sf double med3::double cond\_kgg (long ipp)}

doplnit upravit???

In function {\sf double med3::double cap\_gg (long ipp)}

doplnit upravit???

In function {\sf double med3::double cond\_kgt (long ipp)}

doplnit upravit???

In function {\sf double med3::double cap\_gt (long ipp)}


doplnit upravit???



In function {\sf double med2::double cond\_ktc (long ipp)}

doplnit upravit???

In function {\sf double med2::double cap\_tc (long ipp)}

doplnit upravit???

In function {\sf double med3::double cond\_ktg (long ipp)}

doplnit upravit???

In function {\sf double med3::double cap\_tg (long ipp)}

doplnit upravit???



In function {\sf double med2::double cond\_ktt (long ipp)}

doplnit upravit???

In function {\sf double med2::double cap\_tt (long ipp)}

doplnit upravit???


doplnit prave strany a toky ???


%%%%%%%%%%%%%%%%%%%%%%%%%%%%%%%%%%%%%%%%%%%%%%%%%%%%%%%%%%%%
%%%%%%%%%%%%%%%%%%%%%%%%%%%%%%%%%%%%%%%%%%%%%%%%%%%%%%%%%%%%
\section{Kombinace materialovych modelu}


%%%%%%%%%%%%%%%%%%%%%%%%%%%%%%%%%%%%%%%%%%%%%%%%%%%%%%%%%%%%
%%%%%%%%%%%%%%%%%%%%%%%%%%%%%%%%%%%%%%%%%%%%%%%%%%%%%%%%%%%%
\section{Rozsireni o jedno medium}