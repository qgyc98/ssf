
{\bf Liquid and gas transport}\\

If moisture convection is neglected, the liquid and gas (moist air) transport and 
the vapor diffusion taking place in the gas are the remaining driving mechanisms.
The transport of the vapor phase is governed by Fick's law (\ref{fick})
\begin{eqnarray}\label{fick_2}
\tenss{J}^{gw}_s &=& n(1-S_w) \rho^{gw} \big( \tenss{v} ^{gw} - \tenss{v}^s \big) = 
- \frac{k^{gw}}{\nu^{gw}} {\rm grad} p^{gw}\nonumber\\
&=&-\delta^{gw}(S) {\rm grad} p^{gw}\nonumber\\
&=&-\delta^{gw}(S)\Big[\frac{\partial p^{gw}}{\partial T} {\rm grad} T + \frac{\partial p^{gw}}{\partial S} {\rm grad} 
S \Big].
\end{eqnarray}
The flux of the liquid phase (Darcy's law (\ref{darcy})) is expressed as
\begin{eqnarray}\label{darcy_2}
\tenss{J}_s^{w} &=& n S_w \rho^w \big( \tenss{v} ^{w} - \tenss{v}^s \big) = - \frac{K^w(s)}{g} {\rm grad} p^w\nonumber\\
&=& - \frac{K^w(S)}{g} {\rm grad} (p^{gw} - p^c)\nonumber\\
&=& - \frac{K^w(S)}{g}\Big[\Big(
\frac{\partial p^{gw}}{\partial T} - \frac{\partial p^{c}}{\partial T}\Big) {\rm grad} T
+ \Big(\frac{\partial p^{gw}}{\partial S} - \frac{\partial p^{c}}{\partial S}\Big) {\rm grad} S \Big].
\end{eqnarray}
It remains to calculate the partial derivatives of $p^{gw}$ and $p^c$ with respect to $T$ and $S$.

As $p^{gw} = h p^{gws}$ we have
\begin{eqnarray}
\frac{\partial p^{gw}}{\partial T} = \frac{\partial}{\partial T} \Big(h p^{gws}(T)\Big)\Big|_{h = {\rm const}} 
= h \frac{\partial p^{gws}(T)}{\partial T}.
\end{eqnarray}
Making use of the Kelvin-Laplace law (\ref{kelvinlapl}) yields
\begin{eqnarray}
\frac{\partial p^{c}}{\partial T} = \frac{\partial}{\partial T} \Big[-\frac{\rho^w R T}{M_w}\ln h \Big]_{h = {\rm const}} 
= -\frac{\rho^w R}{M_w}\ln h.
\end{eqnarray}
If $h$ approaches one, $S > S_{\rm irr}$ (II. and III. region) then 
\begin{eqnarray}\label{region}
\frac{\partial p^{gw}}{\partial T} \rightarrow \frac{{\rm d} p^{gws}(T)}{{\rm d} T} 
\qquad {\rm and} \quad \frac{\partial p^{c}}{\partial T}\rightarrow 0.
\end{eqnarray}
\begin{eqnarray}
\frac{{\rm d} w(S)}{{\rm d} S}  &=& \frac{n(\rho^w - \rho^g)}{(1-n)\rho^s} 
\approx \frac{n}{1-n}\frac{(\rho^w - \rho^{gw})}{\rho^s} \doteq {\rm const}.
\end{eqnarray}
Proceeding in a standard way, we find the partial derivative
\begin{eqnarray}
\frac{\partial p^{gw}}{\partial S}\Big|_{T = {\rm const}} &=& 
\frac{\partial p^{gw}}{\partial h} \cdot \frac{\partial h}{\partial w}
\cdot \frac{\partial w}{\partial S}\nonumber\\
&=& \frac{\partial p^{gw}}{\partial w} \frac{n}{1-n}\frac{(\rho^w - \rho^{gw})}{\rho^s}
\end{eqnarray}
where
\begin{eqnarray}
\frac{\partial p^{gw}}{\partial w} = p^{gws}(T) \frac{{\rm d} h(w)}{{\rm d} w}.
\end{eqnarray}
There are two possibilities how to evaluate the partial derivative of $p^c$ with respect to $\varPsi = S, w, h$:
\begin{itemize}
\item{by starting from formula (\ref{Sw}) to get}
\begin{eqnarray}
\frac{\partial p^{c}}{\partial \varPsi} = \frac{\partial p^{c}(S)}{\partial S} \cdot \frac{\partial S}{\partial \varPsi}
\end{eqnarray}
\item{or by exploiting the Kelvin - Laplace equation (\ref{kelvinlapl})}
\end{itemize}
\begin{eqnarray}
\frac{\partial p^{c}}{\partial \varPsi} = \frac{\partial p^{c}}{\partial h}\Big|_{T = {\rm const}} 
\cdot \frac{\partial h}{\partial \varPsi},
\quad {\rm where} \quad \frac{\partial p^{c}}{\partial h}\Big|_{T = {\rm const}} = -\frac{\rho^w R T}{M_w} 
\frac{{\rm d}(\ln h)}{{\rm d} h} = 
- \frac{\rho^w R T}{M_w} \frac{1}{h}
\end{eqnarray}

{\bf Mass balance equation for two-phase zone}\\

The mass balance equation for the liquid phase reads 
\begin{eqnarray}\label{masswater}
\frac{\partial \big( n S_w \rho^w\big)}{\partial t} + {\rm div} \big[n S_w \rho^w \tenss{v}^w\big]
 = -\dot{m}.
\end{eqnarray}
The effect of dry air can be neglected in the transition region for $S > S_{\rm irr}$ (two-phase zone). 
Then $S_{gw} \doteq 1-S_{w} = 1-S$ and the mass balance equation for vapor assumes this form
\begin{eqnarray}\label{massvapor_2}
\frac{\partial \big[n(1- S_w) \rho^{gw}\big]}{\partial t} + {\rm div} \big[n (1-S_w)
\rho^{gw} \tenss{v}^{gw}\big] = \dot{m}.
\end{eqnarray}
By adding Eqs. (\ref{masswater}) and (\ref{massvapor_2}) considering Eq. (\ref{w_S}) and substituting from 
Eqs. (\ref{fick_2}) and (\ref{darcy_2}), 
we arrive at the mass balance equation in two-phase zone of a porous material:
\begin{eqnarray}\label{twophase}
\rho_s \frac{\partial w}{\partial t} = \frac{\partial w(\varPsi)}{\partial \varPsi} 
\cdot \frac{\partial \varPsi}{\partial t} = 
{\rm div} \big( a_{\varPsi T}(\varPsi, T) {\rm grad} T + a_{\varPsi \varPsi}(\varPsi, T){\rm grad} \varPsi \big),
\end{eqnarray}
where $\rho_s = (1 - n)\rho^s$ (see Eq.~\eqref{av_dens}), $\varPsi = S, w, h$, and
\begin{eqnarray}\label{a_phi}
a_{\varPsi T}(\varPsi, T) &=& \Big[\Big( \delta^{gw}(\varPsi) + \frac{K^w(\varPsi)}{g} \Big) 
\frac{\partial p^{gw}}{\partial T} - \frac{K^w(\varPsi)}{g} \frac{\partial p^{c}}{\partial T} \Big]_{\varPsi = 
{\rm const}}\nonumber\\
a_{\varPsi \varPsi}(\varPsi, T) &=& \Big[\Big( \delta^{gw}(\varPsi) + \frac{K^w(\varPsi)}{g} \Big) 
\frac{\partial p^{gw}}{\partial \varPsi} - \frac{K^w(\varPsi)}{g} \frac{\partial p^{c}}{\partial \varPsi} \Big]_{T = 
{\rm const}}\quad .
\end{eqnarray}
If water vapor diffusion is the only driving mechanism, the preceding equations (\ref{a_phi}) simplify as
\begin{eqnarray}
a_{\varPsi T}(\varPsi, T) &=& \delta^{gw}(\varPsi) \frac{\partial p^{gw}}{\partial T}\Big|_{\varPsi = 
{\rm const}}\nonumber\\
a_{\varPsi \varPsi}(\varPsi, T) &=& \delta^{gw}(\varPsi) \frac{\partial p^{gw}}{\partial \varPsi} \Big|_{T = 
{\rm const}}\quad .
\end{eqnarray}
In the two-phase zone $S$ is always greater than $S_{\rm irr}$ (i.e. $h > 0.9$) and, in addition to Eqs. (\ref{region}), 
$\partial p^{gw}/\partial S \rightarrow 0$.\\

{\bf Energy balance equation}\\

The complete set of equations describing the coupled moisture and heat transport in porous media comprises 
the energy balance equation and the continuity equations for the liquid water and gas. 
The {\it energy balance equation} has already been derived in Paragraph~\ref{keynote} (Eq. (\ref{heat})).

The effect of evaporation (latent heat) is represented by the last term appearing on the right-hand 
side of Eq. (\ref{heat}). Neglecting convective terms and assuming rigid porous matrix yields 
the resulting form of the equation of energy conservation:
\begin{eqnarray}\label{enrg_con_first}
( \rho C)_{\rm eff} \frac{\partial T}{\partial t} &=& {\rm div} \Big\{ \Big(\chi_{\rm eff} (S, 
T) + \Delta h_{\rm vap} \delta^{gw}(S) \frac{\partial p^{gw}}{\partial T} \Big) {\rm grad} T +\\ 
& &\qquad \quad + \Delta h_{\rm vap} \delta^{gw}(S) \frac{\partial p^{gw}}{\partial S} {\rm grad} S \Big\} 
- \frac{\partial}{\partial t}
\Big[\Delta h_{\rm vap} n(1-S) \rho^{gw} \Big].\nonumber
\end{eqnarray}
Since $h = \rho^{gw}/\rho^{gws}$, the source term takes this form
\begin{eqnarray}
\frac{\partial}{\partial t} \Big[\Delta h_{\rm vap} n(1-S) \rho^{gw} \Big] &=& \Delta h_{\rm vap} n h \rho^{gws} 
\frac{\partial S}{\partial t} + \Delta h_{\rm vap} n (1 - S)\rho^{gws} \frac{\partial h}{\partial t}\\
&=& \Delta h_{\rm vap} n \rho^{gws} \Big(h + (1-S)\frac{\partial h}{\partial w} \frac{\partial w}{\partial S} \Big) 
\frac{\partial S}{\partial t} = \Delta h_{\rm vap} b(S) \frac{\partial S}{\partial t}\nonumber.
\end{eqnarray}
When dealing with the coupled moisture and heat transfer, three quantities can be used to describe the moisture field. 
Denote them by $\varPsi$; 
$\varPsi = S, w, h$. 
Eq. (\ref{enrg_con_first}) can be then rewritten into this contracted form
\begin{eqnarray}
( \rho C)_{\rm eff} \frac{\partial T}{\partial t} = {\rm div} \big\{ a_{T T} (T, \Psi) {\rm grad} T + 
a_{T \Psi} (T, \Psi) {\rm grad} \Psi \big\} - \Delta h_{\rm vap} b(\Psi)\frac{\partial \Psi}{\partial t},
\end{eqnarray}
\begin{eqnarray}
a_{T T} (T, \Psi) &=& \chi_{\rm eff} (\Psi,T) + \Delta h_{\rm vap} \delta^{gw}(\Psi) \frac{\partial p^{gw}}{\partial T},
\nonumber\\
a_{T S} (T, \Psi) &=& \Delta h_{\rm vap} \delta^{gw}(\Psi) \frac{\partial p^{gw}}{\partial \Psi}.\nonumber
\end{eqnarray}

{\bf FEM formulation for coupled moisture and heat transfer}\\

The balance equation for moisture transfer $(\Psi = w(\tenss{x},t))$ reads.
\begin{eqnarray}\label{couple}
\rho_s \frac{\partial w}{\partial t} &=& - {\rm div} \big(\tenss{J}_s^{gw}+ \tenss{J}_s^w\big) \nonumber\\
&=& {\rm div} \Big\{ \delta^{gw}(w) \Big[ \frac{\partial p^{gw}}{\partial T} {\rm grad} T 
+ \frac{\partial p^{gw}}{\partial w} {\rm grad} w \Big] + \nonumber\\ &+& \frac{K^w(w)}{g} \Big[\Big(
\frac{\partial p^{gw}}{\partial T} - \frac{\partial p^{c}}{\partial T}\Big) {\rm grad} T
+ \Big(\frac{\partial p^{gw}}{\partial w} - \frac{\partial p^{c}}{\partial w}\Big) {\rm grad} w \Big]\Big\}.
\end{eqnarray}
The fluid is transferred across interfaces with the surrounding environment by means of convection 
fluxes, $\tenss{\nu}^T\tenss{J}_c^{\alpha} \dots$ $\alpha = w, gw$. This phenomenon can be expressed 
by the boundary conditions pertaining to Eq. (\ref{couple}):
\begin{itemize}
\item{either given values (an essential condition)}
\begin{eqnarray}
w=\overline{w} \qquad {\rm on} \quad \Gamma_w = \Gamma_{1w}
\end{eqnarray}
\item{or imposed fluxes (natural boundary conditions)}
\begin{eqnarray}
\tenss{\nu}^T \tenss{J}_c^{gw} = - \tenss{\nu}^T {\delta}^{gw}(w) \Big[ \frac{\partial 
p^{gw}}{\partial T} {\rm grad} T + \frac{\partial p^{gw}}{\partial w} {\rm grad} w \Big]
= \overline{q}^{gw} \quad \Gamma^{gw}
\end{eqnarray}
\end{itemize}
and
\begin{eqnarray}
\tenss{\nu}^T \tenss{J}_c^{w} = - \tenss{\nu}^T \frac{K^w(w)}{g} \Big[\Big(
\frac{\partial p^{gw}}{\partial T} - \frac{\partial p^{c}}{\partial T}\Big) {\rm grad} T
+ \Big(\frac{\partial p^{gw}}{\partial w} - \frac{\partial p^{c}}{\partial w}\Big) {\rm grad} w \Big]
= \overline{q}^{w} \quad {\rm on} \quad \Gamma^{w}\nonumber.
\end{eqnarray}
\begin{eqnarray}
{\Gamma^{gw} \cup \Gamma^w = \Gamma_{2w}}\nonumber
\end{eqnarray}	
The water vapor flux prescribed in the direction of the outward normal $\tenss{\nu}$ 
on $\Gamma^{gw}$ 
is obtained from the following formula \cite{ctu}
\begin{eqnarray}
\overline{q}^{gw} = \beta^{gw} (p_{\rm surf}^{gw} - p_{\rm air}).
\end{eqnarray}

The equation of energy conservation is of the form
\begin{eqnarray}\label{enrg_con}
( \rho C)_{\rm eff} \frac{\partial T}{\partial t} &=& {\rm div} \Big\{ \Big(\chi_{\rm eff} (w, 
T) +  \Delta h_{\rm vap} \delta^{gw}(w) \frac{\partial p^{gw}}{\partial T} \Big) {\rm grad} T +\\
&& \qquad  +\Delta h_{\rm vap} \delta^{gw}(w) \frac{\partial p^{gw}}{\partial w} {\rm grad} w \Big\} +  \Delta h_{\rm vap} b(w) 
\frac{\partial}{\partial t}\nonumber
\end{eqnarray}
where
\begin{eqnarray}
 \Delta h_{\rm vap} b(w) =  \Delta h_{\rm vap} n \rho^{gws} \Big( h(w) + (1-S)\frac{{\rm d} h}{{\rm d} w}\Big )
\overline{q}^{gw} = \beta^{gw} (p_{\rm surf}^{gw} - p_{\rm air}).
\end{eqnarray}
The boundary conditions belonging to Eq. (\ref{enrg_con}) are
\begin{itemize}
\item{either the given values}
\begin{eqnarray}
T = \overline{T} \qquad {\rm on} \quad \Gamma_{T} = \Gamma_{1 T}
\end{eqnarray}
\item{or imposed flux. It is given as the sum of three individual fluxes - a convection exchange 
of heat between outer surfaces and the air, the absorption of a fraction of the sun radiation 
heat, and the loss of heat due to latent heat of moisture vaporization. The result reads}
\end{itemize}
\begin{eqnarray}
\tenss{\nu}^T q = - \tenss{\nu}^T \chi_{\rm eff}(w,T) {\rm grad} T  = 
\big(\beta_c + \beta_r(T)\big) \big(T_{\rm surf} - T_{\rm air}\big) + \beta_c \tenss{\nu}^T 
\tenss{J}_c^{gw}= \overline{q}_{T}\nonumber\\
{\rm on} \quad \Gamma^{T} = \Gamma_{2 T}.
\end{eqnarray}

{\bf Numerical solution}\\

Applying Galerkin's method (see Paragraph~\ref{secnonstationary}) we obtain a system of equations 
for coupled moisture and heat transfer:
\begin{eqnarray}\label{final_eq}
\left[ \begin{array}{cc}
\tenss{K}_{ww} & \tenss{K}_{wT} \\
\tenss{K}_{Tw} & \tenss{K}_{TT}
\end{array} \right]
\left\{ \begin{array}{c}
\tenss{r}_w \\
\tenss{r}_{T}
\end{array} \right\} + 
\left[ \begin{array}{cc}
\tenss{C}_{ww} & \tenss{C}_{wT} \\
\tenss{C}_{Tw} & \tenss{C}_{TT}
\end{array} \right]
\left\{ \begin{array}{c}
\dot{\tenss{r}}_w \\
\dot{\tenss{r}}_{T}
\end{array} \right\} = 
\left\{ \begin{array}{c}
\tenss{q}_{w} + \tenss{q}_{gw} \\
\tenss{q}_{T}
\end{array} \right\},
\end{eqnarray}

where
\begin{eqnarray}
\tenss{K}_{ww} = \int_{\Omega} \tenss{B}^T k_{ww} \tenss{B}{\rm d}\Omega,
\qquad \tenss{K}_{wT} = \int_{\Omega} \tenss{B}^T k_{wT} \tenss{B}{\rm d}\Omega,
\end{eqnarray}
\begin{eqnarray}
\tenss{K}_{Tw} = \int_{\Omega} \tenss{B}^T k_{Tw} \tenss{B}{\rm d}\Omega,
\qquad \tenss{K}_{TT} = \int_{\Omega} \tenss{B}^T k_{TT} \tenss{B}{\rm d}\Omega,
\end{eqnarray}
\begin{eqnarray}
\tenss{C}_{ww} = \int_{\Omega} \tenss{N}^T c_{ww}  \tenss{N} {\rm d}\Omega,\qquad
\tenss{C}_{wT} = \int_{\Omega} \tenss{N}^T c_{wT}  \tenss{N} {\rm d}\Omega,\qquad
\end{eqnarray}
\begin{eqnarray}
\tenss{C}_{Tw} = \int_{\Omega} \tenss{N}^T c_{Tw}  \tenss{N} {\rm d}\Omega,\qquad
\tenss{C}_{TT} = \int_{\Omega} \tenss{N}^T c_{TT}  \tenss{N} {\rm d}\Omega,\qquad
\end{eqnarray}
\begin{eqnarray}
\tenss{q}_w = \int_{\Gamma_2} \tenss{N}^T  \overline{q}^{w}{\rm d}\Gamma,\qquad
\tenss{q}_{gw} = \int_{\Gamma_2} \tenss{N}^T  \overline{q}^{gw}{\rm d}\Gamma,\qquad
\tenss{q}_T = \int_{\Gamma_2} \tenss{N}^T  \overline{q}_{T}{\rm d}\Gamma,\qquad
\end{eqnarray}

where
\begin{eqnarray}
 k_{ww} &=& \delta^{gw}(w)\frac{\partial p^{gw}}{\partial w} + \frac{K^w(w)}{g}\Big(\frac{\partial p^{gw}}{\partial w} - \frac{\partial p^{c}}{\partial w}\Big),\\
 k_{wT} &=& \delta^{gw}(w)\frac{\partial p^{gw}}{\partial T} + \frac{K^w(w)}{g}\Big(\frac{\partial p^{gw}}{\partial T} - \frac{\partial p^{c}}{\partial T}\Big),
\end{eqnarray}
\begin{eqnarray}
 k_{Tw} &=& \Delta h_{\rm vap} \delta^{gw}(w) \frac{\partial p^{gw}}{\partial w},\\
 k_{TT} &=& \chi_{\rm eff} (w,T) +  \Delta h_{\rm vap} \delta^{gw}(w) \frac{\partial p^{gw}}{\partial T},
\end{eqnarray}
\begin{eqnarray}
 c_{ww} = \rho_s,
\qquad c_{wT} = 0,
\end{eqnarray}
\begin{eqnarray}
 c_{Tw} = \Delta h_{\rm vap} n \rho^{gws} \Big( h(w) + (1-S)\frac{{\rm d} h}{{\rm d} w}\Big)\overline{q}^{gw} ,
\qquad c_{TT} = (\rho C)_{\rm eff},
\end{eqnarray}
