%%%%%%%%%%%%%%%%%%%%%%%%%%%%%%%%%%%%%%%%%%%%%%%%%%%%%%%%%%%%%%%%%%%%%%%%%%%%%%%%%%%%%%%%%%%%%%%%%%%%%
%%%%%%%%%%%%%%%%%%%%%%%%%%%%%%%%%%%%%%%%%%%%%%%%%%%%%%%%%%%%%%%%%%%%%%%%%%%%%%%%%%%%%%%%%%%%%%%%%%%%%
\chapter{Description of the code}
%%%%%%%%%%%%%%%%%%%%%%%%%%%%%%%%%%%%%%%%%%%%%%%%%%%%%%%%%%%%%%%%%%%%%%%%%%%%%%%%%%%%%%%%%%%%%%%%%%%%%%%
%%%%%%%%%%%%%%%%%%%%%%%%%%%%%%%%%%%%%%%%%%%%%%%%%%%%%%%%%%%%%%%%%%%%%%%%%%%%%%%%%%%%%%%%%%%%%%%%%%%

\section{Allocation and Initiation}

The METR starts with function metr\_init, where all data are read
from input file or files and many of objects and arrays are allocated.

The main objects of the code are allocated at the beginning
of the function metr\_init.
\begin{center}
\begin{tabular}{ll}
Cp = new probdescc;  & problem description
\\
Mp  = new probdesc;  & problem description
\\
Tp  = new probdesct;  & problem description
\\
Gtm = new gtopology; & general topology for mechanics
\\
Gtt = new gtopology; & general topology for transport
\\
Gtu = new gtopology; & general topology for unified mesh
\\
Mt  = new mechtop; & mechanical topology
\\
Tt  = new transtop; & transport topology
\\
Ct = new couptop; & coupled topology
\\
Mm  = new mechmat; & mechamical materials
\\
Tm  = new transmat; & transport materials
\\
Cmu = new coupmatu; & coupled materials
\\
Cml = new coupmatl; & coupled materials
\\
Mc  = new mechcrsec; & mechanical cross sections
\\
Tc  = new transcrsec; & transport cross sections
\\
Cb = new coupbclc; & coupled cross sections
\\
Mb  = new mechbclc; & mechanical boundary conditions
\\
Tb  = new transbclc; & transport boundary conditions
\\
Outdm = new outdriverm; & description of output
\\
Outdt = new outdrivert; & description of output
\\
Outdc = new outdriverc; & description of output
\end{tabular}
\end{center}

\begin{itemize}
\item
Cp--$>$read (in); - reads coupled problem description

\item
Mp--$>$read (in); - reads mechanical problem description

\item
Tp--$>$read (in); -  reads transport problem description


\item
Mt--$>$read (in); - reads mechanical topology
\item
Tt--$>$read (in); -  reads transport topology
\item
Ct--$>$read (in); -  reads coupled topology


\item
Ndofm = Gtm--$>$codenum\_generation (Out); - generates code numbers for mechanical problem

\item
Ndoft = Gtt--$>$codenum\_generation (Outt); - generates code numbers for transport problem

\item
Mm--$>$read (in); - reads mechanical materials

\item
Tm--$>$read (in); - reads transport materials

\item
Tm--$>$intpointalloc (); - allocates array of integration points

\item
Tm--$>$intpointinit (); - initiates variables on integration points,
it defines number of blocks, number of components, type of materials and material id

\item
Mm--$>$init\_ip\_1 (); - initiates variables on integration points, namely, stress/strain state,
number of strain/stress components

\item
Mc--$>$read (in); - reads mechanical cross sections

\item
Tc--$>$read (in); - reads transport cross sections

\item
Mb--$>$read (in); - reads mechanical boundary conditions, it means
prescribed displacements, forces, moments

\item
Tb--$>$read (in); - reads transport boundary conditions, it means
Dirichlet, Neumann and Newton

\item
Tt--$>$elemprescvalue(); - determines elements influenced by prescribed
values, it is used for assembling of nodal fluxes

\item
Tb--$>$elemsource (); - determines elements with sources


\item
Lsrs--$>$alloc (); - allocates arrays lhs (array with solution),
rhs (array with right hand side of the system of equations),
w (array containing eigenvalues)
 
\item
Lsrs--$>$initcond (in); - reads initial conditions

\item
Lsrst--$>$alloc (); - allocates arrays lhs (array with solution),
rhs (array with right hand side of the system of equations),

\item
Lsrst--$>$initcond (in); - reads initial conditions

\item
Outdc--$>$read(in); - reads description of output from the code, namely description
of output to the output file, file for graphical postprocessors (e.g. GiD) and
diagrams

\item
Outdm--$>$read(in); - reads description of output from the code, namely description
of output to the output file, file for graphical postprocessors (e.g. GiD) and
diagrams

\item
Outdt--$>$read(in); - reads description of output from the code, namely description
of output to the output file, file for graphical postprocessors (e.g. GiD) and
diagrams

\item
Mm--$>$init\_ip\_2 (); - allocates arrays on integration points, namely array
for strains, stresses, other values, nonlocal values;
function also allocates the array tempr for storage of temperatures

\item
Mt--$>$alloc\_nodes\_arrays (); - allocates arrays on nodes for
strains, stresses, other values

\item
Tt--$>$alloc\_nodes (); - allocates arrays grad and flux

\item
Mt--$>$give\_maxncompstr(Mm--$>$max\_ncompstrn, Mm--$>$max\_ncompstre); - determines
maximum number of strain/stress components on integration points
it is used for output

\item
Mt--$>$give\_maxncompo(Mm--$>$max\_nncompo, Mm--$>$max\_encompo); - determines
maximum number of components in array other on integration points
it is used for output

\item
Mt--$>$elemprescdisp (); - determines elements influenced by prescribed
displacements, it is used for assembling of nodal forces

\item
Mm--$>$alloceigstrains (); - allocates array eigstrains 

\item
Mm--$>$alloctempstrains (); - allocates array tempstrains

\end{itemize}

%%%%%%%%%%%%%%%%%%%%%%%%%%%%%%%%%%%%%%%%%%%%%%%%%%%%%%%%%%%%%%%
%%%%%%%%%%%%%%%%%%%%%%%%%%%%%%%%%%%%%%%%%%%%%%%%%%%%%%%%%%%%%%%
%%%%%%%%%%%%%%%%%%%%%%%%%%%%%%%%%%%%%%%%%%%%%%%%%%%%%%%%%%%%%%%
%%%%%%%%%%%%%%%%%%%%%%%%%%%%%%%%%%%%%%%%%%%%%%%%%%%%%%%%%%%%%%%
\section{data transfer between TRFEL and MEFEL}
Transfer on the solver level is provided by copy_data function
where two principal procedures are called:
\begin{itemize}
\item {\tt mefel\_trfel()} - transfer of mechanical (non-transport) quantities from MEFEL to TRFEL
\item {\tt trfel\_mefel()} - transfer of transport (non-mechanical) quantities from TRFEL to MEFEL 
\end{itemize}
The names (aliases) of transfered quantities are given in GEFEL header file by
{\tt galias.h} enumerated types {\tt nonmechquant} and {\tt nontransquant}.
They must be stored in GEFEL in order to preserve access to these data types both from 
TRFEL and MEFEL.


\subsection {{\tt mefel\_trfel()}}
There is a loop over all non-transport quantities required by TRFEL whose number is given
by {\tt transmat::nntq} and names (aliases) are defined by {\tt transmat::ntqo}.
In this loop, the nodal values of required non-transport (mechanical) quantity are obtained for 
each TRFEL element by the call of MEFELs {\tt elem\_mechq\_nodval}. The nodal values
of required quantity are on the given element recalculated into integration points by
TRFEL {\tt elem\_intpointvalt} and int. point values are stored in {\tt transmat::nontranq} array
with help of {\tt mehcmat::storenontransq}.

Scheme of {\tt mefel\_trfel()}
{\tt
for(i=0; i<nntq; i++)
{
  for(j=0; j<Tt->ne; j++)
  {
    // j je index prvku TRFELovske site => MEFEL musi mit stejnou sit (max. jine nasady)
    elem_mechq_nodval(j, nodval, ntqo[i]) // elemswitch.cpp  MEFEL
      Element[j]::mechq_nodval(j, nodval, ntqo[i])
        for(k=0; k<Element[j]::nne; k++)
          mechmat::givemechq(ntqo[i], ipp)
 
    elem_intpointvalt (j, nodval, ipval) // elemswitcht.cpp TRFEL
      Elementt[j]::intpointval(j, nodval, ipval)
        #### linbart::intpointval - k cemu je cyklus k ???

    transmat::storenontransq(ntqo[i], ipp, ipval);
      transmat::nontransq[transmat::ntqid[transmat::ntqo[i]-1]][ipp] = ipval
  }
}
}

Scheme of {\tt trfel\_mefel()}
{\tt
for(i=0; i<nnmq; i++)
{
  // vytahne danou neznamou z TRFELu na vsech uzlech
  // gnodv ma mechtop::nn slozek => TRFEL musi mit stejne uzly jako MEFEL
  nodal_nodal_values(gnodv, nmqo[i])
  {
    nnt = transtop::nnt
    give_transq_nodval(gnodv, nnt, nmqo[i]) // elemswitcht.cpp TRFEL
    {
      for(j=0; j<Tt->ne; j++)
      {
        // j je index prvku MEFELovske site => TRFEL musi mit stejnou sit (max. jine nasady)
        elem_transq_nodval(j, nodval, nmqo[i]) // elemswitcht.cpp TRFEL
          Elementt[j]::transq_nodval(j, nodval, nmqo[i])
            for(k=0; k<Elementt[j]::nne; k++)
              nodval[j] = transmat::givetransq(nmqo[i], ipp)
            
        for(k=0; k<Elementt[j]::nne; k++)
          gnodval[Elementt::nodes[j]] = nodval[k]
      }
    }
  }

  if (Cp->lbb==quad_lin)
    intpointval2 (gnodv, nmqo[i]); // elemswitch.cpp MEFEL
  else
    intpointval (gnodv, nmqo[i], 1.0); // elemswitch.cpp MEFEL

}
}




\section{Partially coupled problems}

\subsection{mechanical part}

\begin{itemize}
\item
approximation\_temper (); - (METR/SRC/globmatc.cpp) - approximates nodal temperatures computed in TRFEL
to integration points of MEFEL, temperatures are stored in array Mm-$>$tempr
\item
nodal_nodal_temper_values (gv);
\item
intpointval () or intpointval2 () - (MEFEL/SRC/elemswitch.cpp) - calls 
elem\_intpointval (MEFEL/SRC/elemswitch.cpp) calls Pelq--$>$intpointval (eid,nodval,ipval) (MEFEL/SRC/plelemlq.cpp)
\end{itemize}

humidity is approximated in a similar manner

Mm--$>$updateipval(); - function updates values stored at integration
points, usually it copies array other to the array eqother, in the case
of creep models, it computes some necessary data

mefel_right_hand_side (lcid,f); - computes the right hand side
in this approach, it calls function Mb->dlc[lcid].assemble (lcid,rhs+lcid*Ndofm,Ndofm,Mp->time);
This function uses one of the implemented version of load definition. The load is defined
by time independent load case (loadcase.cpp), where all components are multiplied by
scale factor which depends on time. The load can be defined also by fully time
dependent load case (dloadcase.cpp), where particular components have its own time
function

If the loadcase.cpp is used, it assembles contributions from nodal load, element load,
prescribed displacements and defined temperature changes. Contributions from temperature are
calculated by the function loadcase::tempercontrib (long lcid,double *rhs,double scale)

\item
Mm->temprstrains (lcid) (MEFEL/SRC/loadcase.cpp) - computes temperature
strains, in the case of time independent problem, it computes directly
strain caused by given temperature increment (it is read from input
file). In the case of time
dependent problems, it computes strains caused by temperature
increment which is calculated as a difference of actual temperature
and temperature at the beginning of computation. Increment
of forces caused by temperature is computed in solver,
where prescribed forces from previous time step are subtracted
from actual prescribed forces. Function stores temperature
strains in array Mm--$>$tempstrains.






\end{itemize}