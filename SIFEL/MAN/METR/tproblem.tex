\documentclass[12pt]{book}
\newcommand{\mbf}[1]{\mbox{\boldmath$#1$}}
% definice zlomku
\newcommand{\del}[2]{\mbox{$\displaystyle\frac{#1}{#2}$}}
% definice prvni parcialni derivace funkce
\newcommand{\ppd}[2]{\del{\partial{#1}}{\partial{#2}}}
% definice druhe parcialni derivace funkce podle x a y
\newcommand{\dpd}[3]{\del{\partial^{2}\>\!\!{#1}}{\partial{#2}\ \partial{#3}}}
% definice n-te parcialni derivace funkce
\newcommand{\npd}[3]{\del{\partial^{#1}\>\!\!{#2}}{\partial{#3}^{#1}}}
% definice prvni derivace funkce
\newcommand{\od}[2]{\del{{\rm d}\>\!{#1}}{{\rm d}{#2}}}
% definice n-te obycejne derivace funkce
\newcommand{\nod}[3]{\del{{\rm d}^{#1}\>\!\!{#2}}{{\rm d}{#3}^{#1}}}
\usepackage{makeidx}
\makeindex

\setcounter{secnumdepth}{4}
\setcounter{totalnumber}{4}
\oddsidemargin=5mm
\evensidemargin=5mm
\textwidth=160mm
\topmargin=-2.0cm
\textheight=250mm
\headheight=1.5cm


\begin{document}
\tableofcontents
%\chapter{Numerical solution of nonlinear nonstationary problems}
\chapter{Numerical solution}
General discretized problem of hydro-thermo-mechanical coupling is described by equation
\begin{equation}
\mbf{C}(\mbf{d}) \od{\mbf{d}}{t} + \mbf{K}(\mbf{d}) \mbf{d} = \mbf{F}(\mbf{d}) \ .
\end{equation}
The nonlinearity is caused by the dependency of vector $\mbf{F}$ and matrices $\mbf{C}$ and $\mbf{K}$
on vector of unknowns $\mbf{d}$. First step consists in time discretization by the backward Euler
formula which leads to form
\begin{equation}\label{genprobdiscr}
\mbf{C}(\mbf{d}_{n+1}) \left(\mbf{d}_{n+1} - \mbf{d}_{n}\right)\del{1}{\Delta t} +
\mbf{K}(\mbf{d}_{n+1}) \mbf{d}_{n+1} = \mbf{F}(\mbf{d}_{n+1}) \ ,
\end{equation}
where subscript $_n$ stands for time $t=t_n$ in which variable is evaluated.
Thanks to nonlinear feature of the Eqn. (\ref{genprobdiscr}), standard methods for nonstationary problems
cannot be used because it is not possible to satisfy the equality directly. Therefore any iterative
method for nonlinear problems must be employed during each time step $\Delta$. Sequence of approximations
$\mbf{d}_{n+1}^{m}$ is calculated where superscript $^m$ indicates number of iteration.
Suitable method is Newton-Raphson method which will described further. Important role in iterative methods
plays residuum or vector of residuals. The vector of residuals of problem (\ref{genprobdiscr}) can be written
\begin{equation}\label{defresiduum}
\mbf{R}(\mbf{d}_{n+1}^m) = \mbf{C}(\mbf{d}_{n+1}^m) \left(\mbf{d}_{n+1}^m - \mbf{d}_{n}\right)\del{1}{\Delta t} +
\mbf{K}(\mbf{d}_{n+1}^m) \mbf{d}_{n+1}^m - \mbf{F}(\mbf{d}_{n+1}^m) \ .
\end{equation}
The iteration process is stopped when a suitable norm of residuum is less than required tolerance.
New approximation $\mbf{d}_{n+1}^{m+1}$ is calculated from known expression
\begin{equation}
\mbf{R}(\mbf{d}_{n+1}^{m+1}) = \mbf{0} = \mbf{R}(\mbf{d}_{n+1}^{m}) + \od{\mbf{R}(\mbf{d}_{n+1}^{m})}{\mbf{d}_{n+1}^{m}}
\left(\mbf{d}_{n+1}^{m+1} - \mbf{d}_{n+1}^{m}\right)
\end{equation}
in the form
\begin{equation}\label{gennewton}
\mbf{d}_{n+1}^{m+1} = \mbf{d}_{n+1}^{m} - \left(\od{\mbf{R}(\mbf{d}_{n+1}^{m})}{\mbf{d}_{n+1}^{m}}\right)^{-1}
\mbf{R}(\mbf{d}_{n+1}^{m})\ .
\end{equation}

Equation (\ref{gennewton}) is clear but the evaluation of the term
$\left(\od{\mbf{R}(\mbf{d}_{n+1}^{m})}{\mbf{d}_{n+1}^{m}}\right)^{-1}$ is not so easy and it is usually very demanding.
Matrix notation is not always suitable for detail description and therefore also indices will be used.
The $j$-th component of the vector $\mbf{d}_{n+1}^{m}$ is denoted as $d_{n+1,j}^{m}$.

With the help of indices the vector of residuals has form
\begin{equation}
R_{i}(d_{n+1,j}^{m}) = C_{ik}(d_{n+1,j}^{m}) \left(d_{n+1,k}^{m} - d_{n,k}\right)\del{1}{\Delta t} +
K_{ik}(d_{n+1,j}^{m}) d_{n+1,k}^{m} - F_i(d_{n+1,j}^{m}) \ .
\end{equation}
Derivative of the vector of residuals with respect to the vector of the $m$-th approximation is
\begin{eqnarray}\label{derres}
\ppd{R_{i}(d_{n+1,j}^{m})}{d_{n+1,l}^{m}} &=&
\ppd{C_{ik}(d_{n+1,j}^{m})}{d_{n+1,l}^{m}} \left(d_{n+1,k}^{m} - d_{n,k}\right)\del{1}{\Delta t} +
C_{ik}(d_{n+1,j}^{m}) \delta_{kl}\del{1}{\Delta t} +
\\ \nonumber
&+&
\ppd{K_{ik}(d_{n+1,j}^{m})}{d_{n+1,l}^{m}} d_{n+1,k}^{m} + K_{ik}(d_{n+1,j}^{m}) \delta_{kl} - 
\ppd{F_i(d_{n+1,j}^{m})}{d_{n+1,l}^{m}} \ .
\end{eqnarray}
Evaluation of terms $C_{ik}(d_{n+1,j}^{m}) \delta_{kl}\del{1}{\Delta t}$ and $K_{ik}(d_{n+1,j}^{m}) \delta_{kl}$
from Equation (\ref{derres}) is obvious. The remaining terms deserve attention and will be analysed in the following
text.

The subscripts $n+1$ denoting the time interval and the superscripts $m$ denoting the number of iteration
will be omitted in the following analysis. Subscript will indicate components of matrices and vectors.
Let start with the derivative of conductivity matrix with respect to vector of unknowns. The conductivity matrix
is the sum of contributions from elements and can be written as
\begin{equation}
K_{ik}(d_{j}) = \sum_{\alpha=1}^{ne} L_{is}^{\alpha} K_{sr}^{\alpha}(d_{j}) L_{kr}^{\alpha}\ ,\ \ \ \ \
\mbf{K}(\mbf{d}) = \sum_{\alpha=1}^{ne} \mbf{L}^{\alpha} \mbf{K}^{\alpha}(\mbf{d}) \left(\mbf{L}^{\alpha}\right)^T\ ,
\end{equation}
where $ne$ stands for the number of finite elements in the mesh, $L_{is}^{\alpha}$ and $\mbf{L}^{\alpha}$ denote the
localization matrix of the $\alpha$-th element and $K_{sr}^{\alpha}$ and $\mbf{K}^{\alpha}$ are the conductivity
matrix of the $\alpha$-th element.
Localization matrices $L_{is}^{\alpha}$ do not depend on unknown
components $d_{j}$ and therefore the derivative of conductivity matrix with respect to $d_{l}$ has form
\begin{eqnarray}
\ppd{K_{ik}(d_{j})}{d_{l}} = \sum_{\alpha=1}^{ne} L_{is}^{\alpha} \ppd{K_{sr}^{\alpha}(d_{j})}{d_{l}} L_{kr}^{\alpha}\ ,
\end{eqnarray}
Conductivity matrix of one finite element is defined as
\begin{equation}
K_{sr} = \sum_{t=1}^{nip} B_{ps}(\mbf{\xi}_{t}) D_{pq}(\mbf{d}) B_{qr}(\mbf{\xi}_{t})
\end{equation}
Therefore the derivative with respect to vector of nodal values only concerns with the matrix $D_{pq}(\mbf{d})$.
It is clear that
\begin{eqnarray}
\ppd{K_{sr}^{\alpha}(d_{j})}{d_{l}} = 0
\end{eqnarray}
if the nodal value $d_{l}$ is not connected to the $\alpha$-th element. The derivative is nonzero only if
the nodal value $d_{l}$ is connected to the $\alpha$-th element. With help of Equation (\ref{defderapprfun})
the derivative
\begin{eqnarray}\label{dercondmatelem}
\ppd{K_{sr}(d_u)}{d_l} &=& \ppd{K_{sr}(d_u)}{d_v^{(w)}} =
\sum_{t=1}^{nip} B_{ps}(\mbf{\xi}_{t}) \ppd{D_{pq}(\mbf{d},\ \mbf{\xi}_{t})}{d_v^{(w)}} B_{qr}(\mbf{\xi}_{t}) =
\\ \nonumber &=&
\sum_{t=1}^{nip} B_{ps}(\mbf{\xi}_{t}) \ppd{D_{pq}(\mbf{d},\ \mbf{\xi}_{t})}{f^{(w)}} N^{(w)}_{v}(\mbf{\xi}_{t}) B_{qr}(\mbf{\xi}_{t})\ ,
\end{eqnarray}
where relation (\ref{indcompnodval}) has been applied. Bounds are mentioned once again for better clarity. Subscript $l$ is
from the set $\{1,\ \ldots,\ n\}$ while superscript $w$ is from the set $\{1,\ \ldots,\ n_f\}$ and subscript $v$ is from
the set $\{1,\ \ldots,\ n^{(w)}\}$. Number of combinations of $w$ and $v$ is equal to $n$.
Required term has finally form
\begin{eqnarray}
\ppd{K_{sr}(d_u)}{d_l}\ d_r =
\left(\sum_{t=1}^{nip} B_{ps}(\mbf{\xi}_{t}) \ppd{D_{pq}(\mbf{d},\ \mbf{\xi}_{t})}{f^{(w)}}
N^{(w)}_{v}(\mbf{\xi}_{t}) B_{qr}(\mbf{\xi}_{t})\right)\ d_r
\end{eqnarray}
which expresses matrix with indices $s$ and $l$. Other indices are summation indices and they disappear.
For fixed value of $l$ the derivative from Equation (\ref{dercondmatelem}) is matrix of same order
(same number of rows and columns) as the conductivity matrix. There are $n$ such matrices and they will be denoted
$\mbf{K}_{l}$. With this notation is possible to write contribution from one finite element as a matrix
\begin{equation}
\left(\mbf{K}_{1} \mbf{d},\ \mbf{K}_{2} \mbf{d},\ \ldots,\ \mbf{K}_{n} \mbf{d}\right)\ ,
\end{equation}
where columns are created by matrix-vector multiplications.

Newton-Rapson algorithm starts at time $t_n$ when all fluxes are in equilibrium and therefore equation
\begin{equation}
\mbf{C}(\mbf{d}_{n}) \od{\mbf{d_{n}}}{t} + \mbf{K}(\mbf{d}_{n}) \mbf{d}_{n} = \mbf{F}(\mbf{d}_{n})
\end{equation}
is valid. The vector of nodal unknowns $\mbf{d}$ is not known at time $t_{n+1}$ and trial solution
is computed from relation
\begin{equation}
\mbf{C}(\mbf{d}_{n}) (\mbf{d}_{n+1} - \mbf{d}_{n})\del{1}{\Delta t} + \mbf{K}(\mbf{d}_{n}) \mbf{d}_{n+1} = \mbf{F}(\mbf{d}_{n})
\end{equation}
in form
\begin{equation}
\mbf{d}_{n+1}^{1} = \left(\mbf{C}(\mbf{d}_{n}) \del{1}{\Delta t} + \mbf{K}(\mbf{d}_{n})\right)^{-1} 
\left(\mbf{F}(\mbf{d}_{n}) + \mbf{C}(\mbf{d}_{n})\mbf{d}_{n}\del{1}{\Delta t}\right)\ .
\end{equation}
Vector of residuum is calculated for trial vector $\mbf{d}_{n+1}^{1}$ with help of (\ref{defresiduum})
\begin{equation}
\mbf{R}_{n+1}^{1} = \mbf{R}(\mbf{d}_{n+1}^{1}) =
\mbf{C}(\mbf{d}_{n+1}^{1}) (\mbf{d}_{n+1}^{1} - \mbf{d}_{n})\del{1}{\Delta t} +
\mbf{K}(\mbf{d}_{n+1}^{1}) \mbf{d}_{n+1}^{1} - \mbf{F}(\mbf{d}_{n+1}^{1})
\end{equation}
New approximation $\mbf{d}_{n+1}^{2}$ of final vector $\mbf{d}_{n+1}$ is computed from relation (\ref{gennewton})
where previously defined notation is used
\begin{equation}
\mbf{d}_{n+1}^{2} = \mbf{d}_{n+1}^{1} - \left(\od{\mbf{R}_{n+1}^{1}}{\mbf{d}_{n+1}^{1}}\right)^{-1}
\mbf{R}_{n+1}^{1}\ .
\end{equation}

Newton-Rapson algorithm can be summarized in Table
\begin{center}
\begin{tabular}{|l|}
\hline
iterate over time interval $\langle t_{beginning},\ t_{end}\rangle$
and for each time step compute
\\ \hline
\hspace{10mm} 1. $t_{n+1}=t_{n}+\Delta t_{n}$
\\[3mm]
\hspace{10mm} 2. trial solution (vector $\mbf{d}_{n}$ is known)
\\
\hspace{10mm} $\mbf{d}_{n+1}^{1} = \left(\mbf{C}(\mbf{d}_{n}) \del{1}{\Delta t} + \mbf{K}(\mbf{d}_{n})\right)^{-1} 
\left(\mbf{F}(\mbf{d}_{n}) + \mbf{C}(\mbf{d}_{n})\mbf{d}_{n}\del{1}{\Delta t}\right)$
\\[3mm]
\hspace{10mm} 3. first residuum vector
\\
\hspace{10mm} $\mbf{R}_{n+1}^{1} = \mbf{R}(\mbf{d}_{n+1}^{1}) =
\mbf{C}(\mbf{d}_{n+1}^{1}) (\mbf{d}_{n+1}^{1} - \mbf{d}_{n})\del{1}{\Delta t} +
\mbf{K}(\mbf{d}_{n+1}^{1}) \mbf{d}_{n+1}^{1} - \mbf{F}(\mbf{d}_{n+1}^{1})$
\\[3mm]
\hspace{10mm} 4. if $\mbf{R}_{n+1}^{1} \leq tolerance$ define $\mbf{d}_{n+1} = \mbf{d}_{n+1}^{1}$ and go to the step 1.
\\[3mm]
\hspace{10mm} 5. otherwise set $m=1$ and iterate
\\[3mm]
\hspace{20mm} 5a. new approximation of $\mbf{d}_{n+1}$
\\
\hspace{20mm} $\mbf{d}_{n+1}^{m+1} = \mbf{d}_{n+1}^{m} - \left(\od{\mbf{R}_{n+1}^{m}}{\mbf{d}_{n+1}^{m}}\right)^{-1}
\mbf{R}_{n+1}^{m}$
\\[3mm]
\hspace{20mm} 5b. residuum vector
\\
\hspace{20mm} $\mbf{R}_{n+1}^{m+1} = \mbf{C}(\mbf{d}_{n+1}^{m+1}) (\mbf{d}_{n+1}^{m+1} - \mbf{d}_{n})\del{1}{\Delta t} +
\mbf{K}(\mbf{d}_{n+1}^{m+1}) \mbf{d}_{n+1}^{m+1} - \mbf{F}(\mbf{d}_{n+1}^{m+1})$
\\[3mm]
\hspace{20mm} 5c. if $\mbf{R}_{n+1}^{m+1} \leq tolerance$ define $\mbf{d}_{n+1} = \mbf{d}_{n+1}^{m+1}$ and go to the step 1.
\\[3mm]
\hspace{20mm} 5d. otherwise go to the step 5a.
\\ \hline
\end{tabular}
\end{center}
%%%%%%%%%%%%%%%%%%%%%%%%%%%%%%%%%%%%%%%%%%%%%%%%%%%%%%%%%%%%%%%%%%%%%%%%%%%%%
%%%%%%%%%%%%%%%%%%%%%%%%%%%%%%%%%%%%%%%%%%%%%%%%%%%%%%%%%%%%%%%%%%%%%%%%%%%%%
%%%%%%%%%%%%%%%%%%%%%%%%%%%%%%%%%%%%%%%%%%%%%%%%%%%%%%%%%%%%%%%%%%%%%%%%%%%%%
%%%%%%%%%%%%%%%%%%%%%%%%%%%%%%%%%%%%%%%%%%%%%%%%%%%%%%%%%%%%%%%%%%%%%%%%%%%%%
\section{General transport process}
In general case, there are $n_f$ fluxes and therefore $n_f$ basic variables which will be denoted as $f$.
If the basic variables are collected into the vector, it will be denoted as $\mbf{f}^{(i)}$.
Typically, hydro-thermo-mechanical coupling is described by three basic variables:
temperature $T$, pore pressure in water $p_w$ and pore pressure in gas $p_g$.
Each basic variable is approximated by shape functions and nodal values. From implementation point of
view is suitable to define several vectors and matrices. All nodal values defined on $\alpha$-th element
are collected in vector $\mbf{d}$ which is composed from $n_f$ blocks and contains $n$ components. Each basic variable is
approximated by its own nodal values collected in vector $\mbf{d}^{(i)}$ which contains $n^{(i)}$ components.
It is clear that following relation holds
\begin{equation}
n = \sum_{i=1}^{n_f} n^{(i)}\ .
\end{equation}
The vector of nodal values has form
\begin{equation}\label{defnodval}
\mbf{d} = \left(\mbf{d}^{(1)},\ \mbf{d}^{(2)},\ \ldots,\ \mbf{d}^{(n_f)}\right)\ ,
\end{equation}
which can be rewritten in more detail
\begin{equation}\label{compnodval}
\mbf{d} = \left(d^{(1)}_1,\ \ldots,\ d^{(1)}_{n^{(1)}},\ d^{(2)}_1,\ \ldots,\ d^{(2)}_{n^{(2)}},\
d^{(n_f)}_1,\ \ldots,\ d^{(n_f)}_{n^{(n_f)}}\right)\ .
\end{equation}
Indicial form
\begin{equation}\label{indcompnodval}
d_l = d_j^{(i)}
\end{equation}
follows from Equation (\ref{compnodval}) by reordering of components.
The vector $\mbf{f}$ can be written as
\begin{equation}
\mbf{f} = \mbf{N} \mbf{d}
\end{equation}
but it is not convenient for implementation and therefore block decomposition is used
\begin{equation}\label{defapproxmat}
\left(\begin{array}{c}
f^{(1)}
\\
f^{(2)}
\\
\vdots
\\
f^{(n_f)}
\end{array}\right)
=
\left(\begin{array}{cccc}
\mbf{N}^{(1)} & \mbf{0}       & \ldots & \mbf{0}
\\
\mbf{0}       & \mbf{N}^{(2)} &        &
\\
\vdots        &               & \ddots &
\\
\mbf{0}       & \ldots        &        & \mbf{N}^{(n_f)}
\\
\end{array}\right)
\left(\begin{array}{c}
\mbf{d}^{(1)}
\\
\mbf{d}^{(2)}
\\
\vdots
\\
\mbf{d}^{(n_f)}
\end{array}\right)\ .
\end{equation}
Particular matrices depend on type of finite element. Matrix $\mbf{N}^{(i)}$ for quadrilateral finite
element with bilinear approximation functions used for approximation of temperature is denoted and defined
by relation
\begin{equation}
f^{(i)} = f^{(T)} = \mbf{N}^{(i)} \mbf{d}^{(i)} = \mbf{N}^{(T)} \mbf{d}^{(T)} = (N_1,\ N_2,\ N_3,\ N_4)
\left(\begin{array}{c}
d_1^{(T)}
\\
d_2^{(T)}
\\
d_3^{(T)}
\\
d_4^{(T)}
\end{array}\right)
\end{equation}

Fluxes of particular variables are defined from constitutive relations where gradients of basic variables occur.
The gradients and their approximation is main aim of this paragraph. Gradient of $i$-th basic variable is defined
\begin{equation}
\left(\mbf{g}^{(i)}\right)^T = \left(\ppd{f^{(i)}}{x},\ \ppd{f^{(i)}}{y},\ \ppd{f^{(i)}}{z},\ \right)\ .
\end{equation}
With help of definition of approximation functions is possible to write
\begin{equation}
\mbf{g}^{(i)} =
\left(\begin{array}{c}
\ppd{\mbf{N}^{(i)}}{x}
\\
\ppd{\mbf{N}^{(i)}}{y}
\\
\ppd{\mbf{N}^{(i)}}{z}
\end{array}\right)\mbf{d}^{(i)}
= \mbf{B}^{(i)} \mbf{d}^{(i)}\ .
\end{equation}
Another notation can be used for simple formal description
\begin{equation}\label{defgradmat}
\mbf{g} = \left(\begin{array}{c}
\mbf{g}^{(1)}
\\
\mbf{g}^{(2)}
\\
\vdots
\\
\mbf{g}^{(n_f)}
\end{array}\right)
= \mbf{B} \mbf{d} =
\left(\begin{array}{cccc}
\mbf{B}^{(1)} & \mbf{0}       & \ldots & \mbf{0}
\\
\mbf{0}       & \mbf{B}^{(2)} &        &
\\
\vdots        &               & \ddots &
\\
\mbf{0}       & \ldots        &        & \mbf{B}^{(n_f)}
\\
\end{array}\right)
\left(\begin{array}{c}
\mbf{d}^{(1)}
\\
\mbf{d}^{(2)}
\\
\vdots
\\
\mbf{d}^{(n_f)}
\end{array}\right)\ .
\end{equation}

Everything is ready for definition of fluxes now. In general case each flux depends
on each gradient
\begin{equation}
\mbf{q}^{(i)} = \sum_{j=1}^{n_f} \mbf{D}^{(ij)} \mbf{g}^{(j)}
\end{equation}
Formal description is
\begin{equation}\label{defcondmatmat}
\mbf{q} = \left(\begin{array}{c}
\mbf{q}^{(1)}
\\
\mbf{q}^{(2)}
\\
\vdots
\\
\mbf{q}^{(n_f)}
\end{array}\right)
= \mbf{D}\ \mbf{B}\ \mbf{d} =
\left(\begin{array}{cccc}
\mbf{D}^{(11)}   & \mbf{D}^{(12)} & \ldots & \mbf{D}^{(1n_f)}
\\
\mbf{D}^{(21)}   & \mbf{D}^{(22)} &        &
\\
\vdots           &                & \ddots &
\\
\mbf{D}^{(n_f1}) & \ldots         &        & \mbf{D}^{(n_fn_f)}
\\
\end{array}\right)
\mbf{B}\ \mbf{d}\ .
\end{equation}

The matrix of conductivity of the material $\mbf{D}$ plays very important role in the analysis of nonlinear phenomena.
Therefore it will be studied more carefully. The matrix $\mbf{D}$ contains material parameters which depend on
actual value of basic functions collected in the vector $\mbf{f}$. Thanks to approximation of $\mbf{f}$ by shape functions
and nodal values $\mbf{d}$, the matrix $\mbf{D}$ depends on nodal values and one can write $\mbf{D}(\mbf{d})$, where
$\mbf{d}$ stands for nodal values connected to the studied finite element. The derivatives of the matrix $\mbf{D}(\mbf{d})$
will be usefull in Newton-Raphson algorithm which will be used in nonlinear nonstationary calculations.
Before differentiation of the matrix $\mbf{D}$ with respect to nodal values it is necessary to mention derivatives
of basic variables with respect to nodal values. In accordance with Equation (\ref{defapproxmat}), the derivatives have form
\begin{eqnarray}\label{defderapprfun}
\ppd{f^{(k)}}{d_i^{(j)}} = \left\{\begin{array}{ccc}
0 & \ \ \ \ \ & {\rm for}\ j \neq k
\\[5mm]
N^{(j)}_{i} & & {\rm for}\ j= k
\end{array}\right.
\end{eqnarray}
All necessary relations are ready and the derivative of conductivity matrix of material $\mbf{D}$ with respect to
nodal values results in
\begin{eqnarray}
\ppd{\mbf{D}}{d_i^{(j)}} = \sum_{k=1}^{n_f}\ \ppd{\mbf{D}}{f^{(k)}} \ppd{f^{(k)}}{d_i^{(j)}} =
\ppd{\mbf{D}}{f^{(j)}} N^{(j)}_{i}\ ,\ \ \ \ \ \ ({\rm no\ sum\ over}\ j)\ .
\end{eqnarray}

Conductivity matrix is defined with help of compact formulation as
\begin{equation}\label{defcondmat}
\mbf{K} = \int_{\Omega_e} \mbf{B}^T\ \mbf{D}\ \mbf{B}\ {\rm d}\Omega\ .
\end{equation}
Definition of the conductivity matrix (\ref{defcondmat}) can be rewritten with help of
Equations (\ref{defgradmat}) and (\ref{defcondmatmat}) into the form
\begin{equation}
\mbf{K} = \int_{\Omega_e}
\left(\begin{array}{cccc}
(\mbf{B}^{(1)})^T \mbf{D}^{(11)} \mbf{B}^{(1)}     & (\mbf{B}^{(1)})^T \mbf{D}^{(12)} \mbf{B}^{(2)} &
\ldots                                             & (\mbf{B}^{(1)})^T \mbf{D}^{(1n_f)} \mbf{B}^{(n_f)}
\\
(\mbf{B}^{(2)})^T \mbf{D}^{(21)} \mbf{B}^{(1)}     & (\mbf{B}^{(2)})^T \mbf{D}^{(22)} \mbf{B}^{(2)} &
                                                   &
\\
\vdots                                             &                                                &
\ddots                                             &
\\
(\mbf{B}^{(n_f)})^T \mbf{D}^{(n_f1)} \mbf{B}^{(1}) & \ldots                                         &
                                                   & (\mbf{B}^{(n_f)})^T \mbf{D}^{(n_fn_f)} \mbf{B}^{(n_f)}
\\
\end{array}\right)\ {\rm d}\Omega\ .
\end{equation}
It is clear that the conductivity matrix of one element is assembled from particular blocks
$(\mbf{B}^{(i)})^T \mbf{D}^{(ij)} \mbf{B}^{(j)}$.

%%%%%%%%%%%%%%%%%%%%%%%%%%%%%%%%%%%%%%%%%%%%%%%%%%%%%%%%%%%%%%%%%%%%%%%%%%%%%
%%%%%%%%%%%%%%%%%%%%%%%%%%%%%%%%%%%%%%%%%%%%%%%%%%%%%%%%%%%%%%%%%%%%%%%%%%%%%
%%%%%%%%%%%%%%%%%%%%%%%%%%%%%%%%%%%%%%%%%%%%%%%%%%%%%%%%%%%%%%%%%%%%%%%%%%%%%
%%%%%%%%%%%%%%%%%%%%%%%%%%%%%%%%%%%%%%%%%%%%%%%%%%%%%%%%%%%%%%%%%%%%%%%%%%%%%
\section{Thermo-hydro coupling}
Thermo-hydro coupling is described by three basic variables: temperature $T$, pore pressure in water $p_w$
and pore pressure in gas $p_g$. Everything in this section will be derived for threedimensional case and
twodimensional problems are obtained after neglecting terms connected with the $z$ coordinate. The vector
of basic variables has for this problem form
\begin{equation}
\mbf{f} = \left(T,\ p_w,\ p_g\right)
\end{equation}
and the $n_f$ is equl to 3. Basic variables are approximated
\begin{equation}
\mbf{f} =
\left(\begin{array}{c}
f^{(T)}
\\
f^{(p_w)}
\\
f^{(p_g)}
\end{array}\right)
=
\left(\begin{array}{ccc}
\mbf{N}^{(T)} & \mbf{0}         & \mbf{0}
\\
\mbf{0}       & \mbf{N}^{(p_w)} & \mbf{0}
\\
\mbf{0}       & \mbf{0}         & \mbf{N}^{(p_g)}
\\
\end{array}\right)
\left(\begin{array}{c}
\mbf{d}^{(T)}
\\
\mbf{d}^{(p_w)}
\\
\mbf{d}^{(p_g)}
\end{array}\right)\ .
\end{equation}

Gradients have form
\begin{equation}
\left(\mbf{g}^{(T)}\right)^T = \left(\ppd{T}{x},\ \ppd{T}{y},\ \ppd{T}{z}\right)\ ,
\end{equation}
\begin{equation}
\left(\mbf{g}^{(p_w)}\right)^T = \left(\ppd{p_w}{x},\ \ppd{p_w}{y},\ \ppd{p_w}{z}\right)\ ,
\end{equation}
\begin{equation}
\left(\mbf{g}^{(p_g)}\right)^T = \left(\ppd{p_g}{x},\ \ppd{p_g}{y},\ \ppd{p_g}{z}\right)\ .
\end{equation}
Another notation can be used for simple formal description
\begin{equation}\label{defgradmatth}
\mbf{g} = \left(\begin{array}{c}
\mbf{g}^{(T)}
\\
\mbf{g}^{(p_w)}
\\
\mbf{g}^{(p_g)}
\end{array}\right)
= \mbf{B} \mbf{d} =
\left(\begin{array}{ccc}
\mbf{B}^{(T)} & \mbf{0}         & \mbf{0}
\\
\mbf{0}       & \mbf{B}^{(p_w)} & \mbf{0}
\\
\mbf{0}       & \mbf{0}         & \mbf{B}^{(p_g)}
\\
\end{array}\right)
\left(\begin{array}{c}
\mbf{d}^{(T)}
\\
\mbf{d}^{(p_w)}
\\
\mbf{d}^{(p_g)}
\end{array}\right)\ .
\end{equation}

Formal description is
\begin{equation}\label{defcondmatmatth}
\mbf{q} = \left(\begin{array}{c}
\mbf{q}^{(T)}
\\
\mbf{q}^{(p_w)}
\\
\mbf{q}^{(p_g)}
\end{array}\right)
= \mbf{D}\ \mbf{B}\ \mbf{d} =
\left(\begin{array}{ccc}
\mbf{D}^{(TT)}   & \mbf{D}^{(Tp_w)}   & \mbf{D}^{(Tp_g)}
\\
\mbf{D}^{(p_wT)} & \mbf{D}^{(p_wp_w)} & \mbf{D}^{(p_wp_g)}
\\
\mbf{D}^{(p_gT}) & \mbf{D}^{(p_gp_w)} & \mbf{D}^{(p_gp_g)}
\\
\end{array}\right)
\mbf{B}\ \mbf{d}\ .
\end{equation}

Conductivity matrix is defined with help of compact formulation as
\begin{equation}\label{defcondmatth}
\mbf{K} = \int_{\Omega_e} \mbf{B}^T\ \mbf{D}\ \mbf{B}\ {\rm d}\Omega\ .
\end{equation}
Definition of the conductivity matrix (\ref{defcondmatth}) can be rewritten with help of
Equations (\ref{defgradmatth}) and (\ref{defcondmatmatth}) into the form
\begin{equation}
\mbf{K} = \int_{\Omega_e}
\left(\begin{array}{ccc}
(\mbf{B}^{(T)})^T \mbf{D}^{(TT)} \mbf{B}^{(T)}         & (\mbf{B}^{(T)})^T \mbf{D}^{(Tp_w)} \mbf{B}^{(p_w)} &
(\mbf{B}^{(T)})^T \mbf{D}^{(Tp_g)} \mbf{B}^{(p_g)}
\\
(\mbf{B}^{(p_w)})^T \mbf{D}^{(p_wT)} \mbf{B}^{(T)}     & (\mbf{B}^{(p_w)})^T \mbf{D}^{(p_wp_w)} \mbf{B}^{(p_w)} &
(\mbf{B}^{(p_w)})^T \mbf{D}^{(p_wp_g)} \mbf{B}^{(p_g)}
\\
(\mbf{B}^{(p_g)})^T \mbf{D}^{(p_gT)} \mbf{B}^{(T)}     & (\mbf{B}^{(p_g)})^T \mbf{D}^{(p_gp_w)} \mbf{B}^{(p_w)} &
(\mbf{B}^{(p_g)})^T \mbf{D}^{(p_gp_g)} \mbf{B}^{(p_g)}
\\
\end{array}\right)\ {\rm d}\Omega\ .
\end{equation}

%%%%%%%%%%%%%%%%%%%%%%%%%%%%%%%%%%%%%%%%%%%%%%%%%%%%%%%%%
%%%%%%%%%%%%%%%%%%%%%%%%%%%%%%%%%%%%%%%%%%%%%%%%%%%%%%%%%
%%%%%%%%%%%%%%%%%%%%%%%%%%%%%%%%%%%%%%%%%%%%%%%%%%%%%%%%%
%%%%%%%%%%%%%%%%%%%%%%%%%%%%%%%%%%%%%%%%%%%%%%%%%%%%%%%%%
%%%%%%%%%%%%%%%%%%%%%%%%%%%%%%%%%%%%%%%%%%%%%%%%%%%%%%%%%
%%%%%%%%%%%%%%%%%%%%%%%%%%%%%%%%%%%%%%%%%%%%%%%%%%%%%%%%%


\printindex
\end{document}
