%%%%%%%%%%%%%%%%%%%%%%%%%%%%%%%%%%%%%%%%%%%%%%%%%%%%%%%%%%%%%%%%%%%%%%%%%%%%%%%%%%%%%%%%%%%%%%%%%%%%%
%%%%%%%%%%%%%%%%%%%%%%%%%%%%%%%%%%%%%%%%%%%%%%%%%%%%%%%%%%%%%%%%%%%%%%%%%%%%%%%%%%%%%%%%%%%%%%%%%%%%%
\chapter{Description of MEFEL Philosophy}
%%%%%%%%%%%%%%%%%%%%%%%%%%%%%%%%%%%%%%%%%%%%%%%%%%%%%%%%%%%%%%%%%%%%%%%%%%%%%%%%%%%%%%%%%%%%%%%%%%%%%
%%%%%%%%%%%%%%%%%%%%%%%%%%%%%%%%%%%%%%%%%%%%%%%%%%%%%%%%%%%%%%%%%%%%%%%%%%%%%%%%%%%%%%%%%%%%%%%%%%%%%

\section{Strategy of Nonlinear Static Problems}

Nonlinear static problems are described by (\ref{eqnnonlinstat}). Dependence of stiffness matrix on actual
configuration represented by vector of displacements $\mbf{d}$ is the main difficulty. System of nonlinear
algebraic equations (\ref{eqnnonlinstat}) is solved by the arc-length method or by the Newton-Raphson method.
Increments of displacements and loads are computed from the attained state which is in equilibrium. New
vectors of displacements and loads define new state which is generally not in equilibrium. Therefore vector
of residuals is assembled and corrections of mentioned vectors are computed. The last step is repeated until
equilibrium state is obtained.


%%%%%%%%%%%%%%%%%%%%%%%%%%%%%%%%%%%%%%%%%%%%%%%%%%%%%%%%%%%%%%%%%%%%%%%%%%%%%%%%%%%%%%%%%%%%%%%%%%%%%
\section{Integration Points}
\index{integration points}

Integration points play significant role in the code. They contain information about used material models,
strains and stresses, important material parameters etc.


There are the following arrays at every integration point
\begin{itemize}
\item{{\sf tm} - contains type of material or materials,}
\item{{\sf idm} - contains identification number of material or materials,}
\item{{\sf stress} - contains stress components,}
\item{{\sf strain} - contains strain components,}
\item{{\sf other} - contains all other parameters, which are not in the equilibrium during the iteration,}
\item{{\sf eqother} - contains equilibriated parameters stored in the array {\sf other},}
\item{{\sf nonloc} - contains nonlocal averaged values (allocated only for nonlocal material models).}
\end{itemize}
\index{array!other}\index{array!eqother}\index{array!stress}\index{array!strain}\index{array!nonloc}

The array {\sf other} contains all necessary material parameters. Elastic models do not use it. Plastic
models require to store plastic strains, consistency parameter and optionally hardening parameters.
Damage models require to store damage parameter etc. Stored parameters in the array {\sf other} are
defined by used material model. The material model should describe an order of parameters in the
array {\sf other}.

The array {\sf eqother} contains the same parameters as the array {\sf other}. The array {\sf other}
is changed in each iteration and contains values of material parameters which can be unequilibriated.
On the other hand, the array {\sf eqother} contains values of material parameters which are
equilibriated and this array is changed only when the new global equilibrium is attained.
Components of the array {\sf eqother} are changed by the function {\sf void mechmat::updateipval (void)}
which is called by the solver (arc-length, Newton-Raphson method etc.).

The situation is more complicated in the case of combination of several material models. Since it is
not clear the maximum number of combined material models, the arrays {\sf other} and {\sf eqother}
are not multiplied. The arrays are allocated longer. The number of components of the arrays is
equal to the sum of number of material parameters of combined models. The number of components
of the arrays can be obtained by the function {\sf long mechmat::givencompother (long ipp,long im)}.
The {\it ipp} denotes the number of integration point and {\it im} stands for the number of
material model used at the integration point (it is index in the array {\sf tm}).
The defualt value of {\it im} is 0.

%%%%%%%%%%%%%%%%%%%%%%%%%%%%%%%%%%%%%%%%%%%%%%%%%%%%%%%%%%%%%%%%%%%%%%%%%%%%%%%%%%%%%%%%%%%%%%%%%%%%%
%%%%%%%%%%%%%%%%%%%%%%%%%%%%%%%%%%%%%%%%%%%%%%%%%%%%%%%%%%%%%%%%%%%%%%%%%%%%%%%%%%%%%%%%%%%%%%%%%%%%%
\section{Nodes}

Nodes of finite element mesh are described by the class {\sf node} in file
{\tt node.h} and {\tt node.cpp}. Some of information are also located in
the class {\sf gnode} in file {\tt gnode.h} and {\tt gnode.cpp} in GEFEL.

\begin{itemize}
\item{{\sf crst} - contains type of cross section,}
\item{{\sf idcs} - contains id of cross section,}
\item{{\sf transf} - local coordinate system indicator,}
\item{{\sf react} - reaction indicator,}
\item{{\sf ncompstr} - number of strain/stress components,}
\item{{\sf ncompother} - number of components in array other,}
\end{itemize}
%\index{array!other}\index{array!eqother}\index{array!stress}\index{array!strain}\index{array!nonloc}


{\sf transf} is set to zero in constructor.

\begin{center}
\begin{tabular}{|c|l|}
\hline
0 & global coordinate system is used
\\
  & no basis vectors are defined at node
\\[2mm] \hline
2 & local coordinate system with two basis vectors
\\
  & used in twodimensional problems (plane stress/strain, plates)
\\[2mm] \hline
3 & local coordinate system with three basis vectors
\\
  & used in threedimensional problems, shells
\\ \hline
\end{tabular}
\end{center}



%%%%%%%%%%%%%%%%%%%%%%%%%%%%%%%%%%%%%%%%%%%%%%%%%%%%%%%%%%%%%%%%%%%%%%%%%%%%%%%%%%%%%%%%%%%%%%%%%%%%%
%%%%%%%%%%%%%%%%%%%%%%%%%%%%%%%%%%%%%%%%%%%%%%%%%%%%%%%%%%%%%%%%%%%%%%%%%%%%%%%%%%%%%%%%%%%%%%%%%%%%%
\section{Combination of Material Models}

The code contains many material models in the following areas:
\begin{itemize}
\item{elasticity}
\item{plasticity}
\item{damage}
\item{viscosity}
\item{microplane models}
\end{itemize}

In order to keep the code general in the area of material modelling, the code enables many combinations
of implemented material models. For this purpose, artificial material models are introduced. They are called
artificial because they do not contain any material parameters. Their main and only purpose is that they
call appropriate functions of combined models.

The array {\sf tm} of the class {\sf intpoints} play important role in the case of combination of several
material models. The main material model is located at the first position. This model controls whole
computation. It means, that it calls appropriate methods of classes, stores and restores required data etc.
The artificial material models are always located at the first position of the array {\sf tm}.

The type elastic material model is always located at the last position of the array {\sf tm}. The only exception
is the case with temperature load which requires a model of thermal dilatancy. The type of the thermal dilatancy
is located at the last position of the array {\sf tm} and the type of elastic model is at the last but one position.
The type of elastic material can be obtained by the function {\sf long intpoints::gemid(void)}. It returns
the index of the elastic model in the array {\sf tm}.

%%%%%%%%%%%%%%%%%%%%%%%%%%%%%%%%%%%%%%%%%%%%%%%%%%%%%%%%%%%%%%%%%%%%%%%%%%%%%%%%%%%%%%%%%%%%%%%%%%%%%
\subsection{Combination of damage and plasticity}

Combination of damage and plasticity models is guaranteed by the artificial model defined in the class {\sf damplast}.
It enables combination of any model of plasticity with any model of damage. The user is responsible for
meaningful choice of particular material models.

If the artificial model defined by the class {\sf damplast} is used, the array {\sf tm}
at integration points has the following components:

\begin{center}
\begin{tabular}{cl}
tm[0] =& damplast\\
tm[1] =& type of plasticity\\
tm[2] =& type of damage\\
tm[3] =& type of elasticity\\
\end{tabular}
\end{center}

From the ordering of material models in the array {\sf tm} follows that the arrays {\sf other}
and {\sf eqother} contain material parameters of plasticity model at the beginning and
material parameters of damage model at the end. Therefore, any function of plasticity model
is called with parameters {\it im}=1 (the position with zero index contains damplast model) and {\it ido}=0.
The function {\sf Mm--$>$givencompother (ipp,1)} returns the number of components of plasticity model.
Any function of damage model is called with parameters {\it im}=2 and {\it ido} is equal to the
number of components of plasticity model.
The function {\sf Mm--$>$givencompother (ipp,2)} returns the number of components of damage model.
The function {\sf Mm--$>$givencompother (ipp,0)} returns the total number of components in the arrays
{\sf other} and {\sf eqother}.

The function {\sf void damplast::nlstresses (long ipp)} of the class {\sf damplast}
calls the function {\sf Mm--$>$computenlstresses(long ipp, long im, long ido);} of the class {\sf mechmat}.
The type of plasticity model is defined at the integration point in the array {\sf tm[1]}, therefore {\it im}=1.
Parameters of the plasticity models are at the beginning of the array {\sf other}, therefore {\it ido}=0.
At this stage, the plasticity is solved and the damage will be computed.

Then, the function {\sf void damplast::nlstresses (long ipp)} of the class {\sf damplast}
calls solution of the damage problem. It is nearly identical to the function
{\sf nlstresses (long ipp, long im, long ido)} of damage models. The only difference is based
on the computation of elastic strains which are obtained from the total strains and currently
computed plastic strains (see comments in the code).

%%%%%%%%%%%%%%%%%%%%%%%%%%%%%%%%%%%%%%%%%%%%%%%%%%%%%%%%%%%%%%%%%%%%%%%%%%%%%%%%%%%%%%%%%%%%%%%%%%%%%
%%%%%%%%%%%%%%%%%%%%%%%%%%%%%%%%%%%%%%%%%%%%%%%%%%%%%%%%%%%%%%%%%%%%%%%%%%%%%%%%%%%%%%%%%%%%%%%%%%%%%
\section{Nonlocal Computations}

Nonlocal computations are used for various problems which suffer by difficulties during the classical
local approximation. The typical example is description of damage which has to be localized in a narrow
band.

The application of material models based on nonlocal approximation is clearly more complicated than
the classical material models. For better understanding, the order of method calling is described in this section.

Solver of nonlinear algebraic equations (arc-length, Newton-Raphson) calls function
\newline {\sf internal\_forces (lcid,fi);} from the file {\sf globmat.cpp}.
\newline With respect to the nonlocal model, the function
\newline {\sf nonloc\_internal\_forces (lcid,intfor);} from the file {\sf globmat.cpp}
\newline is called. This function calls the function
\newline {\sf local\_values(lcid, i, 0, 0);}
\newline with respect to the type of finite element. The finite element calls
\newline {\sf Mm--$>$computenlstresses (ipp);} from the file {\sf mechmat.cpp}
\newline in order to obtain correct stresses. With respect to the used material model, the function
\newline {\sf nlstresses (ipp);} or {\sf nonloc\_nlstresses (ipp,im,ido);}
\newline is called. Then the code is going back to the function
\newline {\sf nonloc\_internal\_forces (long lcid,double *intfor)} from the file {\sf globmat.cpp}
\newline where the averaging is performed. Then the function
\newline {\sf nonloc\_internal\_forces (lcid, i, 0, 0, ifor);}
\newline is called with respect to the used element. This function calls
\newline {\sf Mm--$>$compnonloc\_nlstresses (ipp);} from the file {\sf mechmat.cpp}
\newline With respect to the type of material model, the function
\newline {\sf nonloc\_nlstresses (ipp,im,ido);}
\newline is called. After its execution, the stresses are known and nodal forces
are computed at the element level. The resulting vector of nodal forces is obtained
after execution
\newline {\sf nonloc\_internal\_forces (long lcid,double *intfor)}



%%%%%%%%%%%%%%%%%%%%%%%%%%%%%%%%%%%%%%%%%%%%%%%%%%%%%%%%%%%%%%%%%%%%%%%%%%%%%%%%%%%%%%%%%%%%%%%%%%%%%
%%%%%%%%%%%%%%%%%%%%%%%%%%%%%%%%%%%%%%%%%%%%%%%%%%%%%%%%%%%%%%%%%%%%%%%%%%%%%%%%%%%%%%%%%%%%%%%%%%%%%
\section{Handling with plasticity}

Theory of plasticity applied to problems with small strains is roughly described in Section \ref{sectplasmatmodels}.
In the MEFEL code, material models of plasticity are used in connection with nonlinear static problems.
The main aim of any material model of plasticity is to compute correct stresses from given strains.
Strains are computed by any method which solves system of nonlinear algebraic equations. Correct stresses
are extremely important for evaluation of internal forces which are used in computation of the vector of residuals.
If the norm of vector of residuals is less than required error, equilibrium state is obtained.

There are several methods of the class {\sf mechmat} which serve to mentioned main aim of any model of plasticity.
Method {\sf computenlstresses (long ipp,long im,long ido)} of the class {\sf mechmat} is the most important one.
It is called by finite elements during computation of internal forces. Internal forces are obtained by numerical
integration and therefore the number of integration point {\it ipp} is a parameter of the method. The default
value of the two other parameters are equal to 0. They are important only in a case of combination of several
material models. {\it im} stands for the number of material and {\it ido} denotes the position in the array {\sf other}.
The method {\sf computenlstresses (long ipp,long im,long ido)} detects type of material model defined at
required integration point and calls its method {\sf nlstresses (long ipp,long im,long ido)}.

Each class which represents a material model must contain method {\sf nlstresses (long ipp,long im,long ido)} which computes
stresses from given strains.
Method {\sf nlstresses (long ipp,long im,long ido)} of a class representing a material model of plasticity must compute
stresses satisfying yield condition. Given strains and other parameters (plastic strains, hardening/softening parameters, etc.)
are stored at integration point in arrays {\sf strain} and {\sf other}. Number of integration point is a parameter of the
method {\sf nlstresses (long ipp,long im,long ido)}. The method computes stresses and stores them to array {\sf stress} at integration
point.

Stress computation is done by any method which returns trial stresses to yield surface. There is cutting plane method
\index{method!cutting plane} which is described in Section \ref{sectcutplanemet} implemented in the code. Method
{\sf cutting\_plane (long ipp,long im,double \&gamma,vector \&epsn,vector \&epsp,vector \&q)} of the class {\sf mechmat} is
implementation of the cutting plane method. {\it ipp} denotes number of integration point (its id), {\it im} denotes index in the {\it ipp} integration point {\sf tm} array 
- determines material type, {\it gamma} stands for consistency parameter, {\it epsn} contains new strains, {\it epsp} contains plastic strains and {\it q}
contains hardening parameter(s). The cutting plane method requires evaluation of yield function, derivatives of
yield function and plastic potential with respect to stress components and hardening modulus. There are methods
of the class {\sf mechmat} which provide these requirements.

Method {\sf yieldfunction (long ipp, long idpm, matrix \&sig,vector \&q)} of the class {\sf mechmat} evaluates
yield function for given stress stored in {\it sig} and hardening parameter(s) stored in {\it q}. {\it ipp} denotes
number of integration point and {\it idpm} contains id of plastic material model. The method detects type of
material model at required integration point and calls its method {\sf yieldfunction (matrix \&sig,vector \&q)}
which evaluates yield function of the material model. Each class which represents a material model of plasticity
must contain such method. The second parameter {\it q} is optional.

Method {\sf dfdsigma (long ipp, long idpm, matrix \&sig, vector \&q, matrix \&dfds)} of the class {\sf mechmat}
assembles derivatives of yield function with respect to stress components. {\it ipp} denotes number of integration
point, {\it idpm} stands for id of plastic material, {\it sig} contains stress components, {\it q} contains hardening
parameter(s) and {\it dfds} contains required derivatives. The method detects type of material model at required
integration point and calls its method {\sf deryieldfsigma (matrix \&sig, vector \&q, matrix \&dfds)} which
assembles required derivatives of the material model. Each class which represents a material model of plasticity
must contain such method.

Method {\sf dgdsigma (long ipp, long idpm, matrix \&sig, vector \&q, matrix \&dgds)} of the class {\sf mechmat}
assembles derivatives of plastic potential with respect to stress components. {\it ipp} denotes number of
integration point, {\it idpm} stands for id of plastic material, {\it sig} contains stress components, {\it q}
contains hardening parameter(s) and {\it dfds} contains required derivatives. The method detects type of material
model at required integration point and calls its method {\sf deryieldfsigma (matrix \&sig, vector \&q, matrix \&dgds)}
which assembles required derivatives of the model. Each class which represents a material model of plasticity
must contain such method.

Method {\sf plasmodscalar(long ipp, long idpm, matrix \&sig, vector \&eps, vector \&epsp, vector \&qtr)}
of the class {\sf mechmat} computes plastic modulus. {\it ipp} denotes number of integration point, {\it idpm}
stands for id of plastic material, {\it sig} contains stress components, {\it eps} contains strain components,
{\it epsp} contains plastic strain components and {\it qtr} contains hardening parameter(s). The method detects
type of material model at required integration point and calls its method {\sf plasmodscalar (vector \&qtr)}
which computes plastic modulus of the model. Each class which represents a material model of plasticity must
contain such method.

Method {\sf computenlstresses (long ipp, long im, long ido)} of the class {\sf mechmat} is the most important one.
It is called by finite elements during computation of internal forces. Internal forces are obtained by numerical
integration and therefore id of required integration point is a parameter of the method. The method
{\sf computenlstresses (long ipp, long im, long ido)} detects type of the material model given by the index {\it im} of
the required integration point and calls its method {\sf nlstresses (long ipp, long im, long ido)}.

Each class which represents a material model must contain method {\sf nlstresses (long ipp, long im, long ido)} which computes
stresses from given strains.
Method {\sf nlstresses (long ipp, long im, long ido)} of a class representing a material model of plasticity must compute
stresses satisfying yield condition. Given strains and other parameters (plastic strains, hardening/softening parameters, etc.)
are stored at integration point in arrays {\sf strain} and {\sf other}. Number of integration point is a parameter of the
method {\sf nlstresses (long ipp, long im, long ido)}. The method computes stresses and stores them to array {\sf stress} at integration
point.
\begin{center}
{\tt
\begin{tabular}{l}
computenlstresses (long ipp, long im, long ido)
\\
cutting\_plane (long ipp,long im,double \&gamma,vector \&epsn,
\\
vector \&epsp,vector \&q)
\\
yieldfunction (long ipp, long idpm, matrix \&sig,vector \&q)
\\
dfdsigma (long ipp, long idpm, matrix \&sig, vector \&q, matrix \&dfds)
\\
dgdsigma (long ipp, long idpm, matrix \&sig, vector \&q, matrix \&dgds)
\\
plasmodscalar(long ipp, long idpm, matrix \&sig, vector \&eps,
\\
vector \&epsp, vector \&qtr)
\\
\end{tabular}
}
\end{center}

%\begin{center}
{\tt
\begin{tabular}{l}
nlstresses (long ipp, long im, long ido)
\\
yieldfunction (matrix \&sig,vector \&q)
\\
deryieldfsigma (matrix \&sig, vector \&q, matrix \&dfds)
\\
deryieldfsigma (matrix \&sig, vector \&q, matrix \&dgds)
\\
plasmodscalar(vector \&qtr)
\\
\end{tabular}
}
%\end{center}
%%%%%%%%%%%%%%%%%%%%%%%%%%%%%%%%%%%%%%%%%%%%%%%%%%%%%%%%%%%%%%%%%%%%%%%%%%%%%%%%%%%%%%%%
%%%%%%%%%%%%%%%%%%%%%%%%%%%%%%%%%%%%%%%%%%%%%%%%%%%%%%%%%%%%%%%%%%%%%%%%%%%%%%%%%%%%%%%%
\section{Handling with scalar damage}

There are several methods of the class {\sf mechmat} which serve to mentioned main aim of any model of scalar damage.

Method {\sf computenlstresses (long ipp, long im, long ido)} of the class {\sf mechmat} is the most important one.
It is called by finite elements during computation of internal forces. Internal forces are obtained by numerical
integration and therefore id of required integration point is a parameter of the method. The method
{\sf computenlstresses (long ipp, long im, long ido)} detects type of material model defined at required integration point and calls
its method {\sf nlstresses (long ipp, long im, long ido)}.

Each class which represents a material model must contain method {\sf nlstresses (long ipp, long im, long ido)} which computes
stresses from given strains.
Method {\sf nlstresses (long ipp, long im, long ido)} of a class representing a material model of damage must compute
stresses corresponding reached strains. Given strains and other parameters are stored at integration point in arrays
{\sf strain} and {\sf other}. Number of integration point is a parameter of the method {\sf nlstresses (long ipp, long im, long ido)}.
The method computes stresses and stores them to array {\sf stress} at integration ponit.

Algorithm which solves stresses in the scalar damage problem is implemented in the method {\sf scal\_dam\_sol (long ipp, long im, 
vector \&eps,vector \&kappa, vector \&sigma)} of the class {\sf mechmat}. See Section \ref{sectdammatmodels} for the
theoretical background.
{\it ipp} denotes number of integration point (its id), {\it im} denotes index in the {\it ipp} integration point {\sf tm} array - determines material type,
{\it eps} contains new strains, {\it kappa} damage parameter and {\it sigma} is containing returned corresponding stresses. This function requires
evaluation of damage function and evaluation of the parameters for damage function. There are methods of the class {\sf mechmat} which provide 
these requirements.


Method {\sf damfunction(long ipp, long im, vector \&kappa, vector \&eps)} of the class {\sf mechmat} evaluates
damage for given strains stored in {\it eps} and previous reached damage {\it kappa}. {\it ipp} denotes
number of integration point. The method detects type of material model at required integration point by the parameter {\it im} and
calls its method {\sf damfunction(long ipp, vector \&kappa, vector \&eps)} which evaluates damage function
of the material model. Each class which represents a material model of scalar damage must contain such method.

Method {\sf damfuncpar(long ipp, long im, vector \&eps, vector \&kappa)} of the class {\sf mechmat} assembles vector of
damage function parameters for given strains {\it eps}. {\it ipp} denotes number of integration point, {\it im} denotes index in the {\it ipp} 
integration point {\sf tm array} - determines material type, {\it kappa} stands for vector of damage function parameters. The method detects type 
of material model at required integration point and calls its method {\sf damfuncpar(long ipp, vector \&eps, vector \&kappa)} which assembles required 
parameters of the material model. Each class which represents a material model of scalar damage must contain such method.

Each class which represents a material model must contain method {\sf nlstresses (long ipp)} which computes
stresses from given strains.
Method {\sf nlstresses (long ipp, long im, long ido)} of a class representing a material model of scalar damage must compute
stresses for given actual strains. Given strains and other parameters are stored at integration point in arrays
{\sf strain} and {\sf other}. Number of the integration point is a parameter of the method {\sf nlstresses (long ipp, long im, long ido)}.
The method computes stresses and stores them to array {\sf stress} at integration point.

%\begin{center}
{\tt
\begin{tabular}{l}
computenlstresses (long ipp, long im, long ido)
\\
scal\_dam\_sol (long ipp, long im, long ido, vector \&eps,vector \&kappa, vector \&sigma)
\\
damfunction(long ipp, long im, vector \&kappa, vector \&eps)
\\
damfuncpar(long ipp, long im, vector \&eps, vector \&kappa)
\\
\end{tabular}
}
%\end{center}

%\begin{center}
{\tt
\begin{tabular}{l}
nlstresses (long ipp, long im, long ido)
\\
damfuncpar(long ipp, vector \&eps, vector \&kappa)
\\
damfunction(long ipp, vector \&kappa)
\\
\end{tabular}
}
%\end{center}
%%%%%%%%%%%%%%%%%%%%%%%%%%%%%%%%%%%%%%%%%%%%%%%%%%%%%%%%%%%%%%%%%%%%%%%%%%%%%%%%%%%%%%%%
%%%%%%%%%%%%%%%%%%%%%%%%%%%%%%%%%%%%%%%%%%%%%%%%%%%%%%%%%%%%%%%%%%%%%%%%%%%%%%%%%%%%%%%%
\section{Handling with viscoplasticity}

Theory of viscoplasticity applied to problems with small strains is described in Section \ref{sectvisplasmatmodels}.
In the MEFEL code, material of viscoplasticity are used in connection with time dependent mechanical problems with
negligible inertial forces.


\begin{itemize}
\item {\sf void mechmat::readmatchar (FILE *in)}
\item {\sf void mechmat::matstiff (matrix \&d,long ipp)}
\item {\sf void mechmat::damfuncpar(long ipp, vector \&eps, vector \&kappa)}
\item {\sf double mechmat::damfunction(long ipp, vector \&kappa, vector \&eps)}
\item {\sf void updateipval ()}
\item {\sf void mechmat::computenlstresses (long ipp)}
\end{itemize}


%\section{Handling with viscosity}
%\section{Handling with damage}

%%%%%%%%%%%%%%%%%%%%%%%%%%%%%%%%%%%%%%%%%%%%%%%%%%%%%%%%%%%%%%%%%%%%%%%%%%%%%%%%%%%%%%%%
%%%%%%%%%%%%%%%%%%%%%%%%%%%%%%%%%%%%%%%%%%%%%%%%%%%%%%%%%%%%%%%%%%%%%%%%%%%%%%%%%%%%%%%%
\section{Layered problems}

Layered problems must be described more carefully than usual problems because there are many possibilities
of implementation. There are no layered elements because each layer is covered by the finite element mesh
with usual elements. Continuity is enforced by Lagrange multipliers which are defined at nodes. Projection
of meshes from all layers into tangential plane must lead to one mesh. If the number of layers in the problem
is constant everywhere, all meshes in layers must be identical. If the number of layers is variable, there are
missing elements in meshes of particular layers. Consider any node of mesh of any layer and normal vector to
the plate or shell. Then meshes in other layers must contain node on this normal vector. The situation is
depicted in Figure ??. Nodes lying on one normal vector to the plate or shell create layered node. Therefore
two types of nodes are used in layered problems, usual node and layered node. Usual node is represented
by classes {\sf gnode} and {\sf node} while layered node is represented by the class {\sf lgnode}.

%%%%%%%%%%%%%%%%%%%%%%%%%%%%%%%%%%%%%%%%%%%%%%%%%%%%%%%%%%%%%%%%%%%%%%%%%%%%%%%%%%%%%%%%
%%%%%%%%%%%%%%%%%%%%%%%%%%%%%%%%%%%%%%%%%%%%%%%%%%%%%%%%%%%%%%%%%%%%%%%%%%%%%%%%%%%%%%%%
\section{Problems with several subdomains}

Problem type lin\_floating\_subdomain solves mechanical problems where at least two domains
are connected by Lagrange multipliers together. It is tool, e.g., for solution of interaction
of reinforcement and the matrix.

The implementation is based on several subdomains used in the code. The first subdomain
should be constrained properly and it should contain no rigid body motions. All other
subdomains may be unconstrained and may contain rigid body motions. They are connected
to the first subdomain by Lagrange multipliers. The problem is solved by the modification
of the FETI method which is domain decomposition method.


In order to save memory and computational work, several rules have to be used. The rules deal
with node and element ordering.


\subsection{Nodes}

The number of nodes in the problem nn has to be equal to the sum of numbers of nodes defined
on particular subdomains. Each subdomain has to be covered by a finite element mesh with nodes.
The implementation is based on matching meshes. There are doubled nodes along boundary
between two subdomains. The same displacements of the doubled nodes are enforced by Lagrange
multipliers.

Nodes belonging to the first subdomain have to be defined first. Then, nodes belonging to the second
subdomain have to be defined in the list of nodes and so on. The ordering of nodes is important
because the numbering of unknowns depends on it. Wrong ordering of nodes will result in code
failure because submatrices are selected from the whole matrix and improper values will be used.

Definition of interaction is realized by lgnodes. After the list of all nodes, additional
list of generalized nodes is required. Generalized node contain information which two or
more nodes are connected and should be constrained by Lagrange multipliers. Number of connected
nodes and list of connected nodes have to be mentioned for each generalized node.

\subsection{Constrained nodes}

Constrained nodes should be defined only for the fisrt subdomain which has to be supported properly.
It means, the first subdomain contains no rigid body modes.

\subsection{Elements}

The total number of all elements is defined first. 
Then, number of elements for each subdomain is defined. Array of the first elements on
subdomains is assembled. Finally, 

%%%%%%%%%%%%%%%%%%%%%%%%%%%%%%%%%%%%%%%%%%%%%%%%%%%%%%%%%%%%%%%%%%%%%%%%%%%%%%%%%%%%%%%%
%%%%%%%%%%%%%%%%%%%%%%%%%%%%%%%%%%%%%%%%%%%%%%%%%%%%%%%%%%%%%%%%%%%%%%%%%%%%%%%%%%%%%%%%
\section{Modelling of Gradual Construction}

Modelling of gradual construction is based on {\sf problemtype} growing\_mech\_structure.
It is assumed especially for modelling of casting of concrete structures. It can be used
also for any problems with variable number of nodes and elements.

There are small changes in input files in comparison with the classical time dependent
problems ({\sf problemtype} mech\_timedependent\_prob=15).
In the classical problems (with constant numbers of nodes
and elements), only supported nodes or nodes with prescribed displacements are mentioned
in the list of constrained nodes. 0 denotes supported nodes, negative numbers denote
prescribed displacements and positive numbers except 1 denote joined degrees of freedom.
Number 1 may not be read because it is assigned automatically in the code. In problems
with changing numbers of nodes and elements, all nodes must be mentioned at the list of
constrained nodes. Positive numbers are used for definition of time functions which
describe actual state/status of node. Time function returns 0 for nodes which are
switched off and returns 1 for nodes which are switched on. Negative numbers serves
as definition of prescribed displacements and positive numbers except 1 serves for
definition of joined degrees of freedom.
Values of time functions are copied to array cn located at objects gnodes and then the
classical treatment is used. Especially, function gencodnum () from the class gtopology
is used.

Changing status of nodes is not enough for complex description of problems with variable
number of elements and nodes. Therefore, time function must be also defined for elements.
It is important to emphasize, that node time functions and element time functions must be
compatible, otherwise, unexpectable response is computed by the code. Element time functions
id are read after code numbers on element and they are located in objects gelements.

Time functions themself are read after initial values and before outdriver description.
It is convenient to use the same time functions for nodes and elements. This approach
helps to keep compatibility between nodes and elements.


Variable leso is allocated and set up at the function Gtm->lneso\_init (); called in the function 
mechtop::read (FILE *in). At the same time, variables gelements[i].auxinf are set up to zero.
Arrays nodval, nvgraph, nvforce at nodes and the array initdispl at elements are allocated
and cleaned in the function mechtop::alloc\_growstr (). The function is called in the function
mefel\_init (char *argv,stochdriver *stochd).


%%%%%%%%%%%%%%%%%%%%%%%%%%%%%%%%%%%%%%%%%%%%%%%%%%%%%%%%%%%%%%%%%%%%%%%%%%%%%%%%%%%%%%%%
%%%%%%%%%%%%%%%%%%%%%%%%%%%%%%%%%%%%%%%%%%%%%%%%%%%%%%%%%%%%%%%%%%%%%%%%%%%%%%%%%%%%%%%%
\section{Stochastic and Fuzzy Computations}

MEFEL enables solution of stochastic and fuzzy computations. The class probdesc contains variable
stochasticcalc which describes type of computation.

\begin{center}
\begin{tabular}{|c|l|}
\hline
0 & deterministic problem
\\[2mm] \hline
1 & stochastic or fuzzy problem,
\\
  & input data are read all at once at the beginning
\\
  & input data are stored in RAM
\\
  & output data are stored in RAM
\\[2mm] \hline
2 & stochastic or fuzzy problem,
\\
  & input data are read sequentially
\\
  & input data are not stored in RAM
\\
  & output data are printed out after each step
\\[2mm] \hline
3 & fuzzy problem
\\
  & input data are generated automatically in the code
\\
  & output data are stored in the form of fuzzy numbers
\\ \hline
\end{tabular}
\end{center}

There are non-deterministic materials, cross sections and nodal loads.

Output: printed nodal displacements, element output


Number of samples is read or computed then.

Computation of output values

Two important variables are obtained, nstochvar denotes the number
of non-deterministic variables, nprunknowns denotes number of
output variables.
Object Stochdriver contains two arrays, avi and avo. avi stores input
data in one sample while avo contains output data. avi contains nstochvar
components, avo contains nprunknowns unknowns.
