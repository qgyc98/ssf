\documentclass{article}
\usepackage{graphics,epsfig}
\begin{document}

Compute displacements of a brick which is clamped at one face and loaded at the
opossite one. The sizes of the brick are 10 m ($x$ direction), 2 m ($y$ direction) and
3 m ($z$ direction). There are 6 elements along the $x$ direction, 2 elements
along the $y$ direction and 3 elements along the $z$ direction.

This very simple mesh could be generated by our simple generator ghex from the
PRG/PREP/GENER. The statement is

\noindent
ghex brick.top 10 2 3 \hspace{4mm} 6 2 3

There are 84 nodes and 36 elements. The nodes are divided into 5 groups with respect
to their property. All finite elements have property 0. Nodes in plane $yz$ have property
0 and they are clamped, the nodes on the top surface have property 1, the nodes from
the front surface (parallel to $yz$) have property 2 and they are loaded, the nodes
on the bottom surface have property 3 and all other nodes have property 4. It is depicted
in Figure \ref{brick}.

\begin{figure}
\begin{center}
\includegraphics[height=80mm]{brick.eps}
\caption{Scheme of the brick.}
\label{brick}
\end{figure}

\noindent
Figure 1. - it is clear

\noindent
Figure 2. - it is clear

\noindent
Figure 3. - it is clear

\noindent
Figure 4. - one load case is assumed

\noindent
Figure 5. - as was written above, there are 5 groups of nodes, therefore 5 properties are
defined, there is one group of elements

\noindent
Figure 6. - it is clear

\noindent
Figure 7. - you must define the number of degrees of freedom for each group of node, other
definitions are optional

\noindent
Figure 8. - this is 3D problem, therefore 3 degrees of freedom are used

\noindent
Figure 9. - the nodes from the group number 0 are supported, because 3 DOFs were defined at node,
you must define support for them; 0 means supported, 1 means free DOF

\noindent
Figure 10. - also other groups of nodes have 3 DOFs per node

\noindent
Figure 11. - the nodes from the group number 2 are loaded, because 3 DOFs were defined at node,
you must define load for them, you must also chose the load case (there is only 1 load case)

\noindent
Figure 12. - as was mentioned, there is one group of elements

\noindent
Figure 13. - type of element and type of material model must be defined

\noindent
Figure 14. - hexahedral element with tri-linear approximation functions is used; starin and stresses
are computed at integration points

\noindent
Figure 15. - type of material (elastic isotropic material) is defined, in dbmat file should be
material parameters for several material models, the number of set of parameters is required here
(if there are 2 materials in the solved structure, some finite elements have property 0, some of
them property 1, this is the indicator of appropriate material)

\noindent
Figure 16. - the JKTK format of topology file is selected because simple generator is used,
paths to dbmat and dbcrs files is mentioned


\end{document}